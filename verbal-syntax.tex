\documentclass[grammar]{subfiles}
\begin{document}
\chapter{Verbal Syntax}
\label{ch:verbal_syntax}

The preceding chapters dealt primarily with the morphology of Qevesa, with
only occasional references to principles of usage. All major aspects of word
formation have been covered. The focus of this document shifts to syntax: how
the language assembles words into meaningful sentences.

%  Qevesa syntax is fairly fluid, and tends towards being largely
%  left-branching or head-final. The only strict requirement of a sentence is
%  that the verb must occur last, and that the topic, if present, must be
%  first. All other elements may be freely ordered by importance. The general
%  word order is thus \emph{\textsc{topic–comment–verb}}.
%
\section{The Structure of the Verb Phrase} 
\label{sec:vp_structure}

The general structure of a verb phrase is:

\begin{exe}
  \ex \textsc{[adverbs] [main verb] [auxiliary verb]}
\end{exe}

Auxiliary verbs are notable in that they carry all the inflections, with the
main verb preceding them in the infinitive.  Any adverbs and modifiers precede
the main verb, and generally do not agree with it in any way. 

% Verb phrases are always verb-final. Arguments of the verb precede it, as do
% modifiers and auxiliary verbs. The topic of the verb, if present, must occur as
% the first element of the phrase; comments regarding the topic precede the verb
% and its modifiers but follow the topic.

%  \subsection{Verbal Topic} \label{ssec:syn_verbal_topic}
%
%  Qevesa is a \emph{topic-prominent} language, which means that the topic is
%  semantically the most important argument of the verb. The topic is indicated
%  by the noun phrase in the nominative case, with the syntactic role marked on
%  the verb. Any of the constituent phrases can be marked as the topic; it
%  usually consists of the element that the speaker considers to be the most
%  important.
%
%  Voice and valency-adjusting operations are not prominent features of Qevesa.
%  The ‘active’ voice, and most common construction, is to mark the agent or
%  donor of the verb as the topic. The ‘passive’ voice can be indicated by
%  marking the patient, recipient or theme of the verb as the topic.
%
%  \begin{exe} \ex \begin{xlist} \ex EXAMPLE \ex EXAMPLE \ex EXAMPLE
%  \end{xlist} \end{exe}
%
%  As can be seen from the above examples, changing the topic focus is
%  semantically equivalent to the simpler valency-adjusting operations; that is
%  active/passive voice contrasts. The topic focus is not strictly limited to
%  the agent patient or theme, however: the secondary nominal cases can also be
%  promoted to topic, and there are affixes to indicate this on the
%  verb\footnotemark{}.
%
%  \footnotetext{See Section~\ref{sssec:vp_topic_secondary},
%  page~\pageref{sssec:vp_topic_secondary}}
%
%  \begin{exe} \ex \begin{xlist} \ex EXAMPLE \ex EXAMPLE \ex EXAMPLE
%  \end{xlist} \end{exe}
%
%
%  %\section{Noun Phrases} %\label{sec:syn_noun_phrases}
%
%  %Noun phrases 
%
%  %\section{Relative Clauses} %\label{sec:syn_relative_clauses}
%
%  %Qevesa does not employ relative pronouns to related relative clauses to
%  their antecedents. Instead, the relative clause directly modifies the noun
%  phrase as an attributive verb phrase, optionally agreeing in case.
%
%  %\emph{To be written…}


\section{Morphosyntactic Alignment}
\label{sec:vp_trigger_system}


The Eastern Teranean languages possess a fairly unusual morphosyntactic
alignment that blends aspects of nominative-accusative and ergative-absolutive
systems.  Instead of the more usual personal marking, where the personal affix
indicates the subject of the verb, the suffixes in Qevesa indicate the person
and the role, that is, whether they are an agent, a patient, or some other
argument that has been promoted.  The noun in the direct case, which may be
omitted if it is a pronoun, is the argument to which the personal suffix
refers to. 

%The direct case often, though not always, corresponds with the focus or topic of the phrase.

This system is roughly analogous to how other languages use voice.


\subsection{Agent Trigger}
\label{ssec:vp_agt_trigger}

The agent trigger indicates that the noun phrase in the direct case is the
voluntary experiencer of an intransitive verb or the agent of a transitive
verb.  This trigger is equivalent to the active voice in other languages.

\begin{exe}
  \ex \conlang{Japphútin.}
  \gll japphút-in\\
  speak\textsc{\bs prog-3sg.agt}\\
  \glt She is speaking.
  \ex \conlang{Rekáteš jarkúten.}
  \gll rekát-e-š jarkút-en\\
  book\textsc{-indef-abs} write\textsc{\bs prog-1sg.agt}\\
  \glt I am writing a book.
\end{exe}

%  \begin{exe}
%    \ex \conlang{Šaicua sošima jem musal nosiruek.}
%    \glll Šaicu-a sošim-a jem musal nosi-ru-ek\\
%    \textsc{dist.anim-def} girl\textsc{-foc} \textsc{1sg.nom} think\bs\textsc{inf} \textsc{neg\bs cess-pot-obl}\\
%    {that} {girl} {I} {not} {think} {can stop}\\
%    \glt I cannot stop thinking about that girl.
%  \end{exe}

% Generally only animate nouns may be agents; to describe an action involving an
% inanimate noun as agent, a construction using the oblique trigger and the
% instrumental case is used instead. 

\subsection{Patient Trigger}
\label{ssec:vp_pat_trigger}

The patient trigger indicates that the noun phrase in the direct case is the
involuntary experiencer of an intransitive verb; the patient of a transitive
verb; and the recipient of a ditransitive verb.  This trigger is roughly
equivalent to the passive and mediopassive voices in other languages. 

Only animate nouns may be voluntary agents of intransitive verbs; inanimate
nouns are always marked as involuntary experiencers of intransitive verbs.
Furthermore, some intransitive verbs are always involuntary, regardless of
animacy. 

\begin{exe}
  \ex \conlang{Rekáte jem kojuroš.}
  \gll rekát-e-∅ jem kojur-oš\\
  book\textsc{-indef-dir} \textsc{1sg.erg} read\textsc{\bs perf-3sg;inanim.pat}\\
  \glt A book was read by me.
  \ex \conlang{Rekáte kojuroš.}
  \gll rekát-e-∅ kojur-oš\\
  book\textsc{-indef-dir} read\textsc{\bs perf-3sg;inanim.pat}\\
  \glt A book was read.
  \ex \conlang{Mi náčoruš.}
  \gll mi-∅ náčoru-š\\
  \textsc{3sg-dir} sneeze\textsc{\bs perf-3sg.pat}\\
  \glt He sneezed.
\end{exe}

\subsection{Oblique Trigger}
\label{ssec:vp_obl_trigger}

The oblique trigger indicates that the noun phrase in the direct case is
something other than the agent or patient of a transitive verb.  For
ditransitive verbs it normally indicates the theme or direct object.

Another common use of the oblique trigger is to express an inanimate agent of a
verb. In this case, the noun will be double-marked with both the instrumental
case and the direct case. 


\section{Aspect}
\label{vp:sec_aspect}

Qevesa verbal morphology indicates aspect instead of tense, to the extent that
there is no means to indicate tense on the verb phrase; the closest
approximation is periphrastically by means of adverbial phrases referring to
time. 


\subsection{Perfective}
\label{vp:ssec_perfective}

The perfective aspect indicate activities viewed as a single whole.  It is
typically used to speak of singular events completed in the past, but may also
be used to speak of actions without internal structure.

\begin{exe}
  \ex \conlang{Kesselanti tékujen}
  \gll Kessel-anti ték-u-jen\\
  Kessel\textsc{-all} go\textsc{-perf-1sg.agt}\\
  \glt I went to Kessel.
  \ex \conlang{Mi kori lamiztivaš márun.}
  \gll Mi-∅ kori lamizti-v-aš már-u-n\\ 
  \textsc{3sg-dir} three ballgame\textsc{-du-abs} see\textsc{-perf-3sg.agt}\\
  \glt He has watched three ballgames.
\end{exe}

% I wrote / I have written
% I was an architect (and still am, depending on context).


\subsection{Experiential}
\label{vp:ssec_experiential}

The experiential aspect ascribes to a subject the property of having
experienced the event.  There is some overlap between the perfective and
experiential aspects, but the experiential carries connotations of
‘completeness’ that the perfective does not.   

\begin{exe}
  \ex \conlang{Mi kori lamiztivaš máran.}
  \gll Mi-∅ kori lamizti-v-aš már-a-n\\ 
  \textsc{3sg-dir} three ballgame\textsc{-du-abs} see\textsc{-exp-3sg.agt}\\
  \glt He has watched three ballgames [in his entire life].
  \ex \conlang{Kovelnapalli póriaš máratan.}
  \gll ko-velnapa-lli póri-a-š már-a-tan\\
  \textsc{prox}-tomorrow-\textsc{ess} city\textsc{-def-abs} see\textsc{-exp-2sg.agt}\\
  \glt Tomorrow you will have seen [everything in] the city.
\end{exe}

\subsection{Momentane}
\label{vp:ssec_momentane}

The momentane aspect indicates brief single-time activities or states.

% A bolt of lighting struck the tree.
% The mouse squeaked.

\subsection{Progressive and Durative}
\label{vp:ssec_progressive_durative}

The progressive aspect indicates ongoing actions with a change of state.  

\begin{exe}
  \ex\label{ex:vp_putting_on_clothes} \conlang{Veráninaš javrúnen.}
  \gll verán-in-aš javrún-en\\
  clothes\textsc{-part-abs} wear\textsc{\bs prog-1sg.agt}\\
  \glt I am putting on clothes.
\end{exe}

The durative aspect indicates ongoing actions without a change of state, or
actions which last some time.

\begin{exe}
  \ex\label{ex:vp_wearing_clothes} \conlang{Veráninaš javránen.}
  \gll verán-in-aš javrán-en\\
  clothes\textsc{-part-abs} wear\textsc{-dur-1sg.agt}\\
  \glt I am wearing clothes.
\end{exe}

There are a number of verb patterns that imply either the progressive or the durative as their imperfective aspect, 
or have subtly different meanings depending on which is used.  

Adjectival verbs use the progressive aspect to indicate a change to 
the quality described by the adjective, and the durative is used 
%Verbs that imply a change of state (such as \conlang{navronu} “dress oneself”) will use the progressive aspect 

%Some verbs which imply a change of state, such as \conlang{navronu} 

% I am putting on clothes.
% I am hanging the painting on the wall.
% It is raining in Kirua:  Kiruazi apšórak.


% I am wearing clothes.
% The picture is hanging on the wall.

% I can't stop sneezing
% sneeze-PASS-SUBJ stop-I

\subsection{Habitual}
\label{vp:ssec_habitual}

The habitual aspect indicates actions that occur habitually.  

Like the
progressive, it may also describe intermittent actions, but in a general sense.

% I walk to work (every day).

\section{Modality}
\label{sec:vp_modality}


\subsection{Indicative Mood}
\label{ssec:vp_indicative}

The indicative mood is used for factual statements and positive beliefs,
and as such is the default mood.  


\subsection{Mirative Mood}
\label{ssec:vp_mirative}

The mirative mood is used to express surprise and also doubt, irony,
sarcasm.  It is used to express statements contrary to the speaker’s
expectations or state of mind.


\subsection{Conditional Mood}
\label{ssec:vp_conditional}

The conditional mood is used to speak of an event whose realization is
dependent upon another condition. 


\subsection{Optative Mood}
\label{ssec:vp_optative}

The optative mood is used to express hopes, wishes and desires.


\subsection{Potential Mood}
\label{ssec:vp_potential}

The potential mood indicates that, in the opinion of the speaker, the
action or occurrence is considered likely.  It can also be used to express that
one has the ability to do something.


\subsection{Imperative Mood}
\label{ssec:vp_imperative}

The imperative mood is used for commands and requests. 


\section{Voice and Valency-Altering Operations}
\label{sec:vp_voice}


%\section{Auxiliary Verbs}
%\label{sec:vp_auxiliary}
%
%Auxiliary verbs are used to form periphrastic constructions.  

%\subsection{The Copulae}
%\label{ssec:vp_copulae}
%
%The most commonly used auxiliary verbs are the copulae, which are used to form a
%variety of constructions.  The transitive and intransitive forms have different
%roots, \conlang{vaku} and \conlang{azu}, both of which conjugate similarly to
%ordinary verbs.  The conjugated forms of the copula are listed in
%\cref{tab:vp_copulae}.
%
%\begin{table}[h!]\small\capstart
%  \subfloat[Transitive]{
%  \begin{tabular}{BFl Sl -l -l -l -l -l}
%    \toprule
%    \multicolumn{2}{Fc}{\rowstyle{\bfseries}Aspect} & \multicolumn{5}{-c}{Mood} \\
%    \rowstyle{\scshape} & & \acs{ind} & \acs{mir} & \acs{cond} & \acs{opt} & \acs{pot} \\
%    \midrule
%    Perfective  & \acs{perf} & viku & vikun & vikus & vikut & vikuru \\
%    Momentane   & \acs{momt} & vika & vikan & vikas & vikat & vikaru \\
%    Progressive & \acs{prog} & vaku & vakun & vakus & vakut & vakuru \\
%    Durative    & \acs{dur}  & vaki & vakin & vakis & vakit & vakiru \\
%    Habitual    & \acs{hab}  & voku & vokun & vokus & vokut & vokuru \\
%    Inchoative  & \acs{inch} & voka & vokan & vokas & vokat & vokaru \\
%    Cessative   & \acs{cess} & voki & vokin & vokis & vokit & vokiru \\
%    \bottomrule
%  \end{tabular}
%  }\\
%  \subfloat[Intransitive]{
%  \begin{tabular}{BFl Sl -l -l -l -l -l}
%    \toprule
%    \multicolumn{2}{Fc}{\rowstyle{\bfseries}Aspect} & \multicolumn{5}{-c}{Mood} \\
%    \rowstyle{\scshape} & & \acs{ind} & \acs{mir} & \acs{cond} & \acs{opt} & \acs{pot} \\
%    \midrule
%    Perfective  & \acs{perf} & izu & izun & izus & izut & izuru \\
%    Momentane   & \acs{momt} & iza & izan & izas & izat & izaru \\
%    Progressive & \acs{prog} & azu & azun & azus & azut & azuru \\
%    Durative    & \acs{dur}  & azi & azin & azis & azit & aziru \\
%    Habitual    & \acs{hab}  & ozu & ozun & ozus & ozut & ozuru \\
%    Inchoative  & \acs{inch} & oza & ozan & ozas & ozat & ozaru \\
%    Cessative   & \acs{cess} & ozi & ozin & ozis & ozit & oziru \\
%    \bottomrule
%  \end{tabular}
%  }
%  \caption{Conjugation of the copulae \label{tab:vp_copulae}}
%\end{table}

%  \begin{exe}
%    \ex \conlang{Šaicua sošima jem musal nosiruek.}
%    \glll Šaicu-a sošim-a jem musal nosi-ru-ek\\
%    \textsc{dist.anim-def} girl\textsc{-foc} \textsc{1sg.nom} think\bs\textsc{inf} \textsc{neg\bs cess-pot-obl}\\
%    {that} {girl} {I} {not} {think} {can stop}\\
%    \glt I cannot stop thinking about that girl.
%  \end{exe}

% \subsection{Negation}
% \label{ssec:vp_negation}
% 
% Verbs in Qevesa are negated by using a negative auxiliary verb, \conlang{nasu}.
% The main verb appears in the infinitive, with the copula taking its
% inflections, as in a standard auxiliary construction.  The conjugated forms of
% the negative verb are listed in \cref{tab:vp_negation}.
% 
% 
% \begin{table}[h!]\small\capstart
%   \begin{tabular}{BFl Sl -l -l -l -l -l -l}
%     \toprule
%     \multicolumn{2}{Fc}{\rowstyle{\bfseries}Aspect} & \multicolumn{6}{-c}{Mood} \\
%     \rowstyle{\scshape} & & \acs{ind} & \acs{imp} & \acs{mir} & \acs{cond} & \acs{opt} & \acs{pot} \\
%     \midrule
%     Perfective  & \acs{perf} & nisu & nisuk & nisun & nisus & nisut & nisuru \\
%     Momentane   & \acs{momt} & nisa & nisak & nisan & nisas & nisat & nisaru \\
%     Progressive & \acs{prog} & nasu & nasuk & nasun & nasus & nasut & nasuru \\
%     Durative    & \acs{dur}  & nasi & nasik & nasin & nasis & nasit & nasiru \\
%     Habitual    & \acs{hab}  & nosu & nosuk & nosun & nosus & nosut & nosuru \\
%     Inchoative  & \acs{inch} & nosa & nosak & nosan & nosas & nosat & nosaru \\
%     Cessative   & \acs{cess} & nosi & nosik & nosin & nosis & nosit & nosiru \\
%     \bottomrule
%   \end{tabular}
%   \caption{Conjugation of the negative verb \label{tab:vp_negation}}
% \end{table}

% V    nasu - to not              
% 1SG  nisuen   nisunen    nisusen    nisuten    nisuruen
% 2SG  nisutan  nisuntan   nisustan   nisuttan   nisurutan
% 3SG  nisun    nisunin    nisusin    nisutin    nisurun
% 2DU  nisutun  nisuntun   nisustun   nisuttun   nisurutun
% 3DU  nisumin  nisunumin  nisusumin  nisutumin  nisurumin
% 1PL  nisusán  nisunsán   nisussán   nisutsán   nisurusán
% 1EX  nisučen  nisunčen   nisusčen   nisutčen   nisuručen
% 2PL  nisután  nisuntán   nisustán   nisuttán   nisurután
% 3PL  nisumin  nisunamin  nisusamin  nisutamin  nisurumin

% 1SG  nisuken   nisunen    nisusen    nisuten    nisuruen
% 2SG  nisuktan  nisuntan   nisustan   nisuttan   nisurutan
% 3SG  nisukin    nisunin    nisusin    nisutin    nisurun
% 2DU  nisuktun  nisuntun   nisustun   nisuttun   nisurutun
% 3DU  nisukumin  nisunumin  nisusumin  nisutumin  nisurumin
% 1PL  nisuksán  nisunsán   nisussán   nisutsán   nisurusán
% 1EX  nisukčen  nisunčen   nisusčen   nisutčen   nisuručen
% 2PL  nisuktán  nisuntán   nisustán   nisuttán   nisurután
% 3PL  nisukamin  nisunamin  nisusamin  nisutamin  nisurumin





%  \subsection{Evidentiality}
%  \label{ssec:vp_evidentiality}

%  Evidentiality may also be expressed by means of auxiliary verbs.  Qevesa
%  possesses a set of auxiliary verbs which distinguish four degrees of
%  evidentiality: witness, reportative, inferential, and assumptive. 

%  All of the roots of the evidential auxiliaries are also verbs in their own
%  right.  However, they conjugate as Form VIII verbs, with some slightly
%  irregular pattern forms.  Their conjugation is given in
%  \cref{tab:vp_evidentiality_conjugation}.

%  \subsubsection{Witness}
%  \label{sssec:vp_evd_witness}

%  The witness degree of evidentiality is denoted by the verb \conlang{murru},
%  meaning ‘to see’.  It is used when the speaker was a witness to the event.

%  \subsubsection{Reportative}
%  \label{sssec:vp_evd_reportative}

%  The reportative degree of evidentiality is denoted by the verb
%  \conlang{łukšu}, which has the same consonantal root as the verb
%  \conlang{łukuš} ‘to hear (speech)’.

%  \subsubsection{Inferential}
%  \label{sssec:vp_evd_inferential}

%  The inferential degree of evidentiality is denoted by the verb
%  \conlang{kučtu}.  It is used when the speaker infers that the event occurred
%  but was not a witness.

%  \subsubsection{Assumptive}
%  \label{sssec:vp_evd_assumption}

%  The assumption degree of evidentiality is denoted by the verb
%  \conlang{quspu}.  It is used when the speaker is making an assumption about
%  the occurrence of the event.

\end{document}
