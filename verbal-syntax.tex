\documentclass[grammar]{subfiles}
\begin{document}
	\chapter{Verbal Syntax}
	\label{ch:verbal-syntax}

	% Copied into constituent-order-typology.tex
	The preceding chapters dealt primarily with the morphology of Qevesa, with only occasional references to principles of usage. All major aspects of word formation have been covered. The focus of this document shifts to syntax: how the language assembles words into meaningful sentences.

	Qevesa verbs must agree in person and number with the topic of the sentence. Verbs are marked for the syntactic role of the topic; when this marking indicates a sufficient degree of information, such as a pronoun in the first or second person, the topical phrase may be omitted.

	Qevesa syntax is fairly fluid, and tends towards being largely left-branching or head-final. The only strict requirement of a sentence is that the verb must occur last, and that the topic, if present, must be first. All other elements may be freely ordered by importance. The general word order is thus \emph{\textsc{topic–comment–verb}}.

	\section{Verb Phrases}
	\label{sec:syn_verb_phrases}

	Verb phrases are always verb-final. Arguments of the verb precede it, as do modifiers and auxiliary verbs. The topic of the verb, if present, must occur as the first element of the phrase; comments regarding the topic precede the verb and its modifiers but follow the topic.

	\subsection{Verbal Topic}
	\label{ssec:syn_verbal_topic}

	Qevesa is a \emph{topic-prominent} language, which means that the topic is semantically the most important argument of the verb. The topic is indicated by the noun phrase in the nominative case, with the syntactic role marked on the verb. Any of the constituent phrases can be marked as the topic; it usually consists of the element that the speaker considers to be the most important.

	Voice and valency-adjusting operations are not prominent features of Qevesa. The ‘active’ voice, and most common construction, is to mark the agent or donor of the verb as the topic. The ‘passive’ voice can be indicated by marking the patient, recipient or theme of the verb as the topic.

	\begin{exe}
		\ex
		\begin{xlist}
			\ex EXAMPLE
			\ex EXAMPLE
			\ex EXAMPLE
		\end{xlist}
	\end{exe}

	As can be seen from the above examples, changing the topic focus is semantically equivalent to the simpler valency-adjusting operations; that is active/passive voice contrasts. The topic focus is not strictly limited to the agent patient or theme, however: the secondary nominal cases can also be promoted to topic, and there are affixes to indicate this on the verb\footnotemark{}.

	\footnotetext{See Section~\ref{sssec:vm_topic_secondary}, page~\pageref{sssec:vm_topic_secondary}}

	\begin{exe}
		\ex
		\begin{xlist}
			\ex EXAMPLE
			\ex EXAMPLE
			\ex EXAMPLE
		\end{xlist}
	\end{exe}


	%\section{Noun Phrases}
	%\label{sec:syn_noun_phrases}

	%Noun phrases 

	%\section{Relative Clauses}
	%\label{sec:syn_relative_clauses}

	%Qevesa does not employ relative pronouns to related relative clauses to their antecedents. Instead, the relative clause directly modifies the noun phrase as an attributive verb phrase, optionally agreeing in case.

	%\emph{To be written…}

\end{document}
