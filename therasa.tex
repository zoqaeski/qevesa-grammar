\documentclass[grammar]{subfiles}
\begin{document}
  \chapter{Common Therasa}
  \label{app:therasa}

  \section{Phonology}
  \label{sec:th_phonology}

  \subsection{Vowel inventory}
  \label{ssec:th_vowels}

  \begin{table}[htpb]\small\capstart
        \begin{tabular}{BFl -c -c -c}
          \toprule
          \SetRowStyle{\bfseries} & Front & Central & Back \\
          \midrule
          Close & i &      & u \\
          Mid   & e &      & o \\
          Open  &   & a \\
          \bottomrule
        \end{tabular}
      \caption{Therasa vowel phonemes\label{tab:th_vowels}}
  \end{table}


  There were five distinct vowel phonemes in Therasa, listed in Table~\ref{tab:th_vowels}.  
  %The diphthongs were /ai ei ou au 
  
  \subsection{Consonants}
  \label{ssec:th_consonants}

  \begin{table}[htpb]\small\capstart
      \begin{tabular}{BFl -l -c -c -c -c -c -c}
        \toprule
        \multicolumn{2}{Fl}{\SetRowStyle{\bfseries}} & Bilabial & Denti-alveolar & Postalveolar & Palatal & Velar & Glottal \\
        \midrule
        Nasal                     &           & m  & \tsb{n}   &    & ɲ \\
        \multirow{3}{*}{Plosive}  & Unvoiced  & p  & \tsb{t}   &    & c & k  \\
                                  & Aspirated & pʰ & \tsb{t}ʰ  &    &   & kʰ \\
                                  & Voiced    & b  & \tsb{d}   &    & ɟ & g  \\
        Affricate                 &           &    & ts dz     & tʃ \\
        Fricative                 &           &    & s         & ʃ  & & & h  \\
        Lateral                   &           &    & l \\
        Rhotic                    &           &    & r \\
        \bottomrule
      \end{tabular}
      \caption{Consonants\label{tab:th_consonants}}
  \end{table}

  Therasa possessed twenty-two consonants, realised as in Table~\ref{tab:th_consonants}.  

\end{document}

