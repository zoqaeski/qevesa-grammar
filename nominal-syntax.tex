\documentclass[grammar]{subfiles}
\begin{document}
\chapter{Nominal Syntax}
\label{ch:nominal_syntax}


\section{Structure of the Noun Phrase}
\label{sec:ns_structure}

The noun phase…


\section{Number}
\label{sec:ns_number}

Number marking in Qevesa function in a somewhat unusual manner in that every
noun has an inherent “natural” number, which is its default, unmarked form.
The suffixes are appended to indicate that the quantity (and definiteness)
differs from what is expected.  Most nouns default to the implicit singular;
some nouns, such as body parts and items of clothing that come in pairs are
implicitly dual (\conlang{méri} “eyes”); and other nouns may be implicitly plural or
partial (particularly uncountable nouns). 

The dual number functions to indicate exact quantities.  By itself, it
indicates exactly two of the noun; however, it is also used when the noun is
preceded by a modifier that indicates an exact quantity, such as a number word.

In contrast to the dual, the plural number is used for unspecified quantities
greater than the singular.  The plural suffix may also encode definiteness,
especially for those nouns whose unmarked form has an implicit number. 

The partitive is used to express partialness or inexact quantities. 
% It may also be used in an atelic sense. ???


\section{Cases}
\label{sec:ns_cases}

%


\subsection{Direct}
\label{ssec:ns_direct_case}

The direct case marks the topic of the verb phrase.  This may be the
experiencer (both voluntary and involuntary) of an intransitive verb, the agent
or patient of a transitive verb, or (less commonly) some other argument of the
verb.  In this latter case, the direct suffix is stacked onto the other case
suffix. 


Typically, animate nouns in the direct case are the voluntary experiencers or
agents of verbs, and inanimate nouns in the direct case are experiencers or
patients. 


\subsection{Nominative}
\label{ssec:ns_nominative_case}

The nominative case marks the voluntary experiencer of an intransitive verb, or
the agent of a transitive verb.  Inanimate nouns cannot be marked with the
nominative case, because an inanimate entity is considered incapable of acting
of its own accord. 


\subsection{Absolutive}
\label{ssec:ns_absolutive_case}

The absolutive case marks the involuntary experiencer of an intransitive verb,
the patient of a transitive verb or the recipient of ditransitive verb.


\subsection{Secundative}
\label{ssec:ns_secundative_case}

Qevesa is a secundative language, that is, the recipient of a ditransitive verb
is treated the same as the patient of a monotransitive verb. The secundative
case marks the theme of a ditransitive verb.


\subsection{Genitive}
\label{ssec:ns_genitive_case}

The genitive case indicates the possessor of another noun.  Animate pronomial
possessors are usually indicated by means of a suffix on the possessed noun.


\subsection{Essive}
\label{ssec:ns_essive_case}

The essive case is used to indicate duration and time, as well as temporary
states of being or existence.  


\subsection{Instrumental}
\label{ssec:ns_instrumental_case}

The instrumental case indicates the means by which the action is
performed.  Inanimate agents of verbs are also marked with the instrumental case.  


\subsection{Inessive}
\label{ssec:ns_inessive_case}

The inessive case indicates internal location.  


\subsection{Adessive}
\label{ssec:ns_adessive_case}

The adessive case indicates external location.


\subsection{Illative}
\label{ssec:ns_illative_case}

The illative case indicates motion from the exterior to the interior.


\subsection{Allative}
\label{ssec:ns_allative_case}

The allative case indicates motion towards the noun. 


\subsection{Elative}
\label{ssec:ns_elative_case}

The elative case indicates motion from the interior to the exterior.


\subsection{Ablative}
\label{ssec:ns_ablative_case}

The ablative case indicates motion away from the noun.  It can also be used
in expressions of time and emotion to indicate the beginning of the event or
state. 


\subsection{Comparative}
\label{ssec:ns_comparative_case}

The comparative case indicates a likeness to something, or the
standard to which something is compared.


% \subsection{Use of the Locative Cases}
% \label{ssec:ns_locative_cases}
% 
% The locative cases are logically grouped.  There are two positions (internal
% and external) and three directions (static, movement towards and movement
% away).  Combining these results in the six cases, illustrated in
% \cref{tab:ns_locative_cases}.
% 
% \begin{table}[h!]\small\capstart
%   \begin{tabular}{BFl -l -l}
%     \toprule
%     \rowstyle{\bfseries} & Interior & Exterior \\
%     \midrule
%     Static           & Inessive & Adessive \\
%     Movement towards & Illative & Allative \\
%     Movement away    & Elative  & Ablative \\
%     \bottomrule
%   \end{tabular}
%   \caption{Locative cases\label{tab:ns_locative_cases}}
% \end{table}
% 
% Finer distinctions in location are given with postpositions, which are
% described in \cref{sec:ns_postpositions}.

\end{document}

