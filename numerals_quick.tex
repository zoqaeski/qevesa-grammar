\documentclass[grammar]{subfiles}
\begin{document}
	%\setcounter{secnumdepth}{0}
	\section*{Numerals}
	\label{ch:numerals}

	Qevesa, in common with other Teralo languages, uses a duodecimal or base-12 number system for both integers and fractions Although the primary base is duodecimal, evidence of other bases can be seen in some number forms. It is interesting to note that the words for 1\dec\ to 20\dec\ (with the exception of 15\dec) are unanalysable, but the numerals for 21\dec\ to 23\dec\ are vigesimal\footnotemark{}, as are a number of others.
	\footnotetext{Historically, Qevesa used a vigesimal number system, but influence from neighbouring languages caused a switch to a duodecimal system instead.}

	%\section{Cardinals}
	%\label{sec:num_cardinals}

	The base number words are the cardinal numerals. The stems for numerals are unique in that they include vowels, which leads to their classification as \emph{stems} and not \emph{roots}; if the stems are to be used as consonantal roots, the vowels are dropped. Note that consonant clusters are treated as single consonants for numeric roots. The cardinals from 0\dec\ to 21\dec\ are listed in Table~\ref{tab:num_cardinals}.

	\begin{table}[htpb]\small\capstart
		\begin{center}
			\subfloat{
			\begin{tabular}{|fc|-c|-c|}
				\hline
				\multicolumn{2}{|fc|}{\SetRowStyle{\bfseries}Cardinal} & Pronunciation \tnl
				\hline
				0 & zom				& /θom/	\tnl
				1 & sen				& /sen/	\tnl
				2 & ěti				& /je.ti/	\tnl
				3 & köl				& /kɵl/	\tnl
				4 & qese			& /tɕe.se/	\tnl
				5 & inü				& /i.nʉ/	\tnl
				6 & von				& /ʋon/	\tnl
				7 & ikuş			& /i.kuʂ/	\tnl
				8 & sopri			& /sop.ri/	\tnl
				9 & jok				& /jok/	\tnl
				10\dec & méri	& /meːri/	\tnl
				\hline
			\end{tabular}}\qquad
		\subfloat{
			\begin{tabular}{|fc|-c|-c|}
				\hline
				\multicolumn{2}{|fc|}{\SetRowStyle{\bfseries}Cardinal} & Pronunciation \tnl
				\hline
				11\dec & türe		& /tʉ.re/ \tnl
				12\dec & żel		& /tθel/ \tnl
				13\dec & scem		& /stsem/ \tnl
				14\dec & zpet		& /θpet/ \tnl
				15\dec & żeköl	& /tθekɵl/ \tnl
				16\dec & ksoż		& /ksotθ/ \tnl
				17\dec & pedła	& /peð.wa/ \tnl
				18\dec & pseňa	& /pse.ɲa/ \tnl
				19\dec & sevča	& /seʋ.tɕa/ \tnl
				20\dec & vaudi	& /ʋau.di/ \tnl
				21\dec & vaudi-sen & /ʋau.di sen/  \tnl
				\hline
			\end{tabular}}
			\caption{Cardinal numerals from 0\dec\ to 21\dec\label{tab:num_cardinals}}
		\end{center}
	\end{table}

	\newpage
	The numerals from 19\duo\ to 1B\duo\ are vigesimal, as are some multiples of 50\duo\ (60\dec). Numerals from 20\duo\ to BB\duo\ are suffixed with \emph{-za} /-θa/:

	\begin{exe}
		\ex
		\begin{tabular}[t]{r >{\itshape}l >{/}l<{/}}
			19\duo & vaudi-sen & ʋau.ði sen\\
			1A\duo & vaudi-ěti & ʋau.ði je.ti \\
			1B\duo & vaudi-köl & ʋau.ði kɵl\\
			20\duo & ětiza & je.ti.θa\\
			30\duo & kölza & kɵl.θa\\
			40\duo & qeseza & tɕe.se.θa\\
			50\duo & kölvád \textup{or} inüza & kɵl.ʋaːð/ or /i.nʉ.θa\\
			65\duo & vonza-inü & ʋon.θa i.nʉ\\
			A0\duo & mériza & meːri.θa\\
			BB\duo & türeza-türe & tʉ.re.θa tʉ.re\\
		\end{tabular}
	\end{exe}

	Numerals from 100\duo\ to BBB\duo\ are suffixed with \emph{-toc}:

	\begin{exe}
		\ex
		\begin{tabular}[t]{r >{\itshape}l >{/}l<{/}}
			100\duo & sentoc & sen.tots\\
			200\duo & ěttoc & jetːots\\
			300\duo & költoc & kɵl.tots\\
			409\duo & qesetoc-jok & tɕe.se.tots jok\\
			752\duo & ikuştoc-kölvád-ěti & i.kuʂ.tots kɵl.ʋaːð je.ti\\
		\end{tabular}
	\end{exe}

	%\newpage
	Numerals from 1000\duo\ to BBBB\duo\  use the suffix \emph{-síva}:

	\begin{exe}
		\ex
		\begin{tabular}[t]{r >{\itshape}l >{/}l<{/}}
			1000\duo    & sensíva & sen.siːʋa\\
			2000\duo    & ětsíva & jet.siːʋa\\
			4000\duo    & qessíva \textup{(*\emph{qesesíva} )} & tɕesːiːʋa\\
			8603\duo    & soprisíva-vontoc-köl & sop.ri.siːʋa ʋon.tots kɵl\\
			10,000\duo  & żelsíva & tθel.siːʋa\\
			17,029\duo  & pedłasíva-ětiza-jok & peð.wa.siːʋa je.ti.θa jok\\
			50,000\duo  & kölvádsíva & kɵl.ʋaːð.siːʋa\\
			93,487\duo  & jokza-kölsíva qesetoc-sopriza-ikuş & jok.θa kɵl.siːʋa tɕe.se.tots sop.ri.θa i.kuʂ\\
			100,000\duo & sentossíva & sen.tosːiːʋa\\
			682,006\duo & vontoc-sopriza-ětsíva von & von.tots sop.ri.θa jet.siːʋa ʋon\\
		\end{tabular}
	\end{exe}

	\newpage
	Numerals from 10\sup6\duo\ to 10\sup{12}\duo−1 are formed by the addition of the suffix \emph{-műl}:

	\begin{exe}
		\ex
		\begin{tabular}[t]{r >{\itshape\raggedright}m{5cm} >{/}m{4cm}<{/}}
			1·10\sup6\duo       & semműl \textup{(*\emph{senműl} )} & semːʉːl \\
			2·10\sup6\duo       & ětiműl & je.ti.myːl\\
			70·10\sup6\duo      & ikuşzaműl & i.kuʂ.θa.mʉːl\\
			300·10\sup6\duo     & költocműl & kɵl.tots.mʉːl\\
			419,203,62A\duo     & qesetoc-vaudi-semműl ěttoc-kölsíva vontoc-ětiza-méri & \raggedright tɕe.se.tots ʋau.ði semːʉːl jetːots kɵl.siːʋa ʋon.tots je.ti.θa meːri\tnl
			900,000,000,000\duo & joktocsívaműl & jok.tots.siːʋa.mʉːl\\
		\end{tabular}
	\end{exe}

	Using this system alone, it is possible to count up to BBB,BBB,BBB,BBB\duo, or 8,916,100,448,255\dec. Larger numerals, if needed, use a system of powers \emph{which I haven't thought of yet}.

%	\section{Ordinals}
%	\label{sec:num_ordinals}
%	
%	The ordinal numerals are formed by appending the suffix \emph{-ik} to the number word. For large numerals, the suffix is applied to the last word in the sequence. The ordinals from *0\sup{th} to 21\dec\sup{st} are given in Table~\ref{tab:num_ordinals}.
%
%	\begin{table}[htpb]\small\capstart
%		\begin{center}
%			\subfloat{
%			\begin{tabular}{|c|c|c|}
%				\hline
%				\multicolumn{2}{|c|}{\bfseries Ordinal} & {\bfseries Pronunciation} \tnl
%				\hline
%				*0\sup{th} & *zomik & /ðo.mik/ \tnl
%				1\sup{st} & senik   & /se.nik/ \tnl
%				2\sup{nd} & ětík    & /je.tiːk/ \tnl
%				3\sup{rd} & kölik   & /kɵ.lik/ \tnl
%				4\sup{th} & qeseik  & /tɕe.seik/ \tnl
%				5\sup{th} & inüik   & /i.nʉ.ik/ \tnl
%				6\sup{th} & vonik   & /ʋo.nik/ \tnl
%				7\sup{th} & ikuşik  & /i.ku.ʂik/ \tnl
%				8\sup{th} & soprík  & /sop.riːk/ \tnl
%				9\sup{th} & jokik   & /jo.kik/ \tnl
%				10\dec\sup{th} & mérik & /meːrik/ \tnl
%				\hline
%			\end{tabular}}\qquad
%			\subfloat{
%			\begin{tabular}{|c|c|c|}
%				\hline
%				\multicolumn{2}{|c|}{\bfseries Ordinal} & {\bfseries Pronunciation} \tnl
%				\hline
%				11\dec\sup{th} & türeik   & /tʉ.reik/ \tnl
%				12\dec\sup{th} & żelik    & /tθe.lik/ \tnl
%				13\dec\sup{th} & scemik   & /stse.mik/ \tnl
%				14\dec\sup{th} & zpetik   & /θpe.tik/ \tnl
%				15\dec\sup{th} & żekölik  & /tθe.kɵ.lik/ \tnl
%				16\dec\sup{th} & ksożik   & /kso.tθik/ \tnl
%				17\dec\sup{th} & pedłaik  & /peð.waik/ \tnl
%				18\dec\sup{th} & pseňaik  & /pse.ɲaik/ \tnl
%				19\dec\sup{th} & sevčaik  & /seʋ.tɕaik/ \tnl
%				20\dec\sup{th} & vaudík   & /ʋau.ðiːk/ \tnl
%				21\dec\sup{th} & vaudi-senik & /ʋau.ði se.nik/ \tnl
%				\hline
%			\end{tabular}}
%			\caption{Ordinal numerals from 0\dec\ to 21\dec\label{tab:num_ordinals}}
%		\end{center}
%	\end{table}
%
%	\section{Multiplicatives}
%	\label{sec:num_multiplicatives}
%
%	Numerals in Qevesa also have a special form for multiplicatives, formed by appending the suffix \emph{-zmi}. If the numeral stem ends in a consonant, and epenthetic vowel identical to the nucleus vowel of the previous syllable is inserted. The multiplicative numbers from 0\dec\ to 21\dec\ are listed in Table~\ref{tab:num_multiplicatives}.
%	
%	\begin{table}[htpb]\small\capstart
%		\begin{center}
%			\subfloat{
%			\begin{tabular}{|c|c|c|}
%				\hline
%				\multicolumn{2}{|c|}{\bfseries Multiplicative} & {\bfseries Pronunciation} \tnl
%				\hline
%				0× & zomozmi	& /ðo.moθ.mi/ \tnl
%				1× & senezmi	& /se.neθ.mi/ \tnl
%				2× & ětízmi		& /je.tiːθ.mi/ \tnl
%				3× & kölözmi	& /kɵ.lɵθ.mi/ \tnl
%				4× & qesezmi	& /tɕe.seθ.mi/ \tnl
%				5× & inüzmi		& /i.nʉθ.mi/ \tnl
%				6× & vonozmi	& /ʋo.noθ.mi/ \tnl
%				7× & ikuşuzmi	& /i.ku.ʂuθ.mi/ \tnl
%				8× & soprizmi	& /sop.riθ.mi/ \tnl
%				9× & jokozmi	& /jo.koθ.mi/ \tnl
%				10\dec × & mérizmi & /meːriθ.mi/ \tnl
%				\hline
%			\end{tabular}}\qquad
%			\subfloat{
%			\begin{tabular}{|c|c|c|}
%				\hline
%				\multicolumn{2}{|c|}{\bfseries Multiplicative} & {\bfseries Pronuncication} \tnl
%				\hline
%				11× & türezmi   & /tʉ.reθ.mi/ \tnl
%				12× & żelezmi   & /tθe.leθ.mi/ \tnl
%				13× & scemezmi  & /stse.meθ.mi/ \tnl
%				14× & zpetezmi  & /θpe.teθ.mi/ \tnl
%				15× & żekölözmi & /tθe.kɵ.lɵθ.mi/ \tnl
%				16× & ksożozmi  & /kso.tθoθ.mi/ \tnl
%				17× & pedłazmi  & /peð.waθ.mi/ \tnl
%				18× & pseňazmi  & /pse.ɲaθ.mi/ \tnl
%				19× & sevčazmi  & /seʋ.tɕaθ.mi/ \tnl
%				20× & vaudizmi  & /ʋau.ðiθ.mi/ \tnl
%				21× & vaudi-senezmi & /ʋau.ði se.neθ.mi/ \tnl
%				\hline
%			\end{tabular}}
%			\caption{Multiplicative numerals from 0\dec\ to 21\dec\label{tab:num_multiplicatives}}
%		\end{center}
%	\end{table}
%
%	The multiplicative forms are used both in a repetitive and mathematical sense:
%
%	\begin{exe}
%		\ex \emph{EXAMPLES}
%	\end{exe}
%
%	\section{Fractions}
%	\label{sec:num_fractions}
%
%	Fractions are formed by appending the suffix \emph{-Vna} where \emph{V} is the nucleus vowel of the previous syllable — numerals ending in a vowel have this vowel lengthened instead. The fractional numbers from 0\dec\ to 21\dec\ are listed in Table~\ref{tab:num_cardinals}.
%	
%	\begin{table}[htpb]\small\capstart
%		\begin{center}
%			\subfloat{
%				\begin{tabular}{|c|c|c|}
%				\hline
%				\multicolumn{2}{|c|}{\bfseries Fractional} & {\bfseries Pronunciation} \tnl
%				\hline
%				*\sup1⁄\sub0 & *zomona & /ðo.mo.na/ \tnl
%				*\sup1⁄\sub1 & *senena & /se.ne.na/ \tnl
%				\sup1⁄\sub2 & ětína    & /je.tiːna/ \tnl
%				\sup1⁄\sub3 & kölöna   & /kɵ.lɵ.na/ \tnl
%				\sup1⁄\sub4 & qeséna   & /tɕe.seːna/ \tnl
%				\sup1⁄\sub5 & inűna    & /i.nʉːna/ \tnl
%				\sup1⁄\sub6 & vonona   & /ʋo.no.na/ \tnl
%				\sup1⁄\sub7 & ikuşuna  & /i.ku.ʂu.na/ \tnl
%				\sup1⁄\sub8 & soprína  & /sop.riːna/ \tnl
%				\sup1⁄\sub9 & jokona   & /jo.ko.na/ \tnl
%				\sup1⁄\sub{10} & mérina & /meːri.na/ \tnl
%				\hline
%			\end{tabular}}\qquad
%			\subfloat{
%				\begin{tabular}{|c|c|c|}
%				\hline
%				\multicolumn{2}{|c|}{\bfseries Fractional} & {\bfseries Pronunciation} \tnl
%				\hline
%				\sup1⁄\sub{11} & türéna   & /tʉ.reːna/ \tnl
%				\sup1⁄\sub{12} & żelena   & /tθe.le.na/ \tnl
%				\sup1⁄\sub{13} & scemena  & /stse.me.na/ \tnl
%				\sup1⁄\sub{14} & zpetena  & /θpe.te.na/ \tnl
%				\sup1⁄\sub{15} & żekölöna & /tθe.kɵ.lɵ.na/ \tnl
%				\sup1⁄\sub{16} & ksożona  & /kso.tθo.na/ \tnl
%				\sup1⁄\sub{17} & pedłána  & /peð.waːna/ \tnl
%				\sup1⁄\sub{18} & pseňána  & /pse.ɲaːna/ \tnl
%				\sup1⁄\sub{19} & sevčána  & /seʋ.tɕaːna/ \tnl
%				\sup1⁄\sub{20} & vaudína  & /ʋau.ðiːna/ \tnl
%				\sup1⁄\sub{21} & vaudi-senena & /ʋau.ði se.ne.na/ \tnl
%				\hline
%			\end{tabular}}
%			\caption{Fractional numerals from 0\dec\ to 21\dec\label{tab:num_fractional}}
%		\end{center}
%	\end{table}
%
%	\newpage
%	The numerator of a fraction precedes the denominator as an additional modifier, for example:
%
%	\begin{exe}
%		\ex
%		\begin{xlist}
%			\ex \textit{ikuş żelena}
%			\glll ikuş żel-ena\\
%			seven twelve\textsc{-frac}\\
%			seven twelfth\\
%			\glt seven-twelfths
%			\ex \textit{ěti kölöna litaseva}
%			\glll ěti köl-öna litas-ev-a\\
%			two three\textsc{-frac} bread\textsc{-du-nom}\\
%			two third bread\\
%			\glt two-thirds of bread
%		\end{xlist}
%	\end{exe}
%
%	If the denominator of a fraction is a compound number, the fractional suffix is appended to the final word in the sequence:
%
%	\begin{exe}
%		\ex
%		\begin{xlist}
%			\ex \textit{kölvádana}
%			\glll kölvád-ana\\
%			sixty\textsc{-frac}\\
%			sixtieth\\
%			\glt (a) sixtieth
%			\ex \textit{sopri ětiza-vonona}
%			\glll sopri ěti-za=von-ona\\
%			eight two=dozen\textsc{-frac}\\
%			eight twenty-fourths\\
%			\glt eight twenty-fourths
%		\end{xlist}
%	\end{exe}
%
%
%	More complex fractions {\em are yet to be written about… in particular, I need:
%		\begin{itemize}
%			\item Integer ± unit fraction
%			\item Integer × unit fraction
%		\end{itemize}
%	}
%
%
	\end{document}

