\documentclass[grammar]{subfiles}
\begin{document}
  \section*{Qevesa Alphabet}
  \label{sec:romanisation}

  The usual transcription system used for the Latin alphabet is as follows:

  \begin{center}
    \begin{tabularx}{0.9 \textwidth}{fC*{8}{-C}}
      \SetRowStyle{\bfseries} A a & Á á  & C c   & Č č  & D d   & E e & É é  & H h \\
                              /a/ & /aː/ & /ts/  & /tʃ/ & /ð/   & /e/ & /eː/ & /h/ \\
      \SetRowStyle{\bfseries} I i & Í í  & J j   & K k  & KH kh & L l & M m  & N n \\
                              /i/ & /iː/ & /j/   & /k/  & /x/   & /l/ & /m/  & /n/ \\
      \SetRowStyle{\bfseries} Ň ň & O o  & Ó ó   & P p  & PH ph & Q q & R r  & S s \\
                              /ɲ/ & /o/  & /oː/  & /p/  & /f/   & /c/ & /r/  & /s/ \\
      \SetRowStyle{\bfseries} Š š & T t  & TH th & U u  & Ú ú   & V v & X x  & Z z \\
                              /ʃ/ & /t/  & /θ/   & /ʉ/  & /ʉː/  & /v/ & /s ks/ & /z dz/ \\
    \end{tabularx}
    %\caption[Romanisation of Qevesa]{\label{tab:transcription}}
  \end{center}

  %  a  á  c   č  d   e  é  h
  %  i  í  j   k  kh  l  m  n
  %  ň  o  ó   p  ph  q  r  s
  %  š  t  th  u  ú   v  x  z

  %\pagebreak[2]
  The Latin orthography is largely phonemic, and makes use of a number of diacritics and digraphs.  The diacritics indicate the following features:

  \begin{description}
    \item[Háček/Caron] The \foreign{háček} or caron indicates a palatalised consonant variant.  It is used with ⟨c⟩, ⟨n⟩ and ⟨s⟩, producing ⟨č⟩, ⟨ň⟩ and ⟨š⟩.  
    \item[Acute] The acute accent is used to indicate a long vowel, and is used with ⟨a⟩, ⟨e⟩, ⟨i⟩, ⟨o⟩ and ⟨u⟩ to produce ⟨á⟩, ⟨é⟩, ⟨í⟩, ⟨ó⟩ and ⟨ú⟩.  
  \end{description}

  The digraphs ⟨kh⟩, ⟨ph⟩ and ⟨th⟩ represent the phonemes /x/, /f/ and /θ/. These phonemes were originally pronounced as aspirated stops in Common Therasa, and became fricatives in Qevesa. The letters ⟨x⟩ and ⟨z⟩ represent the affricates /ks/ and /dz/.

\end{document}
