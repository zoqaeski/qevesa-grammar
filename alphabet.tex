\documentclass[grammar]{subfiles}
\begin{document}
  \section*{Qevesa Alphabet}
  \label{sec:romanisation}

  The usual transcription system used for the Latin alphabet is as follows:

  \begin{center}
    \begin{tabularx}{0.9 \textwidth}{fC*{10}{-C}}
      \SetRowStyle{\bfseries} A a & Á á & C c & Č č & D d & E e & É é & Ë ë & F f & H h \\
      /a/ & /aː/ & /ts/ & /tʃ/ & /ð/ & /e/ & /eː/ & /\superj e/ & /f/ & /h/ \\		
      \SetRowStyle{\bfseries} I i & Í í & J j & K k & L l & M m & N n & Ň ň & O o & Ó ó \\
      /i/ & /iː/ & /j/ & /k/ & /l/ & /m/ & /n/ & /ɲ/ & /o/ & /oː/ \\
      \SetRowStyle{\bfseries} O o & Ó ó & P p & Q q & R r & S s & Š š & T t & U u & Ú ú \\
      /ɵ/ & /ɵː/ & /p/ & /c/ & /r/ & /s/ & /ʃ/ & /t/ & /u/ & /uː/ &
      \SetRowStyle{\bfseries} U u & Ú ú & V v & X x & Z z \\
      /y/ & /yː/ & /v~ʋ/ & /x/ & /θ/ \\
    \end{tabularx}
    %\caption[Romanisation of Qevesa]{\label{tab:transcription}}
  \end{center}

  \pagebreak[2]
  The Latin orthography makes use of a number of diacritics.  The diacritics on consonants indicate the following features:

  \begin{description}
    %\item[Cedilla/Comma] The cedilla or comma indicates a retroflex variant, and is used with ⟨s⟩ and ⟨c⟩, forming ⟨ş⟩ and ⟨ç⟩.  In handwritten texts, the comma is preferred, but typeset documents normally use the cedilla, due to a lack of typefaces that include the comma as a diacritic.
    \item[Háček/Caron] The \foreign{háček} or caron indicates a palatalised consonant variant.  It is used with ⟨c⟩, ⟨n⟩ and ⟨s⟩, producing ⟨č⟩, ⟨ň⟩ and ⟨š⟩.
    %\item[Dot above] This diacritic indicates an affricate varient of a fricative, and is only used with ⟨z⟩, resulting in ⟨ż⟩.
  \end{description}

  \pagebreak[2]
  Vowels use a similar set of diacritics:

  \begin{description}
    \item[Trema] The trema has two separate uses.  With ⟨o⟩ and ⟨u⟩, it indicates a fronted variant, forming ⟨o⟩ and ⟨u⟩.  When used with ⟨e⟩, it indicates a palatalised variant, usually pronounced as /je/.  The long variant of ⟨ë⟩ is written as ⟨ië⟩ except when word-initial, when it is written ⟨jé⟩. 

      % ëto jéto pëma piéma

    %\item[Háček]\label{def:hacek} The \foreign{háček} or caron indicates an iotated or palatalised variant.  It is most commonly used with ⟨e⟩ to produce ⟨ě⟩, but may be used with other vowels.  The ⟨j-⟩ spelling is used in some situations, such as across a syllable break or between two vowels (in which the inherent /j/ becomes the onset of the next syllable), so *⟨aě⟩ is written as ⟨aje⟩.  Generally, ⟨ě⟩ is preferred when following a consonant or as a nucleus vowel of a syllable, and ⟨je⟩ is used when the /e/ is lengthened ⟨jé⟩, but both representations are interchangeable.
    \item[Acute] The acute accent is used to indicate a long vowel, and is used with ⟨a⟩, ⟨e⟩, ⟨i⟩, ⟨o⟩ and ⟨u⟩ to produce ⟨á⟩, ⟨é⟩, ⟨í⟩, ⟨ó⟩ and ⟨ú⟩.  Long variants of ⟨o⟩ and ⟨u⟩ use a doubled acute, resulting in ⟨ó⟩ and ⟨ú⟩. 
  \end{description}

  Although the orthography is largely morphophonemic, a number of phonemes may be written in more than one way:

  \begin{itemize*}
    %\item Palatalisation is indicated by a following ⟨j⟩, an i-glide diphthong, or a \foreign{háček} above the vowel.  %/ji/ is represented with ⟨y⟩ or ⟨í⟩ and realised as /\superj iː/.
    %\item /i-/glides and /j/ soften (palatalise) preceding consonants: therefore /ɕ/ /tɕ/ /\textltailn/ may be represented by ⟨š⟩ ⟨č⟩ ⟨ń⟩ or ⟨si-⟩ ⟨ci-⟩ ⟨ni-⟩ before another vowel.  /\superj e/ ⟨ě⟩ will also cause palatalisation in this manner.
    \item /v/ may be realised as an approximant in some situations, and digraphs involving ⟨v⟩ or ⟨f⟩ such as ⟨sf⟩ or ⟨zv⟩ may result in the labiodental fricative being realised as anything between [f] [v] and [ʋ].
    %\item /iː/ is usually written as ⟨í⟩ except when word-initial, when it is written as ⟨y⟩.
  \end{itemize*}


\end{document}
