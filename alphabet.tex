\documentclass[grammar]{subfiles}
\begin{document}
  \section*{Qevesa Alphabet}
  \label{sec:alphabet}

  The usual transcription system used for the Latin alphabet is as follows:

  \begin{center}
    \begin{tabularx}{0.9 \textwidth}{FC*{8}{-C}}
      \SetRowStyle{\bfseries} A a & Á á  & C c  & Č č   & CH ch & D d   & E e & É é  \\ 
                              /a/ & /aː/ & /ts/ & /tʃ/  & /ç/   & /ð/   & /e/ & /eː/ \\ 
      \SetRowStyle{\bfseries} H h & I i  & Í í  & J j   & K k   & Kh kh & L l & M m  \\
                              /h/ & /i/  & /iː/ & /j/   & /k/   & /x/   & /l/ & /m/  \\
      \SetRowStyle{\bfseries} N n & Ň ň  & O o  & Ó ó   & P p   & Ph ph & Q q & R r  \\
                              /n/ & /ɲ/  & /o/  & /oː/  & /p/   & /f/   & /c/ & /r/  \\
      \SetRowStyle{\bfseries} S s & Š š  & T t  & TH th & U u   & Ú ú   & V v & Z z \\
                              /s/ & /ʃ/  & /t/  & /θ/   & /u/   & /uː/  & /v/ & /z dz/ \\
    \end{tabularx}
    %\caption[Romanisation of Qevesa]{\label{tab:transcription}}
  \end{center}

  %  a  á  c   č  d   e  é  h
  %  i  í  j   k  kh  l  m  n
  %  ň  o  ó   p  ph  q  r  s
  %  š  t  th  u  ú   v  x  z

  %\pagebreak[2]
  The Latin orthography is largely phonemic, and makes use of a number of
  diacritics and digraphs.  The diacritics indicate the following features:

  \begin{description}
    \item[Háček/Caron] The \foreign{háček} or caron indicates a palatalised
      consonant variant.  It is used with ⟨c⟩, ⟨n⟩ and ⟨s⟩, producing ⟨č⟩, ⟨ň⟩
      and ⟨š⟩.  
    \item[Acute] The acute accent is used to indicate a long vowel, and is used
      with ⟨a⟩, ⟨e⟩, ⟨i⟩, ⟨o⟩ and ⟨u⟩ to produce ⟨á⟩, ⟨é⟩, ⟨í⟩, ⟨ó⟩ and ⟨ú⟩.  
  \end{description}

  The digraphs ⟨ch⟩, ⟨kh⟩, ⟨ph⟩ and ⟨th⟩ represent the phonemes /ç/, /x/, /f/
  and /θ/.  These phonemes were originally pronounced as aspirated stops in
  Common Therasa, and became fricatives in Qevesa. The letter ⟨z⟩ represents
  the affricate /dz/.

  Geminate consonants are doubled, except for the digraphs which only double
  the first consonant.  

\end{document}
