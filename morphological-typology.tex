\documentclass[grammar]{subfiles}
\begin{document}

	\chapter{Morphological Typology}
	\label{ch:morphological_typology}

	% I REALLY REALLY don't like this paragraph
	%Morphology pertains to the organisation, rules, and processes concerning meaningful units of language, whether they be words themselves, or parts of words, such as affixes of various sorts. The minimal meaningful unit of morphological processes is the morpheme; words and phrases are all constructed from morphemes in successive stages.
	
	Qevesa morphology differs quite significantly from English. The lexemes, or roots, are based around discontinuous clusters of two to five consonantal phonemes. These roots interlock with patterns of vowels (and sometimes other consonants) to form words or word stems.

	\begin{exe}
		\ex \emph{EXAMPLE}
	\end{exe}

	These words, or word stems, can be further modified by the addition of inflexional affixes, such as suffixes, prefixes, and occasionally infixes. The triliteral root represents the semantic field or abstract concept; the patterns represent specific lexical or inflectional derivations. Both roots and patterns are bound morphemes, each conveying specific and essential types of information. Neither can exist independently because both are abstract mental representations. 

	\section{Definition of Root}
	\label{sec:definition_of_root}

	A root is a relatively invariable discontinuous bound morpheme, represented by two to five phonemes in a certain order, which interlocks with a pattern to form a stem, and which has lexical meaning. The root morpheme is discontinuous because vowels can be interspersed between the consonants; however, the consonants of a root must always be present and in the same sequence. The usual number of consonants in a Qevesa root is three; however, there are also two-consonantal (biliteral), four-consonantal (quadriliteral) and five-consonantal (quinquiliteral), although the latter are extremely rare. Quadriliteral and quinquiliteral roots always contain a consonant cluster as a root phoneme that cannot be split

	The root is said to contain lexical meaning because it communicates the idea of a real-world concept. It is useful to consider the root as denoting a semantic field because it is within that field that actual words come into existence. The exact number of lexical roots in Qevesa ranges from 6000 to 7500; phonologically there are two to five times that number of permissable roots.\footnote{There are 24 consonants, and as no root can contain more than two of the same consonant, there are 13248\footnotemark{} permissable triliteral roots. Quadriliteral and quinquiliteral roots contain consonant clusters, which makes calculation of the number of permissable roots considerably more difficult. Furthermore, some biliteral roots actually consist of three consonants, with two of those bound as a cluster. }

	\footnotetext{24 × 24 × 23 = 13248}

	%\footnotetext{}

	\section{Definition of Pattern}
	\label{sec:definition_of_pattern}

	A pattern is a bound and often discontinuous morpheme consisting of a sequence of one or more vowels and slots for root phonemes, which either alone or in conjunction with other affixes, interlocks with a root to form a stem, and which generally has a grammatical meaning. The pattern is discontinuous because it intersperses itself among the root consonants, and can be considered as a type of template onto which different roots can be mapped. The derivational affixes include the use of consonants that mark grammatical functions, and	these consonants may be uses as suffixes, prefixes, or infixes. A further component of pattern marking is the gemination or lengthening of existing or already-inserted consonants or vowels.

	Patterns are said to contain grammatical meaning because they signify grammatical or language-internal information; that is, they distinguish word types such as verbal forms, nominal forms, and adjectival forms. They can also signify very specific information about subclasses of the basic word types, such as aspect, number, and case.

	\subsection{Transfix positions}
	\label{ssec:transfix_positions}

	To aid in the description of the patterns or transfixes used to form base stems of verbs, nouns, and adjectives, the positions within a root are labeled as follows: the three consonants are referred to as C\sub1, C\sub2, C\sub3, and the positions adjacent to them are P\sub0, P\sub{12}, P\sub{23}, P\sub4.
	
	\section{Dictionary Ordering}
	\label{sec:dictionary_ordering}

	Qevesa dictionaries are sorted by lexical root and not spelling. Instead of relying on the exact orthography of a word, Qevesa dictionaries are organised by the root or consonant core of a word, providing under that entry every word derived from that particular lexical root. In this regard, a Qevesa dictionary is more akin to a thesaurus, locating all possible variations of a semantic concept under a single entry.

	\section{Other Lexical Types}
	\label{sec:other_lexical_types}

	Other word formation processes in Qevesa include compounding and solid stems.

	\subsection{Compounding}
	\label{ssec:morph_compounding}

	Compounding is the second-most common means of word formation. There are several variations on compounding: roots (and patterns) may be concatenated to form new roots of more consonants; stems may be concatenated to construct new meanings; and words may be strung together as phrases to introduce variations on a theme.

	Some lexical roots consist of solid stems; that is, they possess inherent vowels and generally cannot be reduced into the root-pattern paradigm. Such words fall into one of four categories: pronouns, function words, irregular stems, or loan words. The latter category is fairly sparse, as Qevesa tends to rely on substitution of terms, calquing or coinage of new terms. Sometimes, a loan word may be reanalysed as a root, often with an inherent vowel pattern.
	
	\section{Head/Dependent Marking}
	\label{sec:head_dependent_marking}

	Qevesa tends towards dependent marking, although it also exhibits cases of head-marking.

	\ToBeWritten

	%from which verbal forms, nominal forms, and adjectival forms are constructed. 
	
	%by inserting vowels or consonants, and adding transfixes, suffixes, prefixes or infixes. The vast majority of roots are triliteral, or formed from three consonants, although two-~and four-~consonant roots also exist. These roots are categorised into morphological classes determined by the number of consonants and the specific phonemes that comprise the root.



	%\section{Morphological Processes}
	%\label{sec:morphological_processes}

	%\subsection{Transfixation}
	%\label{ssec:transfixation}

	%As a language based on triliteral roots, a large proportion of the grammatical marking is done by insertion of discontinuous vowel patterns. These transfixes are primarily used to indicate verbal aspects and grammatical categories.

	%\begin{exe}
		%\ex \emph{EXAMPLES}
	%\end{exe}

	%\subsection{Suffixation}
	%\label{ssec:suffixation}

	%The second most productive morphological process is suffixation. Almost all other grammatical marking is done by addition of largely agglutinative suffixes.

	%\begin{exe}
		%\ex \emph{EXAMPLES}
	%\end{exe}

	%\subsection{Prefixation}
	%\label{ssec:prefixation}

	%\ToBeWritten

\end{document}
