\documentclass[grammar]{subfiles}
\begin{document}
	\chapter{Syntax}
	\label{ch:syntax}

	Qevesa syntax is fairly fluid, and tends towards being largely left-branching or head-final. The only strict requirement of a sentence is that the verb must occur last, and that the topic, if present, must be first. All other elements may be freely ordered by importance. The general word order is thus \emph{\textsc{topic–comment–verb}}.

	\section{Verb Phrases}
	\label{sec:syn_verb_phrases}

	Verb phrases are always verb-final. Arguments of the verb precede it, as do modifiers and auxiliary verbs. The topic of the verb, if present, must occur as the first element of the phrase; comments regarding the topic precede the verb and its modifiers but follow the topic.

	\subsection{Verbal Topic}
	\label{ssec:syn_verbal_topic}

	Qevesa is a \emph{topic-prominent} language, which means that the topic is semantically the most important argument of the verb. The topic is indicated by the noun phrase in the nominative case, with the syntactic role marked on the verb. Any of the constituent phrases can be marked as the topic; it usually consists of the element that the speaker considers to be the most important.

	As a result, voice and valency-adjusting operations are not prominent features of Qevesa. The ‘active’ voice, and most common construction, is to mark the topic as the agent of the verb.

	\begin{exe}
		\ex
		\begin{xlist}
			\ex EXAMPLE
			\ex EXAMPLE
			\ex EXAMPLE
		\end{xlist}
	\end{exe}

	\section{Noun Phrases}
	\label{sec:syn_noun_phrases}

	Noun phrases 

	\section{Relative Clauses}
	\label{sec:syn_relative_clauses}

	Qevesa does not employ relative pronouns to related relative clauses to their antecedents. Instead, the relative clause directly modifies the noun phrase as an attributive verb phrase, optionally agreeing in case.

	\emph{To be written…}

\end{document}
