\documentclass[grammar]{subfiles}
\begin{document}
  \chapter{Adjectival Morphology}
  \label{ch:adjectival-morphology}

  Qevesa possesses two types of words that could be loosely described as adjectives:

  \begin{description}
    \item[Adjectival Verbs] are verb-like forms, that are formed by the Form VIII or IX roots.
    \item[Attributives] are plain adjectives, also derived from the Form VIII or IX roots, which may only occur before nouns. In a predicative position, their corresponding adjectival verb form will be used.
    %\item[Adjectival Nouns] are adjectives derived from nouns that attach to a form of the copula when used predicatively. The copula then inflects to the appropriate verbal conjugations.
  \end{description}

  Adjectives possess a number of unique features: they can be directly marked for polarity, and they may also be marked for degree.

  \section{Types of Adjectival Forms}
  \label{sec:am_adjectival_forms}

  \subsection{Adjectival Verbs}
  \label{ssec:am_adjectival_verbs}

  Adjectival verbs are, as the name suggests, a set of verb-like forms, derived from Form VIII or IX verbal roots. 
  They may predicate sentences, and conjugate to aspect, topical agreement and mood in the same manner as ordinary verbs, differing in some inflections, notably aspect. % Not sure about that one
  The transfix patterns used to indicate aspect are the primary means of deriving attributive verbs; these are given in Table~\ref{tab:am_attributive_verb_aspect}.

  \begin{table}[htpb]\small\capstart
      \begin{tabular}{|>{\bfseries}fc->{\scshape}c|-c|-c|-c|-c|}
        \hline
        \SetRowStyle{\bfseries} & & \multicolumn{2}{-c|}{Triliteral Patterns} & \multicolumn{2}{-c|}{Biliteral Patterns} \tabularnewline
        %\cline{3-4}
        %\SetRowStyle{\bfseries} & & Triliteral & Biliteral \tabularnewline
        \hline
        Imperfective & ipfv & 
        C\sub1{u}C\sub2C\sub3{i} & 
        {ě}C\sub1C\sub2{u}C\sub2C\sub3{i} & 
        C\sub1{u}C\sub2C\sub2{i} &
        {ě}C\sub1{u}C\sub2C\sub2{i}
        \tabularnewline
        Stative & stat & 
        C\sub1{ui}C\sub2C\sub3{e} & 
        {ě}C\sub1C\sub2{ui}C\sub2C\sub3{e} & 
        C\sub1{ui}C\sub2C\sub2{e} &
        {ě}C\sub1{ui}C\sub2C\sub2{e}
        \tabularnewline
        Durative & dur;ipfv & 
        C\sub1{u}C\sub2C\sub3{ú} & 
        {ě}C\sub1C\sub2{u}C\sub2C\sub3{ú} & 
        C\sub1{u}C\sub2C\sub2{ú} &
        {ě}C\sub1{u}C\sub2C\sub2{ú}
        \tabularnewline
        Frequentative & freq & 
        C\sub1{u}C\sub2C\sub3{o} & 
        {ě}C\sub1C\sub2{u}C\sub2C\sub3{o} & 
        C\sub1{u}C\sub2C\sub2{o} &
        {ě}C\sub1{u}C\sub2C\sub2{o}
        \tabularnewline
        Habitual & hab & 
        C\sub1{u}C\sub2C\sub3{a} &
        {ě}C\sub1C\sub2{u}C\sub2C\sub3{a} &
        C\sub1{u}C\sub2C\sub2{a} &
        {ě}C\sub1{u}C\sub2C\sub2{a}
        \tabularnewline
        \hline\hline
        Perfective & pfv &
        C\sub1{io}C\sub2C\sub3{a} & 
        {ě}C\sub1C\sub2{io}C\sub2C\sub3{a} & 
        C\sub1{io}C\sub2C\sub2{a} &
        {ě}C\sub1{io}C\sub2C\sub2{a}
        \tabularnewline
        Inchoative & inch & 
        C\sub1{iu}C\sub2C\sub3{o} & 
        {ě}C\sub1C\sub2{iu}C\sub2C\sub3{o} & 
        C\sub1{iu}C\sub2C\sub2{o} &
        {ě}C\sub1{iu}C\sub2C\sub2{o}
        \tabularnewline
        Cessative & cess & 
        C\sub1{í}C\sub2C\sub3{a} & 
        {ě}C\sub1C\sub2{í}C\sub2C\sub3{a} & 
        C\sub1{í}C\sub2C\sub2{a} &
        {ě}C\sub1{í}C\sub2C\sub2{a}
        \tabularnewline
        Durative & dur;pfv & 
        C\sub1{ia}C\sub2C\sub3{u} & 
        {ě}C\sub1C\sub2{ia}C\sub2C\sub3{u} & 
        C\sub1{ia}C\sub2C\sub2{u} &
        {ě}C\sub1{ia}C\sub2C\sub2{u}
        \tabularnewline
        Momentane & momt & 
        C\sub1{iu}C\sub2C\sub3{a} &
        {ě}C\sub1C\sub2{iu}C\sub2C\sub3{a} &
        C\sub1{iu}C\sub2C\sub2{a} &
        {ě}C\sub1{iu}C\sub2C\sub2{a}
        \tabularnewline
        \hline
      \end{tabular}
      \caption{Adjectival verb aspectual conjugation\label{tab:am_attributive_verb_aspect}}
  \end{table}

%  \subsubsection{Adverbs}
%  \label{sssec:am_adverbs}
%
%  Adverbs are a derived class of attributive verbs, formed by appending the suffix \textit{-żi} to the aspectual form that agrees with their head verb.

  \subsection{Attributives}
  \label{ssec:am_attributives}

  Attributives may be derived from the Form III, VIII or IX roots, and have a number of transfix patterns. Common patterns include the \emph{passive participle} \qevesa{C\sub1{o}C\sub2C\sub3{i}}, and the \emph{verbal noun} \qevesa{C\sub1{a}C\sub2C\sub3{u}}. 
  However, it is impossible to predict which form a root will take as the distribution is entirely arbitrary.

%  \subsection{Adjectival Nouns}
%  \label{ssec:am_adjectival_nouns}
%
%  Unlike adjectival verbs, attributive nouns are derived from nominalisations of verb forms, as well as other nouns. The most common nominalisations used to derive adjectival nouns are the verbal nouns and the active and passive participles. Nominal forms can be turned into adjectives by appending the suffix \textit{-mne}.
%
%  \begin{exe}
%    \ex\label{exe:am_adjectival_nouns}
%    \begin{tabular}[t]{f>{\itshape}l l ->{\itshape}l l}\small
%      işşakku  & ‘narcissism’ & işşakkumne  & ‘narcissistic’\\
%    \end{tabular}
%  \end{exe}

  \section{Adjectival Inflection}
  \label{sec:am_adjectival_inflection}

  Adjectives inflect for polarity and degree. The structure of an adjective is:

  \begin{exe}
    \ex\label{ex:am_adjective_structure} \textsc{supl-}\textit{stem}\textsc{-comp-polarity}
  \end{exe}	

  The adjectival stem is its base conjugated form, so for an attributive verb, this would include the aspectual, topical and modal marking. 

  \subsection{Degree}
  \label{ssec:am_degree}

  Qevesa adjectives inflect to three degrees of comparison: comparative, superlative and exaggerated. 
  These are indicated by a combination of prefixes and suffixes, which are listed in Table~\ref{tab:am_degree}. 
  Alternatively, the affixes can precede the adjective as an adverbial construction; this is preferred for predicative attributive sentences.

  \begin{table}[htpb]\small\capstart
      \begin{tabular}{|>{\bfseries}fc->{\scshape}fc|-c|-c|-c|}
        \hline
        & & \SetRowStyle{\bfseries}Prefix & Suffix & Adverb \tabularnewline
        \hline
        Comparative & comp & ∅    & -vén & vén      \tabularnewline
        Superlative & supl & ko-  & -vén & kovén   \tabularnewline
        Exaggerated & exag & los- & -vén & losvén \tabularnewline
        \hline
      \end{tabular}
      \caption{Adjectival degree adverbs\label{tab:am_degree}}
  \end{table}

  Table~\ref{tab:am_degree_inflection} gives the adjective comparison marking for the word \textit{tomsi} (tall), and Example~\ref{exe:am_degree} shows two sentences demonstrating the different styles of comparitive marking.

  \begin{table}[htpb]\small\capstart
      \begin{tabular}[t]{|>{\bfseries}fc|-c|-c|-c|-c|}
        \hline
        & \SetRowStyle{\bfseries}Adjective & Comparative & Superlative & Exaggerated \\
        \hline
        \multirow{4}{*}{Affixes} & \SetRowStyle{\itshape}tomsi & tomsivén & kotomsivén & lostomsivén \\
        & \SetRowStyle{\itshape}tomsi & tomsi-vén & ko-tomsi-vén & los-tomsi-vén \\
        & tall & tall\textsc{-comp} & \textsc{supl-}tall\textsc{-comp} & \textsc{exag-}tall\textsc{-comp} \\
        & ‘tall’ & ‘taller’ & ‘tallest’ & ‘most tallest’\\
        \hline
        \multirow{3}{*}{Adverbs} & \SetRowStyle{\itshape}tomsi & vén tomsi & kovén tomsi & losvén tomsi \\
        & tall & \textsc{comp} tall & \textsc{supl} tall & \textsc{exag} tall \\
        & ‘tall’ & ‘taller’ & ‘tallest’ & ‘most tallest’\\
        \hline
      \end{tabular}
      \caption{Adjectival degree inflection\label{tab:am_degree_inflection}}
  \end{table}

  \newpage
  \begin{exe}
    \ex\label{exe:am_degree} 
    \begin{xlist}
      \ex \textit{Cavoikě ní vén tuimseşu.}
      \glll Cavoik-ě-∅ ní vén tuimse-ş-u\\
      friend\textsc{-1sg;pos-foc} \textsc{comp.1sg} \textsc{comp} tall\textsc{\bs stat-asg;abs-ind}\\
      {friend my} {than me} {(more)} {tall is}\\
      \glt My friend is taller than me.
      \ex \textit{Cavoikě ní tuimseşuvén.}
      \glll Cavoik-ě-∅ ní tuimse-ş-u-vén\\
      friend\textsc{-1sg;pos-foc} \textsc{comp.1sg} tall\textsc{\bs stat-asg;abs-ind-comp}\\
      {friend my} {than me} {taller is}\\
      \glt My friend is taller than me.
    \end{xlist}
  \end{exe}

  \subsection{Polarity}
  \label{ssec:am_polarity}

  Unlike verbs, adjectives can be directly inflected for polarity. Both affirmative and negative suffixes exist, although the affirmative form is only used when a emphasising the existence of the adjectival property. The suffixes for polarity are given in Table~\ref{tab:am_polarity}.

  \begin{table}[htpb]\small\capstart
      \begin{tabular}{|>{\bfseries}fc->{\scshape}fc|-c|}
        \hline
        & & \bfseries Suffix \tabularnewline
        \hline
        Affirmative & aff & -rési \tabularnewline
        Negative & neg & -zúmi \tabularnewline
        \hline
      \end{tabular}
      \caption{Adjectival polarity suffixes\label{tab:am_polarity}}
  \end{table}

  \begin{exe}
    \ex\label{exe:am_polarity} 
    \begin{xlist}\ex
      \textsc{t-m-s} \textit{tumsu}, ‘to be tall’:\\[2\parskip]\small
      \begin{tabular}[t]{fc -c -c -c}
        \SetRowStyle{\itshape}tomsi & tomsirési & tomsizúmi \\
        \SetRowStyle{\itshape}tomsi & tomsi-rési & tomsi-zúmi \\
        tall & tall\textsc{-aff} & tall\textsc{-neg} \\
        ‘tall’ & ‘very tall’ & ‘not tall’\\
      \end{tabular}
      \ex \textit{Cavoikě tuimseşuzúmi.}
      \glll Cavoik-ě-∅ tuimse-ş-u-zúmi\\
      friend\textsc{-1sg;pos-foc} tall\textsc{\bs stat-asg;abs-ind-neg}\\
      {friend my} {tall is not}\\
      \glt My friend is not tall.
    \end{xlist}
  \end{exe}

\end{document}
