\documentclass[grammar]{subfiles}
\begin{document}
  \chapter{Adjectival Morphology}
  \label{ch:adjectival-morphology}

  Qevesa possesses two types of words that could be loosely described as adjectives:

  \begin{description}
    \item[Adjectival Verbs] are stative verbs, that are derived from the Form 7 root.
    \item[Attributives] are plain adjectives, and may be derived from a number of different root forms.
    %\item[Adjectival Nouns] are adjectives derived from nouns that attach to a form of the copula when used predicatively. The copula then inflects to the appropriate verbal conjugations.
  \end{description}

  Adjectives possess a number of unique features: they can be directly marked for polarity, and they may also be marked for degree.

  \section{Types of Adjectival Forms}
  \label{sec:am_adjectival_forms}

  \subsection{Adjectival Verbs}
  \label{ssec:am_adjectival_verbs}

  Adjectival verbs are, as the name suggests, a set of verb-like forms, derived from the Form 7 verbal roots. 
  They may predicate sentences, and conjugate to aspect, topical agreement and mood in the same manner as ordinary verbs, differing in some inflections, notably aspect. % Not sure about that one
  The transfix patterns used to indicate aspect are the primary means of deriving attributive verbs; these are given in Table~\ref{tab:am_attributive_verb_aspect}.

  \begin{table}[htpb]\small\capstart
      \begin{tabular}{|>{\bfseries}fc->{\scshape}c|-c|-c|-c|-c|}
        \hline
        \SetRowStyle{\bfseries} & & \multicolumn{2}{-c|}{Plain} & \multicolumn{2}{-c|}{Intensive} \tnl
        \hline
        Imperfective & \acs{ipfv} & 
        C\sub1{e}C\sub2{u}C\sub3{i} & 
        {ě}C\sub1{u}C\sub2{i} &
        C\sub1{e}C\sub2C\sub2{u}C\sub3{i} & 
        {ě}C\sub1{u}C\sub2C\sub2{i} 
        \tnl
        Stative & \acs{stat} & 
        C\sub1{e}C\sub2{ui}C\sub3{e} & 
        {ě}C\sub1{ui}C\sub2{e} &
        C\sub1{e}C\sub2C\sub2{ui}C\sub3{e} & 
        {ě}C\sub1{ui}C\sub2C\sub2{e} 
        \tnl
        Durative & \acs{dur};\acs{ipfv} & 
        C\sub1{e}C\sub2{u}C\sub3{ú} & 
        {ě}C\sub1{u}C\sub2{ú} &
        C\sub1{e}C\sub2C\sub2{u}C\sub3{ú} & 
        {ě}C\sub1{u}C\sub2C\sub2{ú} 
        \tnl
        Frequentative & \acs{freq} & 
        C\sub1{e}C\sub2{u}C\sub3{o} & 
        {ě}C\sub1{u}C\sub2{o} &
        C\sub1{e}C\sub2C\sub2{u}C\sub3{o} & 
        {ě}C\sub1{u}C\sub2C\sub2{o} 
        \tnl
        Habitual & \acs{hab} & 
        C\sub1{e}C\sub2{u}C\sub3{a} &
        {ě}C\sub1{u}C\sub2{a} &
        C\sub1{e}C\sub2C\sub2{u}C\sub3{a} &
        {ě}C\sub1{u}C\sub2C\sub2{a} 
        \tnl
        \hline
        Perfective & \acs{pfv} &
        C\sub1{e}C\sub2{io}C\sub3{a} & 
        {ě}C\sub1{io}C\sub2{a} &
        C\sub1{e}C\sub2C\sub2{io}C\sub3{a} & 
        {ě}C\sub1{io}C\sub2C\sub2{a} 
        \tnl
        Inchoative & \acs{inch} & 
        C\sub1{e}C\sub2{iu}C\sub3{o} & 
        {ě}C\sub1{iu}C\sub2{o} &
        C\sub1{e}C\sub2C\sub2{iu}C\sub3{o} & 
        {ě}C\sub1{iu}C\sub2C\sub2{o} 
        \tnl
        Cessative & \acs{cess} & 
        C\sub1{e}C\sub2{y}C\sub3{a} & 
        {ě}C\sub1{y}C\sub2{a} &
        C\sub1{e}C\sub2C\sub2{y}C\sub3{a} & 
        {ě}C\sub1{y}C\sub2C\sub2{a} 
        \tnl
        Durative & \acs{dur};\acs{pfv} & 
        C\sub1{e}C\sub2{ia}C\sub3{u} & 
        {ě}C\sub1{ia}C\sub2{u} &
        C\sub1{e}C\sub2C\sub2{ia}C\sub3{u} & 
        {ě}C\sub1{ia}C\sub2C\sub2{u} 
        \tnl
        Momentane & \acs{momt} & 
        C\sub1{e}C\sub2{iu}C\sub3{a} &
        {ě}C\sub1{iu}C\sub2{a} &
        C\sub1{e}C\sub2C\sub2{iu}C\sub3{a} &
        {ě}C\sub1{iu}C\sub2C\sub2{a} 
        \tnl
        \hline
      \end{tabular}
      \caption{Adjectival verb aspectual conjugation\label{tab:am_attributive_verb_aspect}}
  \end{table}

%  \subsubsection{Adverbs}
%  \label{sssec:am_adverbs}
%
%  Adverbs are a derived class of attributive verbs, formed by appending the suffix \textit{-żi} to the aspectual form that agrees with their head verb.

  \subsection{Attributives}
  \label{ssec:am_attributives}

  Attributives may be derived from a number of different root forms, and accordingly have a number of transfix patterns. 
  Common patterns include the \emph{passive participle} \qevesa{C\sub1{o}C\sub2C\sub3{i}}, and the \emph{verbal noun} \qevesa{C\sub1{a}C\sub2C\sub3{u}}. 
  However, it is impossible to predict which form a root will take as the distribution is entirely arbitrary.

%  \subsection{Adjectival Nouns}
%  \label{ssec:am_adjectival_nouns}
%
%  Unlike adjectival verbs, attributive nouns are derived from nominalisations of verb forms, as well as other nouns. The most common nominalisations used to derive adjectival nouns are the verbal nouns and the active and passive participles. Nominal forms can be turned into adjectives by appending the suffix \textit{-mne}.
%
%  \begin{exe}
%    \ex\label{exe:am_adjectival_nouns}
%    \begin{tabular}[t]{f>{\itshape}l l ->{\itshape}l l}\small
%      iššakku  & ‘narcissism’ & iššakkumne  & ‘narcissistic’\\
%    \end{tabular}
%  \end{exe}

  \section{Adjectival Inflection}
  \label{sec:am_adjectival_inflection}

  Adjectives inflect for polarity and degree. The structure of an adjective is:

  \begin{exe}
    \ex\label{ex:am_adjective_structure} \textsc{supl-}\textit{stem}\textsc{-comp-polarity}
  \end{exe}	

  The adjectival stem is its base conjugated form, so for an attributive verb, this would include the aspectual, topical and modal marking. 

  \subsection{Degree}
  \label{ssec:am_degree}

  Qevesa adjectives inflect to three degrees of comparison: comparative, superlative and exaggerated. 
  These are indicated by a combination of prefixes and suffixes, which are listed in Table~\ref{tab:am_degree}. 
  Alternatively, the affixes can precede the adjective as an adverbial construction. This is preferred for predicative attributive sentences.

  \begin{table}[htpb]\small\capstart
      \begin{tabular}{|>{\bfseries}fc->{\scshape}fc|-c|-c|-c|}
        \hline
        & & \SetRowStyle{\bfseries}Prefix & Suffix & Adverb \tnl
        \hline
        Comparative & \acs{comp} & ∅    & -vén & vén   \tnl
        Superlative & \acs{supl} & ko-  & -vén & kovén  \tnl
        Exaggerated & \acs{exag} & los- & -vén & losvén \tnl
        \hline
      \end{tabular}
      \caption{Adjectival degree adverbs\label{tab:am_degree}}
  \end{table}

  Table~\ref{tab:am_degree_inflection} gives the adjective comparison marking for the word \textit{tomsi} (tall), and Example~\ref{exe:am_degree} shows two sentences demonstrating the different styles of comparitive marking.

  \begin{table}[htpb]\small\capstart
      \begin{tabular}[t]{|>{\bfseries}fc|-c|-c|-c|-c|}
        \hline
        & \SetRowStyle{\bfseries}Adjective & Comparative & Superlative & Exaggerated \\
        \hline
        \multirow{4}{*}{Affixes} & \SetRowStyle{\itshape}tomsi & tomsivén & kotomsivén & lostomsivén \\
        & \SetRowStyle{\itshape}tomsi & tomsi-vén & ko-tomsi-vén & los-tomsi-vén \\
        & tall & tall-\acs{comp} & \acs{supl}-tall-\acs{comp} & \acs{exag}-tall-\acs{comp} \\
        & ‘tall’ & ‘taller’ & ‘tallest’ & ‘most tallest’\\
        \hline
        \multirow{3}{*}{Adverbs} & \SetRowStyle{\itshape}tomsi & vén tomsi & kovén tomsi & losvén tomsi \\
        & tall & \acs{comp} tall & \acs{supl} tall & \acs{exag} tall \\
        & ‘tall’ & ‘taller’ & ‘tallest’ & ‘most tallest’\\
        \hline
      \end{tabular}
      \caption{Adjectival degree inflection\label{tab:am_degree_inflection}}
  \end{table}

  \newpage
  \begin{exe}
    \ex\label{exe:am_degree} 
    \begin{xlist}
      \ex \textit{Cavíkaja náj vén temuisešu.}
      \glll Cavík-aj-a náj vén temuise-š-u\\
      friend-\acs{1p}\acs{sg};\acs{pos}-\acs{foc} \acs{comp}.\acs{1p}\acs{sg} \acs{comp} tall\bs\acs{stat}-\acs{asg};\acs{abs}-\acs{ind}\\
      {friend my} {than me} {(more)} {tall is}\\
      \glt My friend is taller than me.
      \ex \textit{Cavíkaja náj temuisešuvén.}
      \glll Cavík-aj-a náj temuise-š-u-vén\\
      friend-\acs{1p}\acs{sg};\acs{pos}-\acs{foc} \acs{comp}.\acs{1p}\acs{sg} tall\bs \acs{stat}-\acs{asg};\acs{abs}-\acs{ind}-\acs{comp}\\
      {friend my} {than me} {taller is}\\
      \glt My friend is taller than me.
    \end{xlist}
  \end{exe}

  \subsection{Polarity}
  \label{ssec:am_polarity}

  The attributive adjectives can be directly inflected for polarity. Both affirmative and negative suffixes exist, although the affirmative form is only used when a emphasising the existence of the adjectival property. The suffixes for polarity are given in Table~\ref{tab:am_polarity}.

  Adjectival verbs are marked for polarity similarly to other verbs. The second infinitive stem is marked with the affirmative or negative suffix, and the corresponding auxiliary verb is conjugated to the desired aspectual, personal and modal form.

  \begin{table}[htpb]\small\capstart
      \begin{tabular}{|>{\bfseries}fc->{\scshape}fc|-c|}
        \hline
        & & \bfseries Suffix \tnl
        \hline
        Affirmative & \acs{aff} & -rös \tnl
        Negative    & \acs{neg} & -nök \tnl
        \hline
      \end{tabular}
      \caption{Adjectival polarity suffixes\label{tab:am_polarity}}
  \end{table}

  \begin{exe}
    \ex\label{exe:am_polarity} 
    \begin{xlist}\ex
      \textsc{t-m-s} \textit{tumsu}, ‘to be tall’:\\[2\parskip]\small
      \begin{tabular}[t]{fc -c -c -c}
        \SetRowStyle{\itshape}tomsi & tomsirös & tomsinök \\
        \SetRowStyle{\itshape}tomsi & tomsi-rös & tomsi-nök \\
        tall & tall-\acs{aff} & tall-\acs{neg} \\
        ‘tall’ & ‘very tall’ & ‘not tall’\\
      \end{tabular}
      \ex \textit{Cavíkaja temisunök nuikešu.}
      \glll Cavík-aj-a temisu-nök nuike-š-u\\
      friend-\acs{1p}\acs{sg};\acs{pos}-\acs{foc} tall\bs\acs{inf2}-\acs{neg} not\bs\acs{stat}-\acs{asg};\acs{abs}-\acs{ind}-\acs{neg}\\
      {friend my} {tall not} {is not}\\
      \glt My friend is not tall.
    \end{xlist}
  \end{exe}

\end{document}
