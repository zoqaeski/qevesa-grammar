\documentclass[grammar]{subfiles}
\begin{document}
\chapter{Adjectival Morphology}
\label{ch:adjectival-morphology}

\textbf{\ToBeWritten}

Adjectives in Qevesa may behave like verbs or nouns.  

%  Adjectives possess a number of unique features: they can be directly marked
%  for polarity, and they may also be marked for degree.
%
%
%  \section{Types of Adjectival Forms} \label{sec:am_adjectival_forms}
%
%%  \subsection{Adjectival Verbs}
%%  \label{ssec:am_adjectival_verbs}
%%
Adjectival verbs are, as the name suggests, a set of verb-like forms, derived
from the Form VI verbal roots.  They may predicate sentences, and conjugate in
the same manner as ordinary verbs, differing in some inflections.

%  The transfix patterns used to indicate aspect are the primary means of
%  deriving attributive verbs; these are given in
%  \cref{tab:am_attributive_verb_conjugation}.
%%

\begin{table}[h!]\small\capstart
  \begin{tabular}{BFl Sl -l}
    \toprule
    \SetRowStyle{\bfseries} Aspect & & VI \\
    \midrule
    Perfective  & \acs{perf} & C\sub1{e}C\sub2{i}C\sub3C\sub3{u} \\
    Momentane   & \acs{momt} & C\sub1{e}C\sub2{i}C\sub3C\sub3{a} \\
    Progressive & \acs{prog} & C\sub1{e}C\sub2{a}C\sub3C\sub3{u} \\
    Durative    & \acs{dur}  & C\sub1{e}C\sub2{a}C\sub3C\sub3{i} \\
    Habitual    & \acs{hab}  & C\sub1{e}C\sub2{o}C\sub3C\sub3{u} \\
    Inchoative  & \acs{inch} & C\sub1{e}C\sub2{o}C\sub3C\sub3{a} \\
    Cessative   & \acs{cess} & C\sub1{e}C\sub2{o}C\sub3C\sub3{i} \\
    %
    \bottomrule
  \end{tabular}
  \caption{Aspectual transfix patterns\label{tab:am_aspect_patterns}}
\end{table}

%         -
%  PERF   temissu
%  MOMT   temissa
%  PROG   temassu
%  DUR    temassi
%  HAB    temossu
%  *INCH  temossa
%  *CESS  temossi

%  temassuesicvén
%  tall\DUR-COND-ABS-CMPR = would be taller
%  
%  hevarrieticvén
%  good\DUR-OPT-ABS-CMPR = could be better

\section{Adjectival Inflection}
\label{sec:am_adjectival_inflection}

Adjectives primarily inflect for degree.  The structure of an adjective is:

\begin{exe}
  \ex\label{ex:am_adjective_structure} \textsc{supl-}\textit{stem}\textsc{-cmpr}
\end{exe}	

The adjectival stem is its base conjugated form, so for an attributive verb,
this would include the aspectual, modal and personal marking. 

\subsection{Degree}
\label{ssec:am_degree}

Qevesa adjectives inflect to three degrees of comparison: comparative,
superlative and exaggerated.  These are indicated by a combination of prefixes
and suffixes, which are listed in \cref{tab:am_degree}.

\begin{table}[h!]\small\capstart
  \begin{tabular}{BFl -Sl -l -l}
    \toprule
    \SetRowStyle{\bfseries} Degree & & Prefix & Suffix \tnl
    \midrule
    Comparative & \acs{cmpr} & ∅    & -vén \tnl
    Superlative & \acs{supl} & ko-  & -vén \tnl
    Exaggerated & \acs{exag} & los- & -vén \tnl
    \bottomrule
  \end{tabular}
  \caption{Adjectival degree adverbs\label{tab:am_degree}}
\end{table}

\end{document}
