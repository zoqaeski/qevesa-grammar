\documentclass[grammar]{subfiles}
\begin{document}
	\chapter{Adjectival Morphology}
	\label{ch:adjectival-morphology}

	Qevesa possesses three of words that could be loosely described as adjectives:
	
	\begin{description}
		\item[Adjectival Verbs] are verb-like forms, that are almost universally formed by the Form VII root.
		\item[Adjectival Nouns] are adjectives derived from nouns that attach to a form of the copula when used predicatively. The copula then inflects to the appropriate verbal conjugations.
		\item[Attributives] are plain adjectives, also derived from the Form VII root, which may only occur before nouns. In a predicative position, their corresponding adjectival verb form will be used.
		\end{description}
	
	Adjectives possess a number of unique features: they can be directly marked for polarity, and they may also be marked for degree.

	\section{Types of Adjectival Forms}
	\label{sec:am_adjectival_forms}

	\subsection{Adjectival Verbs}
	\label{ssec:am_adjectival_verbs}

	Adjectival verbs are, as the name suggests, a set of verb-like forms, derived from Form VII verbal roots. They may predicate sentences, and conjugate to aspect, topical agreement, mood and politeness in the same manner as ordinary verbs, differing in some inflections, notably aspect. The transfix patterns used to indicate aspect are the primary means of deriving attributive verbs; these are given in Table~\ref{tab:am_attributive_verb_aspect}.

	\begin{table}[htpb]\small\capstart
		\begin{center}
			\begin{tabular}{|>{\bfseries}fc->{\scshape}c|-c|-c|}
				\hline
				\SetRowStyle{\bfseries} & & \multicolumn{2}{-c|}{Pattern} \tabularnewline
				\cline{3-4}
				\SetRowStyle{\bfseries} & & Triliteral & Biliteral \tabularnewline
				\hline
				Imperfective & ipfv & 
				\textbf{e}C\sub1C\sub2\textbf{u}C\sub3C\sub3\textbf{i} & 
				\textbf{es}C\sub1\textbf{u}C\sub2C\sub2\textbf{i}
				\tabularnewline
				Stative & stat & 
				\textbf{e}C\sub1C\sub2\textbf{ui}C\sub3C\sub3\textbf{e} & 
				\textbf{es}C\sub1\textbf{ui}C\sub2C\sub2\textbf{e}
				\tabularnewline
				Durative & dur;ipfv & 
				\textbf{e}C\sub1C\sub2\textbf{u}C\sub3C\sub3\textbf{ú} & 
				\textbf{es}C\sub1\textbf{u}C\sub2C\sub2\textbf{ú}
				\tabularnewline
				Frequentative & freq & 
				\textbf{e}C\sub1C\sub2\textbf{u}C\sub3C\sub3\textbf{o} & 
				\textbf{e}C\sub1\textbf{u}C\sub2C\sub2\textbf{o}
				\tabularnewline
				Habitual & hab & 
				\textbf{e}C\sub1C\sub2\textbf{u}C\sub3C\sub3\textbf{a} &
				\textbf{es}C\sub1\textbf{u}C\sub2C\sub2\textbf{a}
				\tabularnewline
				\hline\hline
				Perfective & pfv &
				\textbf{e}C\sub1C\sub2\textbf{io}C\sub3C\sub3\textbf{a} & 
				\textbf{es}C\sub1\textbf{io}C\sub2C\sub2\textbf{a}
				\tabularnewline
				Inchoative & inch & 
				\textbf{e}C\sub1C\sub2\textbf{iu}C\sub3C\sub3\textbf{o} & 
				\textbf{es}C\sub1\textbf{iu}C\sub2C\sub2\textbf{o}
				\tabularnewline
				Cessative & cess & 
				\textbf{e}C\sub1C\sub2\textbf{í}C\sub3C\sub3\textbf{a} & 
				\textbf{es}C\sub1\textbf{í}C\sub2C\sub2\textbf{a}
				\tabularnewline
				Durative & dur;pfv & 
				\textbf{e}C\sub1C\sub2\textbf{ia}C\sub3C\sub3\textbf{u} & 
				\textbf{es}C\sub1\textbf{ia}C\sub2C\sub2\textbf{u}
				\tabularnewline
				Momentane & momt & 
				\textbf{e}C\sub1C\sub2\textbf{iu}C\sub3C\sub3\textbf{a} &
				\textbf{es}C\sub1\textbf{iu}C\sub2C\sub2\textbf{a}
				\tabularnewline
				\hline
			\end{tabular}
			\caption{Adjectival verb aspectual conjugation\label{tab:am_attributive_verb_aspect}}
		\end{center}
	\end{table}

	\subsubsection{Adverbs}
	\label{sssec:am_adverbs}

	Adverbs are a derived class of attributive verbs, formed by appending the suffix \textit{-żi} to the aspectual form that agrees with their head verb.

	\subsection{Adjectival Nouns}
	\label{ssec:am_adjectival_nouns}

	Unlike adjectival verbs, attributive nouns are derived from nominalisations of verb forms, as well as other nouns. The most common nominalisations used to derive adjectival nouns are the verbal nouns and the active and passive participles. Nominal forms can be turned into adjectives by appending the suffix \textit{-mne}.

	\begin{exe}
		\ex\label{exe:am_adjectival_nouns}
		\begin{tabular}[t]{f>{\itshape}l l ->{\itshape}l l}\small
			teşuqqa  & ‘narcissism’ & teşuqqamne  & ‘narcissistic’\\
		\end{tabular}
	\end{exe}

	% ş-q- love, desire
	% teşuqqu love-REFL = love self
	% teşuqqa narcissim
	% teşuqqamne narcissistic
	
	\subsection{Attributives}
	\label{ssec:am_attributives}

	Attributives are also derived from the Form VII root, and have only a single transfix pattern, in contrast to their corresponding adjectival verbs: \textit{iC\sub1C\sub2eC\sub3C\sub3a} for triliteral roots, and \textit{iC\sub1eC\sub2C\sub2a} for biliteral roots.

	\begin{exe}
		\ex\label{exe:am_attributives}
		% \begin{xlist}
		\begin{tabular}[t]{>{\scshape}l>{\itshape}ll >{\itshape}l l}
			p-l-t & pulut & ‘to beautify, to adorn’: & ipletta & ‘beautiful’ \tabularnewline
			t-m-s & itmussu & ‘to be tall’: & itmessa & ‘tall’ \tabularnewline
			\end{tabular}
		% }
		% \end{xlist}
	\end{exe}

	\section{Adjectival Inflection}
	\label{sec:am_adjectival_inflection}

	Adjectives inflect for polarity and degree; neither adjectival verbs, adjectival nouns or attributives agree with their head in number, case, or aspect. The structure of an adjective is:

	\begin{exe}
		\ex\label{ex:am_adjective_structure} \textsc{supl-}\textit{stem}\textsc{-comp-polarity}
	\end{exe}	
	
	The adjectival stem is its base conjugated form, so for an attributive verb, this would include the aspectual, personal and modal marking. 

	\subsection{Degree}
	\label{ssec:am_degree}

	Qevesa adjectives inflect to three degrees of comparison: comparative, superlative and exaggerated. These are indicated by a combination of prefixes and suffixes, which are listed in Table~\ref{tab:am_degree}. Alternatively, the affixes can precede the adjective as an adverbial construction; this is preferred for predicative attributive sentences.

	\begin{table}[htpb]\small\capstart
		\begin{center}
			\begin{tabular}{|>{\bfseries}fc->{\scshape}fc|-c|-c|-c|}
				\hline
				& & \SetRowStyle{\bfseries}Prefix & Suffix & Adverb \tabularnewline
				\hline
				Comparative & comp & ∅    & -vín & vín      \tabularnewline
				Superlative & supl & ko-  & -vín & kovín   \tabularnewline
				Exaggerated & exag & los- & -vín & losvín \tabularnewline
				\hline
			\end{tabular}
			\caption{Adjectival degree adverbs\label{tab:am_degree}}
		\end{center}
	\end{table}

	Table~\ref{tab:am_degree_inflection} gives the adjective comparison marking for the word \textit{itmessa} (tall), and Example~\ref{exe:am_degree} shows two sentences demonstrating the different styles of comparitive marking.

	\begin{table}[htpb]\small\capstart
				\begin{center}
					\begin{tabular}[t]{|>{\bfseries}fc|-c|-c|-c|-c|}
					\hline
					& \SetRowStyle{\bfseries}Adjective & Comparative & Superlative & Exaggerated \\
					\hline
					\multirow{4}{*}{Affixes} & \SetRowStyle{\itshape}itmessa & itmessavín & koitmessavín & lositmessavín \\
					& \SetRowStyle{\itshape}itmessa & itmessa-vín & ko-itmessa-vín & los-itmessa-vín \\
					& tall & tall\textsc{-comp} & \textsc{supl-}tall\textsc{-comp} & \textsc{exag-}tall\textsc{-comp} \\
					& ‘tall’ & ‘taller’ & ‘tallest’ & ‘most tallest’\\
					\hline
					\multirow{3}{*}{Adverbs} & \SetRowStyle{\itshape}itmessa & vín itmessa & kovín itmessa & losvín itmessa \\
					& tall & \textsc{comp} tall & \textsc{supl} tall & \textsc{exag} tall \\
					& ‘tall’ & ‘taller’ & ‘tallest’ & ‘most tallest’\\
					\hline
				\end{tabular}
			\caption{Adjectival degree inflection\label{tab:am_degree_inflection}}
				\end{center}
				\end{table}
	% \newpage
	\begin{exe}
		\ex\label{exe:am_degree} 
		\begin{xlist}
				\ex \textit{Cavoikě noje vín etmuisseşo.}
			\glll Cavoik-ě-∅ no-je vín etmuisse-ş-o\\
			friend\textsc{-1sg;pos-nom} \textsc{comp-1sg} \textsc{comp} tall\textsc{\bs stat-3sg;acc-ind}\\
			{friend my} {than me} {(more)} {tall is}\\
			\glt My friend is taller than me.
				\ex \textit{Cavoikě noje etmuisseşovín.}
			\glll Cavoik-ě-∅ no-je etmuisse-ş-o-vín\\
			friend\textsc{-1sg;pos-nom} \textsc{comp-1sg} tall\textsc{\bs stat-3sg;acc-ind-comp}\\
			{friend my} {than me} {taller is}\\
			\glt My friend is taller than me.
		\end{xlist}
	\end{exe}

	\subsection{Polarity}
	\label{ssec:am_polarity}

	Unlike both verbs and nouns\footnotemark, adjectives can be directly inflected for polarity. Both affirmative and negative suffixes exist, although the affirmative form is only used when a emphasising the existence of the adjectival property. The suffixes for polarity are given in Table~\ref{tab:am_polarity}.
	\footnotetext{Note that the negative determinative prefix, described in Section~\ref{ssec:nm_demonstrative_pronouns} on page~\pageref{ssec:nm_demonstrative_pronouns}, may function as a polarity marker. }

	\begin{table}[htpb]\small\capstart
		\begin{center}
			\begin{tabular}{|>{\bfseries}fc->{\scshape}fc|-c|}
				\hline
				& & \bfseries Suffix \tabularnewline
				\hline
				Affirmative & aff & -şerí \tabularnewline
				Negative & neg & -demí \tabularnewline
				\hline
			\end{tabular}
			\caption{Adjectival polarity suffixes\label{tab:am_polarity}}
		\end{center}
	\end{table}

	\begin{exe}
		\ex\label{exe:am_polarity} 
		\begin{xlist}\ex
		\textsc{t-m-s} \textit{itmussu}, ‘to be tall’:\\[2\parskip]\small
				\begin{tabular}[t]{fc -c -c -c}
					\SetRowStyle{\itshape}itmessa & itmessaşerí & itmessademí \\
					\SetRowStyle{\itshape}itmessa & itmessa-şerí & itmessa-demí \\
					tall & tall\textsc{-aff} & tall\textsc{-neg} \\
					‘tall’ & ‘very tall’ & ‘not tall’\\
				\end{tabular}
				\ex \textit{Cavoikě etmuisseşodemí.}
			\glll Cavoik-ě-∅ etmuisse-ş-o-demí\\
			friend\textsc{-1sg;pos-nom} tall\textsc{\bs stat-3sg;acc-ind-neg}\\
			{friend my} {tall is not}\\
			\glt My friend is not tall.
		\end{xlist}
	\end{exe}

\end{document}
