\documentclass[grammar]{subfiles}
\begin{document}
\chapter{Adjectival Morphology}
\label{ch:adjectival-morphology}

Qevesa does not possess adjectives in the syntactic sense, though there are
words that function as adjectives in the semantic sense.  These are distributed
into two morphological classes, with some overlap between them:

\begin{itemize}
  \item Adjectival verbs have verbal roots and conjugate as stative verbs.
  \item Adjectival nouns are nouns that combine with the intransitive copula.
\end{itemize}

Unlike adjectives in languages like English, adjectival verbs in Qevesa inflect
for aspect, mood and person.  Every adjective can be used in an attributive
position, and nearly every adjective can be used in a predicative position.
Both the predicative and attributive forms can be reanalysed as verb phrases,
making the attributive forms of adjectival verbs and adjectival nouns relative
clauses. 


\section{Adjectival Inflection}
\label{sec:am_adjectival_inflection}

Adjectival words do have additional inflections that aren’t used with non-adjectival verbs and nouns.  primarily inflect for degree.  The structure of an adjective is:
%
%\begin{exe}
%  \ex\label{ex:am_adjective_structure} \textsc{supl-}\textit{stem}\textsc{-cmpr}
%\end{exe}	
%
%The adjectival stem is its base conjugated form, so for an attributive verb,
%this would include the aspectual, modal and personal marking. 
%
%\subsection{Degree}
%\label{ssec:am_degree}
%
%Qevesa adjectives inflect to three degrees of comparison: comparative,
%superlative and exaggerated.  These are indicated by a combination of prefixes
%and suffixes, which are listed in \cref{tab:am_degree}.
%
%\begin{table}[h!]\small\capstart
%  \begin{tabular}{BFl -Sl -l -l}
%    \toprule
%    \SetRowStyle{\bfseries} Degree & & Prefix & Suffix \tnl
%    \midrule
%    Comparative & \acs{cmpr} & ∅    & -vén \tnl
%    Superlative & \acs{supl} & ko-  & -vén \tnl
%    Exaggerated & \acs{exag} & los- & -vén \tnl
%    \bottomrule
%  \end{tabular}
%  \caption{Adjectival degree adverbs\label{tab:am_degree}}
%\end{table}

\end{document}
