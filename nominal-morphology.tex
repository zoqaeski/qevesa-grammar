\documentclass[grammar]{subfiles}
\begin{document}
	\chapter{Nominal Morphology}
	\label{ch:nominal_morphology}

	\section{Definitions and Features}
	\label{sec:nm_definition_features}

	The basic structure of the Qevesa noun consists of a stem composed of a root and zero or more derivational affixes, plus declensional affixes.

	Nouns, like verbs, are highly regular in their declension. There is no grammatical gender, although some nouns, such as body parts, do possess an inherent gender. Explicit constructions to indicate gender are used only when necessary, and these are seldom used except in the formal or literary language. Nouns may be classed according to animacy, a feature which is only indicated by the declension affixes.

	Qevesa nouns decline to indicate two non-inherent features: number and case. Most nouns have three numbers, a singular, dual or collective, and plural, although a small, closed set have a natural number and receive inverse marking. There are fourteen cases in the standard written language: nominative, ergative, accusative, secundative, genitive, essive, instrumental-commitative, inessive, adessive, illative, allative, elative, ablative and comparative. A fifteenth case, the vocative, exists in some spoken dialects, but this is falling out of use\footnotemark.
	\footnotetext{It is interesting to note that the vocative case is commonly used when insulting people regardless of dialect.}

	The citation form of all nouns is the nominative singular.

	\subsection{Animacy}
	\label{ssec:nm_animacy}

	Nouns in the Teralo family of languages display a property known as animacy, in which nouns referring to humans, animals and other things perceived as having consciousness or life decline differently to other nouns in some forms. The animacy of a noun must be known in order to properly decline it to the primary cases and to indicate pronomial forms.

	Animate nouns refer to humans, animals, spirits, some plants, and some meteorological and geological phenomena. This includes personal names, posessions, and some body parts. Most living but inanimate life forms are not included, such as the majority of plants, as wells as microbial life forms. Animacy is a fixed feature, so nouns may not switch between animate and inanimate declensions. Exceptions to this include named objects as well as some towns and cities.

	\subsection{Proper Nouns}
	\label{ssec:nm_proper_nouns}

	Proper nouns may be formed from words existing in the language\footnotemark{}, often supported by gender markers to disambiguate them from common nouns, especially when used as personal names. A noticeable morphological feature of proper nouns is that their case markers are enclitic rather than suffixed, separated by a colon or a non-breaking space. Proper names are seldom pluralised.
	\footnotetext{See Section~\ref{ssec:dev_proper_nouns} on page~\pageref{ssec:dev_proper_nouns} for derivation of proper nouns.}

	% \subsection{Count and Mass Nouns}
	% \label{ssec:nm_count_mass_nouns}

	% Qevesa distinguishes between count and mass nouns, for example:

	% \begin{exe}
	% 	\ex \emph{EXAMPLES}
	% \end{exe}

	% Materials and abstract qualities cannot be counted, and are grammatically singular. Things that do not usually occur as singular items are sometimes uncountable and possess a natural number, receiving inverse number when it must be explicitly indicated. Body parts are typically included in this set:

	% \begin{exe}
	% 	\ex \emph{EXAMPLES}
	% \end{exe}

	\section{Nominal Declension}
	\label{sec:nm_declension}

	Qevesa noun words consist of the stem, followed by number, possessor and case marking:

	\begin{exe}
		\ex\label{ex:nm_structure} \textit{stem}\textsc{-number-possessor-case}
	\end{exe}

	\subsection{Case}
	\label{ssec:nm_case}

	Qevesa possess fourteen cases (fifteen if the marginal vocative is included), which are divided into two groups. The primary cases, of which there are four, indicate morphosyntactic roles of the noun with respect to the verb; the remaining ten cases are the secondary cases, and these are mostly locative and adverbial cases. 
	
	The case suffixes are listed in Table~\ref{tab:nm_case_suffixes}. Noteworthy is that the suffixes could also be analysed as consisting of an animacy suffix followed by the case suffix.

	\begin{table}[htpb]\small\capstart
		\begin{center}
			\begin{tabular}{|>{\bfseries}fc>{\scshape}c|-c-c|-c-c|}
				\hline
				\multicolumn{2}{|fc|}{\SetRowStyle{\bfseries}\multirow{2}{*}{Case}} & \multicolumn{2}{-c|}{Animate} & \multicolumn{2}{-c|}{Inanimate} \tabularnewline
				\cline{3-6}
				\SetRowStyle{\scshape} & & \multicolumn{2}{-c|}{anim} & \multicolumn{2}{-c|}{inanim} \tabularnewline
				\hline
				Nominative		& nom & -a  & -∅ & -ina  & -na \tabularnewline
				Ergative			& erg & -am & -m & -inom & -nom \tabularnewline
				Accusative		& acc & -aş & -ş & -inoş & -noş \tabularnewline
				Secundative		& sdt & -ot & -t & -inot & -not \tabularnewline
				\hline\hline
				Genitive			& gen  & -aik  & -k     & -inok & -nok \tabularnewline
				Essive				& ess  & -ěl   & -jel   & -ol   & -jol \tabularnewline
				Instrumental (Comitative)    & ins    & -ětti & -jetti & -otti & -notti \tabularnewline
				Inessive			& ine  & -ěssi & -jessi & -ossi & -nossi  \tabularnewline
				Adessive			& ade  & -ědi  & -jedi  & -odi  & -nodi   \tabularnewline
				Illative			& ill  & -ěsto & -jesto & -osto & -nosto  \tabularnewline
				Allative			& all  & -ěfti & -jefti & -ofti & -nofti  \tabularnewline
				Elative				& ela  & -ěspo & -jespo & -ospo & -nospo  \tabularnewline
				Ablative			& abl  & -ěsko & -jesko & -osko & -nosko  \tabularnewline
				Comparative		& comp & -ěnno & -jenno & -onno & -nenno \tabularnewline
				(Vocative)		& voc  & -ó    & -jó    & & \tabularnewline
				\hline
			\end{tabular}
			\caption{Case suffixes\label{tab:nm_case_suffixes}}
		\end{center}
	\end{table}


	\subsubsection{The Primary Cases}
	\label{sssec:nm_primary_cases}

	The primary cases indicate the morphosyntactic role of the noun with respect to the verb.

	\paragraph{Nominative}
	\label{par:nm_nominative_case}

	The nominative case marks the topic of the verb phrase. Its actual role is indicated on the verb, using the topical agreement suffixes as described in Section~\ref{ssec:vm_topical_agreement}.

	\paragraph{Ergative}
	\label{par:nm_ergative_case}

	The ergative case marks the voluntary experiencer of an intransitive verb, the agent of a transitive verb, or the donor of a ditransitive verb.

	\paragraph{Accusative}
	\label{par:nm_accusative_case}

	The accusative case marks the involuntary experiencer of an intransitive verb, the patient of a transitive verb, or the recipient of a ditransitive verb.

	\paragraph{Secundative}
	\label{par:nm_secundative_case}

	The secundative case marks the theme of a ditransitive verb.

	\subsubsection{The Secondary Cases}
	\label{sssec:nm_secondary_cases}

	The secondary cases are mainly adpositional and locative cases.

	\paragraph{Genitive}
	\label{par:nm_genitive_case}

	The genitive case indicates the possessor of another noun. Pronomial possessors are indicated by means of a suffix on the possessed item.

	\paragraph{Essive}
	\label{par:nm_essive_case}

	The essive case indicates duration and time. It also indicates a temporary state of being or existence.

	\paragraph{Instrumental (Comitative)}
	\label{par:nm_instrumental_case}

	The instrumental case indicates the means by which the action is performed. It may also be used in a comitative sense, i.e. to indicate the person in whose company the action is carried out.

	\paragraph{Inessive}
	\label{par:nm_inessive_case}

	The inessive case indicates internal location. 

	\paragraph{Adessive}
	\label{par:nm_adessive_case}

	The adessive case indicates external location.

	\paragraph{Illative}
	\label{par:nm_illative_case}

	The illative case indicates motion from the exterior to the interior.

	\paragraph{Allative}
	\label{par:nm_allative_case}

	The allative case indicates motion towards the noun.

	\paragraph{Elative}
	\label{par:nm_elative_case}

	The elative case indicates motion from the interior to the exterior.

	\paragraph{Ablative}
	\label{par:nm_ablative_case}

	The ablative case indicates motion away from the noun.

	\paragraph{Comparative}
	\label{par:nm_comparative_case}

	The comparative case indicates a likeness to something, or the standard to which something is compared.

	\subsubsection{Use of the Locative Cases}
	\label{sssec:nm_locative_cases}

	The locative cases are logically grouped. There are two positions (internal and external) and three directions (static, movement towards and movement away). Combining these results in the six cases, illustrated in Table~\ref{tab:nm_locative_cases}.

	\begin{table}[htpb]\small\capstart
		\begin{center}
			\begin{tabular}{|>{\bfseries}fc|-c|-c|}
				\hline
				\SetRowStyle{\bfseries} & Interior & Exterior \tabularnewline
				\hline
				Static & Inessive & Adessive \tabularnewline
				Movement towards & Illative & Allative \tabularnewline
				Movement away & Elative & Ablative \tabularnewline
				\hline
			\end{tabular}
			\caption{Locative cases\label{tab:nm_locative_cases}}
		\end{center}
	\end{table}

	Finer distinctions in location are given with postpositions, which are described in Section~\ref{sec:postpositions}.

	\subsection{Number}
	\label{ssec:nm_number}

	Qevesa possesses at least three forms of grammatical number: singular, dual/collective, and plural. In addition, a number of irregular nouns, such as body parts, possess a natural number, for which there is a singulative form to indicate exactly one of the noun. Number is indicated by appending a suffix, inserting an epenthetic \emph{-e-} if the stem ends in a consonant, or lengthening the final vowel if the stem ends in a vowel. Some examples are given on page~\pageref{exe:nm_number} in Example~\ref{exe:nm_number}.
	
	The suffixes for number are given in Table~\ref{tab:nm_number_suffixes}. 
	% eye → eyes:  miara → miaras 
	% friend → friends: cavoik → cavoikes
	% bread → breads: litas → litases
	% boy → boys: tokit → tokites 
	% this-PROX person → these-PROX people: totka → totkás
	% 

	\begin{table}[htpb]\small\capstart
		\begin{center}
			\begin{tabular}{|>{\bfseries}fc->{\scshape}fc|-c|}
				\hline
				 & & \bfseries Suffix \tabularnewline
				 \hline
				 Singular        & sg &  -∅ \tabularnewline
				 Dual/Collective & du & -(e)v \tabularnewline
				 Plural          & pl & -(e)s \tabularnewline
				 Singulative     & sgv & -sen \tabularnewline
				\hline
			\end{tabular}
			\caption{Grammatical number suffixes\label{tab:nm_number_suffixes}}
		\end{center}
	\end{table}

	The dual number is of particular note. By itself, it indicates that there are exactly two of the noun. However, if a quantity is to be specified, such as with a number word or quantifier, the dual form is also used. The singulative is used to indicate exactly one of the specified item, in situations where the expected number differs from the actual number.

	\begin{exe}
		\ex\label{exe:nm_number}
		\begin{tabular}[t]{f>{\itshape}l l ->{\itshape}l l ->{\itshape}l l}\small
			miara  & ‘eye’ & miaráv  & ‘two eyes’ & miarás  & ‘eyes’\\
			tolik  & ‘boy’ & tolikev & ‘two boys’ & tolikes & ‘boys’\\
			cavoik & ‘friend’ & cavoikev & ‘two friends’ & cavoikes & ‘friends’\\
		\end{tabular}
	\end{exe}

	\section{Articles}
	\label{sec:nm_articles}

	Qevesa possesses a definite article but no indefinite article. It normally consists of \textit{ła}, but before a vowel the ‹a› may be elided and the article attached as a proclitic \textit{ł’-}. \ToBeWritten

	\textbf{Move this to syntax?}

	\section{Pronouns and Pronomial forms}
	\label{sec:nm_pronouns}

	Pronouns are roughly equivalent to nouns in terms of syntax and morphology. They serve as substitutes for other nouns or noun phrases that have previously been mentioned or can be inferred from context. There are a number of types of pronouns in Qevesa, including personal pronouns, demonstrative pronouns and interrogative pronouns.

	The class of determiners is a special case, in that they can also act as articles for other nouns or noun phrases.

	\subsection{Personal Pronouns}
	\label{ssec:nm_personal_pronouns}

	The personal pronouns stand in for other nouns, indicating that noun's person, number and case. Most personal pronouns refer only to animate referents: a separate inanimate pronoun is used for inanimate referents. There are two first person plural pronouns, an inclusive, which includes the listener, and an exclusive, which does not. 

	Personal pronouns are declined to the primary cases by suffixation; other case constructions use a stem derived from the case ending combined with the suffix form of the pronoun. Although a genetive form of the personal pronouns exists, the suffix form is preferred. 

	The base forms of the pronouns are given in Table~\ref{tab:nm_pronoun_primary_case}, and the cases with personal suffixes are given in Table~\ref{tab:nm_personal_cases}.

	\begin{table}[htpb]\small\capstart
		\begin{center}
			\begin{tabular}{|>{\scshape}fc|-c|-c|-c|-c|-c|-c|-c|}
				\hline
				\SetRowStyle{\bfseries} & \multicolumn{2}{-c|}{Stem} & \multicolumn{5}{-c|}{Cases}\tabularnewline
				\cline{2-8}
				& Root & Suffix &\SetRowStyle{\scshape} nom & erg & acc & sdt & gen \tabularnewline
				\hline
				1sg & je & -ě/-je & je & jem & jeş & jet & jek \tabularnewline
				2sg & ta & -ta & ta & tam & taş & tajot & tak \tabularnewline
				3sg & mi & -mi & mi & mim & miş & mijot & mik \tabularnewline
				1du;inc & jév & -jév & jéva & jévam & jévaş & jévot & jévaik \tabularnewline
				1du;exc & čév & -čév & čéva & čévam & čévaş & čévot & čévaik \tabularnewline
				2du & táv & -táv & táva & távam & távaş & távot & távaik \tabularnewline
				3du & mív & -mív & míva & mívam & mívaş & mívot & mívaik \tabularnewline
				1du;inc & jés & -jés & jésa & jésam & jésaş & jésot & jésaik \tabularnewline
				1du;exc & čés & -čés & čésa & čésam & čésaş & čésot & čésaik \tabularnewline
				2du & tás & -tás & tása & tásam & tásaş & tásot & tásaik \tabularnewline
				3du & mís & -mís & mísa & mísam & mísaş & mísot & mísaik \tabularnewline
				3;inanim & net & -net & neta & netam & netaş & netot & netaik \tabularnewline
				\hline
			\end{tabular}
			\caption{Personal pronouns\label{tab:nm_pronoun_primary_case}}
		\end{center}
	\end{table}

	\begin{table}[htpb]\small\capstart
		\begin{center}
			\subfloat{
			\begin{tabular}{|>{\scshape}fc|->{\itshape}c|-c|-c|-c|-c|-c|}
				\hline
				\multicolumn{2}{|fc|}{\SetRowStyle{\bfseries}} & \multicolumn{5}{-c|}{Cases}\tabularnewline
				\cline{3-7}
				\multicolumn{2}{|fc|}{\SetRowStyle{\scshape}} & ess & ins & ine & ade & ill \tabularnewline
				\cline{3-7}
				\multicolumn{2}{|fc|}{\SetRowStyle{\itshape}} & ěl- & ětt(i)- & ěss(i)- & ěd(i)- & ěsto- \tabularnewline
				\hline
				1sg & -je & ělje & ěttje & ěssje & ědje & ěstoje \tabularnewline
				2sg & -ta & ělta & ěttita & ěssta & ědita & ěstota \tabularnewline
				3sg & -mi & ělmi & ěttimi & ěssmi & ědmi & ěstomi \tabularnewline
				1du;inc & -jév & ěljév & ěttjév & ěssjév & ědjév & ěstojév \tabularnewline
				1du;exc & -čév & ělčév & ěttčév & ěssčév & ědčév & ěstočév \tabularnewline
				2du & -táv & ěltáv & ěttitáv & ěsstáv & ěditáv & ěstotáv \tabularnewline
				3du & -mív & ělmív & ěttimív & ěssmív & ědmív & ěstomív \tabularnewline
				1pl;inc & -jés & ěljés & ěttjés & ěssjés & ědjés & ěstojés \tabularnewline
				1pl;exc & -čés & ělčés & ěttčés & ěssčés & ědčés & ěstočés \tabularnewline
				2pl & -tás & ěltás & ěttitás & ěsstás & ěditás & ěstotás \tabularnewline
				3pl & -mís & ělmís & ěttimís & ěssmís & ědmís & ěstomís \tabularnewline
				3;inanim & -net & ělnet & ěttinet & ěssnet & ědnet & ěstonet \tabularnewline
				\hline
			\end{tabular}
		}\\
		\subfloat{
			\begin{tabular}{|>{\scshape}fc|->{\itshape}c|-c|-c|-c|-c|}
				\hline
				\multicolumn{2}{|fc|}{\SetRowStyle{\bfseries}} & \multicolumn{4}{-c|}{Cases}\tabularnewline
				\cline{3-6}
				\multicolumn{2}{|fc|}{\SetRowStyle{\scshape}} & all & ela & abl & comp \tabularnewline
				\cline{3-6}
				\multicolumn{2}{|fc|}{\SetRowStyle{\itshape}} & ěft(i)- & ěspo- & ěsko- & no- \tabularnewline
				\hline
				1sg & -je      & ěftije & ěspoje & ěskoje & noje \tabularnewline
				2sg & -ta      & ěftita & ěspota & ěskota & nota \tabularnewline
				3sg & -mi      & ěftimi & ěspomi & ěskomi & nomi \tabularnewline
				1du;inc & -jév & ěftijev & ěspojév & ěskojév & nojév \tabularnewline
				1du;exc & -čév & ěftičév & ěspočév & ěskočév & nočév \tabularnewline
				2du & -táv     & ěftitáv & ěspotáv & ěskotáv & notáv \tabularnewline
				3du & -mív     & ěftimív & ěspomív & ěskomív & nomív \tabularnewline
				1pl;inc & -jés & ěftijés & ěspojés & ěskojés & nojés \tabularnewline
				1pl;exc & -čés & ěftičés & ěspočés & ěskočés & nočés \tabularnewline
				2pl & -tás     & ěftitás & ěspotás & ěskotás & notás \tabularnewline
				3pl & -mís     & ěftimís & ěspomís & ěskomís & nomís \tabularnewline
				3;inanim & -net & ěftinet & ěsponet & ěskonet & nonet \tabularnewline
				\hline
			\end{tabular}
		}
			\caption{Cases with personal suffixes\label{tab:nm_personal_cases}}
		\end{center}
	\end{table}

	\newpage
	\subsection{Reflexive and Reciprocal Pronouns}
	\label{ssec:nm_reflexive_pronouns}

	Qevesa does not possess reflexive or reciprocal pronouns as most verb roots have forms that indicate reflexive\footnote{See  Section~\ref{sec:dev_verb_form_v}, page~\pageref{sec:dev_verb_form_v}} or reciprocal\footnote{See  Section~\ref{sec:dev_verb_form_iii}, page~\pageref{sec:dev_verb_form_iii}} actions. The word \textit{máka} ‘self’, may be used as a reflexive pronoun to avoid ambiguity, but this is rare.
	% Qevesa possesses a single reflexive pronoun, \textit{máka} ‘self’, used to refer to something already mentioned. It inflects with the personal suffixes to agree in person with its antecedent. The resulting set of pronouns is given in Table~\ref{tab:nm_reflexive_pronouns}. A related pronoun is the reciprocal pronoun \textit{mákom}, which does not take personal suffixes.

	% \begin{table}[htpb]\small\capstart
	% 	\begin{center}
	% 		\begin{tabular}{|>{\scshape}fc|-c|}
	% 			\hline
	% 			\SetRowStyle{\bfseries} & Reflexive \tabularnewline
	% 			%\cline{2-2}
	% 			\SetRowStyle{\scshape} & refl \tabularnewline
	% 			\hline
	% 			1sg      & mákaje  \tabularnewline
	% 			2sg      & mákati  \tabularnewline
	% 			3sg      & mákami  \tabularnewline
	% 			1du;inc  & mákajév \tabularnewline
	% 			1du;exc  & mákačév \tabularnewline
	% 			2du      & mákatáv \tabularnewline
	% 			3du      & mákamív \tabularnewline
	% 			1pl;inc  & mákajés \tabularnewline
	% 			1pl;exc  & mákačés \tabularnewline
	% 			2pl      & mákatás \tabularnewline
	% 			3pl      & mákamís \tabularnewline
	% 			3;inanim & mákanet \tabularnewline
	% 			recp     & mákom \tabularnewline
	% 			\hline
	% 		\end{tabular}
	% 		\caption{Reflexive and reciprocal pronouns\label{tab:nm_reflexive_pronouns}}
	% 	\end{center}
	% \end{table}

	% This might fix referencing issues
	%\newpage

	\subsection{Demonstrative and Correlative Pronouns}
	\label{ssec:nm_demonstrative_pronouns}

	Qevesa has three degrees of demonstrative pronouns:

	\begin{description}[style=nextline]
		\item[Proximal] The proximal series is marked by the prefix \textit{to-}, and refers to things closer to the speaker than the listener;
		\item[Medial] The medial series is marked by the prefix \textit{ko-}, and refers to things closer to the listener than the speaker; and
		\item[Distal] The distal series, marked by the prefix \textit{isá-}, refers to things that are far from both speaker and listener.
	\end{description}

	There is also an interrogative series, which is marked with the prefix \textit{qe-}. Demonstrative pronouns must agree in number and case with their antecedent, unlike all other types of modifiers, such as adjectives. 

	A related set of pronouns is formed by prefixes denoting number or quantity. These include the existential, elective, universal and negative series, and combine with the suffixes in a highly regular manner. 

	The demonstrative and correlative pronouns are are listed in Table~\ref{tab:nm_demonstrative_correlative_pronouns}.

	\begin{table}[htpb]\small\capstart
		\begin{center}
			\subfloat[][Demonstrative pronouns]{
			\begin{tabular}{|>{\bfseries}fc->{\scshape}c|->{\itshape}c|-c|-c|-c|-c|}
				\hline
				\multicolumn{3}{|fc|}{\SetRowStyle{\bfseries}} & Proximal & Medial & Distal & Interrogative \tabularnewline
				%\cline{3-6}
				\multicolumn{3}{|fc|}{\SetRowStyle{\scshape}} & prox & med & dist & int \tabularnewline
				\cline{4-7}
				\multicolumn{3}{|fc|}{\SetRowStyle{\itshape}} & to- & ko- & isá- & qe- \tabularnewline
				\hline
				Human				& hum  & -tka & totka & kotka & isátka & qetka \tabularnewline
				Nonhuman		& nh   & -ra  & tora  & kora  & isára  & qera \tabularnewline
				Location		& loc  & -zól & tozól & kozól & isázól & qezól \tabularnewline
				Source			& src  & -ská & toská & koská & isáská & qeská \tabularnewline
				Destination & dest & -rve & torve & korve & isárve & qerve \tabularnewline
				Time				& time & -lti & tolti & kolti & isálti & qelti \tabularnewline
				Manner			& man  & -ttu & tottu & kottu & isáttu & qettu \tabularnewline
				Reason			& rsn  & -rte & torte & korte & isárte & qerte \tabularnewline
				\hline
			\end{tabular}}\\
			\subfloat[][Correlative pronouns]{
				\begin{tabular}{|>{\bfseries}fc->{\scshape}c|->{\itshape}c|-c|-c|-c|-c|}
				\hline
				\multicolumn{3}{|fc|}{\SetRowStyle{\bfseries}} & Existential & Elective & Universal & Negative \tabularnewline
				%\cline{3-6}
				\multicolumn{3}{|fc|}{\SetRowStyle{\scshape}} & exist & elect & univ & neg \tabularnewline
				\cline{4-7}
				\multicolumn{3}{|fc|}{\SetRowStyle{\itshape}} & ane- & via- & minű- & domo- \tabularnewline
				\hline
				Human				& hum  & -tka & anetka & viatka & minűtka & domotka \tabularnewline
				Nonhuman		& nh   & -ra  & anera  & viara  & minűra  & domora  \tabularnewline
				Location		& loc  & -zól & anezól & viazól & minűzól & domozól \tabularnewline
				Source			& src  & -ská & aneská & viaská & minűská & domoská \tabularnewline
				Destination & dest & -rve & anerve & viarve & minűrve & domorve \tabularnewline
				Time				& time & -lti & anelti & vialti & minűlti & domolti \tabularnewline
				Manner			& man  & -ttu & anettu & viattu & minűttu & domottu \tabularnewline
				Reason			& rsn  & -rte & anerte & viarte & minűrte & domorte \tabularnewline
				\hline
			\end{tabular}}
			\caption{Demonstrative and correlative pronouns\label{tab:nm_demonstrative_correlative_pronouns}}
		\end{center}
	\end{table}

	The prefixes for each series of demonstratives can also be combined with case suffixes, to produce pronouns of specific direction or location, for example:

	\begin{exe}
		\ex\label{tab:nm_demonstrative_prefixes}
		\begin{tabular}[t]{fl -l}
			\SetRowStyle{\itshape} tojessi & qejefti \tabularnewline
			to-jessi & qe-jefti \tabularnewline
			\SetRowStyle{\scshape} prox-ine & int-all \tabularnewline
			in(side) here & towards where?
		\end{tabular}
	\end{exe}

\end{document}
