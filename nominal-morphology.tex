\documentclass[grammar]{subfiles}
\begin{document}
\chapter{Nominal Morphology}
\label{ch:nominal_morphology}


\section{Definitions and Features}
\label{sec:nm_definition_features}

Qevesa nouns, like verbs, are highly regular in their declension.  They
inflect for two non-inherent features: number and case.  They are also
occasionally marked for animacy, though this is inherent in the noun, and
thus is usually only indicated by the declension affixes. 

Unlike in some languages, there is no grammatical gender.  Instead, Qevesa
uses natural gender, and this is an inherent feature of the noun that is
neither marked nor affects declension.  Explicit constructions to distinguish
gender may be used when necessary.

Most nouns have three numbers, a singular, dual or quantitative, and plural,
although a small, closed set have a natural number and receive inverse
marking. 

There are seven cases in the standard written language: direct, ergative,
accusative, instrumental, genitive, essive, locative. 
% There are fourteen cases in the standard written language: direct, ergative,
% accusative, secundative, genitive, essive, instrumental-commitative,
% inessive, adessive, illative, allative, elative, ablative and comparative.
%
Nouns can also be marked for four states, which are different types of determinateness.

The citation form of all nouns is the unmarked form, that is, with no suffixes or prefixes.


\subsection{Animacy}
\label{ssec:nm_animacy}

Nouns in the Teranean family of languages display a property known as animacy,
in which nouns referring to humans, animals and other things perceived as
having consciousness or life decline differently to other nouns in some
forms.  The animacy of a noun must be known in order to properly decline it
to the primary cases and to indicate pronomial forms.

Animate nouns refer to humans, animals, spirits, some plants, and some
meteorological and geological phenomena.  This includes personal names,
possessions, and some body parts.  Most living but inanimate life forms are
not included, such as the majority of plants, as wells as microbial life
forms.  Animacy is a fixed feature, so nouns may not switch between animate
and inanimate declensions.  Exceptions to this include named objects as well
as some towns and cities.

%  \subsection{Proper Nouns}
%  \label{ssec:nm_proper_nouns}

%  Proper nouns may be formed from words existing in the language, often
%  supported by gender markers to disambiguate them from common nouns,
%  especially when used as personal names.  

% \subsection{Count and Mass Nouns}
% \label{ssec:nm_count_mass_nouns}

% Qevesa distinguishes between count and mass nouns, for example:

% \begin{exe}
% 	\ex \emph{EXAMPLES}
% \end{exe}

% Materials and abstract qualities cannot be counted, and are grammatically
% singular.  Things that do not usually occur as singular items are sometimes
% uncountable and possess a natural number, receiving inverse number when it
% must be explicitly indicated.  Body parts are typically included in this
% set:

% \begin{exe}
% 	\ex \emph{EXAMPLES}
% \end{exe}


\section{Nominal Declension}
\label{sec:nm_declension}

Qevesa noun words consist of the stem, followed by number, possessor and case marking:

\begin{exe}
  \ex\label{ex:nm_structure} \textit{stem}\textsc{-number-possessor-case}
\end{exe}


\subsection{Number}
\label{ssec:nm_number}

Qevesa nouns have four numbers, singular, dual, plural and partitive, which are
typically indicated by the suffixes listed in \cref{tab:nm_number_suffixes}.  A
small, closed set of nouns has suppletive plural forms; these may be so-called 
\emph{broken plurals} or separate roots entirely. 

Number marking in Qevesa functions in a somewhat unusual manner in that every
noun has an inherent “natural” number, which is its default, unmarked form.
The suffixes are appended to indicate that the quantity differs from what is
expected.  Most nouns default to the implicit singular; some nouns, such as
body parts and items of clothing that come in pairs are implicitly dual
(\conlang{méri} “eyes”); and other nouns may be implicitly plural or partial
(particularly uncountable nouns). 

% The indefinite suffix is marked with an \conlang{-e} after a consonant, and is
% unmarked on nouns that end with a vowel, except if the vowel is \conlang{-i} in which
% case the indefinite suffix replaces it.  

% Definiteness may also be indicated by the prepositional
% articles \conlang{a} or \conlang{az}.

The singulative suffix is \conlang{-r}. If the noun stem ends in a vowel, different
epenthetic vowels occur between the suffix and the stem depending on whether
the noun is animate, inanimate, or possessed: animate nouns use \conlang{-a-};
inanimate nouns use \conlang{-o-}; and nouns preceded by a possessive
pronoun use \conlang{-e-}.

The dual number functions to indicate exact quantities.  By itself, it
indicates exactly two of the noun; however, it is also used when the noun is
preceded by a modifier that indicates an exact quantity, such as a number word.

In contrast to the dual, the plural number is used for unspecified quantities
greater than the singular.  The plural suffix may also encode definiteness,
especially for those nouns whose unmarked form has an implicit number. 

The partitive is used to express partialness or inexact quantities.  It may
also be used to indicate telicity, or an incompleteness of the verb, especially
with the perfective aspect.
% It may also be used in an atelic sense. ???

An epenthetic \conlang{-e-} is inserted after a consonant for the dual and plural
suffixes; the partitive uses an \conlang{-i-} instead. 

\begin{table}[h!]\small\capstart
  \begin{tabular}{BFl -Kc -l}
    \toprule
    \multicolumn{2}{Fl}{\rowstyle{\bfseries}Number} & Suffix \\
    \midrule
    % Indefinite        & {\Indef} & -∅, -e \\
    Singulative       & {\Sgv}   & -(a)r, -(o)r, -(e)r \\
    Dual/Quantitative & {\Du}    & -(e)v  \\
    Plural            & {\Pl}    & -(e)s  \\
    Partitive         & {\Part}  & -(i)n  \\
    \bottomrule
  \end{tabular}
  \caption{Grammatical number suffixes\label{tab:nm_number_suffixes}}
\end{table}


\subsection{Case}
\label{ssec:nm_case}

Qevesa possesses seven cases: direct, ergative, accusative, instrumental,
genitive, essive, and locative.  The case suffixes are listed in
\cref{tab:nm_case_suffixes}; the main distinguishing feature between animate
and inanimate nouns is that animate nouns generally use \conlang{-a-} as an epenthic
vowel inserted after a consonant stem, whereas inanimate nouns use \conlang{-o-}
instead.

\begin{table}[h!]\small\capstart
  \begin{tabular}{BFl -Kc -l -l -l -l}
    \toprule
    \multicolumn{2}{Fc}{\rowstyle{\bfseries}Noun Case} & \multicolumn{4}{-c}{Suffixes} \\
    % \rowstyle{\bfseries} & & \multicolumn{2}{-c}{Animate} & {Inanimate} \\
    \rowstyle{\scshape} & & \multicolumn{2}{-c}{{\Anim}} & {{\Inan}} & {{\Pl}} \\
    \midrule
    Direct       & {\Dir}  & \multicolumn{3}{-l}{-a, -∅} & -i \\
    Ergative     & {\Erg}  & -m   & -am  & —    & -im \\
    Accusative   & {\Acc}  & -š   & -aš  & -oš  & -iš \\
    Instrumental & {\Ins}  & -t   & -at  & -ot  & -it \\
    Genitive     & {\Gen}  & -k   & -ak  & -ok  & -ik \\
    Essive       & {\Ess}  & -d   & -ad  & -od  & -id \\
    Locative     & {\Loc}  & -st  & -ast & -ost & -ist \\
    % Vocative     & {\Voc}  & \multicolumn{4}{-c}{-o} \\
    \bottomrule
  \end{tabular}
  \caption{Case suffixes\label{tab:nm_case_suffixes}}
\end{table}

\subsubsection{Direct}
\label{sssec:ns_direct_case}

The direct case marks the topic of the verb phrase.  This may be the
experiencer (both voluntary and involuntary) of an intransitive verb, the agent
or patient of a transitive verb, or (less commonly) some other argument of the
verb.  In this latter case, the direct suffix is stacked onto the other case
suffix. 

Typically, animate nouns in the direct case are the voluntary experiencers or
agents of verbs, and inanimate nouns in the direct case are experiencers or
patients. 

The direct case is only marked if the noun ends with a consonant. 


\subsubsection{Ergative}
\label{sssec:ns_ergative_case}

The ergative case marks the agent of a transitive verb.  Inanimate nouns cannot
be marked with the ergative case, because an inanimate entity is considered
incapable of acting of its own accord. 


\subsubsection{Accusative}
\label{sssec:ns_accusative_case}

The accusative case marks the patient of a transitive verb or the recipient of
ditransitive verb.


\subsubsection{Instrumental}
\label{sssec:ns_instrumental_case}

Qevesa is a secundative language—the recipient of a ditransitive verb
is treated the same as the patient of a monotransitive verb. The instrumental
case case marks the theme of a ditransitive verb, as well as indicating the
means by which the action is performed.  Inanimate agents of verbs are also
marked with the instrumental case.  

\Tbw


\subsubsection{Genitive}
\label{sssec:ns_genitive_case}

The genitive case indicates the possessor of another noun.  Animate pronomial
possessors are usually indicated by means of a suffix on the possessed noun.


\subsubsection{Essive}
\label{sssec:ns_essive_case}
 
The essive case is used to indicate duration and time, as well as temporary
states of being or existence.  It is also used to form adverbs from adjectival
nouns.
 
 
\subsubsection{Locative}
\label{sssec:ns_locative_case}
 
The locative case is used to denote location, and may be used before certain
postpositions with meanings other than location.  It is the only case that
cannot be used without a postposition.

% 
% 
% \subsubsection{Adessive}
% \label{sssec:ns_adessive_case}
% 
% The adessive case indicates external location.
% 
% 
% \subsubsection{Illative}
% \label{sssec:ns_illative_case}
% 
% The illative case indicates motion from the exterior to the interior.
% 
% 
% \subsubsection{Allative}
% \label{sssec:ns_allative_case}
% 
% The allative case indicates motion towards the noun. 
% 
% 
% \subsubsection{Elative}
% \label{sssec:ns_elative_case}
% 
% The elative case indicates motion from the interior to the exterior.
% 
% 
% \subsubsection{Ablative}
% \label{sssec:ns_ablative_case}
% 
% The ablative case indicates motion away from the noun.  It can also be used
% in expressions of time and emotion to indicate the beginning of the event or
% state. 
% 
% 
% \subsubsection{Comparative}
% \label{sssec:ns_comparative_case}
% 
% The comparative case indicates a likeness to something, or the
% standard to which something is compared.



\section{Pronouns and Pronomial forms}
\label{sec:nm_pronouns}

Pronouns are roughly equivalent to nouns in terms of syntax and morphology.
They serve as substitutes for other nouns or noun phrases that have
previously been mentioned or can be inferred from context.  There are a
number of types of pronouns in Qevesa, including personal pronouns,
demonstrative pronouns and interrogative pronouns.

%The class of determiners is a special case, in that they can also act as
%articles for other nouns or noun phrases.


\subsection{Personal Pronouns}
\label{ssec:nm_personal_pronouns}

The personal pronouns stand in for other nouns, indicating that noun's person,
number and case.  Personal pronouns refer only to animate referents:
demonstrative pronouns refer to inanimate referents.  There are two first
person plural pronouns, an inclusive, which includes the listener, and an
exclusive, which does not. 

The base forms of the pronouns are given in \cref{tab:nm_pronoun_primary_case}.

\begin{table}[h!]\small\capstart
  \begin{tabular}{KFl -c -c -c -c -c -c -c -c}
    \toprule
    \rowstyle{\bfseries} & Stem & \multicolumn{6}{-c}{Cases}\\
    & Root & \rowstyle{\scshape} {\Dir} & {\Erg} & {\Acc} & {\Ins} & {\Gen} & {\Ess} & {\Loc} \\
    \midrule
    {\Fsg}             & me  & ma   & mem    & meš   & met   & mek   & med   & mest  \\
    {\Ssg}             & tá  & tá   & tám    & táš   & tát   & ták   & tád   & tást  \\
    {\Tsg}             & ni  & ni   & nim    & niš   & nit   & nik   & nid   & nist  \\
    \midrule
    {\Fdu};{\Incl}   & vi  & vi   & vim    & viš   & vit   & vek   & vid   & vist  \\
    {\Fdu};{\Excl}   & ze  & za   & zem    & zeš   & zet   & zek   & zed   & zest  \\
    {\Sdu}             & kav & káva & kávam  & kávaš & kávet & kávek & káved & kávest \\
    {\Tdu}             & niv & niva & nivam  & nivaš & nivet & nivek & nived & nivest \\
    \midrule
    {\Fpl};{\Incl}   & sa  & sa   & sam    & saš   & set   & sek   & sed   & sest \\
    {\Fpl};{\Excl}   & zes & zesa & zesam  & zesaš & zeset & zesek & zesed & zesest \\
    {\Spl}             & kás & kása & kásam  & kásaš & káset & kásek & kásed & kásest \\
    {\Tpl}             & nis & nisa & nisam  & nisaš & niset & nisek & nised & nisest \\
    \bottomrule
  \end{tabular}
  \caption{Personal pronouns\label{tab:nm_pronoun_primary_case}}
\end{table}


    % \midrule
    % {\Inan};{\Sg} & an  & ano  & (anom) & anoš  & anot  & anok  & anod  & anost \\
    % {\Inan};{\Du} & ava & avo  & (avom) & avoš  & avot  & avok  & avod  & avost \\
    % {\Inan};{\Pl} & asa & aso  & (asom) & asoš  & asot  & asok  & asod  & asost \\

% \subsubsection{Possessive Suffixes}
% \label{sssec:mn_possessive_suffixes}
%
% Pronomial genetive forms are rarely used when the possessor is animate;
% instead, nouns are marked with suffixes that indicate the possessor.  These
% suffixes also influence whether the vowel or consonant form of the following
% case suffix is used.


% \subsection{Reflexive and Reciprocal Pronouns}
% \label{ssec:nm_reflexive_pronouns}
% 
% Qevesa possesses a single reflexive pronoun, \conlang{mech} ‘self’, used to
% refer to something already mentioned.  It inflects with the personal
% suffixes to agree in person with its antecedent.  A related pronoun is the
% reciprocal pronoun \conlang{mocchem}, which does not take personal suffixes.

%   \begin{table}[h!]\small\capstart
%   		\begin{tabular}{Sfc -c}
%   			\hline
%   			\rowstyle{\bfseries} & Reflexive \\
%   			%\cline{2-2}
%   			\rowstyle{\scshape} & refl \\
%   			\hline
%   			{\Fsg}       & mekai   \\
%   			{\Ssg}       & mekata  \\
%   			{\Tsg}       & mekami  \\
%   			{\Fdu};inc   & mekajév \\
%   			{\Fdu};exc   & mekacév \\
%   			{\Sdu}       & mekatáv \\
%   			{\Tdu}       & mekamiv \\
%   			{\Fpl};inc   & mekajés \\
%   			{\Fpl};exc   & mekacés \\
%   			{\Spl}       & mekatás \\
%   			{\Tpl}       & mekamis \\
%        recp      & mokem   \\
%        %\hline
%   			inanim;sg & mekonet \\
%   			inanim;du & mekonev \\
%   			inanim;pl & mekones \\
%   			\hline         
%   		\end{tabular}    
%   		\caption{Reflexive and reciprocal pronouns\label{tab:nm_reflexive_pronouns}}
%   \end{table}

% This might fix referencing issues
%\newpage


\subsection{Demonstrative Pronouns}
\label{ssec:nm_demonstrative_pronouns}

Qevesa has three degrees of demonstrative pronouns.  The basic demonstrative pronouns also act as inanimate personal pronouns.

\begin{itemize}
  \item The \textbf{proximal} series refers to things closer to the speaker than the listener;
  \item The \textbf{medial} series refers to things closer to the listener than the speaker; and
  \item The \textbf{distal} series refers to things that are far from both speaker and listener.
\end{itemize}

The demonstrative pronouns are are listed in \cref{tab:nm_demonstrative_pronouns}.

\begin{table}[h!]\small\capstart
  \begin{tabulary}{0.75 \textwidth}{BFl -C -C -C}
    \toprule
    \rowstyle{\bfseries} Category & Proximal   & Medial    & Distal     \\
    \rowstyle{\scshape}           & {\Prox} & {\Med} & {\Dist} \\
    \midrule
    Basic                         & an         & ko        & iši      \\
    \midrule
    Person                        & nimu       & nikomu    & nišamu    \\
    Animate                       & antan      & kotan     & ištan    \\
    Inanimate                     & anno       & kono      & išano    \\
    \midrule
    Location                      & anist      & kost      & išast    \\
    Direction                     & anvera     & kovera    & išvera   \\
    Origin                        & anuri      & kouri     & išauri   \\
    \midrule
    Manner                        & antu       & kontu     & išintu   \\
    Quantity                      & amiden     & konaden   & išaden   \\
    Quality                       & amizan     & konuzan   & išazan   \\
    Type                          & amneri     & koneri    & išaneri  \\
    Size                          & ammeli     & koneli    & išaheli \\
    \bottomrule
  \end{tabulary}
  \caption{Demonstrative pronouns\label{tab:nm_demonstrative_pronouns}}
\end{table}

%  The prefixes for each series of demonstratives can also be combined with
%  case suffixes, to produce pronouns of specific direction or location, for
%  example:
%
%  \begin{exe}
%    \ex\label{tab:nm_demonstrative_prefixes}
%    \begin{tabular}[t]{fl -l}
%      \rowstyle{\itshape} tossa & qespha \\
%      to-ssa & qe-spha \\
%      \rowstyle{\scshape} prox-ine & int-all \\
%      in(side) here & towards where?
%    \end{tabular}
%  \end{exe}


\section{Postpositions}
\label{sec:nm_postpositions}

As a left-branching language, Qevesa tends to use postpositions almost
exclusively.  Most postpostitions require the head noun to be declined to a
particular case.

\begin{table}[h!]\small\capstart
  \begin{tabulary}{0.9\textwidth}{FIc -L -L}
    \toprule
    \rowstyle{\bfseries} {\upshape Postposition} & \multicolumn{1}{-c}{Meaning} & \multicolumn{1}{-c}{Cases} \\
    \midrule
           & (together) with           & Instrumental, Locative \\
           & around                    & Locative \\
           & away                      & Locative \\
           & before                    & Essive, Locative \\
           & by, beside                & Locative \\
           & from                      & Locative \\
           & inside                    & Locative \\
           & into                      & Locative \\
           & like, as                  & Essive\\
           & on                        & Locative \\
           & onto                      & Locative \\
           & outside                   & Locative \\
           & without                   & Instrumental \\
    evit   & in                        & Locative \\
    kamo   & because of, on account of & Instrumental \\
    kastis & along                     & Locative \\
    kirev  & down, below               & Locative \\
    methi  & except for                & Instrumental \\
    mita   & after                     & Essive, Locative \\
    sapa   & at, near                  & Locative \\
    šesal  & about, concerning         & Instrumental, Essive \\
    ukan   & behind                    & Locative \\
    veki   & to                        & Locative \\
    vileš  & above                     & Locative \\
    véra   & towards                   & Locative \\
    \bottomrule
  \end{tabulary}
  \caption{List of Postpositions\label{tab:nm_postpositions}}
\end{table}

\end{document}
