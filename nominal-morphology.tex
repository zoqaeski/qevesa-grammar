\documentclass[grammar]{subfiles}
\begin{document}
  \chapter{Nominal Morphology}
  \label{ch:nominal_morphology}

  \section{Definitions and Features}
  \label{sec:nm_definition_features}

  Qevesa nouns, like verbs, are highly regular in their declension. They inflect for two non-inherent features: number and case. They are also occasionally marked for animacy, though this is inherent in the noun, and thus is usually only indicated by the declension affixes. 
  
  Unlike in some languages, there is no grammatical gender. Instead, Qevesa uses natural gender, and this is an inherent feature of the noun that is neither marked nor affects declension. Explicit constructions to distinguish gender may be used when necessary.

  %There is no grammatical gender. Explicit constructions to indicate gender are used only when necessary, and these are seldom used except in the formal or literary language. Nouns may be classed according to animacy, a feature which is only indicated by the declension affixes.

  Most nouns have three numbers, a singular, dual or quantitative, and plural, although a small, closed set have a natural number and receive inverse marking. 

  There are fourteen cases in the standard written language: focal, nominative, absolutive, secundative, genitive, essive, instrumental-commitative, inessive, adessive, illative, allative, elative, ablative and comparative. A fifteenth case, the vocative, exists in some spoken dialects, but this is falling out of use\footnote{It is interesting to note that the vocative case is commonly used when insulting people regardless of dialect.}.

  Nouns can also be marked for four states, which are different types of determinateness.

  The citation form of all nouns is the unmarked form, that is, with no suffixes or prefixes.

  \subsection{Animacy}
  \label{ssec:nm_animacy}

  Nouns in the Teralo family of languages display a property known as animacy, in which nouns referring to humans, animals and other things perceived as having consciousness or life decline differently to other nouns in some forms. The animacy of a noun must be known in order to properly decline it to the primary cases and to indicate pronomial forms.

  Animate nouns refer to humans, animals, spirits, some plants, and some meteorological and geological phenomena. This includes personal names, possessions, and some body parts. Most living but inanimate life forms are not included, such as the majority of plants, as wells as microbial life forms. Animacy is a fixed feature, so nouns may not switch between animate and inanimate declensions. Exceptions to this include named objects as well as some towns and cities.

  \subsection{Proper Nouns}
  \label{ssec:nm_proper_nouns}

  Proper nouns may be formed from words existing in the language\footnotemark{}, often supported by gender markers to disambiguate them from common nouns, especially when used as personal names. A noticeable morphological feature of proper nouns is that their case markers are enclitic rather than suffixed, separated by a colon or a non-breaking space. Proper names are seldom pluralised.
  \footnotetext{See Section~\ref{ssec:dev_proper_nouns} on page~\pageref{ssec:dev_proper_nouns} for derivation of proper nouns.}

  % \subsection{Count and Mass Nouns}
  % \label{ssec:nm_count_mass_nouns}

  % Qevesa distinguishes between count and mass nouns, for example:

  % \begin{exe}
  % 	\ex \emph{EXAMPLES}
  % \end{exe}

  % Materials and abstract qualities cannot be counted, and are grammatically singular. Things that do not usually occur as singular items are sometimes uncountable and possess a natural number, receiving inverse number when it must be explicitly indicated. Body parts are typically included in this set:

  % \begin{exe}
  % 	\ex \emph{EXAMPLES}
  % \end{exe}

  \section{Nominal Declension}
  \label{sec:nm_declension}

  Qevesa noun words consist of the stem, followed by number, possessor and case marking:

  \begin{exe}
    \ex\label{ex:nm_structure} \textsc{state-}\textit{stem}\textsc{-number-possessor-case}
  \end{exe}

  The noun may also be marked with a prefix to indicate the state.

  \subsection{Number}
  \label{ssec:nm_number}

  Qevesa nouns have three numbers, singular, dual and plural, which are marked by a series of suffixes that display a form of inverse marking. 
  Every countable noun has an inherent (“natural”) number, which is unmarked, and is only marked for number when the noun occurs in a different number.
  
  The dual number also functions as a quantative number. By itself, it indicates that there are exactly two of the noun. However, if a quantity is to be specified, such as with a number word or quantifier, the dual form is also used.

  The suffixes that indicate number insert an epenthetic \qevesa{-e-} if the stem ends in a consonant; these are given in Table~\ref{tab:nm_number_suffixes}. 
  Some examples are given in Example~\ref{exe:nm_number}.

  \begin{table}[htpb]\small\capstart
      \begin{tabular}{|>{\bfseries}fc->{\scshape}fc|-c|}
        \hline
        & & \bfseries Suffix \tnl
        \hline
        Natural           &          & -∅  \tnl
        %Singular         & \acs{sg} & -∅ \tnl
        Singular          & \acs{sg} & -(e)n \tnl
        Dual/Quantitative & \acs{du} & -(e)v \tnl
        Plural            & \acs{pl} & -(e)s \tnl
        \hline
      \end{tabular}
      \caption{Grammatical number suffixes\label{tab:nm_number_suffixes}}
  \end{table}


  \begin{exe}
    \ex\label{exe:nm_number}
    \begin{tabular}[t]{fc -c -c -c}\small
      \SetRowStyle{\bfseries}Natural & Singular    & Dual         &  Plural \tnl
      \SetRowStyle{\itshape}   tolik & *toliken    & tolikev      & tolikes \tnl
                               ‘boy’ & ‘(a) boy’   & ‘(two) boys’ &  ‘boys’ \tnl
      \SetRowStyle{\itshape}    mari & marin       & *mariv       &   maris \tnl
                        ‘(two) eyes’ & ‘(one) eye’ & ‘(two) eyes’ &  ‘eyes’ \tnl
    \end{tabular}
  \end{exe}

  In Example~\ref{exe:nm_number}, note that the word \qevesa{tolik} ‘boy’ has a singular natural number, but the word \qevesa{mari} has a dual natural number. The suffixes can be applied for emphasis or to indicate quantity (i.e. \qevesa{kör mariv} ‘three eyes’).

  \subsection{Case}
  \label{ssec:nm_case}

  Qevesa possesses fourteen cases (fifteen if the marginal vocative is included), which are divided into two groups. The primary cases, of which there are four, indicate morphosyntactic roles of the noun with respect to the verb; the remaining ten cases are the secondary cases, and these are mostly locative and adverbial cases. 

  The case suffixes are listed in Table~\ref{tab:nm_case_suffixes}. The \textsc{pc} columns list suffixes that follow a consonant, and the \textsc{pv} columns list suffixes that follow a vowel.  

  \begin{table}[htpb]\small\capstart
      \begin{tabular}{|>{\bfseries}fc>{\scshape}c|-c-c|-c-c|}
        \hline
        \multicolumn{2}{|fc|}{\SetRowStyle{\bfseries}\multirow{3}{*}{Case}} & \multicolumn{2}{-c|}{Animate} & \multicolumn{2}{-c|}{Inanimate} \tnl
        \cline{3-6}
        \SetRowStyle{\scshape} & & \multicolumn{2}{-c|}{anim} & \multicolumn{2}{-c|}{inanim} \tnl
        \SetRowStyle{\scshape} & & pc & pv & pc & pv \tnl
        \hline
        \multirow{2}{*}{Focal} & \acs{foc}  & -a    & -∅   & -ina  & -na  \tnl
        & \acs{foc}\sub2  & -a    & -a   & -an   & -n \tnl
        Nominative  & \acs{nom}  & -am   & -m   & -om   & -mm  \tnl
        Absolutive  & \acs{abs}  & -aš   & -š   & -oš   & -niš \tnl
        Secundative & \acs{sdt}  & -ot   & -t   & -ot   & -nt  \tnl
        \hline
        Genitive    & \acs{gen}  & -ek   & -k   & -ok   & -nk  \tnl
        Essive      & \acs{ess}  & -el   & -l   & -ol   & -ll  \tnl
        Instrumental (Comitative) & \acs{ins}  & -etti & -tti & -onta & -nta \tnl
        Inessive    & \acs{ine}  & -essi & -ssi & -ossa & -ssa \tnl
        Adessive    & \acs{ade}  & -edi  & -di  & -oda  & -da  \tnl
        Illative    & \acs{ill}  & -esti & -sti & -osta & -sta \tnl
        Allative    & \acs{all}  & -efti & -fti & -ofta & -fta \tnl
        Elative     & \acs{ela}  & -espi & -spi & -ospa & -spa \tnl
        Ablative    & \acs{abl}  & -eski & -ski & -oska & -ska \tnl
        Comparative & \acs{comp} & -enni & -nni & -onna & -nna \tnl
        (Vocative)  & \acs{voc}  & -ó    & -jó  &       & \tnl
        \hline
      \end{tabular}
      \caption{Case suffixes\label{tab:nm_case_suffixes}}
  \end{table}

  \subsubsection{The Primary Cases}
  \label{sssec:nm_primary_cases}

  The primary cases indicate the morphosyntactic role of the noun with respect to the verb.

  %\paragraph{Focal}
  %\label{par:nm_focal_case}

  The \emph{focal} cases mark the topic of the verb phrase. The role of the noun phrase marked as the focus is indicated on the verb, using the topical agreement suffixes as described in Section~\ref{ssec:vm_topical_agreement}. This case has an additional form which is used when the focus of the verb phrase is already marked with one of the secondary cases, listed in Table~\ref{tab:nm_case_suffixes} as \textsc{foc\sub2}.

  %\paragraph{Nominative}
  %\label{par:nm_nominative_case}

  The \emph{nominative} case marks the voluntary experiencer of an intransitive verb, the agent of a transitive verb, or the donor of a ditransitive verb.

  %\paragraph{Absolutive}
  %\label{par:nm_absolutive_case}

  The \emph{absolutive} case marks the involuntary experiencer of an intransitive verb, the patient of a transitive verb, or the recipient of a ditransitive verb.

  %\paragraph{Secundative}
  %\label{par:nm_secundative_case}

  The \emph{secundative} case marks the theme of a ditransitive verb.

  \subsubsection{The Secondary Cases}
  \label{sssec:nm_secondary_cases}

  The secondary cases are mainly adpositional and locative cases.

  %\paragraph{Genitive}
  %\label{par:nm_genitive_case}

  The \emph{genitive} case indicates the possessor of another noun. Pronomial possessors are indicated by means of a suffix on the possessed item.

  %\paragraph{Essive}
  %\label{par:nm_essive_case}

  The \emph{essive} case indicates duration and time. It also indicates a temporary state of being or existence.

  %\paragraph{Instrumental (Comitative)}
  %\label{par:nm_instrumental_case}

  The \emph{instrumental} case indicates the means by which the action is performed. It may also be used in a comitative sense, i.e. to indicate the person in whose company the action is carried out.

  %\paragraph{Inessive}
  %\label{par:nm_inessive_case}

  The \emph{inessive} case indicates internal location. 

  %\paragraph{Adessive}
  %\label{par:nm_adessive_case}

  The \emph{adessive} case indicates external location.

  %\paragraph{Illative}
  %\label{par:nm_illative_case}

  The \emph{illative} case indicates motion from the exterior to the interior.

  %\paragraph{Allative}
  %\label{par:nm_allative_case}

  The \emph{allative} case indicates motion towards the noun.

  %\paragraph{Elative}
  %\label{par:nm_elative_case}

  The \emph{elative} case indicates motion from the interior to the exterior.

  %\paragraph{Ablative}
  %\label{par:nm_ablative_case}

  The \emph{ablative} case indicates motion away from the noun.

  %\paragraph{Comparative}
  %\label{par:nm_comparative_case}

  The \emph{comparative} case indicates a likeness to something, or the standard to which something is compared.

  A \emph{vocative} case exists in some dialects, and is marginally used in the standard language.

  \subsubsection{Use of the Locative Cases}
  \label{sssec:nm_locative_cases}

  The locative cases are logically grouped. There are two positions (internal and external) and three directions (static, movement towards and movement away). Combining these results in the six cases, illustrated in Table~\ref{tab:nm_locative_cases}.

  \begin{table}[htpb]\small\capstart
      \begin{tabular}{|>{\bfseries}fc|-c|-c|}
        \hline
        \SetRowStyle{\bfseries} & Interior & Exterior \tnl
        \hline
        Static           & Inessive & Adessive \tnl
        Movement towards & Illative & Allative \tnl
        Movement away    & Elative  & Ablative \tnl
        \hline
      \end{tabular}
      \caption{Locative cases\label{tab:nm_locative_cases}}
  \end{table}

  Finer distinctions in location are given with postpositions, which are described in Section~\ref{sec:postpositions}.

  \subsection{State}
  \label{ssec:nm_state}

  Nouns in Qevesa have four possible ‘states’. Nomimal states refer to different conditions of determinateness, which are differentiated primarily by prefixes that attach to the head noun. 

  The \emph{absolute} state (not to be confused with the \emph{absolutive case}) is the default citation form of the noun. 
  It does not mark any form of determination, generally indicating that the noun is indefinite, and has no special markings.

  The \emph{definite} state marked the noun for definiteness, and functions similarly to the definite article in English. 
  It is formed by the prefix \qevesa{aC\sub1}, that is, prefixing \qevesa{a-} and duplicating the first root consonant. % or the clitic \qevesa{a(n)}.

  The \emph{partitive} state makes the noun partitive. 
  It functions broadly similarly to the English determiner ‘some’, but may also be required by some quantifiers. 
  It is formed by the prefix \qevesa{mëC\sub2-}, that is, prefixing \qevesa{më-} and duplicating the second root consonant. %or the clitic \qevesa{më(n)}.

  The \emph{negative} state negates the noun, and is distinct from negating the verb phrase. 
  It is formed by the prefix \qevesa{nak-}. %or the clitic \qevesa{zu(n)}.

  The prefixes that indicate state are given in Table~\ref{tab:nm_state}. %The \textsc{pc} column lists clitics that precede a consonant, and the \textsc{pv} column lists clitics that precede a vowel.

  \begin{table}[htpb]\small\capstart
    \begin{tabular}{|>{\bfseries}fc>{\scshape}c|-c|-c|}
      \hline
      \multicolumn{2}{|fc|}{\SetRowStyle{\bfseries}State} & Prefix \tnl
      \hline
      Absolute  & \acs{abst}  & ∅-        \tnl
      Definite  & \acs{def}   & aC\sub1-  \tnl
      Partitive & \acs{part}  & mëC\sub2- \tnl
      Negative  & \acs{neg}   & nak-      \tnl
      \hline
    \end{tabular}
    \caption{Noun state prefixes\label{tab:nm_state}}
  \end{table}

%  \begin{table}[htpb]\small\capstart
%    \begin{tabular}{|>{\bfseries}fc>{\scshape}c|-c|-c|-c|}
%      \hline
%      \multicolumn{2}{|fc|}{\multirow{2}{*}{\SetRowStyle{\bfseries}State}} & \multicolumn{2}{|c|}{Clitic} \tnl
%      \cline{4-5}
%      \SetRowStyle{\scshape} & & pc & pv \tnl
%      \hline
%      Absolute  & \acs{abst}  & ∅  &  ∅  \tnl
%      Definite  & \acs{def}   & a  & an  \tnl
%      Partitive & \acs{part}  & më & mës \tnl
%      Negative  & \acs{neg}   & nak & nak \tnl
%      \hline
%    \end{tabular}
%    \caption{Noun state prefixes\label{tab:nm_state}}
%  \end{table}

%%   railway = iron road = zeléskölij
%%   Qevelian Railways = Qevel??? Zeléskölijes

%% the railway = azzeléskölij
%% some railway(s) = mëlzeléskölij
%% no railway(s) = nakzeléskölij

%% a house = panem
%% the house = appanem
%% some house(s) = mënpanem
%% no house(s) = nakpanem

  \section{Pronouns and Pronomial forms}
  \label{sec:nm_pronouns}

  Pronouns are roughly equivalent to nouns in terms of syntax and morphology. They serve as substitutes for other nouns or noun phrases that have previously been mentioned or can be inferred from context. There are a number of types of pronouns in Qevesa, including personal pronouns, demonstrative pronouns and interrogative pronouns.

  %The class of determiners is a special case, in that they can also act as articles for other nouns or noun phrases.

  \subsection{Personal Pronouns}
  \label{ssec:nm_personal_pronouns}

  The personal pronouns stand in for other nouns, indicating that noun's person, number and case. Most personal pronouns refer only to animate referents: a separate inanimate pronoun is used for inanimate referents. There are two first person plural pronouns, an inclusive, which includes the listener, and an exclusive, which does not. 

  Personal pronouns are declined to the primary cases by suffixation; other case constructions use a stem derived from the case ending combined with the suffix form of the pronoun. Although a genetive form of the personal pronouns exists, the suffix form is preferred to indicate posession. 

  The base forms of the pronouns are given in Table~\ref{tab:nm_pronoun_primary_case}, and the cases with personal suffixes are given in Table~\ref{tab:nm_personal_cases}.

  \begin{table}[htpb]\small\capstart
      \begin{tabular}{|>{\scshape}fc|-c|-c|-c|-c|-c|-c|-c|}
        \hline
        \SetRowStyle{\bfseries} & \multicolumn{2}{-c|}{Stem} & \multicolumn{5}{-c|}{Cases}\tnl
        \cline{2-8}
        & Root & Suffix &\SetRowStyle{\scshape} \acs{foc} & \acs{nom} & \acs{abs} & \acs{sdt} & \acs{gen} \tnl
        \hline
        \acs{1p}\acs{sg}           & ë  & -(a)j     & ë      & ëm    & ëš    & ët    & ëk \tnl
        \acs{2p}\acs{sg}           & ta  & -ut/-ːt   & ta      & tam    & taš    & tajot  & tak \tnl
        \acs{3p}\acs{sg}           & mi  & -im       & mi      & mim    & miš    & mijot  & mik \tnl
        \acs{1p}\acs{du};\acs{inc} & ëv & -ëva/-iva & ëva    & ëvam  & ëvaš  & ëvot  & ëvek \tnl
        \acs{1p}\acs{du};\acs{exc} & čev & -(e)čev   & čeva    & čevam  & čevaš  & čevot  & čevek \tnl
        \acs{2p}\acs{du}           & tav & -(a)tuv   & tava    & tavam  & tavaš  & tavot  & tavek \tnl
        \acs{3p}\acs{du}           & miv & -(a)miv   & miva    & mivam  & mivaš  & mivot  & mivek \tnl
        \acs{1p}\acs{du};\acs{inc} & ës & -ësa/-isa & ësa    & ësam  & ësaš  & ësot  & ësek \tnl
        \acs{1p}\acs{du};\acs{exc} & čes & -(e)čes   & česa    & česam  & česaš  & česot  & česek \tnl
        \acs{2p}\acs{du}           & tas & -(a)tus   & tasa    & tasam  & tasaš  & tasot  & tasek \tnl
        \acs{3p}\acs{du}           & mis & -(a)mis   & misa    & misam  & misaš  & misot  & misek \tnl
        %\hline
        \acs{inanim};\acs{sg}      & net & -net      & netina  & netom  & netoš  & netot  & netok \tnl
        \acs{inanim};\acs{du}      & nev & -nev      & nevtina & nevtom & nevtoš & nevtot & nevtok \tnl
        \acs{inanim};\acs{pl}      & nes & -nes      & nestina & nestom & nestoš & nestot & nestok \tnl
        \hline
      \end{tabular}
      \caption{Personal pronouns\label{tab:nm_pronoun_primary_case}}
  \end{table}

  \begin{table}[htpb]\small\capstart
      \subfloat{
        \begin{tabular}{|>{\scshape}fc|->{\itshape}c|-c|-c|-c|-c|-c|}
          \hline
          \multicolumn{2}{|fc|}{\SetRowStyle{\bfseries}} & \multicolumn{5}{-c|}{Cases}\tnl
          \cline{3-7}
          \multicolumn{2}{|fc|}{\SetRowStyle{\scshape}} & \acs{ess} & \acs{ins} & \acs{ine} & \acs{ade} & \acs{ill} \tnl
          \cline{3-7}
          \multicolumn{2}{|fc|}{\SetRowStyle{\itshape}} & ël- & ëtt- & ëss- & ëd- & ëst- \tnl
          \hline
          \acs{1p}\acs{sg}           & -(a)j     & ëlaj   & ëttaj   & ëssaj   & ëdaj   & ëstaj \tnl
          \acs{2p}\acs{sg}           & -ut/-ːt   & ëlut   & ëttut   & ëssut   & ëdut   & ëstut \tnl
          \acs{3p}\acs{sg}           & -im/-ːm   & ëlim   & ëttim   & ëssim   & ëdim   & ëstim \tnl
          \acs{1p}\acs{du};\acs{inc} & -ëva/-iva & ëliva  & ëttiva  & ëssiva  & ëdiva  & ëstiva \tnl
          \acs{1p}\acs{du};\acs{exc} & -(e)čev   & ëlečev & ëttečev & ëssečev & ëdečev & ëstečev \tnl
          \acs{2p}\acs{du}           & -(a)tuv   & ëlatuv & ëttatuv & ëssatuv & ëdatuv & ëstatuv \tnl
          \acs{3p}\acs{du}           & -(a)miv   & ëlamiv & ëttamiv & ëssamiv & ëdamiv & ëstamiv \tnl
          \acs{1p}\acs{pl};\acs{inc} & -ësa/-isa & ëlisa  & ëttisa  & ëssisa  & ëdisa  & ëstisa \tnl
          \acs{1p}\acs{pl};\acs{exc} & -(e)čes   & ëlečes & ëttečes & ëssečes & ëdečes & ëstečes \tnl
          \acs{2p}\acs{pl}           & -(a)tus   & ëlaset & ëttatus & ëssatus & ëdatus & ëstatus \tnl
          \acs{3p}\acs{pl}           & -(a)mis   & ëlamis & ëttamis & ëssamis & ëdamis & ëstamis \tnl
          \hline
          \multicolumn{2}{|fc|}{\SetRowStyle{\itshape}} & ola- & onti- & ossi- & od- & osta- \tnl
          \hline
          \acs{inanim};\acs{sg}     & -net   & olanet & ontinet & ossinet & odnet  & ostanet \tnl
          \acs{inanim};\acs{du}     & -nev   & olanev & ontinev & ossinev & odnev  & ostanev \tnl
          \acs{inanim};\acs{pl}     & -nes   & olanes & ontines & ossines & odnes  & ostanes \tnl
          \hline
        \end{tabular}}\\
      \subfloat{
        \begin{tabular}{|>{\scshape}fc|->{\itshape}c|-c|-c|-c|-c|}
          \hline
          \multicolumn{2}{|fc|}{\SetRowStyle{\bfseries}} & \multicolumn{4}{-c|}{Cases}\tnl
          \cline{3-6}
          \multicolumn{2}{|fc|}{\SetRowStyle{\scshape}} & \acs{all} & \acs{ela} & \acs{abl} & \acs{comp} \tnl
          \cline{3-6}
          \multicolumn{2}{|fc|}{\SetRowStyle{\itshape}} & ëft- & ësp- & ësk- & na- \tnl
          \hline
          \acs{1p}\acs{sg}           & -(a)j     & ëftaj   & ëspaj   & ëskaj   & náj \tnl
          \acs{2p}\acs{sg}           & -ut/-ːt   & ëftut   & ësput   & ëskut   & nát \tnl
          \acs{3p}\acs{sg}           & -im/-ːm   & ëftim   & ëspim   & ëskim   & naim \tnl
          \acs{1p}\acs{du};\acs{inc} & -ëva/-iva & ëftiva  & ëspiva  & ëskiva  & naiva \tnl
          \acs{1p}\acs{du};\acs{exc} & -(e)čev   & ëftečev & ëspečev & ëskečev & načev \tnl
          \acs{2p}\acs{du}           & -(a)tuv   & ëftatuv & ëspatuv & ëskatuv & natuv \tnl
          \acs{3p}\acs{du}           & -(a)miv   & ëftamiv & ëspamiv & ëskamiv & namiv \tnl
          \acs{1p}\acs{pl};\acs{inc} & -ësa/-isa & ëftisa  & ëspisa  & ëskisa  & naisa \tnl
          \acs{1p}\acs{pl};\acs{exc} & -(e)čes   & ëftečes & ëspečes & ëskečes & načes \tnl
          \acs{2p}\acs{pl}           & -(a)tus   & ëftatus & ëspatus & ëskatus & natus \tnl
          \acs{3p}\acs{pl}           & -(a)mis   & ëftamis & ëspamis & ëskamis & namis \tnl
          \hline
          \multicolumn{2}{|fc|}{\SetRowStyle{\itshape}} & ofta- & ospa- & oska- & no- \tnl
          \hline
          \acs{inanim};\acs{sg}     & -net   & oftanet & ospanet & oskanet & nonet \tnl
          \acs{inanim};\acs{du}     & -nev   & oftanev & ospanev & oskanev & nonev \tnl
          \acs{inanim};\acs{pl}     & -nes   & oftanes & ospanes & oskanes & nones \tnl
          \hline
        \end{tabular}}
      \caption{Cases with personal suffixes\label{tab:nm_personal_cases}}
  \end{table}

  \subsubsection{Possessive Suffixes}
  \label{sssec:mn_possessive_suffixes}

  \ToBeWritten

  \newpage
  \subsection{Reflexive and Reciprocal Pronouns}
  \label{ssec:nm_reflexive_pronouns}

  %Qevesa does not possess reflexive or reciprocal pronouns as most verb roots have forms that indicate reflexive\footnote{See  Section~\ref{ssec:dev_verb_form_v}, page~\pageref{ssec:dev_verb_form_v}} or reciprocal\footnote{See  Section~\ref{ssec:dev_verb_form_iii}, page~\pageref{ssec:dev_verb_form_iii}} actions. The word \qevesa{máka} ‘self’, may be used as a reflexive pronoun to avoid ambiguity, but this is rare.
   Qevesa possesses a single reflexive pronoun, \qevesa{meka} ‘self’, used to refer to something already mentioned. It inflects with the personal suffixes to agree in person with its antecedent. 
   %The resulting set of pronouns is given in Table~\ref{tab:nm_reflexive_pronouns}. 
   A related pronoun is the reciprocal pronoun \qevesa{mökem}, which does not take personal suffixes.

%   \begin{table}[htpb]\small\capstart
%   		\begin{tabular}{|>{\scshape}fc|-c|}
%   			\hline
%   			\SetRowStyle{\bfseries} & Reflexive \tnl
%   			%\cline{2-2}
%   			\SetRowStyle{\scshape} & refl \tnl
%   			\hline
%   			\acs{1p}\acs{sg}       & mekai   \tnl
%   			\acs{2p}\acs{sg}       & mekata  \tnl
%   			\acs{3p}\acs{sg}       & mekami  \tnl
%   			\acs{1p}\acs{du};inc   & mekajév \tnl
%   			\acs{1p}\acs{du};exc   & mekačév \tnl
%   			\acs{2p}\acs{du}       & mekatáv \tnl
%   			\acs{3p}\acs{du}       & mekamiv \tnl
%   			\acs{1p}\acs{pl};inc   & mekajés \tnl
%   			\acs{1p}\acs{pl};exc   & mekačés \tnl
%   			\acs{2p}\acs{pl}       & mekatás \tnl
%   			\acs{3p}\acs{pl}       & mekamis \tnl
%        recp      & mökem   \tnl
%        %\hline
%   			inanim;sg & mekonet \tnl
%   			inanim;du & mekonev \tnl
%   			inanim;pl & mekones \tnl
%   			\hline         
%   		\end{tabular}    
%   		\caption{Reflexive and reciprocal pronouns\label{tab:nm_reflexive_pronouns}}
%   \end{table}

  % This might fix referencing issues
  %\newpage

  \subsection{Demonstrative and Correlative Pronouns}
  \label{ssec:nm_demonstrative_pronouns}

  Qevesa has three degrees of demonstrative pronouns:

  \begin{description}[style=nextline]
    \item[Proximal] The proximal series is marked by the prefix \qevesa{to-}, and refers to things closer to the speaker than the listener;
    \item[Medial] The medial series is marked by the prefix \qevesa{ko-}, and refers to things closer to the listener than the speaker; and
    \item[Distal] The distal series, marked by the prefix \qevesa{isá-}, refers to things that are far from both speaker and listener.
  \end{description}

  There is also an interrogative series, which is marked with the prefix \qevesa{qe-}. Demonstrative pronouns must agree in number, case and sometimes state with their antecedent, unlike all other types of modifiers, such as adjectives. 

  %A related set of pronouns is formed by prefixes denoting number or quantity. These include the existential, elective, universal and negative series, and combine with the suffixes in a highly regular manner. 

  The demonstrative pronouns are are listed in Table~\ref{tab:nm_demonstrative_pronouns}.

  \begin{table}[htpb]\small\capstart
      %\subfloat[Demonstrative pronouns]{
        \begin{tabular}{|>{\bfseries}fc->{\scshape}c|->{\itshape}c|-c|-c|-c|-c|}
          \hline
          \multicolumn{3}{|fc|}{\SetRowStyle{\bfseries}} & Proximal & Medial & Distal & Interrogative \tnl
          %\cline{3-6}
          \multicolumn{3}{|fc|}{\SetRowStyle{\scshape}} & \acs{prox} & \acs{med} & \acs{dist} & \acs{int} \tnl
          \cline{4-7}
          \multicolumn{3}{|fc|}{\SetRowStyle{\itshape}} & to- & ko- & isá- & qe- \tnl
          \hline
          Human       & \acs{hum}  & -tka & totka & kotka & isátka & qetka \tnl
          Nonhuman    & \acs{nh}   & -ra  & tora  & kora  & isára  & qera  \tnl
          Location    & \acs{loc}  & -zól & tozól & kozól & isázól & qezól \tnl
          Source      & \acs{src}  & -ská & toská & koská & isáská & qeská \tnl
          Destination & \acs{dest} & -rve & torve & korve & isárve & qerve \tnl
          Time        & \acs{time} & -lti & tolti & kolti & isálti & qelti \tnl
          Manner      & \acs{man}  & -ttu & tottu & kottu & isáttu & qettu \tnl
          Reason      & \acs{rsn}  & -rte & torte & korte & isárte & qerte \tnl
          \hline
        \end{tabular}%}\\
%      \subfloat[Correlative pronouns]{
%        \begin{tabular}{|>{\bfseries}fc->{\scshape}c|->{\itshape}c|-c|-c|-c|-c|}
%          \hline
%          \multicolumn{3}{|fc|}{\SetRowStyle{\bfseries}} & Existential & Elective & Universal & Negative \tnl
%          %\cline{3-6}
%          \multicolumn{3}{|fc|}{\SetRowStyle{\scshape}} & exist & elect & univ & neg \tnl
%          \cline{4-7}
%          \multicolumn{3}{|fc|}{\SetRowStyle{\itshape}} & ane- & via- & minű- & zumo- \tnl
%          \hline
%          Human				& hum  & -tka & anetka & viatka & minűtka & zumotka \tnl
%          Nonhuman		& nh   & -ra  & anera  & viara  & minűra  & zumora  \tnl
%          Location		& loc  & -zól & anezól & viazól & minűzól & zumozól \tnl
%          Source			& src  & -ská & aneská & viaská & minűská & zumoská \tnl
%          Destination & dest & -rve & anerve & viarve & minűrve & zumorve \tnl
%          Time				& time & -lti & anelti & vialti & minűlti & zumolti \tnl
%          Manner			& man  & -ttu & anettu & viattu & minűttu & zumottu \tnl
%          Reason			& rsn  & -rte & anerte & viarte & minűrte & zumorte \tnl
%          \hline
%        \end{tabular}}
%     \caption{Demonstrative and correlative pronouns\label{tab:nm_demonstrative_correlative_pronouns}}
      \caption{Demonstrative pronouns\label{tab:nm_demonstrative_pronouns}}
  \end{table}

%  The prefixes for each series of demonstratives can also be combined with case suffixes, to produce pronouns of specific direction or location, for example:
%
%  \begin{exe}
%    \ex\label{tab:nm_demonstrative_prefixes}
%    \begin{tabular}[t]{fl -l}
%      \SetRowStyle{\itshape} tossa & qefta \tnl
%      to-ssa & qe-fta \tnl
%      \SetRowStyle{\scshape} prox-ine & int-all \tnl
%      in(side) here & towards where?
%    \end{tabular}
%  \end{exe}

  \section{Postpositions}
  \label{sec:nm_postpositions}

  As a left-branching language, Qevesa tends to use postpositions almost exclusively. Many postpositions are inflected for case, and require the complement after which they are placed to adopt a particular case form as well.

\end{document}
