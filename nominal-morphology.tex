\documentclass[grammar]{subfiles}
\begin{document}
\chapter{Nominal Morphology}
\label{ch:nominal_morphology}


\section{Definitions and Features}
\label{sec:nm_definition_features}

Qevesa nouns, like verbs, are highly regular in their declension.  They
inflect for two non-inherent features: number and case.  They are also
occasionally marked for animacy, though this is inherent in the noun, and
thus is usually only indicated by the declension affixes. 

Unlike in some languages, there is no grammatical gender.  Instead, Qevesa
uses natural gender, and this is an inherent feature of the noun that is
neither marked nor affects declension.  Explicit constructions to distinguish
gender may be used when necessary.

Most nouns have three numbers, a singular, dual or quantitative, and plural,
although a small, closed set have a natural number and receive inverse
marking. 

There are fourteen cases in the standard written language: direct, nominative,
absolutive, secundative, genitive, essive, instrumental-commitative,
inessive, adessive, illative, allative, elative, ablative and comparative.

Nouns can also be marked for four states, which are different types of determinateness.

The citation form of all nouns is the unmarked form, that is, with no suffixes or prefixes.


\subsection{Animacy}
\label{ssec:nm_animacy}

Nouns in the Teralo family of languages display a property known as animacy,
in which nouns referring to humans, animals and other things perceived as
having consciousness or life decline differently to other nouns in some
forms.  The animacy of a noun must be known in order to properly decline it
to the primary cases and to indicate pronomial forms.

Animate nouns refer to humans, animals, spirits, some plants, and some
meteorological and geological phenomena.  This includes personal names,
possessions, and some body parts.  Most living but inanimate life forms are
not included, such as the majority of plants, as wells as microbial life
forms.  Animacy is a fixed feature, so nouns may not switch between animate
and inanimate declensions.  Exceptions to this include named objects as well
as some towns and cities.

%  \subsection{Proper Nouns}
%  \label{ssec:nm_proper_nouns}

%  Proper nouns may be formed from words existing in the language, often
%  supported by gender markers to disambiguate them from common nouns,
%  especially when used as personal names.  

% \subsection{Count and Mass Nouns}
% \label{ssec:nm_count_mass_nouns}

% Qevesa distinguishes between count and mass nouns, for example:

% \begin{exe}
% 	\ex \emph{EXAMPLES}
% \end{exe}

% Materials and abstract qualities cannot be counted, and are grammatically
% singular.  Things that do not usually occur as singular items are sometimes
% uncountable and possess a natural number, receiving inverse number when it
% must be explicitly indicated.  Body parts are typically included in this
% set:

% \begin{exe}
% 	\ex \emph{EXAMPLES}
% \end{exe}


\section{Nominal Declension}
\label{sec:nm_declension}

Qevesa noun words consist of the stem, followed by number, possessor and case marking:

\begin{exe}
  \ex\label{ex:nm_structure} \textit{stem}\textsc{-number-possessor-case}
\end{exe}

Noun stems that end in a long vowel reduce it before suffixes that begin with a vowel. 


\subsection{Number}
\label{ssec:nm_number}

Qevesa nouns have four numbers, singular, dual, plural and partitive, which are
typically indicated by the suffixes listed in \cref{tab:nm_number_suffixes}.  A
small, closed set of nouns has suppletive plural forms; these may be so-called 
\emph{broken plurals} or separate roots entirely. 

The indefinite suffix is marked with an \Q{-e} after a consonant, and is
unmarked on nouns that end with a vowel, except if the vowel is \Q{-i} in which
case the indefinite suffix replaces it.  

The definite singular suffix is usually \Q{-ia}, and is not used before a
pronomial possessor.  Definiteness may also be indicated by the prepositional
articles \Q{a} or \Q{az}.

An epenthetic \Q{-e-} is inserted after a consonant for the dual and plural
suffixes; the partitive uses an \Q{-i-} instead. 

\begin{table}[h!]\small\capstart
  \begin{tabular}{BFl -Sc -l}
    \toprule
    \multicolumn{2}{Fl}{\SetRowStyle{\bfseries}Number} & Suffix \\
    \midrule
    Indefinite        & \acs{indef} & -∅, -e    \\
    Definite Singular & \acs{sg}    & -ia, -ja  \\
    Dual/Quantitative & \acs{du}    & -(e)v  \\
    Plural            & \acs{pl}    & -(e)s  \\
    Partitive         & \acs{part}  & -(i)n  \\
    \bottomrule
  \end{tabular}
  \caption{Grammatical number suffixes\label{tab:nm_number_suffixes}}
\end{table}

Number marking in Qevesa functions in a somewhat unusual manner in that every
noun has an inherent “natural” number, which is its default, unmarked form.
The suffixes are appended to indicate that the quantity (and definiteness)
differs from what is expected.  Most nouns default to the implicit singular;
some nouns, such as body parts and items of clothing that come in pairs are
implicitly dual (\Q{méri} “eyes”); and other nouns may be implicitly plural or
partial (particularly uncountable nouns). 

The dual number functions to indicate exact quantities.  By itself, it
indicates exactly two of the noun; however, it is also used when the noun is
preceded by a modifier that indicates an exact quantity, such as a number word.

In contrast to the dual, the plural number is used for unspecified quantities
greater than the singular.  The plural suffix may also encode definiteness,
especially for those nouns whose unmarked form has an implicit number. 

The partitive is used to express partialness or inexact quantities. 
% It may also be used in an atelic sense. ???


\subsection{Case}
\label{ssec:nm_case}

Qevesa possesses fourteen cases, which are divided into two groups.  The
primary cases, of which there are four, indicate morphosyntactic roles of the
noun with respect to the verb; the remaining ten cases are the secondary
cases, and these are mostly locative and adverbial cases. 

The case suffixes are listed in \cref{tab:nm_case_suffixes}.  The left
column lists suffixes that follow a vowel, and the right column lists
suffixes that follow a consonant.  

\begin{table}[h!]\small\capstart
  \begin{tabular}{BFl Sl -l -l}
    \toprule
    \multicolumn{2}{Fc}{\SetRowStyle{\bfseries}Noun Case} & \multicolumn{2}{-c}{Suffix} \\
    %\SetRowStyle{\scshape} & & sg & du & pl & indef & part \\
    \midrule
    Direct       & \acs{dir}  & \multicolumn{2}{-l}{-a, -n, -∅}  \\
    Nominative   & \acs{nom}  & -m   & -am    \\
    Absolutive   & \acs{abs}  & -š   & -aš    \\
    Secundative  & \acs{sdt}  & -t   & -at    \\
    \midrule
    Genitive     & \acs{gen}  & -k   & -ak    \\
    Comparative  & \acs{cmpr} & -d   & -ad    \\
    Essive       & \acs{ess}  & -l   & -alli  \\
    Instrumental & \acs{ins}  & -či  & -ači   \\
    Inessive     & \acs{ine}  & -ssi & -assi  \\
    Adessive     & \acs{ade}  & -zi  & -azi   \\
    Illative     & \acs{ill}  & -sti & -asti  \\
    Allative     & \acs{all}  & -nti & -anti  \\
    Elative      & \acs{ela}  & -spi & -aspi  \\
    Ablative     & \acs{abl}  & -mpi & -ampi  \\
    \bottomrule
  \end{tabular}
  \caption{Case suffixes\label{tab:nm_case_suffixes}}
\end{table}

\subsection{Direct}
\label{ssec:ns_direct_case}

The direct case marks the topic of the verb phrase.  This may be the
experiencer (both voluntary and involuntary) of an intransitive verb, the agent
or patient of a transitive verb, or (less commonly) some other argument of the
verb.  In this latter case, the direct suffix is stacked onto the other case
suffix. 

Typically, animate nouns in the direct case are the voluntary experiencers or
agents of verbs, and inanimate nouns in the direct case are experiencers or
patients. 

The direct case suffix takes several forms: \Q{-a} only occurs after a
consonant, or a consonant followed by \Q{u}; \Q{-n} occurs after a diphthong
ending in \Q{u}, or when the direct case is stacked onto a secondary case;
elsewhere, the direct case is unmarked.


\subsection{Nominative}
\label{ssec:ns_nominative_case}

The nominative case marks the voluntary experiencer of an intransitive verb, or
the agent of a transitive verb.  Inanimate nouns cannot be marked with the
nominative case, because an inanimate entity is considered incapable of acting
of its own accord. 


\subsection{Absolutive}
\label{ssec:ns_absolutive_case}

The absolutive case marks the involuntary experiencer of an intransitive verb,
the patient of a transitive verb or the recipient of ditransitive verb.


\subsection{Secundative}
\label{ssec:ns_secundative_case}

Qevesa is a secundative language, that is, the recipient of a ditransitive verb
is treated the same as the patient of a monotransitive verb. The secundative
case marks the theme of a ditransitive verb.


\subsection{Genitive}
\label{ssec:ns_genitive_case}

The genitive case indicates the possessor of another noun.  Animate pronomial
possessors are usually indicated by means of a suffix on the possessed noun.


\subsection{Essive}
\label{ssec:ns_essive_case}

The essive case is used to indicate duration and time, as well as temporary
states of being or existence.  


\subsection{Instrumental}
\label{ssec:ns_instrumental_case}

The instrumental case indicates the means by which the action is
performed.  Inanimate agents of verbs are also marked with the instrumental case.  


\subsection{Inessive}
\label{ssec:ns_inessive_case}

The inessive case indicates internal location.  


\subsection{Adessive}
\label{ssec:ns_adessive_case}

The adessive case indicates external location.


\subsection{Illative}
\label{ssec:ns_illative_case}

The illative case indicates motion from the exterior to the interior.


\subsection{Allative}
\label{ssec:ns_allative_case}

The allative case indicates motion towards the noun. 


\subsection{Elative}
\label{ssec:ns_elative_case}

The elative case indicates motion from the interior to the exterior.


\subsection{Ablative}
\label{ssec:ns_ablative_case}

The ablative case indicates motion away from the noun.  It can also be used
in expressions of time and emotion to indicate the beginning of the event or
state. 


\subsection{Comparative}
\label{ssec:ns_comparative_case}

The comparative case indicates a likeness to something, or the
standard to which something is compared.



\section{Pronouns and Pronomial forms}
\label{sec:nm_pronouns}

Pronouns are roughly equivalent to nouns in terms of syntax and morphology.
They serve as substitutes for other nouns or noun phrases that have
previously been mentioned or can be inferred from context.  There are a
number of types of pronouns in Qevesa, including personal pronouns,
demonstrative pronouns and interrogative pronouns.

%The class of determiners is a special case, in that they can also act as
%articles for other nouns or noun phrases.


\subsection{Personal Pronouns}
\label{ssec:nm_personal_pronouns}

The personal pronouns stand in for other nouns, indicating that noun's
person, number and case.  Most personal pronouns refer only to animate
referents: a separate inanimate pronoun is used for inanimate referents.
There are two first person plural pronouns, an inclusive, which includes the
listener, and an exclusive, which does not. 

Personal pronouns are declined to the some of the cases by suffixation; other
case constructions use a stem derived from the case ending combined with the
suffix form of the pronoun.  The suffix form is generally preferred over the
genetive case to indicate posession, but inanimate pronouns lack a suffix form
so always use the genetive pronoun. 

The base forms of the pronouns are given in
\cref{tab:nm_pronoun_primary_case}, and the cases with personal suffixes
are given in \cref{tab:nm_personal_cases}.

\begin{table}[h!]\small\capstart
  \begin{tabular}{SFl -l -l -l -l -l -l -l -l}
    \toprule
    \SetRowStyle{\bfseries} & \multicolumn{2}{-c}{Stem} & \multicolumn{6}{-c}{Cases}\\
    & Root & Suffix &\SetRowStyle{\scshape} \acs{dir} & \acs{nom} & \acs{abs} & \acs{sdt} & \acs{gen} & \acs{cmpr} \\
    \midrule
    \acs{1p}\acs{sg}           & je   & -(a)i, -e  & je   & jem   & ješ   & jeut  & jek   & jed   \\
    \acs{2p}\acs{sg}           & tá   & -ut        & tá   & tám   & táš   & tait  & ták   & tád   \\
    \acs{3p}\acs{sg}           & mi   & -(i)m      & mi   & mim   & miš   & miot  & miek  & mied  \\
    \acs{1p}\acs{du};\acs{inc} & vu   & -iu, -vu   & vu   & vum   & vuš   & vot   & vek   & vud   \\
    \acs{1p}\acs{du};\acs{exc} & če   & -(e)če     & ča   & čém   & čéš   & čeut  & ček   & čed   \\
    \acs{2p}\acs{du}           & tav  & -(e)tu     & táva & távam & távaš & távet & távek & táved \\
    \acs{3p}\acs{du}           & miv  & -(u)mi     & miva & mivam & mivaš & mivet & mivek & mived \\
    \acs{1p}\acs{pl};\acs{inc} & jis  & -(i)sá     & jisa & jisam & jisaš & jiset & jisek & jised \\
    \acs{1p}\acs{pl};\acs{exc} & čes  & -(e)če     & česa & česam & česaš & česet & česek & česed \\
    \acs{2p}\acs{pl}           & tás  & -(a)tá     & tása & tásam & tásaš & táset & tásek & tásed \\
    \acs{3p}\acs{pl}           & mis  & -(a)mi     & misa & misam & misaš & miset & misek & mised \\
    \midrule
    \acs{inanim};\acs{sg}      & han  &            & hana & hanam & hanaš & hanet & hanek & haned \\
    \acs{inanim};\acs{du}      & hava &            & hava & havam &	havaš & havet & havek & haved \\
    \acs{inanim};\acs{pl}      & hasa &            & hasa & hasam & hasaš & haset & hasek & hased \\
    \bottomrule
  \end{tabular}
  \caption{Personal pronouns\label{tab:nm_pronoun_primary_case}}
\end{table}



\begin{landscape}
  \begin{table}[h!]\small\capstart
    \begin{tabular}{SFl -Il -l -l -l -l -l -l -l -l}
      \toprule
      \multicolumn{2}{Fc}{\SetRowStyle{\bfseries}} & \multicolumn{8}{-c}{Cases}\\
      \multicolumn{2}{Fc}{\SetRowStyle{\scshape}} & \acs{ess} & \acs{ins} & \acs{ine} & \acs{ade} & \acs{ill} & \acs{all} & \acs{ela} & \acs{abl} \\
      \multicolumn{2}{Fc}{\SetRowStyle{\itshape}} & el-, l- & eč-, č-, če- & ess- & ez- & est- & ent- & esp- & emp- \\
      \midrule
      \acs{1p}\acs{sg}                            & -ai         & elai    & ečai   & essai   & ezai   & estai   & entai   & espai   & empai    \\
      \acs{2p}\acs{sg}                            & -ut         & alut    & ačut   & assut   & azut   & astut   & antut   & asput   & amput    \\
      \acs{3p}\acs{sg}                            & -mi         & elim    & ečim   & essim   & ezim   & estim   & entim   & espim   & empim    \\
      \acs{1p}\acs{du};\acs{inc}                  & -ivi        & elivi   & ečivi  & essivi  & ezivi  & estivi  & entivi  & espivi  & empivi    \\
      \acs{1p}\acs{du};\acs{exc}                  & -eči        & eleči   & ečeči  & esseči  & ezeči  & esteči  & enteči  & espeči  & empeči    \\
      \acs{2p}\acs{du}                            & -etu        & eletu   & ečetu  & essetu  & ezetu  & estetu  & entetu  & espetu  & empetu   \\
      \acs{3p}\acs{du}                            & -umi        & lumi    & čumi   & essumi  & ezumi  & estumi  & entumi  & espumi  & empumi    \\
      \acs{1p}\acs{pl};\acs{inc}                  & -ísa, -isa  & lísa    & čeisa  & essísa  & ezísa  & estísa  & entísa  & espísa  & empísa    \\
      \acs{1p}\acs{pl};\acs{exc}                  & -ičes       & ličes   & čičes  & essičes & ezičes & estičes & entičes & espičes & empičes   \\
      \acs{2p}\acs{pl}                            & -atás       & latás   & čatás  & essatás & ezatás & estatás & entatás & espatás & empatás   \\
      \acs{3p}\acs{pl}                            & -mis, -emis & lemis   & čemis  & essemis & ezemis & estemis & entemis & espemis & empemis   \\
      \midrule
      \multicolumn{2}{Fc}{\SetRowStyle{\itshape}}               & -lla    & -či    & -ssi    & -zi    & -sti    & -nti    & -spi    & -mpi    \\
      \midrule
      \acs{inanim};\acs{sg}                       & ha-         & halla   & hači   & hassi   & hazi   & hasti   & hanti   & haspi   & hampi   \\
      \acs{inanim};\acs{du}                       & hav-        & havalla & havači & havassi & havazi & havasti & havanti & havaspi & havampi \\
      \acs{inanim};\acs{pl}                       & has-        & hasalla & hasači & hasassi & hasazi & hasasti & hasanti & hasaspi & hasampi \\
      \bottomrule
    \end{tabular}
    \caption{Cases with personal suffixes\label{tab:nm_personal_cases}}
  \end{table}
\end{landscape}


\subsubsection{Possessive Suffixes}
\label{sssec:mn_possessive_suffixes}

Pronomial genetive forms are rarely used when the possessor is animate;
instead, nouns are marked with suffixes that indicate the possessor.  These
suffixes also influence whether the vowel or consonant form of the following
case suffix is used.


% \subsection{Reflexive and Reciprocal Pronouns}
% \label{ssec:nm_reflexive_pronouns}
% 
% Qevesa possesses a single reflexive pronoun, \Q{mech} ‘self’, used to
% refer to something already mentioned.  It inflects with the personal
% suffixes to agree in person with its antecedent.  A related pronoun is the
% reciprocal pronoun \Q{mocchem}, which does not take personal suffixes.

%   \begin{table}[h!]\small\capstart
%   		\begin{tabular}{Sfc -c}
%   			\hline
%   			\SetRowStyle{\bfseries} & Reflexive \\
%   			%\cline{2-2}
%   			\SetRowStyle{\scshape} & refl \\
%   			\hline
%   			\acs{1p}\acs{sg}       & mekai   \\
%   			\acs{2p}\acs{sg}       & mekata  \\
%   			\acs{3p}\acs{sg}       & mekami  \\
%   			\acs{1p}\acs{du};inc   & mekajév \\
%   			\acs{1p}\acs{du};exc   & mekačév \\
%   			\acs{2p}\acs{du}       & mekatáv \\
%   			\acs{3p}\acs{du}       & mekamiv \\
%   			\acs{1p}\acs{pl};inc   & mekajés \\
%   			\acs{1p}\acs{pl};exc   & mekačés \\
%   			\acs{2p}\acs{pl}       & mekatás \\
%   			\acs{3p}\acs{pl}       & mekamis \\
%        recp      & mokem   \\
%        %\hline
%   			inanim;sg & mekonet \\
%   			inanim;du & mekonev \\
%   			inanim;pl & mekones \\
%   			\hline         
%   		\end{tabular}    
%   		\caption{Reflexive and reciprocal pronouns\label{tab:nm_reflexive_pronouns}}
%   \end{table}

% This might fix referencing issues
%\newpage


\subsection{Demonstrative and Correlative Pronouns}
\label{ssec:nm_demonstrative_pronouns}

Qevesa has three degrees of demonstrative pronouns, as well as an interrogative series.

\begin{itemize}
  \item The \textbf{proximal} series refers to things closer to the speaker than the listener;
  \item The \textbf{medial} series refers to things closer to the listener than the speaker; and
  \item The \textbf{distal} series refers to things that are far from both speaker and listener.
\end{itemize}

Demonstrative pronouns must agree in number and case with their antecedent,
unlike all other types of modifiers, such as adjectives. 

The demonstrative pronouns are are listed in \cref{tab:nm_demonstrative_pronouns}.

\begin{table}[h!]\small\capstart
  \begin{tabularx}{0.75 \textwidth}{BFl -Il -l -l -l -l}
    \toprule
    \SetRowStyle{\bfseries} &        & Proximal   & Medial    & Distal     & Interrogative \\
    \SetRowStyle{\scshape}  &        & \acs{prox} & \acs{med} & \acs{dist} & \acs{int}     \\
    \SetRowStyle{\itshape}  &        & to-        & ko-       & ša-        & ve-           \\
    \midrule
    Person                  & -icu   & toicu      & koicu     & šaicu      & veicu         \\
    Animate                 & -re    & tore       & kore      & šare       & vere          \\
    Inanimate               & -ku    & toku       & koku      & šaku       & veku          \\
    Location                & -ze    & toze       & koze      & šaze       & veze          \\
    Direction               & -chira & tochira    & kochira   & šachira    & vechira       \\
    Manner                  & -ːdu    & tódu       & kódu      & šádu       & védu          \\
    \bottomrule
  \end{tabularx}
  \caption{Demonstrative pronouns\label{tab:nm_demonstrative_pronouns}}
\end{table}

%  The prefixes for each series of demonstratives can also be combined with
%  case suffixes, to produce pronouns of specific direction or location, for
%  example:
%
%  \begin{exe}
%    \ex\label{tab:nm_demonstrative_prefixes}
%    \begin{tabular}[t]{fl -l}
%      \SetRowStyle{\itshape} tossa & qespha \\
%      to-ssa & qe-spha \\
%      \SetRowStyle{\scshape} prox-ine & int-all \\
%      in(side) here & towards where?
%    \end{tabular}
%  \end{exe}


\section{Postpositions}
\label{sec:nm_postpositions}

As a left-branching language, Qevesa tends to use postpositions almost
exclusively.  Many postpositions are inflected for case, and require the
complement after which they are placed to adopt a particular case form as
well.

\end{document}
