\documentclass[grammar]{subfiles}
\begin{document}
  \chapter{Nominal Morphology}
  \label{ch:nominal_morphology}

  \section{Definitions and Features}
  \label{sec:nm_definition_features}

  Qevesa nouns, like verbs, are highly regular in their declension.  They
  inflect for two non-inherent features: number and case.  They are also
  occasionally marked for animacy, though this is inherent in the noun, and
  thus is usually only indicated by the declension affixes. 
  
  Unlike in some languages, there is no grammatical gender.  Instead, Qevesa
  uses natural gender, and this is an inherent feature of the noun that is
  neither marked nor affects declension.  Explicit constructions to distinguish
  gender may be used when necessary.

  %There is no grammatical gender.  Explicit constructions to indicate gender
  %are used only when necessary, and these are seldom used except in the formal
  %or literary language.  Nouns may be classed according to animacy, a feature
  %which is only indicated by the declension affixes.

  Most nouns have three numbers, a singular, dual or quantitative, and plural,
  although a small, closed set have a natural number and receive inverse
  marking. 

  There are fourteen cases in the standard written language: focal, nominative,
  absolutive, secundative, genitive, essive, instrumental-commitative,
  inessive, adessive, illative, allative, elative, ablative and comparative.  A
  fifteenth case, the vocative, exists in some spoken dialects, but this is
  falling out of use\footnote{It is interesting to note that the vocative case
    is commonly used when insulting people regardless of dialect.}.

  Nouns can also be marked for four states, which are different types of determinateness.

  The citation form of all nouns is the unmarked form, that is, with no suffixes or prefixes.

  \subsection{Animacy}
  \label{ssec:nm_animacy}

  Nouns in the Teralo family of languages display a property known as animacy,
  in which nouns referring to humans, animals and other things perceived as
  having consciousness or life decline differently to other nouns in some
  forms.  The animacy of a noun must be known in order to properly decline it
  to the primary cases and to indicate pronomial forms.

  Animate nouns refer to humans, animals, spirits, some plants, and some
  meteorological and geological phenomena.  This includes personal names,
  possessions, and some body parts.  Most living but inanimate life forms are
  not included, such as the majority of plants, as wells as microbial life
  forms.  Animacy is a fixed feature, so nouns may not switch between animate
  and inanimate declensions.  Exceptions to this include named objects as well
  as some towns and cities.

  \subsection{Proper Nouns}
  \label{ssec:nm_proper_nouns}

  Proper nouns may be formed from words existing in the
  language\footnotemark{}, often supported by gender markers to disambiguate
  them from common nouns, especially when used as personal names.  A noticeable
  morphological feature of proper nouns is that their case markers are enclitic
  rather than suffixed, separated by a colon or a non-breaking space.  Proper
  names are seldom pluralised.  
  \footnotetext{See \cref{ssec:dev_proper_nouns} on \cpageref{ssec:dev_proper_nouns}
    for derivation of proper nouns.}

  % \subsection{Count and Mass Nouns}
  % \label{ssec:nm_count_mass_nouns}

  % Qevesa distinguishes between count and mass nouns, for example:

  % \begin{exe}
  % 	\ex \emph{EXAMPLES}
  % \end{exe}

  % Materials and abstract qualities cannot be counted, and are grammatically
  % singular.  Things that do not usually occur as singular items are sometimes
  % uncountable and possess a natural number, receiving inverse number when it
  % must be explicitly indicated.  Body parts are typically included in this
  % set:

  % \begin{exe}
  % 	\ex \emph{EXAMPLES}
  % \end{exe}

  \section{Nominal Declension}
  \label{sec:nm_declension}

  Qevesa noun words consist of the stem, followed by number, possessor and case marking:

  \begin{exe}
    \ex\label{ex:nm_structure} \textsc{article}=\textit{stem}\textsc{-number-possessor-case}
  \end{exe}

  The noun phrase may also be preceded by a clitic to indicate the state.

  \subsection{Number}
  \label{ssec:nm_number}

  Qevesa nouns have three numbers, singular, dual and plural, which are marked
  by a series of suffixes that display a form of inverse marking.  Every
  countable noun has an inherent (“natural”) number, which is unmarked, and is
  only marked for number when the noun occurs in a different number.
  
  The dual number also functions as a quantative number.  By itself, it
  indicates that there are exactly two of the noun.  However, if a
  quantity is to be specified, such as with a number word or quantifier,
  the dual form is also used.

  The suffixes that indicate number are listed in \cref{tab:nm_number_suffixes}. 
  An epenthetic \qevesa{-e-} may be inserted if the suffix follows a consonant,
  but this is somewhat irregular and depends on both the previous consonant and
  the suffixes, if any, that follow.

  In addition, a small closed set of nouns has plural declining forms that are
  different to their base form.
  
 % Some examples are given in \cref{exe:nm_number}.

  \begin{table}[htpb]\small\capstart
      \begin{tabular}{>{\bfseries}Fl->{\scshape}Fc -l}
        \toprule
        \multicolumn{2}{Fl}{\SetRowStyle{\bfseries}Number} & Suffix \\
        \midrule
        Natural           &           & -∅  \\
        Singulative       & \acs{sgv} & -(e)r \\
        Dual/Quantitative & \acs{du}  & -(e)v \\
        Plural            & \acs{pl}  & -(e)s \\
        \bottomrule
      \end{tabular}
      \caption{Grammatical number suffixes\label{tab:nm_number_suffixes}}
  \end{table}


  % WRONG WRONG WRONG
%  \begin{exe}
%    \ex\label{exe:nm_number}
%    \begin{tabular}[t]{Fl -l -l -l}\small
%      \SetRowStyle{\bfseries}Natural & Singular    & Dual         & Plural \\
%      \SetRowStyle{\itshape}   tolik & toliker     & tolikev      & tolikes \\
%                               ‘boy’ & ‘(a) boy’   & ‘(two) boys’ & ‘boys’ \\
%      \SetRowStyle{\itshape}    mari & marír       & marív        & marís \\
%                        ‘(two) eyes’ & ‘(one) eye’ & ‘(two) eyes’ & ‘eyes’ \\
%    \end{tabular}
%  \end{exe}
%
%  In Example~\ref{exe:nm_number}, note that the word \qevesa{tolik} ‘boy’ has a
%  singular natural number, but the word \qevesa{mari} has a dual natural
%  number.  The suffixes can be applied for emphasis or to indicate quantity
%  (i.e. \qevesa{koro marív} ‘three eyes’).

  \subsection{Case}
  \label{ssec:nm_case}

  Qevesa possesses fourteen cases (fifteen if the marginal vocative is
  included), which are divided into two groups.  The primary cases, of which
  there are four, indicate morphosyntactic roles of the noun with respect to
  the verb; the remaining ten cases are the secondary cases, and these are
  mostly locative and adverbial cases. 

  The case suffixes are listed in \cref{tab:nm_case_suffixes}.  The left
  columns list suffixes that follow a consonant, and the right columns list
  suffixes that follow a vowel.  

  \begin{table}[htpb]\small\capstart
      \begin{tabular}{>{\bfseries}Fl>{\scshape}l -l -l -l -l}
        \toprule
        \multicolumn{2}{Fc}{\SetRowStyle{\bfseries}Noun Case} & \multicolumn{2}{-c}{Animate} & \multicolumn{2}{-c}{Inanimate} \\
        \SetRowStyle{\scshape} & & \multicolumn{2}{-c}{anim} & \multicolumn{2}{-c}{inanim} \\
        \midrule
        \multirow{2}{*}{Focal}  & \acs{foc}      & -a    & -∅   & -a    & -na  \\
                                & \acs{foc}\sub2 & -a    & -a   & -on   & -n   \\
        Nominative              & \acs{nom}      & -am   & -m   & (-om) & (-m)  \\
        Absolutive              & \acs{abs}      & -aš   & -š   & -oš   & -š \\
        Secundative             & \acs{sdt}      & -ot   & -t   & -ot   & -t  \\
        \midrule
        Genitive                & \acs{gen}      & -ek   & -k   & -ok   & -k  \\
        Essive                  & \acs{ess}      & -el   & -l   & -ol   & -l  \\
        Instrumental            & \acs{ins}      & -ar   & -r   & -or   & -ra \\
        Inessive                & \acs{ine}      & -essi & -ssi & -ossa & -ssa \\
        Adessive                & \acs{ade}      & -ezi  & -zi  & -oza  & -za  \\
        Illative                & \acs{ill}      & -esti & -sti & -osta & -sta \\
        Allative                & \acs{all}      & -etti & -tti & -otta & -tta \\
        Elative                 & \acs{ela}      & -espi & -spi & -ospa & -spa \\
        Ablative                & \acs{abl}      & -eppi & -ppi & -ompa & -mpa \\
        Comparative             & \acs{comp}     & -echi & -chi & -ocha & -cha \\
        (Vocative)              & \acs{voc}      & -o    & -jo  &       & \\
        \bottomrule
      \end{tabular}
      \caption{Case suffixes\label{tab:nm_case_suffixes}}
  \end{table}

%  \subsubsection{The Primary Cases}
%  \label{sssec:nm_primary_cases}
%
%  The primary cases indicate the morphosyntactic role of the noun with respect to the verb.

  \subsubsection{Focal}
  \label{nm_focal_case}

  The focal cases mark the topic of the verb phrase.  The role of the
  noun phrase marked as the focus is indicated on the verb, using the topical
  agreement suffixes as described in \cref{ssec:vm_topical_agreement}.
  This case has an additional form which is used when the focus of the verb
  phrase is already marked with one of the secondary cases, listed in
  \cref{tab:nm_case_suffixes} as \textsc{foc\sub2}.

  \subsubsection{Nominative}
  \label{nm_nominative_case}

  The nominative case marks the voluntary experiencer of an intransitive verb,
  the agent of a transitive verb, or the donor of a ditransitive verb.  Only
  animate nouns can be agents.

  \subsubsection{Absolutive}
  \label{nm_absolutive_case}

  The absolutive case marks the involuntary experiencer of an
  intransitive verb, the patient of a transitive verb, or the recipient of a
  ditransitive verb.

  \subsubsection{Secundative}
  \label{nm_secundative_case}

  The secundative case marks the theme of a ditransitive verb.

%  \subsubsection{The Secondary Cases}
%  \label{sssec:nm_secondary_cases}
%
%  The secondary cases are mainly adpositional and locative cases.

  \subsubsection{Genitive}
  \label{nm_genitive_case}

  The genitive case indicates the possessor of another noun.  Pronomial
  possessors are indicated by means of a suffix on the possessed item.

  \subsubsection{Essive}
  \label{nm_essive_case}

  The essive case indicates duration and time.  It also indicates a
  temporary state of being or existence.

  \subsubsection{Instrumental}
  \label{nm_instrumental_case}

  The instrumental case indicates the means by which the action is
  performed.  It is also used to indicate inanimate agents, as
  inanimate nouns cannot perform actions of their own volition.  
  
  It may also be used in a comitative sense, i.e. to indicate the
  person in whose company the action is carried out.


  \subsubsection{Inessive}
  \label{nm_inessive_case}

  The inessive case indicates internal location. 

  \subsubsection{Adessive}
  \label{nm_adessive_case}

  The adessive case indicates external location.

  \subsubsection{Illative}
  \label{nm_illative_case}

  The illative case indicates motion from the exterior to the interior.

  \subsubsection{Allative}
  \label{nm_allative_case}

  The allative case indicates motion towards the noun.

  \subsubsection{Elative}
  \label{nm_elative_case}

  The elative case indicates motion from the interior to the exterior.

  \subsubsection{Ablative}
  \label{nm_ablative_case}

  The ablative case indicates motion away from the noun.

  \subsubsection{Comparative}
  \label{nm_comparative_case}

  The comparative case indicates a likeness to something, or the
  standard to which something is compared.

  \subsubsection{Vocative}
  \label{nm_vocative_case}

  A vocative case exists in some dialects, and is marginally used in the
  standard language.

  \subsubsection{Use of the Locative Cases}
  \label{sssec:nm_locative_cases}

  The locative cases are logically grouped.  There are two positions (internal
  and external) and three directions (static, movement towards and movement
  away).  Combining these results in the six cases, illustrated in
  \cref{tab:nm_locative_cases}.

  \begin{table}[htpb]\small\capstart
      \begin{tabular}{>{\bfseries}Fl -l -l}
        \toprule
        \SetRowStyle{\bfseries} & Interior & Exterior \\
        \midrule
        Static           & Inessive & Adessive \\
        Movement towards & Illative & Allative \\
        Movement away    & Elative  & Ablative \\
        \bottomrule
      \end{tabular}
      \caption{Locative cases\label{tab:nm_locative_cases}}
  \end{table}

  Finer distinctions in location are given with postpositions, which are
  described in \cref{sec:postpositions}.

  \subsection{Articles}
  \label{ssec:nm_articles}

  There are three articles in Qevesa, the definite, partitive and negative,
  indicated by clitics that precede the noun and any modifiers.  

  The \emph{definite} article marks the noun for definiteness, and functions
  similarly to the definite article in English (though used far more
  frequently).  It has two forms, \qevesa{a} and \qevesa{az}, the former
  preceding consonants and the latter before vowels.  

  The \emph{partitive} article makes the noun partitive.  It functions broadly
  similarly to the English determiner ‘some’, but may also be required by some
  quantifiers.  Like the definite state, it also has two forms, \qevesa{mie}
  and \qevesa{mies}.

  The \emph{negative} article negates the noun, and is distinct from negating the
  verb phrase.  It is formed by the clitic \qevesa{en}, which is identical to
  the numeral ‘zero’.  

  Count nouns with the partitive or negative articles must also be marked with
  the quantitative number, whereas mass nouns never take a number suffix.  

%  The clitics that indicate state are given in \cref{tab:nm_state_clitic}.
%  The left column lists clitics that precede a consonant, and the right column
%  lists clitics that precede a vowel.
%
%  \begin{table}[htpb]\small\capstart
%    \begin{tabular}{>{\bfseries}Fl>{\scshape}l -l -l -l}
%      \toprule
%      \multicolumn{2}{Fc}{\SetRowStyle{\bfseries}State} & \multicolumn{2}{-c}{Clitic} \\
%      \midrule
%      Absolute  & \acs{abst} & ∅   & ∅    \\
%      Definite  & \acs{def}  & a   & az   \\
%      Partitive & \acs{part} & mie & mies \\
%      Negative  & \acs{neg}  & en  & en   \\
%      \bottomrule
%    \end{tabular}
%    \caption{Noun state clitics\label{tab:nm_state_clitic}}
%  \end{table}

%%   railway = iron road = zeléskolij
%%   Qevelian Railways = Qevel??? Zeléskolijes

%% the railway = a zeléskolij
%% some railway(s) = mie zeléskolij
%% no railway(s) = nák zeléskolij

%% a house = panem
%% the house = a panem
%% some house(s) = mie panem
%% no house(s) = nák panem

  \section{Pronouns and Pronomial forms}
  \label{sec:nm_pronouns}

  Pronouns are roughly equivalent to nouns in terms of syntax and morphology.
  They serve as substitutes for other nouns or noun phrases that have
  previously been mentioned or can be inferred from context.  There are a
  number of types of pronouns in Qevesa, including personal pronouns,
  demonstrative pronouns and interrogative pronouns.

  %The class of determiners is a special case, in that they can also act as
  %articles for other nouns or noun phrases.

  \subsection{Personal Pronouns}
  \label{ssec:nm_personal_pronouns}

  The personal pronouns stand in for other nouns, indicating that noun's
  person, number and case.  Most personal pronouns refer only to animate
  referents: a separate inanimate pronoun is used for inanimate referents.
  There are two first person plural pronouns, an inclusive, which includes the
  listener, and an exclusive, which does not. 

  Personal pronouns are declined to the primary cases by suffixation; other
  case constructions use a stem derived from the case ending combined with the
  suffix form of the pronoun.  The suffix form is used to indicate posession;
  pronouns are not declined to the genetive case.  

  The base forms of the pronouns are given in
  \cref{tab:nm_pronoun_primary_case}, and the cases with personal suffixes
  are given in \cref{tab:nm_personal_cases}.

  \begin{table}[htpb]\small\capstart
      \begin{tabular}{>{\scshape}Fl -l -l -l -l -l -l -l}
        \toprule
        \SetRowStyle{\bfseries} & \multicolumn{2}{-c}{Stem} & \multicolumn{5}{-c}{Cases}\\
        & Root & Suffix &\SetRowStyle{\scshape} \acs{foc} & \acs{nom} & \acs{abs} & \acs{sdt} & \acs{gen} \\
        \midrule
        \acs{1p}\acs{sg}           & je   & -(a)i     & je    & jem   & ješ   & jeut  & jek  \\
        \acs{2p}\acs{sg}           & tá   & -ut       & tá    & tám   & táš   & tait  & ták  \\
        \acs{3p}\acs{sg}           & mi   & -(i)m     & mi    & mim   & miš   & miot  & miek  \\
        \acs{1p}\acs{du};\acs{inc} & ju   & -iu, -iv  & íva   & jum   & juš   & ívot  & ívek \\
        \acs{1p}\acs{du};\acs{exc} & če   & -(e)če    & čia   & čém   & čéš   & čeut  & ček \\
        \acs{2p}\acs{du}           & tav  & -(e)tu    & táva  & távam & távaš & távot & távek \\
        \acs{3p}\acs{du}           & miv  & -(u)mi    & miva  & mivam & mivaš & mivot & mivek \\
        \acs{1p}\acs{pl};\acs{inc} & is   & -isa, -si & ísa   & ísam  & ísaš  & ísot  & ísek \\
        \acs{1p}\acs{pl};\acs{exc} & čes  & -(e)če    & česa  & česam & česaš & česot & česek \\
        \acs{2p}\acs{pl}           & tás  & -(a)tá    & tása  & tásam & tásaš & tásot & tásek \\
        \acs{3p}\acs{pl}           & mis  & -(a)mi    & misa  & misam & misaš & misot & misek \\
        \midrule                                                                      
        \acs{inanim};\acs{sg}      & han  & -an, -ːn  & hanna & hanom & hanoš & hanot & hanek \\
        \acs{inanim};\acs{du}      & vina & -ve, -ːve & vinna & vinom & vinoš & vinot & vinek \\
        \acs{inanim};\acs{pl}      & sina & -se, -ːse & sinna & sinom & sinoš & sinot & sinek \\
        \bottomrule
      \end{tabular}
      \caption{Personal pronouns\label{tab:nm_pronoun_primary_case}}
  \end{table}

  \begin{landscape}
  \begin{table}[htpb]\small\capstart
    \begin{tabular}{>{\scshape}Fl ->{\itshape}l -l -l -l -l -l -l -l -l -l}
      \toprule
      \multicolumn{2}{Fc}{\SetRowStyle{\bfseries}} & \multicolumn{9}{-c}{Cases}\\
      \multicolumn{2}{Fc}{\SetRowStyle{\scshape}} & \acs{ess} & \acs{ins} & \acs{ine} & \acs{ade} & \acs{ill} & \acs{all} & \acs{ela} & \acs{abl} & \acs{comp} \\
      \multicolumn{2}{Fc}{\SetRowStyle{\itshape}} & el- & ed- & ess- & ez- & est- & ett- & esp- & epp- & cha- \\
      \midrule
      \acs{1p}\acs{sg}           & -(a)i   & elai   & erai   & essai   & ezai   & estai   & ettai   & espai   & eppai   & chai   \\
      \acs{2p}\acs{sg}           & -ut     & elut   & erut   & essut   & ezut   & estut   & ettut   & esput   & epput   & chát   \\
      \acs{3p}\acs{sg}           & -im     & elim   & erim   & essim   & ezim   & estim   & ettim   & espim   & eppim   & chaim  \\
      \acs{1p}\acs{du};\acs{inc} & -iva    & eliva  & eriva  & essiva  & eziva  & estiva  & ettiva  & espiva  & eppiva  & chaiva \\
      \acs{1p}\acs{du};\acs{exc} & -(e)čev & elečev & erečev & essečev & ezečev & estečev & ettečev & espečev & eppečev & chačev \\
      \acs{2p}\acs{du}           & -(a)tuv & elatuv & eratuv & essatuv & ezatuv & estatuv & ettatuv & espatuv & eppatuv & chatuv \\
      \acs{3p}\acs{du}           & -(a)miv & elamiv & eramiv & essamiv & ezamiv & estamiv & ettamiv & espamiv & eppamiv & chamiv \\
      \acs{1p}\acs{pl};\acs{inc} & -isa    & elisa  & erisa  & essisa  & ezisa  & estisa  & ettisa  & espisa  & eppisa  & chaisa \\
      \acs{1p}\acs{pl};\acs{exc} & -(e)čes & elečes & erečes & essečes & ezečes & estečes & ettečes & espečes & eppečes & chačes \\
      \acs{2p}\acs{pl}           & -(a)tus & elatus & eratus & essatus & ezatus & estatus & ettatus & espatus & eppatus & chatus \\
      \acs{3p}\acs{pl}           & -(a)mis & elamis & eramis & essamis & ezamis & estamis & ettamis & espamis & eppamis & chamis \\
      \midrule
      \multicolumn{2}{Fc}{\SetRowStyle{\itshape}} & ola- & oda- & ossa- & oza- & osta- & otta- & ospa- & ompa- & cho- \\
      \midrule
      \acs{inanim};\acs{sg}      & -ːn  & olán  & orán  & ossán  & ozán  & ostán  & ottán  & ospán  & ompán  & chón \\
      \acs{inanim};\acs{du}      & -ːve & oláve & oráve & ossáve & ozáve & ostáve & ottáve & ospáve & ompáve & chóve \\
      \acs{inanim};\acs{pl}      & -ːse & oláse & oráse & ossáse & ozáse & ostáse & ottáse & ospáse & ompáse & chóse \\
      \bottomrule
    \end{tabular}
    \caption{Cases with personal suffixes\label{tab:nm_personal_cases}}
  \end{table}
  \end{landscape}

  \subsubsection{Possessive Suffixes}
  \label{sssec:mn_possessive_suffixes}

  Pronomial genetive forms are rarely used when the possessor is animate;
  instead, nouns are marked with suffixes that indicate the possessor.  There
  are several different possessive suffixes for each person, and the rules as
  to which is used when are complex.  In general, the bracketed vowels are
  epenthetic, and are only inserted after a consonant.  However, when the
  possessive suffix follows a number suffix, only one of those suffixes may
  contain an epenthetic vowel, with the exception of the first person singular,
  which always occurs as \qevesa{-ai} after a consonant.  The first person dual
  suffix \qevesa{-iu} precedes a consonant, and \qevesa{-iv} precedes a vowel.

  These suffixes also influence whether the vowel or consonant form of the
  following case suffix is used.

  Complete tables of the regular noun suffix combinations are given in \cref{app:noun_suffix_tables}.

  \subsection{Reflexive and Reciprocal Pronouns}
  \label{ssec:nm_reflexive_pronouns}

%  Qevesa does not possess reflexive or reciprocal pronouns as most verb roots
%  have forms that indicate reflexive\footnote{See
%  \cref{ssec:dev_verb_form_v}, page~\pageref{ssec:dev_verb_form_v}} or
%  reciprocal\footnote{See  \cref{ssec:dev_verb_form_iii},
%  page~\pageref{ssec:dev_verb_form_iii}} actions.  The word \qevesa{máka}
%  ‘self’, may be used as a reflexive pronoun to avoid ambiguity, but this is
%  rare.

   Qevesa possesses a single reflexive pronoun, \qevesa{meicha} ‘self’, used to
   refer to something already mentioned.  It inflects with the personal
   suffixes to agree in person with its antecedent.  A related pronoun is the
   reciprocal pronoun \qevesa{mocchem}, which does not take personal suffixes.

%   \begin{table}[htpb]\small\capstart
%   		\begin{tabular}{>{\scshape}fc -c}
%   			\hline
%   			\SetRowStyle{\bfseries} & Reflexive \\
%   			%\cline{2-2}
%   			\SetRowStyle{\scshape} & refl \\
%   			\hline
%   			\acs{1p}\acs{sg}       & mekai   \\
%   			\acs{2p}\acs{sg}       & mekata  \\
%   			\acs{3p}\acs{sg}       & mekami  \\
%   			\acs{1p}\acs{du};inc   & mekajév \\
%   			\acs{1p}\acs{du};exc   & mekačév \\
%   			\acs{2p}\acs{du}       & mekatáv \\
%   			\acs{3p}\acs{du}       & mekamiv \\
%   			\acs{1p}\acs{pl};inc   & mekajés \\
%   			\acs{1p}\acs{pl};exc   & mekačés \\
%   			\acs{2p}\acs{pl}       & mekatás \\
%   			\acs{3p}\acs{pl}       & mekamis \\
%        recp      & mokem   \\
%        %\hline
%   			inanim;sg & mekonet \\
%   			inanim;du & mekonev \\
%   			inanim;pl & mekones \\
%   			\hline         
%   		\end{tabular}    
%   		\caption{Reflexive and reciprocal pronouns\label{tab:nm_reflexive_pronouns}}
%   \end{table}

  % This might fix referencing issues
  %\newpage

  \subsection{Demonstrative and Correlative Pronouns}
  \label{ssec:nm_demonstrative_pronouns}

  Qevesa has three degrees of demonstrative pronouns, as well as an interrogative series.

  \begin{description}[style=nextline]
    \item[Proximal] The proximal series refers to things closer to the speaker than the listener;
    \item[Medial] The medial series refers to things closer to the listener than the speaker; and
    \item[Distal] The distal series refers to things that are far from both speaker and listener.
  \end{description}

  Demonstrative pronouns must agree in number and case with their antecedent,
  unlike all other types of modifiers, such as adjectives. 

  The demonstrative pronouns are are listed in \cref{tab:nm_demonstrative_pronouns}.

  \begin{table}[htpb]\small\capstart
    \begin{tabularx}{0.75 \textwidth}{>{\bfseries}Fl->{\scshape}l ->{\itshape}l *{3}{f} -l}
      \toprule
      \SetRowStyle{\bfseries} & &  & Proximal & Medial & Distal & Interrogative \\
      \SetRowStyle{\scshape}  & & & \acs{prox} & \acs{med} & \acs{dist} & \acs{int} \\
      \SetRowStyle{\itshape}  & & & co- & ko- & tha- & ve- \\
      \midrule
      Animate     & \acs{anim}   & -uka & couka & kouka & thauka & veuka \\
      Inanimate   & \acs{inanim} & -ina & coina & koina & thaina & veina \\
      Location    & \acs{loc}    & -zie & cozie & kozie & thazie & vezia \\
      Source      & \acs{src}    & -spe & cospe & kospe & thaspe & vespa \\
      Destination & \acs{dest}   & -tte & cotte & kotte & thatte & vetta \\
      Time        & \acs{time}   & -lle & colle & kolle & thalle & vella \\
      Manner      & \acs{man}    & -ru  & coru  & koru  & tharu  & vera \\
      \bottomrule
    \end{tabularx}
    \caption{Demonstrative pronouns\label{tab:nm_demonstrative_pronouns}}
  \end{table}

%  The prefixes for each series of demonstratives can also be combined with case suffixes, to produce pronouns of specific direction or location, for example:
%
%  \begin{exe}
%    \ex\label{tab:nm_demonstrative_prefixes}
%    \begin{tabular}[t]{fl -l}
%      \SetRowStyle{\itshape} tossa & qespha \\
%      to-ssa & qe-spha \\
%      \SetRowStyle{\scshape} prox-ine & int-all \\
%      in(side) here & towards where?
%    \end{tabular}
%  \end{exe}

  \section{Postpositions}
  \label{sec:nm_postpositions}

  As a left-branching language, Qevesa tends to use postpositions almost
  exclusively.  Many postpositions are inflected for case, and require the
  complement after which they are placed to adopt a particular case form as
  well.

\end{document}
