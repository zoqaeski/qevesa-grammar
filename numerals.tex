\documentclass[grammar]{subfiles}
\begin{document}
  \chapter{Numerals}
  \label{ch:numerals}

  Qevesa, in common with other Teralo languages, uses a duodecimal
  or base-12 number system for both integers and fractions.  The basic number
  words are listed in \Cref{tab:num_basic}.

  \begin{table}[htpb]\small\capstart
    \begin{tabular}{Fc -l}
      \hline
      \SetRowStyle{\bfseries} & Cardinal \\
      \hline
      0\duo  & en     \\
      1\duo  & ira    \\
      2\duo  & vít    \\
      3\duo  & kor    \\
      4\duo  & qese   \\
      5\duo  & nicha  \\
      6\duo  & zum    \\
      7\duo  & ikuš   \\
      8\duo  & soppi  \\
      9\duo  & jouka  \\
      A\duo  & mieri  \\
      B\duo  & túre   \\
      10\duo & ševa   \\
      \hline
    \end{tabular}
  \caption{Basic numerals\label{tab:num_basic}}
  \end{table}

  Numerals from 11\duo\ to 2B\duo\ are suffixed with \qevesa{-váš}:

    \begin{longtable}[l]{r >{\itshape}l}
      11\duo & erváš    \\
      12\duo & vítváš   \\
      13\duo & korvás   \\
      14\duo & qeseváš  \\
      15\duo & nichaváš \\
      16\duo & zumváš   \\
      17\duo & ikušváš  \\
      28\duo & soppiváš \\
      29\duo & joukaváš \\
      2A\duo & mieriváš \\
      2B\duo & túreváš  \\
    \end{longtable}

  Numerals from 20\duo\ to B0\duo\ are suffixed with \qevesa{-cet}:

    \begin{longtable}[l]{r >{\itshape}l}
      20\duo & vítcet       \\
      30\duo & korcet       \\
      40\duo & qesecet      \\
      50\duo & nichacet     \\
      70\duo & ikušcet      \\
      A0\duo & miericet     \\
      BB\duo & túrecet-túre \\
    \end{longtable}

  Numerals from 100\duo\ to B00\duo\ are suffixed with \qevesa{-tús}:

    \begin{longtable}[l]{r >{\itshape}l}
      100\duo & ertús                \\
      200\duo & víttús               \\
      300\duo & kortús               \\
      409\duo & qesetús-jouka        \\
      752\duo & ikuštús-nichacet-vít \\
    \end{longtable}

  %\newpage
  Numerals from 1000\duo\ to B000\duo\  use the suffix \qevesa{-mazi}:

    \begin{longtable}[l]{r >{\itshape}l}
      1000\duo    & ermazi                                       \\
      2000\duo    & vítmazi                                      \\
      4000\duo    & qesemazi                                     \\
      8603\duo    & soppimazi-zumtús-kor                         \\
      10,000\duo  & ševamazi                                     \\
      17,029\duo  & ševaikušmazi-vítcet-jouka                    \\
      50,000\duo  & nichatúsmazi                                 \\
      93,487\duo  & joukacet-kormazi qesetús-soppicet-ikuš       \\
      100,000\duo & ertúsmazi                                    \\
      582,196\duo & nichatús-soppicet-vítmazi ertús-joukacet-zum \\
    \end{longtable}

  %\newpage
%  Numerals from 10\sup6\duo\ to 10\sup{12}\duo−1 are formed by the addition of the suffix \qevesa{-múl}:
%
%  \begin{exe}
%    \ex
%    \begin{tabular}[t]{r >{\itshape}l}
%      1·10\sup6\duo       & ermúl\\
%      2·10\sup6\duo       & hetimúl\\
%      70·10\sup6\duo      & ikuššamúl\\
%      300·10\sup6\duo     & kortocmúl\\
%      419,203,52A\duo     & qesetoc-ševasoppimúl hettoc-korsavi nichatoc-hetiša-mieri\\
%      900,000,000,000\duo & joukatocsavimúl\\
%    \end{tabular}
%  \end{exe}
%
%  Using this system alone, it is possible to count up to 1BBB,BBB,BBB,BBB\duo, or 
%17,832,200,896,511\dec\footnotemark.
%  
%  \footnotetext{In full, this is \qevesa{ševatúretoc-túreša-túresavimúl túretoc-túreša-túremúl túretoc-túreša-túresavi túretoc-túreša-túre}}

%  \section{Ordinals}
%  \label{sec:num_ordinals}
%
%  The ordinal numerals are formed by appending the suffix \qevesa{-(i)k} to the number word.  For large numerals, the suffix is applied to the last word in the sequence.  The ordinals from *0\sup{th} to 23\dec\sup{st} are given in Table~\ref{tab:num_ordinals}.
%
%  \begin{table}[htpb]\small\capstart
%      \subfloat{
%        \begin{tabular}{c c}
%          \hline
%          \multicolumn{2}{c}{\bfseries Ordinal} \\
%          \hline
%          0 & naxik		\\
%          1 & erk		\\
%          2 & hetik		\\
%          3 & korok	\\
%          4 & qesek	\\
%          5 & nichak	\\
%          6 & zumik	  \\
%          7 & ikušik	\\
%          8 & soppik	\\
%          9 & joukak		\\
%          10\dec & mierik	\\
%          11\dec & túreik	 \\
%          \hline
%        \end{tabular}}\qquad
%      \subfloat{
%        \begin{tabular}{Fc -l}
%          \hline
%          \multicolumn{2}{fc}{\SetRowStyle{\bfseries}Ordinal} \\
%          \hline
%          12\dec & ševak	      \\
%          13\dec & ševaerk	  \\
%          14\dec & ševahetik	  \\
%          15\dec & ševakorok	  \\
%          16\dec & ševaqesek	  \\
%          17\dec & ševanichak   \\
%          18\dec & ševazumik	  \\
%          19\dec & ševaikušik	  \\
%          20\dec & ševasoppik   \\
%          21\dec & ševajoukak  \\
%          22\dec & ševamierik    \\
%          23\dec & ševatúrek   \\
%          \hline
%        \end{tabular}}
%      \caption{Ordinal numerals from 0\dec\ to 23\dec\label{tab:num_ordinals}}
%  \end{table}
%
%  \section{Multiplicatives}
%  \label{sec:num_multiplicatives}
%
%  Numerals in Qevesa also have a special form for multiplicatives, formed by appending the suffix \qevesa{-mi}.  If the numeral stem ends in a consonant, an epenthetic vowel identical to the nucleus vowel of the previous syllable is inserted.  The multiplicative numbers from 0\dec\ to 23\dec\ are listed in Table~\ref{tab:num_multiplicatives}.
%
%  \begin{table}[htpb]\small\capstart
%      \subfloat{
%        \begin{tabular}{c c}
%          \hline
%          \multicolumn{2}{c}{\bfseries Multiplicative} \\
%          \hline
%          0× & naxami	\\
%          1× & ermi	\\
%          2× & hetimi		\\
%          3× & koromi	\\
%          4× & qesemi	\\
%          5× & nichami	\\
%          6× & zumumi	\\
%          7× & ikušumi	\\
%          8× & soppimi	\\
%          9× & joukami	\\
%          10\dec × & mierimi \\
%          11\dec × & túremi \\
%          \hline
%        \end{tabular}}\qquad
%      \subfloat{
%        \begin{tabular}{c c}
%          \hline
%          \multicolumn{2}{c}{\bfseries Multiplicative} \\
%          \hline
%          12× & ševami     \\
%          13× & ševaermi	\\
%          14× & ševahetimi		\\
%          15× & ševakoromi	\\
%          16× & ševaqesemi	\\
%          17× & ševanichami	\\
%          18× & ševazumumi	\\
%          19× & ševaikušumi	\\
%          20× & ševasoppimi	\\
%          21× & ševajoukami	\\
%          22× & ševamierimi	\\
%          23× & ševatúremi	\\
%          \hline
%        \end{tabular}}
%      \caption{Multiplicative numerals from 0\dec\ to 23\dec\label{tab:num_multiplicatives}}
%  \end{table}
%
%  The multiplicative forms are used both in a repetitive and mathematical erse:
%
%  \begin{exe}
%    \ex \emph{EXAMPLES}
%  \end{exe}
%
%  \section{Fractions}
%  \label{sec:num_fractions}
%
%  Fractions are formed by appending the suffix \qevesa{-Vna} where \textit{V} is the nucleus vowel of the previous syllable.  The fractional numbers from 0\dec\ to 21\dec\ are listed in Table~\ref{tab:num_cardinals}.
%
%  \begin{table}[htpb]\small\capstart
%      \subfloat{
%        \begin{tabular}{c c}
%          \hline
%          \multicolumn{2}{c}{\bfseries Fractional} \\
%          \hline
%          *\sup1⁄\sub{0} & *naxana	\\
%          *\sup1⁄\sub{1} & *erna	\\
%          \sup1⁄\sub{2} & hetina		\\
%          \sup1⁄\sub{3} & korna	\\
%          \sup1⁄\sub{4} & qesena	\\
%          \sup1⁄\sub{5} & nichana	\\
%          \sup1⁄\sub{6} & zumuna	\\
%          \sup1⁄\sub{7} & ikušuna	\\
%          \sup1⁄\sub{8} & soppina	\\
%          \sup1⁄\sub{9} & joukana	\\
%          \sup1⁄\sub{10} & mierina \\
%          \sup1⁄\sub{11} & túrena \\
%          \hline
%        \end{tabular}}\qquad
%      \subfloat{
%        \begin{tabular}{c c}
%          \hline
%          \multicolumn{2}{c}{\bfseries Fractional} \\
%          \hline
%          \sup1⁄\sub{12} & ševana     \\
%          \sup1⁄\sub{13} & ševaerna	\\
%          \sup1⁄\sub{14} & ševahetina		\\
%          \sup1⁄\sub{15} & ševakorna	\\
%          \sup1⁄\sub{16} & ševaqesena	\\
%          \sup1⁄\sub{18} & ševanichana	\\
%          \sup1⁄\sub{17} & ševazumuna	\\
%          \sup1⁄\sub{19} & ševaikušuna	\\
%          \sup1⁄\sub{20} & ševasoppina	\\
%          \sup1⁄\sub{21} & ševajoukana	\\
%          \sup1⁄\sub{22} & ševamierina	\\
%          \sup1⁄\sub{23} & ševatúrena	\\
%          \hline
%        \end{tabular}}
%      \caption{Fractional numerals from 0\dec\ to 23\dec\label{tab:num_fractional}}
%  \end{table}
%
%  \newpage
%  The numerator of a fraction precedes the denominator and is in the ordinal form:
%
%  \begin{exe}
%    \ex
%    \begin{xlist}
%      \ex \qevesa{ikušik ševana}
%      \glll ikuš-ik ševa-na\\
%      seven-\acs{ord} twelve-\acs{frac}\\
%      seven twelfth\\
%      \glt seven-twelfths
%      \ex \qevesa{hetik korna litasevok}
%      \glll het-ik kor-na litas-ev-ok\\
%      two-\acs{ord} three-\acs{frac} bread-\acs{du}-\acs{gen}\\
%      two third bread\\
%      \glt two-thirds of bread
%    \end{xlist}
%  \end{exe}
%
%  If the denominator of a fraction is a compound number, the fractional suffix is appended to the final word in the sequence:
%
%  \begin{exe}
%    \ex
%    \begin{xlist}
%      \ex \qevesa{zumšana}
%      \glll zumša-na\\
%      sixty-\acs{frac}\\
%      sixtieth\\
%      \glt (a) sixtieth
%      \ex \qevesa{soppík hetišana}
%      \glll soppi-ik heti-ša-na\\
%      eight-\acs{ord} two-dozen-\acs{frac}\\
%      eight twenty-fourths\\
%      \glt eight twenty-fourths
%    \end{xlist}
%  \end{exe}
%
%
%  More complex fractions {\em are yet to be written about… in particular, I need:
%    \begin{itemize}
%      \item Integer ± unit fraction
%      \item Integer × unit fraction
%    \end{itemize}
%  }


\end{document}

