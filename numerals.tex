\documentclass[grammar]{subfiles}
\begin{document}
  \chapter{Numerals}
  \label{ch:numerals}

  \ToBeWritten
  Qevesa, in common with other Teralo languages, uses a duodecimal or base-12
  number system for both integers and fractions.

  \section{Cardinals}
  \label{sec:num_cardinals}

  The base number words are the cardinal numerals. 
  %With the exception of a \qevesa{nak} (“zero, none”), the stems for numerals cannot be composed into consonantal roots. 
  The cardinals from 0\duo\ to 21\duo\ are listed in
  Table~\ref{tab:num_cardinals}.

  \begin{table}[htpb]\small\capstart
      \subfloat{
        \begin{tabular}{Fc -l}
          \hline
          \multicolumn{2}{c}{\bfseries Cardinal} \tnl
          \hline
          0 & en   \tnl
          1 & iso   \tnl
          2 & heti  \tnl
          3 & kor  \tnl
          4 & qese  \tnl
          5 & nicha  \tnl
          6 & zum   \tnl
          7 & ikuš  \tnl
          8 & soppi \tnl
          9 & jouka  \tnl
          A & mieri  \tnl
          B & túre  \tnl
          \hline
        \end{tabular}}\qquad
      \subfloat{
        \begin{tabular}{Fc -l}
          \hline
          \multicolumn{2}{fc}{\SetRowStyle{\bfseries}Cardinal} \tnl
          \hline
          10 & ševa	     \tnl
          11 & ševaiso	 \tnl
          12 & ševaheti	 \tnl
          13 & ševakor	 \tnl
          14 & ševaqese	 \tnl
          15 & ševanicha  \tnl
          16 & ševazum	 \tnl
          17 & ševaikuš	 \tnl
          28 & ševasoppi \tnl
          29 & ševajouka	 \tnl
          2A & ševamieri  \tnl
          2B & ševatúre  \tnl
          \hline
        \end{tabular}}
      \caption{Cardinal numerals from 0\dec\ to 23\dec\label{tab:num_cardinals}}
  \end{table}

  Numerals from 20\duo\ to B0\duo\ are suffixed with \qevesa{-ša}:

  \begin{exe}
    \ex
    \begin{tabular}[t]{r >{\itshape}l}
      20\duo & hetiša\\
      30\duo & korša\\
      40\duo & qeseša\\
      50\duo & nichaša\\
      70\duo & ikušša\\
      A0\duo & mieriša\\
      BB\duo & túreša-túre\\
    \end{tabular}
  \end{exe}

  Numerals from 100\duo\ to B00\duo\ are suffixed with \qevesa{-toc}:

  \begin{exe}
    \ex
    \begin{tabular}[t]{r >{\itshape}l}
      100\duo & isotoc \\
      200\duo & hettoc \\
      300\duo & kortoc \\
      409\duo & qesetoc-jouka \\
      752\duo & ikuštoc-nichaša-heti \\
    \end{tabular}
  \end{exe}

  %\newpage
  Numerals from 1000\duo\ to B000\duo\  use the suffix \qevesa{-síva}:

  \begin{exe}
    \ex
    \begin{tabular}[t]{r >{\itshape}l}
      1000\duo    & isosíva\\
      2000\duo    & hetsíva\\
      4000\duo    & qesesíva\\
      8603\duo    & soppisíva-zumtoc-kor\\
      10,000\duo  & ševasíva\\
      17,029\duo  & ševaikušsíva-hetiša-jouka\\
      50,000\duo  & nichašasíva\\
      93,487\duo  & joukaša-korsíva qesetoc-soppiša-ikuš\\
      100,000\duo & isotocsíva\\
      582,196\duo & nichatoc-soppiša-hetsíva isotoc-joukaša-zum\\
    \end{tabular}
  \end{exe}

  %\newpage
  Numerals from 10\sup6\duo\ to 10\sup{12}\duo−1 are formed by the addition of the suffix \qevesa{-múl}:

  \begin{exe}
    \ex
    \begin{tabular}[t]{r >{\itshape}l}
      1·10\sup6\duo       & isomúl\\
      2·10\sup6\duo       & hetimúl\\
      70·10\sup6\duo      & ikuššamúl\\
      300·10\sup6\duo     & kortocmúl\\
      419,203,52A\duo     & qesetoc-ševasoppimúl hettoc-korsíva nichatoc-hetiša-mieri\\
      900,000,000,000\duo & joukatocsívamúl\\
    \end{tabular}
  \end{exe}

  Using this system alone, it is possible to count up to 1BBB,BBB,BBB,BBB\duo, or 
17,832,200,896,511\dec\footnotemark.
  
  \footnotetext{In full, this is \qevesa{ševatúretoc-túreša-túresívamúl túretoc-túreša-túremúl túretoc-túreša-túresíva túretoc-túreša-túre}}

%  \section{Ordinals}
%  \label{sec:num_ordinals}
%
%  The ordinal numerals are formed by appending the suffix \qevesa{-(i)k} to the number word.  For large numerals, the suffix is applied to the last word in the sequence.  The ordinals from *0\sup{th} to 23\dec\sup{st} are given in Table~\ref{tab:num_ordinals}.
%
%  \begin{table}[htpb]\small\capstart
%      \subfloat{
%        \begin{tabular}{c c}
%          \hline
%          \multicolumn{2}{c}{\bfseries Ordinal} \tnl
%          \hline
%          0 & naxik		\tnl
%          1 & isok		\tnl
%          2 & hetik		\tnl
%          3 & korok	\tnl
%          4 & qesek	\tnl
%          5 & nichak	\tnl
%          6 & zumik	  \tnl
%          7 & ikušik	\tnl
%          8 & soppik	\tnl
%          9 & joukak		\tnl
%          10\dec & mierik	\tnl
%          11\dec & túreik	 \tnl
%          \hline
%        \end{tabular}}\qquad
%      \subfloat{
%        \begin{tabular}{Fc -l}
%          \hline
%          \multicolumn{2}{fc}{\SetRowStyle{\bfseries}Ordinal} \tnl
%          \hline
%          12\dec & ševak	      \tnl
%          13\dec & ševaisok	  \tnl
%          14\dec & ševahetik	  \tnl
%          15\dec & ševakorok	  \tnl
%          16\dec & ševaqesek	  \tnl
%          17\dec & ševanichak   \tnl
%          18\dec & ševazumik	  \tnl
%          19\dec & ševaikušik	  \tnl
%          20\dec & ševasoppik   \tnl
%          21\dec & ševajoukak  \tnl
%          22\dec & ševamierik    \tnl
%          23\dec & ševatúrek   \tnl
%          \hline
%        \end{tabular}}
%      \caption{Ordinal numerals from 0\dec\ to 23\dec\label{tab:num_ordinals}}
%  \end{table}
%
%  \section{Multiplicatives}
%  \label{sec:num_multiplicatives}
%
%  Numerals in Qevesa also have a special form for multiplicatives, formed by appending the suffix \qevesa{-mi}.  If the numeral stem ends in a consonant, an epenthetic vowel identical to the nucleus vowel of the previous syllable is inserted.  The multiplicative numbers from 0\dec\ to 23\dec\ are listed in Table~\ref{tab:num_multiplicatives}.
%
%  \begin{table}[htpb]\small\capstart
%      \subfloat{
%        \begin{tabular}{c c}
%          \hline
%          \multicolumn{2}{c}{\bfseries Multiplicative} \tnl
%          \hline
%          0× & naxami	\tnl
%          1× & isomi	\tnl
%          2× & hetimi		\tnl
%          3× & koromi	\tnl
%          4× & qesemi	\tnl
%          5× & nichami	\tnl
%          6× & zumumi	\tnl
%          7× & ikušumi	\tnl
%          8× & soppimi	\tnl
%          9× & joukami	\tnl
%          10\dec × & mierimi \tnl
%          11\dec × & túremi \tnl
%          \hline
%        \end{tabular}}\qquad
%      \subfloat{
%        \begin{tabular}{c c}
%          \hline
%          \multicolumn{2}{c}{\bfseries Multiplicative} \tnl
%          \hline
%          12× & ševami     \tnl
%          13× & ševaisomi	\tnl
%          14× & ševahetimi		\tnl
%          15× & ševakoromi	\tnl
%          16× & ševaqesemi	\tnl
%          17× & ševanichami	\tnl
%          18× & ševazumumi	\tnl
%          19× & ševaikušumi	\tnl
%          20× & ševasoppimi	\tnl
%          21× & ševajoukami	\tnl
%          22× & ševamierimi	\tnl
%          23× & ševatúremi	\tnl
%          \hline
%        \end{tabular}}
%      \caption{Multiplicative numerals from 0\dec\ to 23\dec\label{tab:num_multiplicatives}}
%  \end{table}
%
%  The multiplicative forms are used both in a repetitive and mathematical isose:
%
%  \begin{exe}
%    \ex \emph{EXAMPLES}
%  \end{exe}
%
%  \section{Fractions}
%  \label{sec:num_fractions}
%
%  Fractions are formed by appending the suffix \qevesa{-Vna} where \textit{V} is the nucleus vowel of the previous syllable.  The fractional numbers from 0\dec\ to 21\dec\ are listed in Table~\ref{tab:num_cardinals}.
%
%  \begin{table}[htpb]\small\capstart
%      \subfloat{
%        \begin{tabular}{c c}
%          \hline
%          \multicolumn{2}{c}{\bfseries Fractional} \tnl
%          \hline
%          *\sup1⁄\sub{0} & *naxana	\tnl
%          *\sup1⁄\sub{1} & *isona	\tnl
%          \sup1⁄\sub{2} & hetina		\tnl
%          \sup1⁄\sub{3} & korna	\tnl
%          \sup1⁄\sub{4} & qesena	\tnl
%          \sup1⁄\sub{5} & nichana	\tnl
%          \sup1⁄\sub{6} & zumuna	\tnl
%          \sup1⁄\sub{7} & ikušuna	\tnl
%          \sup1⁄\sub{8} & soppina	\tnl
%          \sup1⁄\sub{9} & joukana	\tnl
%          \sup1⁄\sub{10} & mierina \tnl
%          \sup1⁄\sub{11} & túrena \tnl
%          \hline
%        \end{tabular}}\qquad
%      \subfloat{
%        \begin{tabular}{c c}
%          \hline
%          \multicolumn{2}{c}{\bfseries Fractional} \tnl
%          \hline
%          \sup1⁄\sub{12} & ševana     \tnl
%          \sup1⁄\sub{13} & ševaisona	\tnl
%          \sup1⁄\sub{14} & ševahetina		\tnl
%          \sup1⁄\sub{15} & ševakorna	\tnl
%          \sup1⁄\sub{16} & ševaqesena	\tnl
%          \sup1⁄\sub{18} & ševanichana	\tnl
%          \sup1⁄\sub{17} & ševazumuna	\tnl
%          \sup1⁄\sub{19} & ševaikušuna	\tnl
%          \sup1⁄\sub{20} & ševasoppina	\tnl
%          \sup1⁄\sub{21} & ševajoukana	\tnl
%          \sup1⁄\sub{22} & ševamierina	\tnl
%          \sup1⁄\sub{23} & ševatúrena	\tnl
%          \hline
%        \end{tabular}}
%      \caption{Fractional numerals from 0\dec\ to 23\dec\label{tab:num_fractional}}
%  \end{table}
%
%  \newpage
%  The numerator of a fraction precedes the denominator and is in the ordinal form:
%
%  \begin{exe}
%    \ex
%    \begin{xlist}
%      \ex \qevesa{ikušik ševana}
%      \glll ikuš-ik ševa-na\\
%      seven-\acs{ord} twelve-\acs{frac}\\
%      seven twelfth\\
%      \glt seven-twelfths
%      \ex \qevesa{hetik korna litasevok}
%      \glll het-ik kor-na litas-ev-ok\\
%      two-\acs{ord} three-\acs{frac} bread-\acs{du}-\acs{gen}\\
%      two third bread\\
%      \glt two-thirds of bread
%    \end{xlist}
%  \end{exe}
%
%  If the denominator of a fraction is a compound number, the fractional suffix is appended to the final word in the sequence:
%
%  \begin{exe}
%    \ex
%    \begin{xlist}
%      \ex \qevesa{zumšana}
%      \glll zumša-na\\
%      sixty-\acs{frac}\\
%      sixtieth\\
%      \glt (a) sixtieth
%      \ex \qevesa{soppík hetišana}
%      \glll soppi-ik heti-ša-na\\
%      eight-\acs{ord} two-dozen-\acs{frac}\\
%      eight twenty-fourths\\
%      \glt eight twenty-fourths
%    \end{xlist}
%  \end{exe}
%
%
%  More complex fractions {\em are yet to be written about… in particular, I need:
%    \begin{itemize}
%      \item Integer ± unit fraction
%      \item Integer × unit fraction
%    \end{itemize}
%  }


\end{document}

