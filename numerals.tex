\documentclass[grammar]{subfiles}
\begin{document}
  \chapter{Numerals}
  \label{ch:numerals}

  Qevesa, in common with other Teralo languages, uses a duodecimal or base-12 number system for both integers and fractions.

  \section{Cardinals}
  \label{sec:num_cardinals}

  The base number words are the cardinal numerals. 
  With the exception of a \qevesa{nak} (“zero, none”), the stems for numerals cannot be composed into consonantal roots. 
  The cardinals from 0\dec\ to 21\dec\ are listed in Table~\ref{tab:num_cardinals}.

  \begin{table}[htpb]\small\capstart
      \subfloat{
        \begin{tabular}{|fc|-c|}
          \hline
          \multicolumn{2}{|fc|}{\SetRowStyle{\bfseries}Cardinal} \tnl
          \hline
          0 & nak		\tnl
          1 & sen		\tnl
          2 & ëti		\tnl
          3 & kör		\tnl
          4 & qese	\tnl
          5 & von	\tnl
          6 & neca		\tnl
          7 & ikuš	\tnl
          8 & sopri	\tnl
          9 & jok		\tnl
          10\dec & meri	\tnl
          11\dec & türe	 \tnl
          \hline
        \end{tabular}}\qquad
      \subfloat{
        \begin{tabular}{|fc|-c|}
          \hline
          \multicolumn{2}{|fc|}{\SetRowStyle{\bfseries}Cardinal} \tnl
          \hline
          12\dec & ševa	    \tnl
          13\dec & ševasen	  \tnl
          14\dec & ševajet		\tnl
          15\dec & ševakör	  \tnl
          16\dec & ševaqese	 \tnl
          17\dec & ševavon \tnl
          18\dec & ševaneca	  \tnl
          19\dec & ševaikuš	\tnl
          20\dec & ševasopri \tnl
          21\dec & ševajok	  \tnl
          22\dec & ševameri  \tnl
          23\dec & ševatüre  \tnl
          \hline
        \end{tabular}}
      \caption{Cardinal numerals from 0\dec\ to 23\dec\label{tab:num_cardinals}}
  \end{table}

  Numerals from 20\duo\ to B0\duo\ are suffixed with \qevesa{-ša}:

  \begin{exe}
    \ex
    \begin{tabular}[t]{r >{\itshape}l}
      20\duo & ëtiša\\
      30\duo & körša\\
      40\duo & qeseša\\
      50\duo & vonša\\
      70\duo & ikušša \textup{(*\emph{ikušsa})}\\
      A0\duo & meriša\\
      BB\duo & türeša-türe\\
    \end{tabular}
  \end{exe}

  Numerals from 100\duo\ to B00\duo\ are suffixed with \qevesa{-toc}:

  \begin{exe}
    \ex
    \begin{tabular}[t]{r >{\itshape}l}
      100\duo & sentoc\\
      200\duo & ëttoc\\
      300\duo & körtoc\\
      409\duo & qesetoc-jok\\
      752\duo & ikuštoc-vonša-ëti\\
    \end{tabular}
  \end{exe}

  %\newpage
  Numerals from 1000\duo\ to B000\duo\  use the suffix \qevesa{-síva}:

  \begin{exe}
    \ex
    \begin{tabular}[t]{r >{\itshape}l}
      1000\duo    & sensíva\\
      2000\duo    & ëtsíva\\
      4000\duo    & qesesíva\\
      8603\duo    & soprisíva-necatoc-kör\\
      10,000\duo  & ševasíva\\
      17,029\duo  & ševaikušsíva-ëtiša-jok\\
      50,000\duo  & vonšasíva\\
      93,487\duo  & jokša-körsíva qesetoc-sopriša-ikuš\\
      100,000\duo & sentocsíva\\
      682,196\duo & necatoc-sopriša-ëtsíva sentoc-jokša-neca\\
    \end{tabular}
  \end{exe}

  %\newpage
  Numerals from 10\sup6\duo\ to 10\sup{12}\duo−1 are formed by the addition of the suffix \qevesa{-műl}:

  \begin{exe}
    \ex
    \begin{tabular}[t]{r >{\itshape}l}
      1·10\sup6\duo       & semműl \textup{(*\emph{senműl} )}\\
      2·10\sup6\duo       & ëtiműl\\
      70·10\sup6\duo      & ikuššaműl\\
      300·10\sup6\duo     & körtocműl\\
      419,203,62A\duo     & qesetoc-ševasopriműl ëttoc-körsíva necatoc-ëtiša-meri\\
      900,000,000,000\duo & joktocsívaműl\\
    \end{tabular}
  \end{exe}

  Using this system alone, it is possible to count up to 1BBB,BBB,BBB,BBB\duo, or 
17,832,200,896,511\dec\footnotemark.
  
  \footnotetext{In full, this is \qevesa{ševatüretoc-türeša-türesívaműl türetoc-türeša-türeműl türetoc-türeša-türesíva türetoc-türeša-türe}}

  \section{Ordinals}
  \label{sec:num_ordinals}

  The ordinal numerals are formed by appending the suffix \qevesa{-ik} to the number word.  For large numerals, the suffix is applied to the last word in the sequence.  The ordinals from *0\sup{th} to 23\dec\sup{st} are given in Table~\ref{tab:num_ordinals}.

  \begin{table}[htpb]\small\capstart
      \subfloat{
        \begin{tabular}{|c|c|}
          \hline
          \multicolumn{2}{|c|}{\bfseries Ordinal} \tnl
          \hline
          0 & nakik		\tnl
          1 & senik		\tnl
          2 & ëtik		\tnl
          3 & körik		\tnl
          4 & qeseik	\tnl
          5 & vonik	\tnl
          6 & necaik		\tnl
          7 & ikušik	\tnl
          8 & soprík	\tnl
          9 & jokik		\tnl
          10\dec & merík	\tnl
          11\dec & türeik	 \tnl
          \hline
        \end{tabular}}\qquad
      \subfloat{
        \begin{tabular}{|fc|-c|}
          \hline
          \multicolumn{2}{|fc|}{\SetRowStyle{\bfseries}Ordinal} \tnl
          \hline
          12\dec & ševaik	    \tnl
          13\dec & ševasenik	  \tnl
          14\dec & ševajetik		\tnl
          15\dec & ševakörik	  \tnl
          16\dec & ševaqeseik	 \tnl
          17\dec & ševavonik \tnl
          18\dec & ševanecaik	  \tnl
          19\dec & ševaikušik	\tnl
          20\dec & ševasoprík \tnl
          21\dec & ševajokik	  \tnl
          22\dec & ševamerík  \tnl
          23\dec & ševatüreik  \tnl
          \hline
        \end{tabular}}
      \caption{Ordinal numerals from 0\dec\ to 23\dec\label{tab:num_ordinals}}
  \end{table}

  \section{Multiplicatives}
  \label{sec:num_multiplicatives}

  Numerals in Qevesa also have a special form for multiplicatives, formed by appending the suffix \qevesa{-mi}.  If the numeral stem ends in a consonant, an epenthetic vowel identical to the nucleus vowel of the previous syllable is inserted.  The multiplicative numbers from 0\dec\ to 23\dec\ are listed in Table~\ref{tab:num_multiplicatives}.

  \begin{table}[htpb]\small\capstart
      \subfloat{
        \begin{tabular}{|c|c|}
          \hline
          \multicolumn{2}{|c|}{\bfseries Multiplicative} \tnl
          \hline
          0× & nakami	\tnl
          1× & senemi	\tnl
          2× & ëtími		\tnl
          3× & körömi	\tnl
          4× & qesémi	\tnl
          5× & vonomi	\tnl
          6× & necámi	\tnl
          7× & ikušumi	\tnl
          8× & soprími	\tnl
          9× & jokomi	\tnl
          10\dec × & merími \tnl
          11\dec × & türémi \tnl
          \hline
        \end{tabular}}\qquad
      \subfloat{
        \begin{tabular}{|c|c|}
          \hline
          \multicolumn{2}{|c|}{\bfseries Multiplicative} \tnl
          \hline
          12× & ševámi     \tnl
          13× & ševasenemi	\tnl
          14× & ševajetími		\tnl
          15× & ševakörömi	\tnl
          16× & ševaqesémi	\tnl
          17× & ševavonomi	\tnl
          18× & ševanecámi	\tnl
          19× & ševaikušumi	\tnl
          20× & ševasoprími	\tnl
          21× & ševajokomi	\tnl
          22× & ševamerími	\tnl
          23× & ševatürémi	\tnl
          \hline
        \end{tabular}}
      \caption{Multiplicative numerals from 0\dec\ to 23\dec\label{tab:num_multiplicatives}}
  \end{table}

  The multiplicative forms are used both in a repetitive and mathematical sense:

  \begin{exe}
    \ex \emph{EXAMPLES}
  \end{exe}

  \section{Fractions}
  \label{sec:num_fractions}

  Fractions are formed by appending the suffix \qevesa{-Vna} where \textit{V} is the nucleus vowel of the previous syllable — numerals ending in a vowel have this vowel lengthened instead.  The fractional numbers from 0\dec\ to 21\dec\ are listed in Table~\ref{tab:num_cardinals}.

  \begin{table}[htpb]\small\capstart
      \subfloat{
        \begin{tabular}{|c|c|}
          \hline
          \multicolumn{2}{|c|}{\bfseries Fractional} \tnl
          \hline
          *\sup1⁄\sub{0} & *nakana	\tnl
          *\sup1⁄\sub{1} & *senena	\tnl
          \sup1⁄\sub{2} & ëtína		\tnl
          \sup1⁄\sub{3} & köröna	\tnl
          \sup1⁄\sub{4} & qeséna	\tnl
          \sup1⁄\sub{5} & vonona	\tnl
          \sup1⁄\sub{6} & necána	\tnl
          \sup1⁄\sub{7} & ikušuna	\tnl
          \sup1⁄\sub{8} & soprína	\tnl
          \sup1⁄\sub{9} & jokona	\tnl
          \sup1⁄\sub{10} & merína \tnl
          \sup1⁄\sub{11} & türéna \tnl
          \hline
        \end{tabular}}\qquad
      \subfloat{
        \begin{tabular}{|c|c|}
          \hline
          \multicolumn{2}{|c|}{\bfseries Fractional} \tnl
          \hline
          \sup1⁄\sub{12} & ševana     \tnl
          \sup1⁄\sub{13} & ševasenena	\tnl
          \sup1⁄\sub{14} & ševajetína		\tnl
          \sup1⁄\sub{15} & ševaköröna	\tnl
          \sup1⁄\sub{16} & ševaqeséna	\tnl
          \sup1⁄\sub{17} & ševavonona	\tnl
          \sup1⁄\sub{18} & ševanecána	\tnl
          \sup1⁄\sub{19} & ševaikušuna	\tnl
          \sup1⁄\sub{20} & ševasoprína	\tnl
          \sup1⁄\sub{21} & ševajokona	\tnl
          \sup1⁄\sub{22} & ševamerína	\tnl
          \sup1⁄\sub{23} & ševatüréna	\tnl
          \hline
        \end{tabular}}
      \caption{Fractional numerals from 0\dec\ to 23\dec\label{tab:num_fractional}}
  \end{table}

  \newpage
  The numerator of a fraction precedes the denominator and is in the ordinal form:

  \begin{exe}
    \ex
    \begin{xlist}
      \ex \qevesa{ikušik ševána}
      \glll ikuš-ik ševa-ana\\
      seven-\acs{ord} twelve-\acs{frac}\\
      seven twelfth\\
      \glt seven-twelfths
      \ex \qevesa{ëtik köröna litasevok}
      \glll ët-ik kör-öna litas-ev-ok\\
      two-\acs{ord} three-\acs{frac} bread-\acs{du}-\acs{gen}\\
      two third bread\\
      \glt two-thirds of bread
    \end{xlist}
  \end{exe}

  If the denominator of a fraction is a compound number, the fractional suffix is appended to the final word in the sequence:

  \begin{exe}
    \ex
    \begin{xlist}
      \ex \qevesa{necašána}
      \glll necaša-ana\\
      sixty-\acs{frac}\\
      sixtieth\\
      \glt (a) sixtieth
      \ex \qevesa{soprík ëtišána}
      \glll sopri-ik ëti-ša-ana\\
      eight-\acs{ord} two-dozen-\acs{frac}\\
      eight twenty-fourths\\
      \glt eight twenty-fourths
    \end{xlist}
  \end{exe}


  More complex fractions {\em are yet to be written about… in particular, I need:
    \begin{itemize}
      \item Integer ± unit fraction
      \item Integer × unit fraction
    \end{itemize}
  }


\end{document}

