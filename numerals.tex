\documentclass[grammar]{subfiles}
\begin{document}
  \chapter{Numerals}
  \label{ch:numerals}

  Qevesa, in common with other Teralo languages, uses a duodecimal or base-12 number system for both integers and fractions Although the primary base is duodecimal, evidence of other bases can be seen in some number forms. It is interesting to note that the words for 1\dec\ to 20\dec\ (with the exception of 15\dec) are unanalysable, but the numerals for 21\dec\ to 23\dec\ are vigesimal\footnotemark{}, as are a number of others.
  \footnotetext{Historically, Qevesa used a vigesimal number system, but influence from neighbouring languages caused a switch to a duodecimal system instead.}

  \section{Cardinals}
  \label{sec:num_cardinals}

  The base number words are the cardinal numerals. The stems for numerals are unique in that they include vowels, which leads to their classification as \emph{stems} and not \emph{roots}; if the stems are to be used as consonantal roots, the vowels are dropped. Note that consonant clusters are treated as single consonants for numeric roots. The cardinals from 0\dec\ to 21\dec\ are listed in Table~\ref{tab:num_cardinals}.

  \begin{table}[htpb]\small\capstart
    \begin{center}
      \subfloat{
        \begin{tabular}{|fc|-c|-c|}
          \hline
          \multicolumn{2}{|fc|}{\SetRowStyle{\bfseries}Cardinal} & Root \tabularnewline
          \hline
          0 & dom				& d-m-	\tabularnewline
          1 & sen				& s-n-	\tabularnewline
          2 & ěti				& j-t-	\tabularnewline
          3 & köl				& k-l-	\tabularnewline
          4 & qese			& q-s-	\tabularnewline
          5 & inü				& h-n-	\tabularnewline
          6 & von				& v-n-	\tabularnewline
          7 & ikuş			& h-k-ş	\tabularnewline
          8 & sopri			& s-pr-	\tabularnewline
          9 & jok				& j-k-	\tabularnewline
          10\dec & méri	& m-r-	\tabularnewline
          \hline
        \end{tabular}}\qquad
      \subfloat{
        \begin{tabular}{|fc|-c|-c|}
          \hline
          \multicolumn{2}{|fc|}{\SetRowStyle{\bfseries}Cardinal} & Root \tabularnewline
          \hline
          11\dec & türe		& t-r- \tabularnewline
          12\dec & żel		& ż-l- \tabularnewline
          13\dec & scem		& sc-m- \tabularnewline
          14\dec & zpet		& zp-t- \tabularnewline
          15\dec & żeköl	& ż-k-l \tabularnewline
          16\dec & ksoż		& ks-ż- \tabularnewline
          17\dec & pedła	& p-dł- \tabularnewline
          18\dec & pseňa	& ps-ň- \tabularnewline
          19\dec & sevča	& s-vč- \tabularnewline
          20\dec & vaudi	& v-d- \tabularnewline
          21\dec & vaudi-sen & ∅ \tabularnewline
          \hline
        \end{tabular}}
      \caption{Cardinal numerals from 0\dec\ to 21\dec\label{tab:num_cardinals}}
    \end{center}
  \end{table}

  \newpage
  The numerals from 19\duo\ to 1B\duo\ are vigesimal, as are some multiples of 50\duo\ (60\dec). Numerals from 20\duo\ to BB\duo\ are suffixed with \textit{-za}:

  \begin{exe}
    \ex
    \begin{tabular}[t]{r >{\itshape}l}
      19\duo & vaudi-sen\\
      1A\duo & vaudi-ěti\\
      1B\duo & vaudi-köl\\
      20\duo & ětiza\\
      30\duo & kölza\\
      40\duo & qeseza\\
      50\duo & kölvád \textup{or} inüza\\
      65\duo & vonza-inü\\
      A0\duo & mériza\\
      BB\duo & türeza-türe\\
    \end{tabular}
  \end{exe}

  Numerals from 100\duo\ to BBB\duo\ are suffixed with \textit{-toc}:

  \begin{exe}
    \ex
    \begin{tabular}[t]{r >{\itshape}l}
      100\duo & sentoc\\
      200\duo & ěttoc\\
      300\duo & költoc\\
      409\duo & qesetoc-jok\\
      752\duo & ikuştoc-kölvád-ěti\\
    \end{tabular}
  \end{exe}

  %\newpage
  Numerals from 1000\duo\ to BBBB\duo\  use the suffix \textit{-síva}:

  \begin{exe}
    \ex
    \begin{tabular}[t]{r >{\itshape}l}
      1000\duo    & sensíva\\
      2000\duo    & ětsíva\\
      4000\duo    & qessíva \textup{(*\emph{qesesíva} )}\\
      8603\duo    & soprisíva-vontoc-köl\\
      10,000\duo  & żelsíva\\
      17,029\duo  & pedłasíva-ětiza-jok\\
      50,000\duo  & kölvádsíva\\
      93,487\duo  & jokza-kölsíva qesetoc-sopriza-ikuş\\
      100,000\duo & sentossíva\\
      682,196\duo & vontoc-sopriza-ětsíva sentoc-jokza-von\\
    \end{tabular}
  \end{exe}

  \newpage
  Numerals from 10\sup6\duo\ to 10\sup{12}\duo−1 are formed by the addition of the suffix \textit{-műl}:

  \begin{exe}
    \ex
    \begin{tabular}[t]{r >{\itshape}l}
      1·10\sup6\duo       & semműl \textup{(*\emph{senműl} )}\\
      2·10\sup6\duo       & ětiműl\\
      70·10\sup6\duo      & ikuşzaműl\\
      300·10\sup6\duo     & költocműl\\
      419,203,62A\duo     & qesetoc-vaudi-semműl ěttoc-kölsíva vontoc-ětiza-méri\\
      900,000,000,000\duo & joktocsívaműl\\
    \end{tabular}
  \end{exe}

  Using this system alone, it is possible to count up to BBB,BBB,BBB,BBB\duo, or 8,916,100,448,255\dec. Larger numerals, if needed, use a system of powers \emph{which I haven't thought of yet}.

  \section{Ordinals}
  \label{sec:num_ordinals}

  The ordinal numerals are formed by appending the suffix \textit{-ik} to the number word. For large numerals, the suffix is applied to the last word in the sequence. The ordinals from *0\sup{th} to 21\dec\sup{st} are given in Table~\ref{tab:num_ordinals}.

  \begin{table}[htpb]\small\capstart
    \begin{center}
      \subfloat{
        \begin{tabular}{|c|c|}
          \hline
          \multicolumn{2}{|c|}{\bfseries Ordinal} \tabularnewline
          \hline
          *0\sup{th} & *domik \tabularnewline
          1\sup{st} & senik   \tabularnewline
          2\sup{nd} & ětík    \tabularnewline
          3\sup{rd} & kölik   \tabularnewline
          4\sup{th} & qeseik  \tabularnewline
          5\sup{th} & inüik    \tabularnewline
          6\sup{th} & vonik   \tabularnewline
          7\sup{th} & ikuşik  \tabularnewline
          8\sup{th} & soprík  \tabularnewline
          9\sup{th} & jokik   \tabularnewline
          10\dec\sup{th} & mérik \tabularnewline
          \hline
        \end{tabular}}\qquad
      \subfloat{
        \begin{tabular}{|c|c|}
          \hline
          \multicolumn{2}{|c|}{\bfseries Ordinal} \tabularnewline
          \hline
          11\dec\sup{th} & türeik   \tabularnewline
          12\dec\sup{th} & żelik    \tabularnewline
          13\dec\sup{th} & scemik   \tabularnewline
          14\dec\sup{th} & zpetik   \tabularnewline
          15\dec\sup{th} & żekölik  \tabularnewline
          16\dec\sup{th} & ksożik   \tabularnewline
          17\dec\sup{th} & pedłaik  \tabularnewline
          18\dec\sup{th} & pseňaik  \tabularnewline
          19\dec\sup{th} & sevčaik  \tabularnewline
          20\dec\sup{th} & vaudík   \tabularnewline
          21\dec\sup{th} & vaudi-senik \tabularnewline
          \hline
        \end{tabular}}
      \caption{Ordinal numerals from 0\dec\ to 21\dec\label{tab:num_ordinals}}
    \end{center}
  \end{table}

  \section{Multiplicatives}
  \label{sec:num_multiplicatives}

  Numerals in Qevesa also have a special form for multiplicatives, formed by appending the suffix \textit{-zmi}. If the numeral stem ends in a consonant, and epenthetic vowel identical to the nucleus vowel of the previous syllable is inserted. The multiplicative numbers from 0\dec\ to 21\dec\ are listed in Table~\ref{tab:num_multiplicatives}.

  \begin{table}[htpb]\small\capstart
    \begin{center}
      \subfloat{
        \begin{tabular}{|c|c|}
          \hline
          \multicolumn{2}{|c|}{\bfseries Multiplicative} \tabularnewline
          \hline
          0× & domozmi	\tabularnewline
          1× & senezmi	\tabularnewline
          2× & ětízmi		\tabularnewline
          3× & kölözmi	\tabularnewline
          4× & qesezmi	\tabularnewline
          5× & inüzmi		\tabularnewline
          6× & vonozmi	\tabularnewline
          7× & ikuşuzmi	\tabularnewline
          8× & soprizmi	\tabularnewline
          9× & jokozmi	\tabularnewline
          10\dec × & mérizmi \tabularnewline
          \hline
        \end{tabular}}\qquad
      \subfloat{
        \begin{tabular}{|c|c|}
          \hline
          \multicolumn{2}{|c|}{\bfseries Multiplicative} \tabularnewline
          \hline
          11× & türezmi \tabularnewline
          12× & żelezmi \tabularnewline
          13× & scemezmi \tabularnewline
          14× & zpetezmi \tabularnewline
          15× & żekölözmi \tabularnewline
          16× & ksożozmi \tabularnewline
          17× & pedłazmi \tabularnewline
          18× & pseňazmi \tabularnewline
          19× & sevčazmi \tabularnewline
          20× & vaudizmi \tabularnewline
          21× & vaudi-senezmi \tabularnewline
          \hline
        \end{tabular}}
      \caption{Multiplicative numerals from 0\dec\ to 21\dec\label{tab:num_multiplicatives}}
    \end{center}
  \end{table}

  The multiplicative forms are used both in a repetitive and mathematical sense:

  \begin{exe}
    \ex \emph{EXAMPLES}
  \end{exe}

  \section{Fractions}
  \label{sec:num_fractions}

  Fractions are formed by appending the suffix \textit{-Vna} where \textit{V} is the nucleus vowel of the previous syllable — numerals ending in a vowel have this vowel lengthened instead. The fractional numbers from 0\dec\ to 21\dec\ are listed in Table~\ref{tab:num_cardinals}.

  \begin{table}[htpb]\small\capstart
    \begin{center}
      \subfloat{
        \begin{tabular}{|c|c|}
          \hline
          \multicolumn{2}{|c|}{\bfseries Fractional} \tabularnewline
          \hline
          *\sup1⁄\sub0 & *domona \tabularnewline
          *\sup1⁄\sub1 & *senena \tabularnewline
          \sup1⁄\sub2 & ětína    \tabularnewline
          \sup1⁄\sub3 & kölöna   \tabularnewline
          \sup1⁄\sub4 & qeséna   \tabularnewline
          \sup1⁄\sub5 & inűna    \tabularnewline
          \sup1⁄\sub6 & vonona   \tabularnewline
          \sup1⁄\sub7 & ikuşuna  \tabularnewline
          \sup1⁄\sub8 & soprína  \tabularnewline
          \sup1⁄\sub9 & jokona   \tabularnewline
          \sup1⁄\sub{10} & mérina \tabularnewline
          \hline
        \end{tabular}}\qquad
      \subfloat{
        \begin{tabular}{|c|c|}
          \hline
          \multicolumn{2}{|c|}{\bfseries Fractional} \tabularnewline
          \hline
          \sup1⁄\sub{11} & türéna \tabularnewline
          \sup1⁄\sub{12} & żelena \tabularnewline
          \sup1⁄\sub{13} & scemena \tabularnewline
          \sup1⁄\sub{14} & zpetena \tabularnewline
          \sup1⁄\sub{15} & żekölöna \tabularnewline
          \sup1⁄\sub{16} & ksożona \tabularnewline
          \sup1⁄\sub{17} & pedłána \tabularnewline
          \sup1⁄\sub{18} & pseňána \tabularnewline
          \sup1⁄\sub{19} & sevčána \tabularnewline
          \sup1⁄\sub{20} & vaudína \tabularnewline
          \sup1⁄\sub{21} & vaudi-senena \tabularnewline
          \hline
        \end{tabular}}
      \caption{Fractional numerals from 0\dec\ to 21\dec\label{tab:num_fractional}}
    \end{center}
  \end{table}

  \newpage
  The numerator of a fraction precedes the denominator as an additional modifier, for example:

  \begin{exe}
    \ex
    \begin{xlist}
      \ex \textit{ikuş żelena}
      \glll ikuş żel-ena\\
      seven twelve\textsc{-frac}\\
      seven twelfth\\
      \glt seven-twelfths
      \ex \textit{ěti kölöna litaseva}
      \glll ěti köl-öna litas-ev-a\\
      two three\textsc{-frac} bread\textsc{-du-nom}\\
      two third bread\\
      \glt two-thirds of bread
    \end{xlist}
  \end{exe}

  If the denominator of a fraction is a compound number, the fractional suffix is appended to the final word in the sequence:

  \begin{exe}
    \ex
    \begin{xlist}
      \ex \textit{kölvádana}
      \glll kölvád-ana\\
      sixty\textsc{-frac}\\
      sixtieth\\
      \glt (a) sixtieth
      \ex \textit{sopri ětiza-vonona}
      \glll sopri ěti-za=von-ona\\
      eight two=dozen\textsc{-frac}\\
      eight twenty-fourths\\
      \glt eight twenty-fourths
    \end{xlist}
  \end{exe}


  More complex fractions {\em are yet to be written about… in particular, I need:
    \begin{itemize}
      \item Integer ± unit fraction
      \item Integer × unit fraction
    \end{itemize}
  }


\end{document}

