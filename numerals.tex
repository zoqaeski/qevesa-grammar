\documentclass[grammar]{subfiles}
\begin{document}
  \chapter{Numerals}
  \label{ch:numerals}
  
  Numerals form a separate class in Qevesa, … The counting system is fundamentally duodecimal

  \begin{table}[h!]\small\capstart
    \begin{tabular}{Fr r -l}
      \hline
      \rowstyle{\bfseries} & & Cardinal \\
      \hline
      0\duo  & 0  & en     \\
      1\duo  & 1  & jara    \\
      2\duo  & 2  & vít    \\
      3\duo  & 3  & kor    \\
      4\duo  & 4  & qesa   \\
      5\duo  & 5  & peks  \\
      6\duo  & 6  & zusti    \\
      7\duo  & 7  & kuš   \\
      8\duo  & 8  & soppi  \\
      9\duo  & 9  & jukka  \\
      A\duo  & \textturntwo  & meži  \\
      B\duo  & \textturnthree  & tuva   \\
      10\duo & 10 & veša   \\
      \hline
    \end{tabular}
  \caption{Basic numerals\label{tab:num_basic}}
  \end{table}



%  Numerals from 11\duo\ to 2B\duo\ are suffixed with \Q{-zivsi}:
%
%    \begin{longtable}[l]{r Il}
%      11\duo & jaraziva    \\
%      12\duo & víteziva   \\
%      13\duo & koreziva   \\
%      14\duo & qesaziva  \\
%      15\duo & pesyziva \\
%      16\duo & zustiziva   \\
%      17\duo & kušeziva  \\
%      28\duo & soppiziva \\
%      29\duo & jukkaziva \\
%      2A\duo & mieriziva \\
%      2B\duo & túreziva  \\
%    \end{longtable}

  Numerals from 10\duo\ to B0\duo\ are suffixed with \Q{-vešy}:

    \begin{longtable}[l]{r Il}
      10\duo & javešy        \\
      20\duo & vítvešy       \\
      30\duo & korvešy       \\
      40\duo & qesavešy      \\
      50\duo & peksvešy     \\
      70\duo & kušvešy      \\
      A0\duo & mežavešy     \\
      BB\duo & tuvavešy-tuva \\
    \end{longtable}

  Numerals from 100\duo\ to B00\duo\ are suffixed with \Q{-tus}:

    \begin{longtable}[l]{r Il}
      100\duo & ertus                \\
      200\duo & víttus               \\
      300\duo & kortus               \\
      409\duo & qesetus-jukka        \\
      752\duo & kuštus-peksvešy-vít \\
    \end{longtable}

  %\newpage
  Numerals from 1000\duo\ to B000\duo\  use the suffix \Q{-mazi}:

    \begin{longtable}[l]{r Il}
      1000\duo    & ermazi                                       \\
      2000\duo    & vítmazi                                      \\
      4000\duo    & qesemazi                                     \\
      8603\duo    & soppimazi-zustitus-kor                         \\
      10,000\duo  & vešamazi                                     \\
      17,029\duo  & vešakušmazi-vítvešy-jukka                    \\
      50,000\duo  & pekstusmazi                                 \\
      93,487\duo  & jukkavešy-kormazi qesetus-soppivešy-kuš       \\
      100,000\duo & ertusmazi                                    \\
      582,196\duo & pekstus-soppivešy-vítmazi ertus-jukkavešy-zusti \\
    \end{longtable}

  %\newpage
%  Numerals from 10\sup6\duo\ to 10\sup{12}\duo−1 are formed by the addition of the suffix \Q{-múl}:
%
%  \begin{exe}
%    \ex
%    \begin{tabular}[t]{r Il}
%      1·10\sup6\duo       & ermúl\\
%      2·10\sup6\duo       & hetimúl\\
%      70·10\sup6\duo      & ikuššamúl\\
%      300·10\sup6\duo     & kortocmúl\\
%      419,203,52A\duo     & qesetoc-ševasoppimúl hettoc-korsavi pesyatoc-hetiša-mieri\\
%      900,000,000,000\duo & jukkatocsavimúl\\
%    \end{tabular}
%  \end{exe}
%
%  Using this system alone, it is possible to count up to 1BBB,BBB,BBB,BBB\duo, or 
%17,832,200,896,511\dec\footnotemark.
%  
%  \footnotetext{In full, this is \Q{ševatúretoc-túreša-túresavimúl túretoc-túreša-túremúl túretoc-túreša-túresavi túretoc-túreša-túre}}

%  \section{Ordinals}
%  \label{sec:num_ordinals}
%
%  The ordinal numerals are formed by appending the suffix \Q{-(i)k} to the number word.  For large numerals, the suffix is applied to the last word in the sequence.  The ordinals from *0\sup{th} to 23\dec\sup{st} are given in Table~\ref{tab:num_ordinals}.
%
%  \begin{table}[h!]\small\capstart
%      \subfloat{
%        \begin{tabular}{c c}
%          \hline
%          \multicolumn{2}{c}{\bfseries Ordinal} \\
%          \hline
%          0 & naxik		\\
%          1 & erk		\\
%          2 & hetik		\\
%          3 & korok	\\
%          4 & qesek	\\
%          5 & pesyak	\\
%          6 & zustik	  \\
%          7 & ikušik	\\
%          8 & soppik	\\
%          9 & jukkak		\\
%          10\dec & mierik	\\
%          11\dec & túreik	 \\
%          \hline
%        \end{tabular}}\qquad
%      \subfloat{
%        \begin{tabular}{Fc -l}
%          \hline
%          \multicolumn{2}{fc}{\rowstyle{\bfseries}Ordinal} \\
%          \hline
%          12\dec & ševak	      \\
%          13\dec & ševaerk	  \\
%          14\dec & ševahetik	  \\
%          15\dec & ševakorok	  \\
%          16\dec & ševaqesek	  \\
%          17\dec & ševapesyak   \\
%          18\dec & ševazustik	  \\
%          19\dec & ševaikušik	  \\
%          20\dec & ševasoppik   \\
%          21\dec & ševajukkak  \\
%          22\dec & ševamierik    \\
%          23\dec & ševatúrek   \\
%          \hline
%        \end{tabular}}
%      \caption{Ordinal numerals from 0\dec\ to 23\dec\label{tab:num_ordinals}}
%  \end{table}
%
%  \section{Multiplicatives}
%  \label{sec:num_multiplicatives}
%
%  Numerals in Qevesa also have a special form for multiplicatives, formed by appending the suffix \Q{-mi}.  If the numeral stem ends in a consonant, an epenthetic vowel identical to the nucleus vowel of the previous syllable is inserted.  The multiplicative numbers from 0\dec\ to 23\dec\ are listed in Table~\ref{tab:num_multiplicatives}.
%
%  \begin{table}[h!]\small\capstart
%      \subfloat{
%        \begin{tabular}{c c}
%          \hline
%          \multicolumn{2}{c}{\bfseries Multiplicative} \\
%          \hline
%          0× & naxami	\\
%          1× & ermi	\\
%          2× & hetimi		\\
%          3× & koromi	\\
%          4× & qesemi	\\
%          5× & pesyami	\\
%          6× & zustimi	\\
%          7× & ikušumi	\\
%          8× & soppimi	\\
%          9× & jukkami	\\
%          10\dec × & mierimi \\
%          11\dec × & túremi \\
%          \hline
%        \end{tabular}}\qquad
%      \subfloat{
%        \begin{tabular}{c c}
%          \hline
%          \multicolumn{2}{c}{\bfseries Multiplicative} \\
%          \hline
%          12× & ševami     \\
%          13× & ševaermi	\\
%          14× & ševahetimi		\\
%          15× & ševakoromi	\\
%          16× & ševaqesemi	\\
%          17× & ševapesyami	\\
%          18× & ševazustimi	\\
%          19× & ševaikušumi	\\
%          20× & ševasoppimi	\\
%          21× & ševajukkami	\\
%          22× & ševamierimi	\\
%          23× & ševatúremi	\\
%          \hline
%        \end{tabular}}
%      \caption{Multiplicative numerals from 0\dec\ to 23\dec\label{tab:num_multiplicatives}}
%  \end{table}
%
%  The multiplicative forms are used both in a repetitive and mathematical erse:
%
%  \begin{exe}
%    \ex \emph{EXAMPLES}
%  \end{exe}
%
%  \section{Fractions}
%  \label{sec:num_fractions}
%
%  Fractions are formed by appending the suffix \Q{-Vna} where \textit{V} is the nucleus vowel of the previous syllable.  The fractional numbers from 0\dec\ to 21\dec\ are listed in Table~\ref{tab:num_cardinals}.
%
%  \begin{table}[h!]\small\capstart
%      \subfloat{
%        \begin{tabular}{c c}
%          \hline
%          \multicolumn{2}{c}{\bfseries Fractional} \\
%          \hline
%          *\sup1⁄\sub{0} & *naxana	\\
%          *\sup1⁄\sub{1} & *erna	\\
%          \sup1⁄\sub{2} & hetina		\\
%          \sup1⁄\sub{3} & korna	\\
%          \sup1⁄\sub{4} & qesena	\\
%          \sup1⁄\sub{5} & pesyna	\\
%          \sup1⁄\sub{6} & zustina	\\
%          \sup1⁄\sub{7} & ikušuna	\\
%          \sup1⁄\sub{8} & soppina	\\
%          \sup1⁄\sub{9} & jukkana	\\
%          \sup1⁄\sub{10} & mierina \\
%          \sup1⁄\sub{11} & túrena \\
%          \hline
%        \end{tabular}}\qquad
%      \subfloat{
%        \begin{tabular}{c c}
%          \hline
%          \multicolumn{2}{c}{\bfseries Fractional} \\
%          \hline
%          \sup1⁄\sub{12} & ševana     \\
%          \sup1⁄\sub{13} & ševaerna	\\
%          \sup1⁄\sub{14} & ševahetina		\\
%          \sup1⁄\sub{15} & ševakorna	\\
%          \sup1⁄\sub{16} & ševaqesena	\\
%          \sup1⁄\sub{18} & ševapesyna	\\
%          \sup1⁄\sub{17} & ševazustina	\\
%          \sup1⁄\sub{19} & ševaikušuna	\\
%          \sup1⁄\sub{20} & ševasoppina	\\
%          \sup1⁄\sub{21} & ševajukkana	\\
%          \sup1⁄\sub{22} & ševamierina	\\
%          \sup1⁄\sub{23} & ševatúrena	\\
%          \hline
%        \end{tabular}}
%      \caption{Fractional numerals from 0\dec\ to 23\dec\label{tab:num_fractional}}
%  \end{table}
%
%  \newpage
%  The numerator of a fraction precedes the denominator and is in the ordinal form:
%
%  \begin{exe}
%    \ex
%    \begin{xlist}
%      \ex \Q{ikušik ševana}
%      \glll ikuš-ik ševa-na\\
%      seven-\acs{ord} twelve-\acs{frac}\\
%      seven twelfth\\
%      \glt seven-twelfths
%      \ex \Q{hetik korna litasevok}
%      \glll het-ik kor-na litas-ev-ok\\
%      two-\acs{ord} three-\acs{frac} bread-\acs{du}-\acs{gen}\\
%      two third bread\\
%      \glt two-thirds of bread
%    \end{xlist}
%  \end{exe}
%
%  If the denominator of a fraction is a compound number, the fractional suffix is appended to the final word in the sequence:
%
%  \begin{exe}
%    \ex
%    \begin{xlist}
%      \ex \Q{zustišana}
%      \glll zustiša-na\\
%      sixty-\acs{frac}\\
%      sixtieth\\
%      \glt (a) sixtieth
%      \ex \Q{soppík hetišana}
%      \glll soppi-ik heti-ša-na\\
%      eight-\acs{ord} two-dozen-\acs{frac}\\
%      eight twenty-fourths\\
%      \glt eight twenty-fourths
%    \end{xlist}
%  \end{exe}
%
%
%  More complex fractions {\em are yet to be written about… in particular, I need:
%    \begin{itemize}
%      \item Integer ± unit fraction
%      \item Integer × unit fraction
%    \end{itemize}
%  }


\end{document}

