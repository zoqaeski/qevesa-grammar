\documentclass[grammar]{subfiles}
\begin{document}
  \chapter{Phonology}
  \label{ch:phonology}

  \section{Phonotactics}
  \label{sec:phonotactics}

  \subsection{Vowel inventory}
  \label{ssec:vowels}

  \begin{table}[htpb]\small\capstart
        \begin{tabular}{>{\bfseries}Fl -c -c -c}
          \toprule
          \SetRowStyle{\bfseries} & Front & Central & Back \\
          \midrule
          Close & i iː &      & u uː \\% ʉ ʉː \\
          Mid   & e eː &      & o oː \\
          Open  &      & a aː & \\\hline
        \end{tabular}
      \caption{Qevesa vowel phonemes\label{tab:vowels}}
  \end{table}


  There are ten distinct vowel phonemes in Qevesa, listed in
  Table~\ref{tab:vowels}.  These are divided into five long and five short
  phonemes, differing in length but not quality.  Long vowels are held
  approximately twice as long as their short counterparts.
  
  Although the vowels [e] and [o] are conventionally written using the
  close-mid \textsc{ipa} symbols, they are more accurately transcribed as mid
  vowels [\textlowering{e}] and [\textlowering{o}].  In contrast to the
  consonants, the vowels show very little variation.

  The diphthongs are /i-/ glides /ia ie io iu/ and /u-/ glides /ua ue ui uo/,
  with assimilation of /ii/ and /uu/ to /iː/ and /uː/.  /i-/ glides tend to
  cause palatalisation, and /u-/ glides may cause labialisation, but this is
  dialect-dependent, with palatalisation being far more common.  /-u/ offglides

  \subsection{Consonant inventory}
  \label{ssec:consonants}

  \begin{table}[htpb]\small\capstart
      \begin{tabular}{>{\bfseries}Fl -c -c -c -c -c -c -c}
        \toprule
        \SetRowStyle{\bfseries} & Bilabial & Labiodental & Denti-alveolar & Postalveolar & Palatal & Velar & Glottal \\
        \midrule
        Nasal       & m &     & \tsb{n}                     &    & ɲ & [ŋ] \\
        Plosive     & p &     & \tsb{t}                     &    & c & k  \\ 
        Affricate   &   &     & ts dz & tʃ &   & \\
        Fricative   &   & f v & s θ ð                 & ʃ  & [ç] & x  & h \\
        Approximant &   &     &                                       &    & j \\
        Lateral     &   &     & l                     &    &   \\
        Rhotic      &   &     & r                                     &    &   \\
        \bottomrule
      \end{tabular}
      \caption{Consonants\label{tab:consonants}}
  \end{table}

  Qevesa possesses twenty-two consonants, realised as in
  Table~\ref{tab:consonants}.  Features and allophones of each row are
  described in more detail below.  Consonants are slightly palatalised before
  /i/ (and its associated glides).

  Consonantal length is phonemic, so [mata] and [matːa] are distinguished.  In
  correct speech, geminate consonants should be articulated and released
  separately, although in quick speech they will be pronounced as prolonged.
  Geminates may also appear at in word-initial syllables, but are rare
  word-finally.  Word-medially, syllables will be split at the geminate
  consonant.

  \subsubsection{Nasals}
  \label{sssec:nasals}

  Qevesa has three nasal consonants: /m \textsubbridge{n} ɲ/.
  /\textsubbridge{n}/ is a laminal denti-alveolar nasal, rather than a true
  dental nasal.  These consonants are largely consistent in their realisation,
  though they may assimilate to the articulation point of adjacent plosives in
  clusters. 

  The velar nasal [ŋ] is an allophone of /\textsubbridge{n} ɲ/ before /k/.

  \subsubsection{Plosives}
  \label{sssec:plosives}

  Qevesa has four plosive consonants.  These are spread over four positions
  (labial, denti-alveolar, palatal, velar); voice is not distinguished: /p
  \textsubbridge{t} c k/.  The plosives are often realised with a slight
  aspiration when syllable-final; /c/ may become an affricate [cç].  
  
  %The plosive consonants may be palatalised to [p\superj{} \textsubbridge{t\superj} cç k\superj].

  \subsubsection{Fricatives}
  \label{sssec:fricatives}

  Qevesa has eight fricative consonants: /f v θ ð \textsubbridge{s} ʃ x h/.
  /v/ and /ð/ are commonly realised as approximants. Before /i/ or /j/, /x/ and
  /h/ may be realised as [ç].
%  Palatalisation affects the fricatives in a variety of way: 
% 
%  \begin{itemize*}
%    \item /f/ palatalises to [f\superj];
%    \item /θ/ palatalises to [θ\superj];
%    %\item /θ/ palatalises to [\textsubbridge{s\superj}];
%    \item /s/ palatalises to [s\superj]
%    \item /ʃ/ palatalise to [ɕ];
%    \item /ʋ/ and /ð/ reduce to [j]; and,
%    \item /x/ palatalises to [ç];
%  \end{itemize*}

  \subsubsection{Affricates}
  \label{sssec:affricates}

  Qevesa has three affricates: /\textsubbridge{t}\textsubbridge{s} tʃ dz/.
  /ts/ and /tʃ/ are consistently realised as affricates and behave as though
  they were a single consonant. /dz/ may be realised as a fricative when word
  initial or preceded by another non-fricative consonant.

  %The affricates are affected by palatalisation in a similar manner to the fricatives.

  \subsubsection{Liquids and Glides}
  \label{sssec:liquids}

  Qevesa has two liquid consonants (one lateral and one rhotic) and two to four glides.

  The lateral consonant is the denti-alveolar /\textsubbridge{l}/. When
  preceding an /i-/ glide or /j/, it is realised as [ʎ].  %An allophone of /l/
  is [ɬ] that occurs only in some clusters, such as /tl/.

  The rhotic consonant is the alveolar trill /r/.  It may be realised as a tap [ɾ] when intervocalic.  
  %Palatalised /r/ is realised as [ɾj].

  The glide is the palatal glide /j/.  This shows little allophonic variation,
  tending to induce allophonic changes in other consonants.  The fricatives /v/
  and /ð/ are often realised as approximants.  

  \subsection{Phonemic Restrictions}
  \label{ssec:phonemic_restrictions}

  The main limitations on phonemic distribution are found within the context of
  consonant clusters.  Any single consonant may appear in onset or coda
  position, word-initially, word-medially, or word-finally.  Likewise, any
  vowel may occur in any of the three positions. 

  \subsubsection{Consonant Clusters}
  \label{sssec:consonant_clusters}

  Qevesa is fairly lenient when it comes to word-internal clusters.  Almost any
  combination is permitted, including clusters containing two consonants having
  the same point of articulation. 

%  \begin{itemize*}
%    \item /x/-final clusters result in a [kx] affricate developing
%    \item /h/-initial clusters result in lengthening of the previous vowel and a slight pre-aspiration of the following consonant
%    \item /h/-final clusters are realised as simple clusters; some dialects insert a glottal stop between the initial consonant and the /h/
%  \end{itemize*}

  Initial consonant clusters are not permitted, except for palatal and labial offglides.

%  \begin{itemize*}
%    \item Any non-palatal plosive or /f v/ + /r l/: /pr tr kr fr vr pl tl kl fl vl/ 
%    \item /θ s ʃ/ + a plosive or /m n/: /θp θt θk θm θn sp st sk sm sn ʃp ʃt ʃk ʃm ʃn/
%    \item /θ s ʃ/ + /l/: /θl sl ʃl/
%    \item A fricative + affricate at the same point of articulation: /θtθ sts ʃtʃ/
%    \item Any non-palatal plosive + /θ s ʃ/: /pθ tθ kθ ps ts ks pʃ tʃ kʃ/.  Note that affricates contrast with plosive-fricative sequences.
%    \item Any non-palatal plosive, fricative, or affricate + /f\textasciitilde v/: /pf tf kf θf sf ʃf/.  Note that the labiodental fricative may vary between [f] [v] and [v], regardless of its orthographical representation.
%    \item Any non-palatal plosive or fricative + /w/: /pw tw kw fw θw sw ʃw/
%    \item Any consonant + /i-/ or /u-/ glide.  Note that /i-/ glides and /j/ induce palatalisation of the previous phoneme, according to the allophonic rules described in Sections \ref{sssec:nasals}–\ref{sssec:liquids}, and that /u-/ glides often assimilate to /w/
%    \item /mn mɲ/
%  \end{itemize*}
%
%  Syllable-final clusters are even more restricted than syllable-initial ones:
%
%  \begin{itemize*}
%    %\item A nasal, a plosive, or /r l j/ + /f θ s ʃ/
%    \item /r l w/ + a plosive or /f θ s ʃ/: /rp rt rk rf rs rʃ lp lt lk lf ls lʃ wp wt wk wf ws wʃ/
%    \item a nasal or plosive + /f θ s ʃ/: /mf mθ ms mʃ nf nθ ns nʃ pf pθ ps pʃ tf tθ ts tʃ kf kθ ks kʃ/
%    \item /f θ s ʃ/ + a non-palatal plosive: /fp ft fk θp θt θk sp st sk ʃp ʃt ʃk/
%    \item A fricative + affricate at the same point of articulation: /θtθ sts ʃtʃ/
%    \item /n/ + /t k/: /nt nk/
%    \item /mp/
%  \end{itemize*}
%
%  Though there are a large number of permissable consonant clusters, their actual occurrence is fairly infrequent. 
%  Syllable-initial or syllable-final clusters are to be avoided word-internally: VCCV will always be split into VC.CV\@. 
%  Clusters of three or more consonants are only permitted across syllable breaks, and will always be split to favour an initial cluster over a final one.

  \subsubsection{Syllable Structure}
  \label{sssec:syllables}

  Qevesa syllables are strictly CV(C).

  \ToBeWritten
  %Although a wide variety of initial consonant clusters are permitted, they should be avoided when dividing a word into syllables.  The general rule is that non-word-final consonants are always the onset of syllables unless followed by another consonant or permissable initial cluster.

  \newpage

  \subsection{Romanisation}
  \label{ssec:romanisation}

  The usual transcription system used for the Latin alphabet is as follows:

  \begin{center}
    \begin{tabularx}{0.9 \textwidth}{fC*{8}{-C}}
      \SetRowStyle{\bfseries} A a & Á á  & C c   & Č č  & D d   & E e & É é  & H h \\
                              /a/ & /aː/ & /ts/  & /tʃ/ & /ð/   & /e/ & /eː/ & /h/ \\
      \SetRowStyle{\bfseries} I i & Í í  & J j   & K k  & KH kh & L l & M m  & N n \\
                              /i/ & /iː/ & /j/   & /k/  & /x/   & /l/ & /m/  & /n/ \\
      \SetRowStyle{\bfseries} Ň ň & O o  & Ó ó   & P p  & PH ph & Q q & R r  & S s \\
                              /ɲ/ & /o/  & /oː/  & /p/  & /f/   & /c/ & /r/  & /s/ \\
      \SetRowStyle{\bfseries} Š š & T t  & TH th & U u  & Ú ú   & V v & Z z \\
                              /ʃ/ & /t/  & /θ/   & /u/  & /uː/  & /v/ & /z dz/ \\
    \end{tabularx}
    %\caption[Romanisation of Qevesa]{\label{tab:transcription}}
  \end{center}

  %  a  á  c   č  d   e  é  h
  %  i  í  j   k  kh  l  m  n
  %  ň  o  ó   p  ph  q  r  s
  %  š  t  th  u  ú   v  z

  %\pagebreak[2]
  The Latin orthography is largely phonemic, and makes use of a number of
  diacritics and digraphs.  The diacritics indicate the following features:

  \begin{description}
    \item[Háček/Caron] The \foreign{háček} or caron indicates a palatalised
      consonant variant.  It is used with ⟨c⟩, ⟨n⟩ and ⟨s⟩, producing ⟨č⟩, ⟨ň⟩
      and ⟨š⟩.  
    \item[Acute] The acute accent is used to indicate a long vowel, and is used
      with ⟨a⟩, ⟨e⟩, ⟨i⟩, ⟨o⟩ and ⟨u⟩ to produce ⟨á⟩, ⟨é⟩, ⟨í⟩, ⟨ó⟩ and ⟨ú⟩.  
  \end{description}

  The digraphs ⟨kh⟩, ⟨ph⟩ and ⟨th⟩ represent the phonemes /x/, /f/ and /θ/.
  These phonemes were originally pronounced as aspirated stops in Common
  Therasa, and became fricatives in Qevesa. The letters ⟨z⟩ represents
  the affricate and /dz/.

  Geminate consonants are doubled, except for the digraphs which only double the first consonant.  

%  Although the orthography is largely morphophonemic, a number of phonemes may be written in more than one way:

%  \begin{itemize*}
%    %\item Palatalisation is indicated by a following ⟨j⟩, an i-glide diphthong, or a \foreign{háček} above the vowel.  %/ji/ is represented with ⟨y⟩ or ⟨í⟩ and realised as /\superj iː/.
%    %\item /i-/glides and /j/ soften (palatalise) preceding consonants: therefore /ɕ/ /tɕ/ /\textltailn/ may be represented by ⟨š⟩ ⟨č⟩ ⟨ń⟩ or ⟨si-⟩ ⟨ci-⟩ ⟨ni-⟩ before another vowel.  /\superj e/ ⟨ě⟩ will also cause palatalisation in this manner.
%    %\item /v/ may be realised as an approximant in some situations, and digraphs involving ⟨v⟩ or ⟨f⟩ such as ⟨sf⟩ or ⟨zv⟩ may result in the labiodental fricative being realised as anything between [f] [v] and [ʋ].
%    %\item /iː/ is usually written as ⟨í⟩ except when word-initial, when it is written as ⟨y⟩.
%  \end{itemize*}

  \section{Prosody}
  \label{sec:prosody}

  Qevesa is a syllable-timed language.
  \ToBeWritten

  \subsection{Stress}
  \label{ssec:stress}

  Stress always falls on the penultimate syllable of a word. 
  \ToBeWritten

  \subsection{Intonation}
  \label{ssec:intonation}

  Qevesa possesses a limited pitch-accent.
  \ToBeWritten

\end{document}
