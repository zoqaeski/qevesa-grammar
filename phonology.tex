\documentclass[grammar]{subfiles}
\begin{document}
	\chapter{Phonology}
	\label{ch:phonology}

	\section{Phonotactics}
	\label{sec:phonotactics}

	\subsection{Vowel inventory}
	\label{ssec:vowels}

	\begin{table}[htpb]\small\capstart
		\begin{center}
			\begin{tabular}{|>{\bfseries}fc| c c c c c c|}
				\hline
				\SetRowStyle{\bfseries} & \multicolumn{2}{-c|}{Front} & \multicolumn{2}{-c|}{Central} & \multicolumn{2}{-c|}{Back} \\\hline
				Close & i & y & & & & u \\
				Near-Mid & \superj e\textasciitilde je & \multirow{2}{*}{ø} & & & & \\
				Mid & e & & & & & o \\
				%Near-Open & \multirow{2}{*}{æ} & & & & &  \\
				Open & & & a & & & \\\hline
			\end{tabular}
			\caption{Short vowels\label{tab:vowels}}
		\end{center}
	\end{table}

	Qevesa possesses seven to eight distinct vowels, listed in Table~\ref{tab:vowels}. Although the vowels [e], [ø] and [o] are conventionally written using the close-mid \textsc{ipa} symbols, they are more accurately transcribed as mid vowels [\textlowering{e}], [\textlowering{ø}] and [\textlowering{o}]. In contrast to the consonants, the vowels show very little variation.

	The diphthongs are /ai au ei oi ou øi øy/, as well as /i-/ and /u-/ glides. /u-/ glides may cause labialisation, but this is dialect-dependent.

	A sound change in Proto-Teralo resulted in the appearance of palatal approximant /j/ before a syllable-initial vowel, particularly /e/. This phenomenon, known as iotation, resulted in the development of the phoneme /\superj e\textasciitilde je/. /i-/ glides were similarly affected, and the process induced palatalisation of the preceding consonant.

	Vowels also possess a phonemic length distinction. Each of the eight short vowels has a long equivalent; these are listed in Table~\ref{tab:long_vowels}. A long vowel should be approximately twice as long as a short vowel.

	\begin{table}[htpb]\small\capstart
		\begin{center}
			\begin{tabular}{|>{\bfseries}fc| c c c c c c|}
				\hline
				\SetRowStyle{\bfseries} & \multicolumn{2}{-c|}{Front} & \multicolumn{2}{-c|}{Central} & \multicolumn{2}{-c|}{Back} \\\hline
				Close & iː & yː & & & & uː \\
				Mid & eː & øː & & & & oː \\
				%Near-Open & \multirow{2}{*}{æː} & & & & &  \\
				Open & & & aː & & & \\\hline
			\end{tabular}
			\caption{Long vowels\label{tab:long_vowels}}
		\end{center}
	\end{table}

	Long vowels are also formed through collision of two identical vowels due to morphological marking.

	\subsection{Consonant inventory}
	\label{ssec:consonants}

	\begin{table}[htpb]\small\capstart
		\begin{center}
			\begin{tabular}{|>{\bfseries}fc|c c|c c|c c|c c|c c|c c|c c|c c|c c|}
				\hline
				\SetRowStyle{\bfseries} & \multicolumn{2}{-c|}{Bilabial} & \multicolumn{2}{-c|}{Labiodental} & \multicolumn{2}{-c|}{Dental} & \multicolumn{2}{-c|}{Alveolar} & \multicolumn{2}{-c|}{Retroflex} & \multicolumn{2}{-c|}{Palatal} & \multicolumn{2}{-c|}{Velar} & \multicolumn{2}{-c|}{Glottal} \tabularnewline\hline
				Nasal & & m & & & & \textsubbridge{n} & & & & & & ɲ & & & & \tabularnewline%\hline
				Plosive & p & & & & \textsubbridge{t} & & & & & & & & k & & & \tabularnewline%\hline
				Affricate & & & & & \multicolumn{2}{c|}{\textsubbridge{t}θ} & ts & & ʈʂ & & tɕ & & & & & \tabularnewline%\hline
				Fricative & & & f & \multirow{2}{*}{ʋ} & θ & \multirow{2}{*}{ð} & s & & ʂ & & ɕ & & x & & h & \tabularnewline%\cline{1-4} \cline{6-6} \cline{8-17}
				Approximant & & & \multicolumn{1}{r}{} & & \multicolumn{1}{r}{} & & & & & & & j & & w & & \tabularnewline%\hline
				Lateral & & & & & & & & l & & & & & & & & \tabularnewline%\hline
				Rhotic & & & & & & & & r\textasciitilde\textfishhookr & & & & & & & & \tabularnewline\hline
			\end{tabular}
			\caption{Consonants\label{tab:consonants}}
		\end{center}
	\end{table}

	Qevesa possesses twenty-three consonants, realised as in Table~\ref{tab:consonants}. Features and allophones of each row are described in more detail below. Palatalisation is allophonic and only occurs before iotated vowels (often /i-/ glides) and /-j/.

	Consonantal length is phonemic, so [mata] and [matːa] are distinguished. In correct speech, geminate consonants should be articulated and released separately, although in quick speech they will be pronounced as prolonged. Geminates may also appear at in word-initial syllables, but are rare word-finally. Word-medially, syllables will be split at the geminate consonant.

	\subsubsection{Nasals}
	\label{sssec:nasals}

	Qevesa has three nasal consonants: /m \textsubbridge{n} ɲ/. /\textsubbridge{n}/ is a laminal denti-alveolar nasal, rather than a true dental nasal. The palatal nasal /ɲ/ is realised as an alveolo-palatal nasal in virtually all dialects; as this does not contrast with a true palatal nasal, it is transcribed using the \textsc{ipa} symbol for a palatal nasal, rather than the obsolete and non-standard /\textctn/.

	These consonants are largely consistent in their realisation; however, /\textsubbridge{n}/ may be palatalised to /ɲ/. The velar nasal [ŋ] is an allophone of /\textsubbridge{n} ɲ/ before /k/.

	\subsubsection{Plosives}
	\label{sssec:plosives}

	Qevesa has three plosive consonants. These are spread over three positions (labial, denti-alveolar, velar); voice is not distinguished: /p \textsubbridge{t} k/. A fourth plosive (palatal, [c]) exists in marginal dialects, although this has since merged with the alveolo-palatal affricate in the standard dialect.

	The plosive consonants may be palatalised to [p\superj{} \textsubbridge{t\superj} k\superj]. In most dialects, the plosives are aspirated in an initial and word-final position, as [p\superh{} \textsubbridge{t\superh} k\superh].

	\subsubsection{Fricatives}
	\label{sssec:fricatives}

	Qevesa has nine fricative consonants: /f ʋ θ ð s ʂ ɕ x h/. /ʋ/ and /ð/ are commonly realised as approximants. Palatalisation affects the fricatives in a variety of way: 

	\begin{itemize*}
	\item /f/ palatalises to [f\superj{}];
	\item /s/ palatalises to [ɕ] instead of /s\superj/;
	\item /θ/ to /\textsubbridge{s\superj}/;
	\item /x h/ palatalises to [ç];
	\item /ʋ/ and /ð/ reduce to [j]; and,
	\item /ɕ/ and /ʂ/ are not affected by palatalisation.
	\end{itemize*}

	\subsubsection{Affricates}
	\label{sssec:affricates}

	Qevesa has four affricates: /tθ ts ʈʂ tɕ/. Affricates at other points of articulation are attested in historical texts, but these have since merged with the fricatives in the modern dialects. All of these behave as though they were a single consonant, and so should be represented with a tie-bar ligature; for simplicity this will not be done here, except if necessary to contrast the affricates from sequences of distinct phonemes. 

	The affricates are affected by palatalisation in a similar manner to the fricatives: /ts/ palatalises to /tɕ/ and /tθ/ to /t\textsubbridge{s\superj}/.

	\subsubsection{Liquids and Glides}
	\label{sssec:liquids}

	Qevesa has two liquid consonants (one lateral and one rhotic) and two to four glides.

	The lateral consonant is the alveolar /l/. It is often pronounced with a slight palatalisation, as [l\superj]; when preceding an iotated vowel, /i-/ glide or /j/, it weakens to [j]. A velarised lateral /ɫ/ formerly existed, but this has weakened to /w/ in the majority of dialects. An allophone of /l/ is [ɬ] that occurs only in some clusters, such as /tl/ and occasionally /ʂl/.

	The rhotic consonant is the denti-alveolar trill /r/. It may be realised as a tap [ɾ] when initial or intervocalic. Palatalisation weakens /r/ to [\textsubring{ɾ}j], that is, a palatalised devoiced alveolar tap.

	The two glides are the palatal glide /j/ and labiovelar glide /w/. These show little allophonic variation, tending to induce allophonic changes in other consonants. The fricatives /v/ and /ð/ are often realised as approximants, and in some dialects /ʋ/ and /w/ are merging into [w]. An unusual feature of the palatal glide is that word-finally it is realised as an unvoiced front unrounded vowel that palatalises the previous consonant: [\textsubring{\superj i}].

	\subsection{Phonemic Restrictions}
	\label{ssec:phonemic_restrictions}

	The main limitations on phonemic distribution are found within the context of consonant clusters. Any single consonant may appear in onset or coda position, word-initially, word-medially, or word-finally. Likewise, any vowel may occur in any of the three positions. 

	\subsubsection{Consonant Clusters}
	\label{sssec:consonant_clusters}

	Qevesa is fairly lenient when it comes to word-internal clusters. Almost any combination is permitted, including clusters containing two consonants having the same point of articulation. A limited amount of assimilation will occur: voicing always assimilates to the initial consonant, and pairs of sibilant fricatives (including affricate-initial clusters) assimilate to the point of articulation of the final consonant. Notable exceptions include clusters involving /x/ and /h/: 

	\begin{itemize*}
	\item /x/-final clusters result in a [kx] affricate developing
	\item /h/-initial clusters result in lengthening of the previous vowel and a slight pre-aspiration of the following consonant
	\item /h/-final clusters are realised as simple clusters; some dialects insert a glottal stop between the initial consonant and the /h/
	\end{itemize*}

	Initial consonant clusters are much more restricted. Only the following combinations are permissable:

	\begin{itemize*}
	\item Any non-palatal plosive or /f v/ + /r l/: /pr tr kr fr vr pl tl kl fl vl/ 
	\item /θ ʂ ɕ/ + a plosive or /m n/: /θp θt θk θm θn sp st sk sm sn ʂp ʂt ʂk ʂm ʂn ɕp ɕt ɕk ɕm ɕn/
	\item /θ s ʂ/ + /l/: /θl sl ʂl/
	\item A fricative + affricate at the same point of articulation: /θtθ sts ʂʈʂ ɕtɕ/
	\item Any non-palatal plosive + /θ s ʂ ɕ/: /pθ tθ kθ ps ts ks pʂ ʈʂ kʂ pɕ tɕ kɕ/. Note that affricates contrast with plosive-fricative sequences.
	\item Any non-palatal plosive, fricative, or affricate + /f\textasciitilde ʋ/: /pf tf kf θf sf ʂf ɕf/. Note that the labiodental fricative may vary between [f] [v] and [ʋ], regardless of its othographical representation.
	\item Any non-palatal plosive or fricative + /w/: /pw tw kw fw θw sw ʂw ɕw/
	\item Any consonant + /i-/ or /u-/ glide. Note that /i-/ glides and /j/ induce palatalisation of the previous phoneme, according to the allophonic rules described in Sections \ref{sssec:nasals}–\ref{sssec:liquids}, and that /u-/ glides often assimilate to /w/
	\item /mn mɲ/
	\end{itemize*}

	Syllable-final clusters are even more restricted than syllable-initial ones:

	\begin{itemize*}
	\item /r l/ + a plosive or /f θ s ʂ/: /rp rt rk rf rθ rs rʂ lp lt lp lf ls lʂ/
	\item a plosive + /f θ s ʂ ɕ/: /pf tf kf pθ tθ kθ ps ts ks pʂ ʈʂ kʂ pɕ tɕ kɕ/
	\item /f θ s ʂ ɕ/ + a plosive: /fp ft fk θp θt θk sp st sk ʂp ʂt ʂk ɕp ɕt ɕk/
	\item /n/ + /t k/: /nt nk/
	\item /mp/
	\end{itemize*}

	Note that syllable-initial or syllable-final clusters are to be avoided word-internally—VCCV will always be split into VC.CV\@. Clusters of three consonants are only permitted across syllable breaks, and will always be split to favour an initial cluster over a final one, provided that the combination of consonants is permitted. Such clusters are extremely rare, however.

	Clusters of four or more consonants are never permitted—clusters in loan words will always be simplified or reduced.
	
	\subsubsection{Syllable Structure}
	\label{sssec:syllables}

	Although a wide variety of initial consonant clusters are permitted, they should be avoided when dividing a word into syllables. The general rule is that non-word-final consonants are always the onset of syllables unless followed by another consonant or permissable initial cluster.

	\subsection{Romanisation}
	\label{ssec:romanisation}

	The usual transcription system used for the Latin alphabet is as follows:

	\begin{center}
		\begin{tabularx}{0.9 \textwidth}{fC*{12}{-C}}
			\SetRowStyle{\bfseries} A a & Á á & C c & Ç ç & Č č & D d & E e & É é & Ě ě & F f & H h & I i\\
			/a/ & /aː/ & /ts/ & /ʈʂ/ & /tɕ/ & /ð/ & /e/ & /eː/ & /\superj e/ & /f/ & /h/ & /i/\\		
			\SetRowStyle{\bfseries} Í í & K k & L l & Ł ł & M m & N n & Ň ň & O o & Ó ó & Ö ö & Ő ő & P p \\
			/iː/ &	/k/ & /l/ & /w/ & /m/ & /n/ & /ɲ/ &	/o/ & /oː/ & /ø/ & /øː/ & /p/ \\ 
			\SetRowStyle{\bfseries} Q q & R r & S s & Ş ş & Š š & T t & U u & Ú ú & Ü ü & Ű ű & V v & X x \\
			/tɕ/ & /r/ & /s/ & /ʂ/& /ɕ/ & /t/ & /u/ & /uː/ & /y/ & /yː/ & /ʋ/ & /x/\\
			\SetRowStyle{\bfseries} Y y & Z z & Ż ż\\
			/j/ & /θ/ & /tθ/\\
		\end{tabularx}
		%\caption[Romanisation of Qevesa]{\label{tab:transcription}}
	\end{center}

	\pagebreak[2]
	The orthography makes use of a number of diacritics. The diacritics on consonants indicate the following features:

	\begin{description}
		\item[Cedilla/Comma] The cedilla or comma indicates a retroflex variant, and is used with ‹s› and ‹c›, forming ‹ş› and ‹ç›. In handwritten texts, the comma is preferred, but typeset documents normally use the cedilla, due to a lack of typefaces that include the comma as a diacritic.
		\item[Háček/Caron] The \emph{háček} or caron indicates a palatalised consonant variant. It is used with ‹s› and ‹c›, producing ‹š› and ‹č›.
		\item[Dot above] This diacritic indicates an affricate varient of a fricative, and is only used with ‹z›, resulting in ‹ż›.
		\item[Stroke] The stroke is only used with ‹l›, to indicate the labiovelar approximant, or in some dialects, the velar lateral. Handwritten and stylistic forms normally place the stroke above the \emph l, to distinguish it from lowercase \emph t.
	\end{description}

	Vowels use a similar set of diacritics:

	\begin{description}
		\item[Trema/Umlaut] The trema or umlaut is used to indicate a fronted variant of ‹o› and ‹u›, forming ‹ö› and ‹ü›.
		\item[Háček/Caron]\label{def:hacek} The \emph{háček} or caron indicates an iotated or palatalised variant. It is most commonly used with ‹e› to produce ‹ě›, but may be used with other vowels. The ‹y-› spelling is used in some situations, such as across a syllable break or between two vowels (in which the inherent /j/ becomes the onset of the next syllable), so *‹aě› is written as ‹aye›. Generally, ‹ě› is preferred when following a consonant or as a nucleus vowel of a syllable, and ‹ye› is used when the /e/ is lengthened ‹yé›, but both representations are interchangeable.
		\item[Acute] The acute accent is used to indicate a long vowel, and is used with ‹a›, ‹e›, ‹i›, ‹o› and ‹u› to produce ‹á›, ‹é›, ‹í›, ‹ó› and ‹ú›. Long variants of ‹ö› and ‹ü› use a doubled acute, resulting in ‹ő› and ‹ű›. 
	\end{description}

	Although the orthography is largely morphophonemic, a number of phonemes may be written in more than one way:

	\begin{itemize*}
	\item	/tɕ/ is represented by both ‹q› and ‹č› due to a sound change that merged /c/→/tɕ/
	\item Consonantal allophones caused by palatalisation are not explicitly indicated (a \emph{háček} may be used above the following vowel if ambiguous)
		%\item /i-/glides and /j/ soften (palatalise) preceding consonants: therefore /ɕ/ /tɕ/ /\textltailn/ may be represented by ‹š› ‹č› ‹ń› or ‹si-› ‹ci-› ‹ni-› before another vowel. /\superj e/ ‹ě› will also cause palatalisation in this manner.
	\item /v/ may be realised as an approximant in some situations, and digraphs involving ‹v› or ‹f› such as ‹sf› or ‹zv› may result in the labiodental fricative being realised as anything between [f] [v] and [ʋ].
	\end{itemize*}

	\section{Prosody}
	\label{sec:prosody}

	\ToBeWritten

	\subsection{Stress}
	\label{ssec:stress}

	Stress falls on the first syllable of the root. \ToBeWritten

	\subsection{Intonation}
	\label{ssec:intonation}

	\ToBeWritten

\end{document}
