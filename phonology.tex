\documentclass[grammar]{subfiles}
\begin{document}
\chapter{Phonology}
\label{ch:phonology}


\section{Vowels}
\label{sec:vowels}

There are twelve distinct vowel phonemes in Qevesa, listed in \cref{tab:vowels}.
These are divided into six long and six short phonemes, differing in length
but not quality.  Long vowels are held approximately twice as long as their
short counterparts.

\begin{table}[h!]\small\capstart
  \begin{tabular}{BFl -c -c -c -c -c}
    \toprule
    \rowstyle{\bfseries} & \multicolumn{2}{-c}{Front} & \multicolumn{2}{-c}{Central} & Back \\
    \midrule
    Close & i iː & \multicolumn{2}{c}{y yː} & & u uː \\
    Mid   & \multicolumn{2}{c}{e eː} & & & o oː \\
    Open  & & & a aː \\
    \bottomrule
  \end{tabular}
  \caption{Qevesa vowel phonemes\label{tab:vowels}}
\end{table}

Although the vowels /e/ and /o/ are conventionally written using the close-mid
\textsc{ipa} symbols, they are more accurately transcribed as mid vowels
[e̞] and [o̞].

In addition to the plain vowels, there are eight diphthongs, /ai̯ ei̯ oi̯ ui̯ yi̯ au̯ eu̯ iu̯/.

\subsection{Allophones}
\label{ssec:vowel_allophones}

Stressed vowels show very little variation, with the exception that word
initially, the mid vowels /e/ and /o/ may acquire glides, becoming /je/ and
/wo/. 

Unstressed vowels tend to be reduced and often show a loss in quality:

\begin{itemize}
  \item The high vowels /i u/ tend to centralise towards [ɪ] and [ʊ].
  \item The high front rounded vowel /y/ loses its roundedness as well as
    centralises towards [ɨ].  This vowel is particularly prone to being reduced.
  \item The mid front vowel /e/ centralises towards [ə].
  \item The mid back vowel /o/ is less rounded, more open and also centralised
    to something between [ʌ\tlde ə].
  \item The open vowel /a/ centralises towards [ɐ].
\end{itemize}

Note that these allophones only occur with short vowels in medial or final
positions.  Long vowels are rarely unstressed, and when they aren’t the primary
stress in a word they are always pronounced clearly.

 
% \begin{table}[h]\small\capstart
%   \begin{tabular}{BFl -c -c -c -c}
%     \toprule
%     \rowstyle{\bfseries} & i- & -i & u- & -u \\
%     \midrule
%     a & ia & ai & ua & au \\
%     e & ie & ei & ue & eu \\
%     o & io & oi & uo & ou \\
%     i & iː & iː & ui & iu \\
%     u & iu & ui & uː & uː \\
%     \bottomrule
%   \end{tabular}
%   \caption{Qevesa diphthongs\label{tab:diphthongs}}
% \end{table}


\section{Consonants}
\label{sec:consonants}

\begin{table}[h!]\small\capstart
  \begin{tabular}{BFl -c -c -c -c -c -c}
    \toprule
    \rowstyle{\bfseries} & Labial & Dental/alveolar & Postalveolar & Palatal & Velar & Glottal \\
    \midrule
    Nasal       & m       & \tsb{n} &         & ɲ  \\
    Plosive     & p       & \tsb{t} &         & c            & k \\
    Fricative   & f v     & θ ð s z & ʂ ʐ     & ɕ [ʝ\tlde ʑ] & x & h [ɦ] \\
    Affricate   &         & ts [dz] & tʂ [dʐ] & [tɕ] \\
    Approximant & [w] [ʋ] & [ð̞ ɹ]   &         & j  \\
    Lateral     &         & l       \\
    Rhotic      &         & r       \\
    \bottomrule
  \end{tabular}
  \caption{Consonants\label{tab:consonants}}
\end{table}

Qevesa possesses twenty-three consonants, excluding allophones, which are
listed in \cref{tab:consonants}.  The features and allophones of each row are
described in more detail below.

Consonantal length is phonemic, so [mata] and [matːa] are distinguished.  In
correct speech, geminate consonants should be articulated and released
separately, although in quick speech they will be pronounced as prolonged.
Geminates may only occur in the middle or at the end of words.


\subsection{Nasals}
\label{ssec:nasals}

Qevesa has three nasal consonants: /m \tsb{n} ɲ/.  /\tsb{n}/ is a laminal
denti-alveolar nasal, rather than a true dental nasal.  These consonants are
largely consistent in their realisation, though they may assimilate to the
articulation point of adjacent plosives in clusters. 


\subsection{Plosives}
\label{ssec:plosives}

Qevesa has four plosive consonants, spread over four positions (labial,
denti-alveolar, palatal, velar): /p \tsb{t} c k/.  They are pronounced
unaspirated in all positions except word-finally, where they can acquire
a slight aspiration.

Before the stressed rounded vowels /o u y/, all plosives become slightly
labialised.

The exact realisation of the palatal consonant /c/ varies quite a bit.  [c] is
considered the most proper form, but a slight affricate often occurs when
syllable-final: [c\sup{ç}].  In some regional dialects [c] and the former
phoneme [cʰ] have completely merged into [tɕ] (in the standard dialect they
remain separate phonemes), and in regions where Qevesa is widely spoken as a
second language a palatalised velar [kʲ] is generally regarded as an acceptable
variant. 
%although [t] may also be heard, particularly by Cavasko speakers
%on the north coast plains near the border with Cavaskia.

It is very common for back vowels preceding [c] to acquire a slight offglide:
/ac/ → [a(ɪ̯)c].


\subsection{Fricatives and affricates}
\label{ssec:fricatives}

Qevesa has eleven fricative consonants: /f v θ ð s z ʂ ʐ ɕ x h/.  /v/ and /ð/
are commonly realised as approximants.  Before front vowels /x/ and /h/ may be
realised as [ç], and an intervocalic /h/ may be realised as /ɦ/. Unique amongst
the consonantal sounds, /h/ cannot be geminated—a long /h/ induces lengthening
of the previous vowel and is realised as /ɦ/.

The postalveolar fricatives /ʂ/ and /ʐ/ are realised as laminal retroflex
fricatives, and are transcribed as such.

There are four affricate consonants, /ts tʂ tɕ dz dʐ/, the latter three of which
are in free variation of the phonemes /ɕ z ʐ/.  They primarily occur in
geminates and (occasionally) when intervocalic.  The phoneme /ts/ only occurs
in loan words.


\subsection{Liquids and Glides}
\label{ssec:liquids}

Qevesa has two liquid consonants (one lateral and one rhotic) and two glides.

The lateral consonant is the denti-alveolar /l/.  When preceding front vowels
or /j/, it is often palatalised to [lʲ] and occasionally realised as [ʎ].
Conversely, when syllable-final—especially when following back vowels—it may be
realised as the “dark L” [ɫ].

The rhotic consonant is the alveolar trill /r/, which is often realised as the
tap [ɾ] between vowels.  Immediately adjacent to /s ʂ ʐ/, /r/ is usually
realised as an approximant [ɹ], and after /n ɲ/ it may be realised as [ʐ].
Adjacent to a lateral, the rhotic assimilates such that /rl lr/ are pronounced
[lː] or [ɫ] depending on the following vowel.

The glide is the palatal glide /j/, which alternates between [j\tlde ʝ\tlde ʑ].
Initially and intervocalically it is usually pronounced as an approximant, but
when final it may be pronounced as a fricative, especially before a stop or
nasal consonant.  

The fricatives /v/ and /ð/ are also often realised as approximants [ʋ] and [ð̞].  


\section{Syllables}
\label{sec:syllables}

There are three weights of syllable in Qevesa.  Light syllables consist of an
onset and a short vowel; heavy syllables consist of an onset, a short vowel and
coda, or a long vowel; and superheavy syllables consist of an onset, a long
vowel, and a coda. 

The onset is optional for all three weights, and any consonant may occur in
this position.  The coda may consist of any single consonant, a geminate
consonant, or one of the following clusters:

\begin{itemize}
  \item /r l/ + /s ʂ t/: [rs rʂ ls lʂ rt lt]
  \item /m n ɲ p t c k/ + /s ʂ/: [ms mʂ ns nʂ ɲɕ ps pʂ ts tʂ cç\tlde tɕ ks kʂ]
  \item /s ʂ/ + /p t c k/ : [sp ʂp st ʂt ɕc sk ʂk]
  \item /mp nt ɲc nk/
  % \item A fricative + affricate at the same point of articulation: /sts ʂtʂ/
\end{itemize}

Though there are a large number of permissable consonant clusters, their actual
occurrence is fairly infrequent.  Syllable-final clusters are to be avoided
word-internally where possible: VCCV will always be split into VC.CV\@. 

\begin{itemize}
  \item Light syllables are (C)V
  \item Heavy syllables are (C)Vː or (C)VC
  \item Superheavy syllables are (C)VːC(C) or (C)VCC
\end{itemize}

\section{Stress}
\label{sec:stress}

Stress in Qevesa is not phonemically contrastive, and bears a strong
relationship to vowel length and syllable weight.  The basic rules are as
follows: 

\begin{itemize}
  \item Only one of the last three syllables may be stressed.
  \item If all three syllables are of equal weight, stress falls on the penultimate syllable.
  \item If two of these syllables are heavier than the other, primary stress falls on the first of those two, and 
  \item Otherwise, stress falls on the heaviest syllable.
\end{itemize}

These rules apply regardless of morphology changes, so the stress of a given
word will move depending on what affixes (if any) are attached.

\section{Intonation}
\label{sec:intonation}

Qevesa possesses a limited pitch-accent.

%\section{Prosody}
%\label{sec:prosody}
%
%Qevesa is a syllable-timed language.
%\tbw
%
%


%\newpage
\section{Romanisation}
\label{sec:romanisation}

The usual transcription system used for the Latin alphabet is as follows:

\begin{center}
  \begin{tabulary}{0.9 \textwidth}{Fc*{7}{-c}}
    \rowstyle{\bfseries} A a   & Á á  & C c      & Č č  & Ch ch  & D d    & E e     \\
                         /a/   & /aː/ & /c/ [ts] & /tʂ/ & /ɕ tɕ/ & /ð/    & /e/     \\
    \rowstyle{\bfseries} É é   & F f  & H h      & I i  & Í í    & J j    & K k     \\
                         /eː/  & /f/  & /h/      & /i/  & /iː/   & /j/    & /k/     \\
    \rowstyle{\bfseries} Kh kh & L l  & M m      & N n  & Ň ň    & O o    & Ó ó     \\
                         /x/   & /l/  & /m/      & /n/  & /ɲ/    & /o/    & /oː/    \\
    \rowstyle{\bfseries} P p   & Q q  & R r      & S s  & Š š    & T t    & Th th   \\
                         /p/   & /c/  & /r/      & /s/  & /ʂ/    & /t/    & /θ/     \\
    \rowstyle{\bfseries} U u   & Ú ú  & V v      & Y y  & Ý ý    & Z z    & Ž ž     \\
                         /u/   & /uː/ & /v/      & /y/  & /yː/   & /z dz/ & /ʐ dʐ/  \\
  \end{tabulary}
  %\caption[Romanisation of Qevesa]{\label{tab:transcription}}
\end{center}

%  a   á   c   ch  d   e   é
%  h   i   í   j   k   kh  l
%  m   n   ň   o   ó   p   ph
%  q   qh  r   s   š   t   th
%  u   ú   v   y   ý   z   ž

% Cevesa

%\pagebreak[2]
The Latin orthography is largely phonemic, although not a one-to-one
transliteration of the native script, and makes use of a number of diacritics
and digraphs.  The diacritics indicate the following features:

\begin{description}
  \item[Háček/Caron] The \foreign{háček} or caron indicates a palatalised
    variant.  It is used with \conlang{c}, \conlang{n}, \conlang{s} and
    \conlang{z}, producing \conlang{č}, \conlang{ň}, \conlang{š} and
    \conlang{ž}.
  \item[Acute] The acute accent is used to indicate a long vowel, and is used
    with \conlang{a}, \conlang{e}, \conlang{i}, \conlang{o}, \conlang{u} and
    \conlang{y} to produce \conlang{á}, \conlang{é}, \conlang{í}, \conlang{ó},
    \conlang{ú} and \conlang{ý}.  In handwriting, the acute accent is usually
    written more like a macron with an almost horizontal line. 
\end{description}

The digraphs \conlang{ch}, \conlang{kh}, and \conlang{th} represent the
phonemes /ɕ/, /x/, and /θ/.  These phonemes were originally pronounced as
aspirated stops in Common Therasa, and became fricatives or affricates in
Qevesa.  The grapheme \conlang{q} is a stylistic variation on \conlang{c}, and
is pronounced identically.
% Originally ‹c› was used before back vowels representing /ts/ as distinguished
% from /c/ (written as ‹q›) before front vowels, and the modern pronunciation
% results from a historical hypercorrection to /c/.  The neighbouring language
% \conlang{Cavasko} went the other way, that is /c/ → /ts/ in all positions.

Geminate consonants are doubled, except for the digraphs which only double the
first consonant.  

\end{document}
