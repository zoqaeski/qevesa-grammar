\documentclass[grammar]{subfiles}
\begin{document}
  \chapter{Phonology}
  \label{ch:phonology}

  \section{Phonotactics}
  \label{sec:phonotactics}

  \subsection{Vowel inventory}
  \label{ssec:vowels}

  %	\begin{table}[htpb]\small\capstart
  %			\begin{tabular}{|>{\bfseries}fc| c c c c c c|}
  %				\hline
  %				\SetRowStyle{\bfseries} & \multicolumn{2}{-c|}{Front} & \multicolumn{2}{-c|}{Central} & \multicolumn{2}{-c|}{Back} \\\hline
  %				Close & i & y & & & & u \\
  %				Near-Mid & \superj e\textasciitilde je & \multirow{2}{*}{ø} & & & & \\
  %				Mid & e & & & & & o \\
  %				%Near-Open & \multirow{2}{*}{æ} & & & & &  \\
  %				Open & & & a & & & \\\hline
  %			\end{tabular}
  %			\caption{Short vowels\label{tab:vowels}}
  %	\end{table}

  \begin{table}[htpb]\small\capstart
      \subfloat[Short vowels]{\label{tab:short_vowels}
        \begin{tabular}{|>{\bfseries}fc|c c c|}
          \hline
          \SetRowStyle{\bfseries} & \multicolumn{1}{-c|}{Front} & \multicolumn{1}{-c|}{Central} & \multicolumn{1}{-c|}{Back} \\\hline
          Close & i & y & u \\
          Mid   & e & ɵ & o \\
          Open  &   & a &   \\\hline
        \end{tabular}
      }
      \subfloat[Long vowels]{\label{tab:long_vowels}
        \begin{tabular}{|>{\bfseries}fc|c c c|}
          \hline
          \SetRowStyle{\bfseries} & \multicolumn{1}{-c|}{Front} & \multicolumn{1}{-c|}{Central} & \multicolumn{1}{-c|}{Back} \\\hline
          Close & iː & yː & uː \\
          Mid   & eː & ɵː & oː \\
          Open  &    & aː &    \\\hline
        \end{tabular}
      }
      \caption{Qevesa vowel phonemes\label{tab:vowels}}
  \end{table}


  Qevesa possesses seven distinct vowels, listed in Table~\ref{tab:vowels}.  Although the vowels [e], [ɵ] and [o] are conventionally written using the close-mid \textsc{ipa} symbols, they are more accurately transcribed as mid vowels [\textlowering{e}], [\textlowering{ɵ}] and [\textlowering{o}].  In contrast to the consonants, the vowels show very little variation.

  The vowels [ɵ] and [y] are front-central rounded vowels.  [ɵ] is a mid front-central rounded vowel, and [y] is a close front-central rounded vowel.

  The diphthongs are /i-/ glides /ia ie io iɵ iu iy/ and /u-/ glides /ua ue ui uo/, with assimilation of /ii/, /uu/ and /uy/ to /iː/, /uː/ and /yː/.  /i-/ glides tend to cause palatalisation, and /u-/ glides may cause labialisation, but this is dialect-dependent, with palatalisation being far more common.  

  % This is not reflected in the morphology!
  %A sound change in Proto-Teralo resulted in the appearance of palatal approximant /j/ before a syllable-initial vowel, particularly /e/.  This phenomenon, known as iotation, resulted in the development of the phoneme /\superj e\textasciitilde je/.  /i-/ glides were similarly affected, and the process induced palatalisation of the preceding consonant.

  Vowels also possess a phonemic length distinction.  Each of the seven short vowels has a long equivalent; these are listed in Table~\ref{tab:long_vowels}.  A long vowel should be approximately twice as long as a short vowel.

  Long vowels are also formed through collision of two identical vowels due to morphological marking.

  \subsection{Consonant inventory}
  \label{ssec:consonants}

  \begin{table}[htpb]\small\capstart
      \begin{tabular}{|>{\bfseries}fc|c c|c c|c c|c c|c c|c c|c c|}
        \hline
        \SetRowStyle{\bfseries} & \multicolumn{2}{-c|}{Bilabial} & \multicolumn{2}{-c|}{Labiodental} & \multicolumn{2}{-c|}{Denti-alveolar} & \multicolumn{2}{-c|}{Postalveolar} & \multicolumn{2}{-c|}{Palatal} & \multicolumn{2}{-c|}{Velar} \tnl\hline
        Nasal & & m & & & & \textsubbridge{n} & & & & ɲ & & \tnl%\hline
        Plosive & p & & & & \textsubbridge{t} & & & & c & & k & \tnl%\hline
        Affricate & & & & & \textsubbridge{t}\textsubbridge{s} & & tʃ & & & & & \tnl%\hline
        Fricative & & & f & \multirow{2}{*}{ʋ} & \textsubbridge{s} θ & \multirow{2}{*}{ð} & ʃ & & ç & & x & \tnl%\cline{1-4} \cline{6-6} \cline{8-17}
        Approximant & & & \multicolumn{1}{r}{} & & \multicolumn{1}{r}{} & & & & & j & & \tnl%\hline
        Lateral & & & & & & \textsubbridge{l} & & & & & & \tnl%\hline
        Rhotic & & & & & & r\textasciitilde\textfishhookr & & & & & & \tnl\hline
      \end{tabular}
      \caption{Consonants\label{tab:consonants}}
  \end{table}

  Qevesa possesses twenty-two consonants, realised as in Table~\ref{tab:consonants}.  Features and allophones of each row are described in more detail below.  Consonants are slightly palatalised before /i/ (and its associated glides).

  Consonantal length is phonemic, so [mata] and [matːa] are distinguished.  In correct speech, geminate consonants should be articulated and released separately, although in quick speech they will be pronounced as prolonged.  Geminates may also appear at in word-initial syllables, but are rare word-finally.  Word-medially, syllables will be split at the geminate consonant.

  \subsubsection{Nasals}
  \label{sssec:nasals}

  Qevesa has three nasal consonants: /m \textsubbridge{n} ɲ/.  /\textsubbridge{n}/ is a laminal denti-alveolar nasal, rather than a true dental nasal. 
  %The palatal nasal /ɲ/ is realised as an alveolo-palatal nasal in virtually all dialects; as this does not contrast with a true palatal nasal, it is transcribed using the \textsc{ipa} symbol for a palatal nasal, rather than the obsolete and non-standard /\textctn/.

  These consonants are largely consistent in their realisation. 
  The velar nasal [ŋ] is an allophone of /\textsubbridge{n} ɲ/ before /k/.

  \subsubsection{Plosives}
  \label{sssec:plosives}

  Qevesa has four plosive consonants.  These are spread over four positions (labial, denti-alveolar, palatal, velar); voice is not distinguished: /p \textsubbridge{t} c k/.  The plosives may be aspirated when word-final, /c/ often being realised as the affricate /cç/.
  
  %The plosive consonants may be palatalised to [p\superj{} \textsubbridge{t\superj} cç k\superj].

  \subsubsection{Fricatives}
  \label{sssec:fricatives}

  Qevesa has eight fricative consonants: /f ʋ θ ð \textsubbridge{s} ʃ x/.  /ʋ/ and /ð/ are commonly realised as approximants, and /x/ may be realised as /h/. Before /i/ or /j/, /x/ may also be realised as [ç].
%  Palatalisation affects the fricatives in a variety of way: 
% 
%  \begin{itemize*}
%    \item /f/ palatalises to [f\superj];
%    \item /θ/ palatalises to [θ\superj];
%    %\item /θ/ palatalises to [\textsubbridge{s\superj}];
%    \item /s/ palatalises to [s\superj]
%    \item /ʃ/ palatalise to [ɕ];
%    \item /ʋ/ and /ð/ reduce to [j]; and,
%    \item /x/ palatalises to [ç];
%  \end{itemize*}

  \subsubsection{Affricates}
  \label{sssec:affricates}

  Qevesa has two affricates: /\textsubbridge{t}\textsubbridge{s} tʃ/.  Affricates at other points of articulation are attested in historical texts, but these have since merged with the fricatives in the modern dialects.  All of these behave as though they were a single consonant, and so should be represented with a tie-bar ligature; for simplicity this will not be done here, except if necessary to contrast the affricates from sequences of distinct phonemes. 

  %The affricates are affected by palatalisation in a similar manner to the fricatives.

  \subsubsection{Liquids and Glides}
  \label{sssec:liquids}

  Qevesa has two liquid consonants (one lateral and one rhotic) and two to four glides.

  The lateral consonant is the denti-alveolar /\textsubbridge{l}/. When preceding an /i-/ glide or /j/, it is realised as [ʎ].  %An allophone of /l/ is [ɬ] that occurs only in some clusters, such as /tl/.

  The rhotic consonant is the alveolar trill /r/.  It may be realised as a tap [ɾ] when intervocalic.  
  %Palatalised /r/ is realised as [ɾj].

  The glide is the palatal glide /j/.  This shows little allophonic variation, tending to induce allophonic changes in other consonants.  The fricatives /v/ and /ð/ are often realised as approximants.  

  \subsection{Phonemic Restrictions}
  \label{ssec:phonemic_restrictions}

  The main limitations on phonemic distribution are found within the context of consonant clusters.  Any single consonant may appear in onset or coda position, word-initially, word-medially, or word-finally.  Likewise, any vowel may occur in any of the three positions. 

  \subsubsection{Consonant Clusters}
  \label{sssec:consonant_clusters}

  Qevesa is fairly lenient when it comes to word-internal clusters. 
  Almost any combination is permitted, including clusters containing two consonants having the same point of articulation. 

%  \begin{itemize*}
%    \item /x/-final clusters result in a [kx] affricate developing
%    \item /h/-initial clusters result in lengthening of the previous vowel and a slight pre-aspiration of the following consonant
%    \item /h/-final clusters are realised as simple clusters; some dialects insert a glottal stop between the initial consonant and the /h/
%  \end{itemize*}

  Initial consonant clusters are not permitted, except for palatal offglides.

%  \begin{itemize*}
%    \item Any non-palatal plosive or /f v/ + /r l/: /pr tr kr fr vr pl tl kl fl vl/ 
%    \item /θ s ʃ/ + a plosive or /m n/: /θp θt θk θm θn sp st sk sm sn ʃp ʃt ʃk ʃm ʃn/
%    \item /θ s ʃ/ + /l/: /θl sl ʃl/
%    \item A fricative + affricate at the same point of articulation: /θtθ sts ʃtʃ/
%    \item Any non-palatal plosive + /θ s ʃ/: /pθ tθ kθ ps ts ks pʃ tʃ kʃ/.  Note that affricates contrast with plosive-fricative sequences.
%    \item Any non-palatal plosive, fricative, or affricate + /f\textasciitilde v/: /pf tf kf θf sf ʃf/.  Note that the labiodental fricative may vary between [f] [v] and [v], regardless of its orthographical representation.
%    \item Any non-palatal plosive or fricative + /w/: /pw tw kw fw θw sw ʃw/
%    \item Any consonant + /i-/ or /u-/ glide.  Note that /i-/ glides and /j/ induce palatalisation of the previous phoneme, according to the allophonic rules described in Sections \ref{sssec:nasals}–\ref{sssec:liquids}, and that /u-/ glides often assimilate to /w/
%    \item /mn mɲ/
%  \end{itemize*}
%
%  Syllable-final clusters are even more restricted than syllable-initial ones:
%
%  \begin{itemize*}
%    %\item A nasal, a plosive, or /r l j/ + /f θ s ʃ/
%    \item /r l w/ + a plosive or /f θ s ʃ/: /rp rt rk rf rs rʃ lp lt lk lf ls lʃ wp wt wk wf ws wʃ/
%    \item a nasal or plosive + /f θ s ʃ/: /mf mθ ms mʃ nf nθ ns nʃ pf pθ ps pʃ tf tθ ts tʃ kf kθ ks kʃ/
%    \item /f θ s ʃ/ + a non-palatal plosive: /fp ft fk θp θt θk sp st sk ʃp ʃt ʃk/
%    \item A fricative + affricate at the same point of articulation: /θtθ sts ʃtʃ/
%    \item /n/ + /t k/: /nt nk/
%    \item /mp/
%  \end{itemize*}
%
%  Though there are a large number of permissable consonant clusters, their actual occurrence is fairly infrequent. 
%  Syllable-initial or syllable-final clusters are to be avoided word-internally: VCCV will always be split into VC.CV\@. 
%  Clusters of three or more consonants are only permitted across syllable breaks, and will always be split to favour an initial cluster over a final one.

  \subsubsection{Syllable Structure}
  \label{sssec:syllables}

  \ToBeWritten
  %Although a wide variety of initial consonant clusters are permitted, they should be avoided when dividing a word into syllables.  The general rule is that non-word-final consonants are always the onset of syllables unless followed by another consonant or permissable initial cluster.

  \newpage

  \subsection{Romanisation}
  \label{ssec:romanisation}

  The usual transcription system used for the Latin alphabet is as follows:

  \begin{center}
    \begin{tabularx}{0.9 \textwidth}{fC*{10}{-C}}
      \SetRowStyle{\bfseries} A a & Á á & C c & Č č & D d & E e & É é & Ë ë & F f & H h \\
      /a/ & /aː/ & /ts/ & /tʃ/ & /ð/ & /e/ & /eː/ & /\superj e/ & /f/ & /x/ \\		
      \SetRowStyle{\bfseries} I i & Í í & J j & K k & L l & M m & N n & Ň ň & O o & Ó ó \\
      /i/ & /iː/ & /j/ & /k/ & /l/ & /m/ & /n/ & /ɲ/ & /o/ & /oː/ \\
      \SetRowStyle{\bfseries} Ö ö & Ő ő & P p & Q q & R r & S s & Š š & T t & U u & Ú ú \\
      /ɵ/ & /ɵː/ & /p/ & /c/ & /r/ & /s/ & /ʃ/ & /t/ & /u/ & /uː/ \\
      \SetRowStyle{\bfseries} Ü ü & Ű ű & V v & Z z \\
      /y/ & /yː/ & /v~ʋ/ & /θ/ \\
    \end{tabularx}
    %\caption[Romanisation of Qevesa]{\label{tab:transcription}}
  \end{center}

  %\pagebreak[2]
  The Latin orthography is largely phonemic, and makes use of a number of diacritics.  The diacritics indicate the following features:

  \begin{description}
    %\item[Cedilla/Comma] The cedilla or comma indicates a retroflex variant, and is used with ⟨s⟩ and ⟨c⟩, forming ⟨ş⟩ and ⟨ç⟩.  In handwritten texts, the comma is preferred, but typeset documents normally use the cedilla, due to a lack of typefaces that include the comma as a diacritic.
    \item[Háček/Caron] The \foreign{háček} or caron indicates a palatalised consonant variant.  It is used with ⟨c⟩, ⟨n⟩ and ⟨s⟩, producing ⟨č⟩, ⟨ň⟩ and ⟨š⟩.
    %\item[Dot above] This diacritic indicates an affricate varient of a fricative, and is only used with ⟨z⟩, resulting in ⟨ż⟩.
    \item[Trema] The trema has two separate uses.  With ⟨o⟩ and ⟨u⟩, it indicates a fronted variant, forming ⟨ö⟩ and ⟨ü⟩.  When used with ⟨e⟩, it indicates a palatalised variant, usually pronounced as /je/.  The long variant of ⟨ë⟩ is sometimes written as ⟨ě⟩, though ⟨é⟩ is more common. 

    %\item[Háček]\label{def:hacek} The \foreign{háček} or caron indicates an iotated or palatalised variant.  It is most commonly used with ⟨e⟩ to produce ⟨ě⟩, but may be used with other vowels.  The ⟨j-⟩ spelling is used in some situations, such as across a syllable break or between two vowels (in which the inherent /j/ becomes the onset of the next syllable), so *⟨aě⟩ is written as ⟨aje⟩.  Generally, ⟨ě⟩ is preferred when following a consonant or as a nucleus vowel of a syllable, and ⟨je⟩ is used when the /e/ is lengthened ⟨jé⟩, but both representations are interchangeable.
    \item[Acute] The acute accent is used to indicate a long vowel, and is used with ⟨a⟩, ⟨e⟩, ⟨i⟩, ⟨o⟩ and ⟨u⟩ to produce ⟨á⟩, ⟨é⟩, ⟨í⟩, ⟨ó⟩ and ⟨ú⟩.  Long variants of ⟨ö⟩ and ⟨ü⟩ use a doubled acute, resulting in ⟨ő⟩ and ⟨ű⟩. 
  \end{description}

  Although the orthography is largely morphophonemic, a number of phonemes may be written in more than one way:

  \begin{itemize*}
    %\item Palatalisation is indicated by a following ⟨j⟩, an i-glide diphthong, or a \foreign{háček} above the vowel.  %/ji/ is represented with ⟨y⟩ or ⟨í⟩ and realised as /\superj iː/.
    %\item /i-/glides and /j/ soften (palatalise) preceding consonants: therefore /ɕ/ /tɕ/ /\textltailn/ may be represented by ⟨š⟩ ⟨č⟩ ⟨ń⟩ or ⟨si-⟩ ⟨ci-⟩ ⟨ni-⟩ before another vowel.  /\superj e/ ⟨ě⟩ will also cause palatalisation in this manner.
    \item /v/ may be realised as an approximant in some situations, and digraphs involving ⟨v⟩ or ⟨f⟩ such as ⟨sf⟩ or ⟨zv⟩ may result in the labiodental fricative being realised as anything between [f] [v] and [ʋ].
    %\item /iː/ is usually written as ⟨í⟩ except when word-initial, when it is written as ⟨y⟩.
  \end{itemize*}

  \section{Prosody}
  \label{sec:prosody}

  Qevesa is a syllable-timed language.
  \ToBeWritten

  \subsection{Stress}
  \label{ssec:stress}

  Stress falls on the first syllable of the root, unless a following syllable within the root contains a long vowel.  \ToBeWritten

  \subsection{Intonation}
  \label{ssec:intonation}

  Qevesa possesses a limited pitch-accent.
  \ToBeWritten

\end{document}
