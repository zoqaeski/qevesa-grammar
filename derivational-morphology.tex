\documentclass[grammar]{subfiles}
\begin{document}
\chapter{Derivational Morphology}
\label{ch:derivational-morphology}

As a highly synthetic language, derivation plays a major role in the
formation of words in Qevesa.  Due to its triliteral roots, the majority of
words are in fact derived by productive transfixes, suffixes, and prefixes,
as well as compounding operations.


\section{Nominalisation}
\label{sec:dev_nominalisation}


\subsection{Discontinuous Patterns}
\label{ssec:dev_discontinuous_patterns}

A large number of nouns in Qevesa are derived from the root + vowel pattern
framework of the verbal system.  


The pattern \conlang{*C₁aC₂C₂aC₃} is commonly used to form
professions from verbal roots.  It is no longer highly productive, so most
nouns with this pattern represent professions that have existed for a very
long time.  
% Examples of this pattern are given in \cref{tab:dev_profession_caccoc}. 

\begin{center}\small
  \begin{tabular}{FIl -l @{\hspace{1em}→\hspace{1em}}-Il -l}
    \toprule
    \rowstyle{\bfseries\upshape} Root/Base & Meaning & Profession & Meaning \\
    \midrule
    foruk & cut [wood, etc] & farrak & carpenter \\
    kolun & heal            & kallan & doctor \\
    losut & study, teach    & lassat & teacher \\
    rokut & write           & rakkat & scribe \\
    sotur & govern          & sattar & governor, lord \\
    zomur & guard, watch    & zammar & guard \\
    \bottomrule
  \end{tabular}
  %\caption{Nouns of profession\label{tab:dev_profession_caccoc}}
\end{center}


The pattern \conlang{*C₁áC₂iC₃an} is the most common pattern used to form professions
(as well as many other role-like agentives) in modern-day Qevesa.  It actually
consists of the active participle without the initial \conlang{e}, suffixed with the
third person agentive suffix.
% Some examples of this pattern are given in \Cref{tab:dev_profession_caccin}. 

\begin{center}\small
  \begin{tabular}{FIl -l @{\hspace{1em}→\hspace{1em}}-Il -l}
    \toprule
    \rowstyle{\bfseries\upshape} Root/Base & Meaning & Profession & Meaning \\
    \midrule
    jonuv & steal        & jánivan & thief \\
    lokuj & trick        & lákijan & trickster \\
    mosul & think        & másilan & philosopher \\
    satuk & send, travel & sátikan & messenger, envoy \\
    sonuš & count        & sánišan & accountant \\
    \bottomrule
  \end{tabular}
  %\caption{Nouns of profession\label{tab:dev_profession_caccin}}
\end{center}


The pattern \conlang{*taC₁C₂eC₃} creates agentives from activities
that are social in nature, that is, typically involve more than one person
and are not done on their own.  
% Examples are given in \Cref{tab:dev_agent_miccec}.

\begin{center}\small
  \begin{tabular}{FIl -l @{\hspace{1em}→\hspace{1em}}-Il -l}
    \toprule
    \rowstyle{\bfseries\upshape} Root/Base & Meaning & Agentive & Meaning \\
    \midrule
    hocuv & sit   & tácev  & resident \\
    losut & learn & talset & student \\
    rovud & work  & tarved & worker, employee \\
    toruz & come  & tatrez & guest \\
    šél   & love  & tašle  & lover \\
    \bottomrule
  \end{tabular}
  %\caption{Nouns of profession\label{tab:dev_agent_miccec}}
\end{center}


The pattern \conlang{*zeC₁C₂VC₃}, where ‘V’ represents the long root vowel, typically forms nouns of place or
location, such as physical features or buildings.  
% Some examples are listed in \Cref{tab:dev_location_zeccac}.

\begin{center}\small
  \begin{tabular}{FIl -l @{\hspace{1em}→\hspace{1em}}-Il -l}
    \toprule
    \rowstyle{\bfseries\upshape} Root/Base & Meaning & Location & Meaning \\
    \midrule
    khonus & get up, stand         & zekhnós & place, location \\
    losut  & learn                 & zelsót  & school \\
    rosuc  & bathe                 & zersóc  & bath, bathtub \\
    vesuk  & lay down              & zevsék  & bed \\
    \bottomrule
  \end{tabular}
  %\caption{\label{tab:dev_location_zeccac}}
\end{center}


The pattern \conlang{*C₁eC₂C₃i} is also used to form nouns of place or location.  

\begin{center}\small
  \begin{tabular}{FIl -l @{\hspace{1em}→\hspace{1em}}-Il -l}
    \toprule
    \rowstyle{\bfseries\upshape} Root/Base & Meaning & Location & Meaning \\
    \midrule
    % lamut  & learn                 & lemti   & university \\
    khedus & be special            & khedsi  & temple \\
    tosun  & house, shelter        & tesni   & house \\
    veluj  & rise [sun, moon, etc] & velí    & east \\
    keruv  & set [sun, moon, etc]  & kervi   & west \\
    \bottomrule
  \end{tabular}
  %\caption{\label{tab:dev_location_cececca}}
\end{center}


The pattern \conlang{*mVC₁C₂eC₃}, where ‘V’ is the short root vowel, is used to form
nouns describing tools or instruments used to perform an action. 

\begin{center}\small
  \begin{tabular}{FIl -l @{\hspace{1em}→\hspace{1em}}-Il -l}
    \toprule
    \rowstyle{\bfseries\upshape} Root/Base & Meaning & Instrument & Meaning \\
    \midrule
    choput & open  & mochpet & key \\
    jakun  & boil  & mažken  & kettle \\
    rokut  & write & morket  & pen \\
    sonut  & weigh & mosnet  & scale \\
    šomú   & shave & mošmé   & razor \\
    šovuc  & burn  & mošvec  & lighter \\
    \bottomrule
  \end{tabular}
  %\caption{\label{tab:dev_instrument_miccoc}}
\end{center}


The pattern \conlang{*C₁eC₂áC₃} is similarly used to form names of
tools and other physical objects.  These nouns are typically, but not always,
the resulting product of the action.

\begin{center}\small
  \begin{tabular}{FIl -l @{\hspace{1em}→\hspace{1em}}-Il -l}
    \toprule
    \rowstyle{\bfseries\upshape} Root/Base & Meaning & Object & Meaning \\
    \midrule
    rokut  & write     & rekát & book \\
    sakut  & send      & sekát & letter, message \\
    vorun  & wear      & verán & garment \\
    zotú   & bind, tie & zetá  & belt, sash \\
    \bottomrule
  \end{tabular}
  %\caption{\label{tab:dev_result_cicac}}
\end{center}


A related pattern is \conlang{*šeC₁C₂éC₃}, or the passive participle without the
final \conlang{-y}. It is used to form nouns resulting from the action, or nouns
possessing qualities of the action.

\begin{center}\small
  \begin{tabular}{FIl -l @{\hspace{1em}→\hspace{1em}}-Il -l}
    \toprule
    \rowstyle{\bfseries\upshape} Root/Base & Meaning & Object & Meaning \\
    \midrule
    % cícavu & freeze    & šeccév & ice \\
    rokut  & write     & šerkét & document \\
    zotú   & bind, tie & šezté  & knot \\
    \bottomrule
  \end{tabular}
  %\caption{\label{tab:dev_result_cicac}}
\end{center}


The pattern \conlang{*meC₁VːC₂iC₃}, where ‘V’ is the long root vowel, is used to form
abstract nouns, primarily from adjectives and stative roots which denote physical
or temporal characteristics.

\begin{center}\small
  \begin{tabular}{FIl -l @{\hspace{1em}→\hspace{1em}}-Il -l}
    \toprule
    \rowstyle{\bfseries\upshape} Root/Base & Meaning & Noun & Meaning \\
    \midrule
    korum & wish luck     & mekórim & luck \\
    ňevuc & new           & meňévic & age  \\
    rovud & work          & meróvid & work \\
    \bottomrule
  \end{tabular}
  %\caption{\label{tab:dev_result_cicac}}
\end{center}


The pattern \conlang{*C₁iC₂C₃a} also forms abstract nouns denoting physical quantities.

\begin{center}\small
  \begin{tabular}{FIl -l @{\hspace{1em}→\hspace{1em}}-Il -l}
    \toprule
    \rowstyle{\bfseries\upshape} Root/Base & Meaning & Noun & Meaning \\
    \midrule
    sonuc & measure, count & sinca & quantity \\
    leput & be wide        & lipta & area  \\
    móš   & be hot         & mimša & temperature \\
    želut & be strong      & žilta & iron \\
    \bottomrule
  \end{tabular}
  %\caption{\label{tab:dev_result_cicac}}
\end{center}

%
%  \subsection{Common Nouns}
%  \label{ssec:dev_common_nouns}
%
%  \tbw
%
%  \subsection{Generic and Specific Nouns}
%  \label{ssec:dev_generic_nouns}
%
%  The generic noun is a general nominalisation which represents the concept,
%  process, activity or ability denoted by the root. 
%  This contrasts with the pattern that denotes a specific instance of the
%  generic concept. 
%  Both patterns are related, and in many cases, the specific pattern is itself
%  a derivation of the generic pattern. 
%  The patterns are listed in Table~\ref{tab:dev_generic_specific}.
%
%  \begin{table}[h!]\small\capstart
%    \subfloat[Generic nominalisation]{
%      \begin{tabular}{BFl -l -l}
%        \toprule
%        \rowstyle{\bfseries} Form & Pattern \\
%        \midrule
%        1 & C₁{e}C₂{é}C₃ \\
%        2 & C₁{e}C₂C₂{é}C₃ \\
%        3 & C₁{e}C₂C₃{é} \\
%        4 & {i}C₁C₂{e}C₃{é} \\
%        5 & {me}C₁{e}C₂{é}C₃ \\
%        6 & {ta}C₁C₂{e}C₃{é} \\
%        7 & C₁{ie}C₂{e}C₃{é} \\
%        \bottomrule
%      \end{tabular}
%    }
%    \subfloat[Specific nominalisation]{
%      \begin{tabular}{BFl -l -l}
%        \toprule
%        \rowstyle{\bfseries} Form & Pattern \\
%        \midrule
%        1 & C₁{e}C₂{ú}C₃ \\
%        2 & C₁{e}C₂C₂{ú}C₃ \\
%        3 & C₁{e}C₂C₃{ú} \\
%        4 & {i}C₁C₂{e}C₃{ú} \\
%        5 & {me}C₁{e}C₂{ú}C₃ \\
%        6 & {ta}C₁C₂{e}C₃{ú} \\
%        7 & C₁{ie}C₂{e}C₃{ú} \\
%        \bottomrule
%      \end{tabular}
%    }
%    \caption{Generic and specific noun forms\label{tab:dev_generic_specific}}
%  \end{table}
%
%  \begin{exe}
%    \ex \emph{EXAMPLES}
%  \end{exe}

% \subsection{Abstract Nouns}
% \label{ssec:dev_abstract_nouns}

% \tbw

% \subsection{Collective Nouns, Mass Nouns, and Unit Nouns}
% \label{ssec:dev_collective_mass_unit}

% \tbw

% \subsection{Dimunitive Nouns}
% \label{ssec:dev_dimunitive_nouns}

% \tbw

% \subsection{Proper Nouns}
% \label{ssec:dev_proper_nouns}

% \tbw

\end{document}

