\documentclass[grammar]{subfiles}
\begin{document}
  \chapter{Derivational Morphology}
  \label{ch:derivational-morphology}

  As a highly synthetic language, derivation plays a major role in the formation of words in Qevesa. Due to its triliteral roots, the majority of words are in fact derived by productive transfixes, suffixes, and prefixes, as well as compounding operations.

  \section{Verb Root Forms}
  \label{sec:dev_verb_root_forms}

  Although the arrangement of consonants in a root is generally fixed, there are regular processes to derive subtle semantic variations on the meaning of the root, such as causatives and reflexives. These root variants are called forms, or \qevesa{méttüses} (“constructions”), from the root \qevesa{mutus} (“build, construct”). There are nine commonly-used forms, listed as Forms I–IX, although not every root can be shaped into each form. These are listed in Table~\ref{tab:dev_root_forms}.

  Note that the forms affect only the grouping and gemination of root consonants, and not the vowel patterns that are applied to create meaningful words. In those forms where consonants are grouped into clusters, the consonant pairs are subsequently treated as a single consonant.

  % REWRITE ROOT FORMS TO THIS:
  % transitive / intransitive
  % normal / intense
  % causative
  %
  % So:
  % Form I     Normal                  1u2u3     1u2u
  % Form II    Intensive               1u22u3    1u22u
  % Form III   Passive                 ja1u23u   ja1u22u
  % Form IV    Reciprocal              te1u2u3   te1u2u
  % Form V     Causative               i12u33u   i11u22u
  % Form VI    Reciprocal (Causative)  ta12u33u  ta11u22u
  % Form VII   Reciprocal (Intensive)  sa1u22u3  sa1u22u
  % Form VIII  Stative (Attributive)   1u23u     1u22u
  % Form IX    Intensive Stative       ě12u23u   ě1u22u

  \begin{table}[htpb]\small\capstart
    \begin{tabular}{|>{\bfseries}fc|-c|-c|}
      \hline
      \SetRowStyle{\bfseries} Root Form & \multicolumn{2}{-c|}{Pattern} \tabularnewline
      \cline{2-3}
      \SetRowStyle{\bfseries} & Triliteral & Biliteral \tabularnewline
      \hline
      I & 
      C\sub1{u}C\sub2{u}C\sub3 & 
      C\sub1{u}C\sub2{u} 
      \tabularnewline
      II & 
      C\sub1{u}C\sub2C\sub2{u}C\sub3 &
      C\sub1{u}C\sub2C\sub2{u} 
      \tabularnewline
      III & 
      {ja}C\sub1C\sub2{u}C\sub3{u} & 
      {ja}C\sub1C\sub2{u}C\sub2{u} 
      \tabularnewline
      IV & 
      {te}C\sub1{u}C\sub2{u}C\sub3	& 
      {te}C\sub1{u}C\sub2C\sub2{u} 
      \tabularnewline
      V & 
      {i}C\sub1C\sub2{u}C\sub3C\sub3{u} & 
      {i}C\sub1C\sub1{u}C\sub2C\sub2{u} 
      \tabularnewline
      VI & 
      {ta}C\sub1C\sub2{u}C\sub3C\sub3{u}	& 
      {ta}C\sub1C\sub1{u}C\sub2C\sub2{u} 
      \tabularnewline
      VII & 
      {sa}C\sub1{u}C\sub2C\sub2{u}C\sub3	& 
      {sa}C\sub1{u}C\sub2C\sub2{u} 
      \tabularnewline
      VIII & 
      C\sub1{u}C\sub2C\sub3{u} & 
      C\sub1{u}C\sub2C\sub2{u} 
      \tabularnewline
      IX & 
      {ě}C\sub1C\sub2{u}C\sub2C\sub3{u} & 
      {ě}C\sub1{u}C\sub2C\sub2{u} 
      \tabularnewline
      \hline
    \end{tabular}
    \caption{Verb root forms\label{tab:dev_root_forms}}
  \end{table}

  \subsection{Form I}
  \label{ssec:dev_verb_form_i}

  Form I is the most common consonantal root form, containing no preformative affixes or pairing of consonants as occurs in the other forms. It is typically the closest indicator to the lexical meaning of the root, and though it has no particular semantic function associated with it, verbs in Form I are often transitive.

  \subsection{Form II}
  \label{ssec:dev_verb_form_ii}

  Form II is the \emph{intensive} stem. It typically indicates an intensive, frequentative or causative meaning, and may also be used to form transitive verbs from intransitive roots. 

  It is constructed by geminating the second consonant; a limited number of verbs replace the gemination with two root consonants. % WHICH??

  \subsection{Form III}
  \label{ssec:dev_verb_form_iii}

  Form III is commonly known as the \emph{passive} stem.
  It is commonly used to make the passive intransitive of the Form I root, and may also be used to describe participles.
  Another use of the Form III root is to form adjectives and attributes, though this is generally non-productive in modern Qevesa, this function having been assumed by Forms VIII and IX.

  It is formed by pairing the second and third consonants and prefixing \qevesa{ja-}; biliteral roots geminate the second consonant.

  \subsection{Form IV}
  \label{ssec:dev_verb_form_iv}

  Form IV is commonly known as the \emph{reciprocal} stem. 
  It commonly conveys meanings of a reciprocal or reflexive nature, and is often used to create verbs denoting social interactions.

  This form is constructed by prefixing the Form I stem with \qevesa{te-}.

  \subsection{Form V}
  \label{ssec:dev_verb_form_v}

  Form IV is commonly known as the \emph{causative} stem. 
  Its most common function is causative; it may also convert transitive verbs into ditransitive ones.
  It can also have a causative meaning on verbs whose Form I root is intransitive, and for some verbs, may convey an assistive or factitive meaning.

  Triliteral roots construct this form by by pairing the first and second consonants, geminating the third, and prefixing with \qevesa{i-}.
  Biliteral roots geminate both consonants and prefix with \qevesa{i-}.

  \subsection{Form VI}
  \label{ssec:dev_verb_form_vi}

  Form V is the \emph{reciprocal causative} stem, so called for historical reasons as it also includes a number of other intransitive meanings. 
  It is subject to much unpredictable metaphorical and semantic and drift, so actual meanings may vary quite a lot from the Form I verb.
  True reflexives account for only a portion of the verbs in this form. 
  Its main functions are: 

  \begin{itemize*}
    \item Forming reflexives from transitive roots
    \item Forming verbs denoting accompaniment
    \item Forming \emph{autoreflexive} verbs, that is, intransitive actions performed on one’s body
  \end{itemize*}

  The only functions which are still fully productive are the forming of reflexives from transitive roots and the verbs of accompaniment. 
  The group of autoreflexives are a closed class, overlapping with similar verbs in Form VI.

  Triliteral roots construct this form by pairing the first and second consonants, geminating the third, and prefixing with \qevesa{ta-}. 
  Biliteral roots geminate both consonants and prefix with \qevesa{ta-}.

  \subsection{Form VII}
  \label{ssec:dev_verb_form_vii}

  Form VII is the \emph{intensive reciprocal} stem, generally indicating an intensive variant of Form IV.
  Like Form VI, Form VII roots are also subject to unpredictable metaphorical and semantic drift.

  This form is constructed by prefixing the Form II root with \qevesa{sa-}.

  \subsection{Form VIII}
  \label{ssec:dev_verb_form_viii}

  Form VII is the \emph{attributive} stem, indicating attributes, physical traits, or colours, and is always intransitive. 
  It is often used as the base form from which adjectives may be derived.

  Triliteral roots construct this form by pairing the second and third consonants. 
  Biliteral roots construct this form by geminating the second consonant.

  \subsection{Form IX}
  \label{ssec:dev_verb_form_ix}

  Form IX is the \emph{intensive stative} stem. 
  It generally indicates intensive variants of Form VIII, and is also always intransitive.
  All verbs which possess a Form IX root also possess a Form VIII root.

  Triliteral roots construct this form by pairing the first and second consonants, duplicating the second consonant and pairing it with the third, and prefixing with \qevesa{ě-}. Biliteral roots simply prefix the Form VIII root with \qevesa{ě-}.

%  \subsection{Other Verb Forms}
%  \label{ssec:dev_other_verb_forms}
%
%  There are a small number of irregular stem forms, mainly used to construct auxiliary verbs, such as those indicating evidentiality or polarity. They are effectively a closed class, and are generally not productive.

  \section{Nominalisation}
  \label{sec:dev_nominalisation}

  Most Qevesa nouns are derived from biliteral, triliteral or quadriliteral lexical roots, and all nouns derived from a particular root are listed in a dictionary under that root entry. 
  Some nouns, however, have solid stems, unanalysable into roots and patterns, although their consonants may be adapted into roots for derivation of new terms. 
  Derived nouns are formed through application of particular morphological patterns; the use of patterns interlocking with root phonemes allows the formation of actual words or stems. 
  The nominal patterns themselves carry meaning, such as “place where action is performed,” “person who performs action,” “name of action,” or “instrument used to carry out action.” 
  The most frequently occurring noun patterns are listed in the following sections.

  It is important to note that not all root forms have all nominalisation patterns.

  \subsection{Verbal Nouns}
  \label{ssec:dev_verbal_nouns}

  Verbal nouns are systematically related to verb forms. 
  The verbal noun names the action denoted by its corresponding verb; they are often abstract in meaning, but some of them have specific, concrete reference. 
  Verbal nouns may also express infinitive forms. 
  Typically, the infinitive verbal noun is the citation form of a root.

  The verbal noun pattern is generally formed by the pattern \qevesa{C\sub1{a}C\sub2{u}C\sub3}, that is, by replacing the \qevesa{-u-} in P\sub{12} in the citation form with \qevesa{-a-}.

  \begin{table}[htpb]\small\capstart
    \begin{tabular}{|>{\bfseries}fc|-c|-c|}
      \hline
      \SetRowStyle{\bfseries} Root Form & \multicolumn{2}{-c|}{Pattern} \tabularnewline
      \cline{2-3}
      \SetRowStyle{\bfseries} & Triliteral & Biliteral \tabularnewline
      \hline
      I & 
      C\sub1{a}C\sub2{u}C\sub3 & 
      C\sub1{a}C\sub2{u} 
      \tabularnewline
      II & 
      C\sub1{a}C\sub2C\sub2{u}C\sub3 &
      C\sub1{a}C\sub2C\sub2{u} 
      \tabularnewline
      III & 
      {ja}C\sub1C\sub2{a}C\sub3{u} & 
      {ja}C\sub1C\sub2{a}C\sub2{u} 
      \tabularnewline
      IV & 
      {te}C\sub1{a}C\sub2{u}C\sub3	& 
      {te}C\sub1{a}C\sub2C\sub2{u} 
      \tabularnewline
      V & 
      {i}C\sub1C\sub2{a}C\sub3C\sub3{u} & 
      {i}C\sub1C\sub1{a}C\sub2C\sub2{u} 
      \tabularnewline
      VI & 
      {ta}C\sub1C\sub2{a}C\sub3C\sub3{u}	& 
      {ta}C\sub1C\sub1{a}C\sub2C\sub2{u} 
      \tabularnewline
      VII & 
      {sa}C\sub1{a}C\sub2C\sub2{u}C\sub3	& 
      {sa}C\sub1{a}C\sub2C\sub2{u} 
      \tabularnewline
      VIII & 
      C\sub1{a}C\sub2C\sub3{u} & 
      C\sub1{a}C\sub2C\sub2{u} 
      \tabularnewline
      IX & 
      {ě}C\sub1C\sub2{a}C\sub2C\sub3{u} & 
      {ě}C\sub1{a}C\sub2C\sub2{u} 
      \tabularnewline
      \hline
    \end{tabular}
    \caption{Verbal noun paradigms\label{tab:dev_verbal_nouns}}
  \end{table}

  \begin{exe}
    \ex \emph{EXAMPLES}
  \end{exe}

  \subsection{Active and Passive Participles}
  \label{ssec:dev_active_passive_participles}

  Participles are descriptive terms derived from verbs. 
  The active participle describes the doer or the agent of the action, and the passive participle describes or refers to the object or patient of the action. 
  Both participles are predictably derived according to the verbal root forms; the most common patterns are listed in Table~\ref{tab:dev_nominal_participles}.

\begin{table}[htpb]\small\capstart
  \subfloat[Active participles]{
    \begin{tabular}{|>{\bfseries}fc|-c|-c|}
      \hline
      \SetRowStyle{\bfseries} Root Form & \multicolumn{2}{-c|}{Pattern} \tabularnewline
      \cline{2-3}
      \SetRowStyle{\bfseries} & Triliteral & Biliteral \tabularnewline
      \hline
      I & 
      C\sub1{a}C\sub2{oi}C\sub3 & 
      C\sub1{a}C\sub2{oi} 
      \tabularnewline
      II & 
      C\sub1{a}C\sub2C\sub2{oi}C\sub3 &
      C\sub1{a}C\sub2C\sub2{oi} 
      \tabularnewline
      III & 
      {ja}C\sub1C\sub2{a}C\sub3{oi} & 
      {ja}C\sub1C\sub2{a}C\sub2{oi} 
      \tabularnewline
      IV & 
      {te}C\sub1{a}C\sub2{oi}C\sub3	& 
      {te}C\sub1{a}C\sub2C\sub2{oi} 
      \tabularnewline
      V & 
      {i}C\sub1C\sub2{a}C\sub3C\sub3{oi} & 
      {i}C\sub1C\sub1{a}C\sub2C\sub2{oi} 
      \tabularnewline
      VI & 
      {ta}C\sub1C\sub2{a}C\sub3C\sub3{oi}	& 
      {ta}C\sub1C\sub1{a}C\sub2C\sub2{oi} 
      \tabularnewline
      VII & 
      {sa}C\sub1{a}C\sub2C\sub2{oi}C\sub3	& 
      {sa}C\sub1{a}C\sub2C\sub2{oi} 
      \tabularnewline
      VIII & 
      C\sub1{a}C\sub2C\sub3{oi} & 
      C\sub1{a}C\sub2C\sub2{oi} 
      \tabularnewline
      IX & 
      {ě}C\sub1C\sub2{a}C\sub2C\sub3{oi} & 
      {ě}C\sub1{a}C\sub2C\sub2{oi} 
      \tabularnewline
      \hline
    \end{tabular}
  }\\
  \subfloat[Passive participles]{
    \begin{tabular}{|>{\bfseries}fc|-c|-c|}
      \hline
      \SetRowStyle{\bfseries} Root Form & \multicolumn{2}{-c|}{Pattern} \tabularnewline
      \cline{2-3}
      \SetRowStyle{\bfseries} & Triliteral & Biliteral \tabularnewline
      \hline
      I & 
      C\sub1{o}C\sub2{i}C\sub3 & 
      C\sub1{o}C\sub2{i} 
      \tabularnewline
      II & 
      C\sub1{o}C\sub2C\sub2{i}C\sub3 &
      C\sub1{o}C\sub2C\sub2{i} 
      \tabularnewline
      III & 
      {ja}C\sub1C\sub2{o}C\sub3{i} & 
      {ja}C\sub1C\sub2{o}C\sub2{i} 
      \tabularnewline
      IV & 
      {te}C\sub1{o}C\sub2{i}C\sub3	& 
      {te}C\sub1{o}C\sub2C\sub2{i} 
      \tabularnewline
      V & 
      {i}C\sub1C\sub2{o}C\sub3C\sub3{i} & 
      {i}C\sub1C\sub1{o}C\sub2C\sub2{i} 
      \tabularnewline
      VI & 
      {ta}C\sub1C\sub2{o}C\sub3C\sub3{i}	& 
      {ta}C\sub1C\sub1{o}C\sub2C\sub2{i} 
      \tabularnewline
      VII & 
      {sa}C\sub1{o}C\sub2C\sub2{i}C\sub3	& 
      {sa}C\sub1{o}C\sub2C\sub2{i} 
      \tabularnewline
      VIII & 
      C\sub1{o}C\sub2C\sub3{i} & 
      C\sub1{o}C\sub2C\sub2{i} 
      \tabularnewline
      IX & 
      {ě}C\sub1C\sub2{o}C\sub2C\sub3{i} & 
      {ě}C\sub1{o}C\sub2C\sub2{i} 
      \tabularnewline
      \hline
    \end{tabular}
  }
  \caption{Nominal participles\label{tab:dev_nominal_participles}}
\end{table}

  %\subsubsection{Active Participle}

  %\ToBeWritten

  \subsection{Location}
  \label{ssec:dev_nouns_location}

  Another noun pattern specifies the location in which an action is performed. 
  Only Forms I–VI have locative patterns, and of these, not all are productive or even valid. 
  The patterns for location are given in Table~\ref{tab:dev_nominal_location}.

  \begin{table}[htpb]\small\capstart
    \begin{tabular}{|>{\bfseries}fc|-c|-c|}
      \hline
      \SetRowStyle{\bfseries} Root Form & \multicolumn{2}{-c|}{Pattern} \tabularnewline
      \cline{2-3}
      \SetRowStyle{\bfseries} & Triliteral & Biliteral \tabularnewline
      \hline
      I & 
      C\sub1{a}C\sub2{e}C\sub3 & 
      C\sub1{a}C\sub2{e} 
      \tabularnewline
      II & 
      C\sub1{a}C\sub2C\sub2{e}C\sub3 &
      C\sub1{a}C\sub2C\sub2{e} 
      \tabularnewline
      III & 
      {ja}C\sub1C\sub2{a}C\sub3{e} & 
      {ja}C\sub1C\sub2{a}C\sub2{e} 
      \tabularnewline
      IV & 
      {te}C\sub1{a}C\sub2{e}C\sub3	& 
      {te}C\sub1{a}C\sub2C\sub2{e} 
      \tabularnewline
      V & 
      {i}C\sub1C\sub2{a}C\sub3C\sub3{e} & 
      {i}C\sub1C\sub1{a}C\sub2C\sub2{e} 
      \tabularnewline
      VI & 
      {ta}C\sub1C\sub2{a}C\sub3C\sub3{e}	& 
      {ta}C\sub1C\sub1{a}C\sub2C\sub2{e} 
      \tabularnewline
      VII & 
      {sa}C\sub1{a}C\sub2C\sub2{e}C\sub3	& 
      {sa}C\sub1{a}C\sub2C\sub2{e} 
      \tabularnewline
      \hline
    \end{tabular}
    \caption{Nouns of location\label{tab:dev_nominal_location}}
  \end{table}

  Some examples:

  \begin{exe}
    \ex \emph{EXAMPLES}
  \end{exe}

  \subsection{Instrument}
  \label{ssec:dev_nouns_instrument}

  A specific derivational pattern is used to indicate nouns of instrument; that is, nouns that denote items used in accomplishing a particular action. 
  These patterns are only used with Forms I–V, and are listed in Table~\ref{tab:dev_nominal_instrument}.

  \begin{table}[htpb]\small\capstart
    \begin{tabular}{|>{\bfseries}fc|-c|-c|}
      \hline
      \SetRowStyle{\bfseries} Root Form & \multicolumn{2}{-c|}{Pattern} \tabularnewline
      \cline{2-3}
      \SetRowStyle{\bfseries} & Triliteral & Biliteral \tabularnewline
      \hline
      I & 
      C\sub1{ö}C\sub2{e}C\sub3 & 
      C\sub1{ö}C\sub2{e} 
      \tabularnewline
      II & 
      C\sub1{ö}C\sub2C\sub2{e}C\sub3 &
      C\sub1{ö}C\sub2C\sub2{e} 
      \tabularnewline
      III & 
      {ja}C\sub1C\sub2{ö}C\sub3{e} & 
      {ja}C\sub1C\sub2{ö}C\sub2{e} 
      \tabularnewline
      IV & 
      {te}C\sub1{ö}C\sub2{e}C\sub3	& 
      {te}C\sub1{ö}C\sub2C\sub2{e} 
      \tabularnewline
      V & 
      {i}C\sub1C\sub2{ö}C\sub3C\sub3{e} & 
      {i}C\sub1C\sub1{ö}C\sub2C\sub2{e} 
      \tabularnewline
      VI & 
      {ta}C\sub1C\sub2{ö}C\sub3C\sub3{e}	& 
      {ta}C\sub1C\sub1{ö}C\sub2C\sub2{e} 
      \tabularnewline
      VII & 
      {sa}C\sub1{ö}C\sub2C\sub2{e}C\sub3	& 
      {sa}C\sub1{ö}C\sub2C\sub2{e} 
      \tabularnewline
      \hline
    \end{tabular}
    \caption{Nouns of instrument\label{tab:dev_nominal_instrument}}
  \end{table}

  Some examples:

  \begin{exe}
    \ex \emph{EXAMPLES}
  \end{exe}

  \subsection{Intensity, Repetition, Profession}
  \label{ssec:dev_nouns_intensity_repetition_profession}

  A noun pattern exists to denote intensity or repeated actions; it also often denotes professions. 
  The patterns are given in Table~\ref{tab:dev_nominal_intensity_repetition}; note that Forms I and II have merged in this pattern.

  \begin{table}[htpb]\small\capstart
    \subfloat[Intensity/Repetition]{
      \begin{tabular}{|>{\bfseries}fc|-c|-c|}
        \hline
        \SetRowStyle{\bfseries} Root Form & \multicolumn{2}{-c|}{Pattern} \tabularnewline
        \cline{2-3}
        \SetRowStyle{\bfseries} & Triliteral & Biliteral \tabularnewline
        \hline
        I & 
        C\sub1{o}C\sub2{á}C\sub3 & 
        C\sub1{o}C\sub2{á} 
        \tabularnewline
        II & 
        C\sub1{o}C\sub2C\sub2{á}C\sub3 &
        C\sub1{o}C\sub2C\sub2{á} 
        \tabularnewline
        III & 
        {ja}C\sub1C\sub2{o}C\sub3{á} & 
        {ja}C\sub1C\sub2{o}C\sub2{á} 
        \tabularnewline
        IV & 
        {te}C\sub1{o}C\sub2{á}C\sub3	& 
        {te}C\sub1{o}C\sub2C\sub2{á} 
        \tabularnewline
        V & 
        {i}C\sub1C\sub2{o}C\sub3C\sub3{á} & 
        {i}C\sub1C\sub1{o}C\sub2C\sub2{á} 
        \tabularnewline
        VI & 
        {ta}C\sub1C\sub2{o}C\sub3C\sub3{á}	& 
        {ta}C\sub1C\sub1{o}C\sub2C\sub2{á} 
        \tabularnewline
        VII & 
        {sa}C\sub1{o}C\sub2C\sub2{á}C\sub3	& 
        {sa}C\sub1{o}C\sub2C\sub2{á} 
        \tabularnewline
        \hline
      \end{tabular}
    }\\
    \subfloat[Repetition/Habitual/Intermittent]{
      \begin{tabular}{|>{\bfseries}fc|-c|-c|}
        \hline
        \SetRowStyle{\bfseries} Root Form & \multicolumn{2}{-c|}{Pattern} \tabularnewline
        \cline{2-3}
        \SetRowStyle{\bfseries} & Triliteral & Biliteral \tabularnewline
        \hline
        I & 
        C\sub1{ű}C\sub2{a}C\sub3 & 
        C\sub1{ű}C\sub2{a} 
        \tabularnewline
        II & 
        C\sub1{ű}C\sub2C\sub2{a}C\sub3 &
        C\sub1{ű}C\sub2C\sub2{a} 
        \tabularnewline
        III & 
        {ja}C\sub1C\sub2{ű}C\sub3{a} & 
        {ja}C\sub1C\sub2{ű}C\sub2{a} 
        \tabularnewline
        IV & 
        {te}C\sub1{ű}C\sub2{a}C\sub3	& 
        {te}C\sub1{ű}C\sub2C\sub2{a} 
        \tabularnewline
        V & 
        {i}C\sub1C\sub2{ű}C\sub3C\sub3{a} & 
        {i}C\sub1C\sub1{ű}C\sub2C\sub2{a} 
        \tabularnewline
        VI & 
        {ta}C\sub1C\sub2{ű}C\sub3C\sub3{a}	& 
        {ta}C\sub1C\sub1{ű}C\sub2C\sub2{a} 
        \tabularnewline
        VII & 
        {sa}C\sub1{ű}C\sub2C\sub2{a}C\sub3	& 
        {sa}C\sub1{ű}C\sub2C\sub2{a} 
        \tabularnewline
        \hline
      \end{tabular}
    }
    \caption{Nouns of intensity and/or repetition\label{tab:dev_nominal_intensity_repetition}}
  \end{table}

  \begin{exe}
    \ex \emph{EXAMPLES}
  \end{exe}

  The abstract noun denoting the name of a profession is often given by the patterns \qevesa{C\sub1{oi}C\sub2C\sub2{á}C\sub3} and \qevesa{C\sub1{oi}C\sub2C\sub2{á}}:

  \begin{exe}
    \ex \emph{EXAMPLES}
  \end{exe}

  \subsection{Common Nouns}
  \label{ssec:dev_common_nouns}

  \ToBeWritten

  \subsection{Generic and Specific Nouns}
  \label{ssec:dev_generic_nouns}

  The generic noun is a general nominalisation which represents the concept, process, activity or ability denoted by the root. 
  This contrasts with the pattern that denotes a specific instance of the generic concept. 
  Both patterns are related, and in many cases, the specific pattern is itself a derivation of the generic pattern. 
  The patterns are listed in Table~\ref{tab:dev_generic_specific}.

  \begin{table}
    \subfloat[Generic nominalisation]{
      \begin{tabular}{|>{\bfseries}fc|-c|-c|}
        \hline
        \SetRowStyle{\bfseries} Root Form & \multicolumn{2}{-c|}{Pattern} \tabularnewline
        \cline{2-3}
        \SetRowStyle{\bfseries} & Triliteral & Biliteral \tabularnewline
        \hline
        I & 
        C\sub1{e}C\sub2{é}C\sub3 & 
        C\sub1{e}C\sub2{é} 
        \tabularnewline
        II & 
        C\sub1{e}C\sub2C\sub2{é}C\sub3 &
        C\sub1{e}C\sub2C\sub2{é} 
        \tabularnewline
        III & 
        {ja}C\sub1C\sub2{e}C\sub3{é} & 
        {ja}C\sub1C\sub2{e}C\sub2{é} 
        \tabularnewline
        IV & 
        {te}C\sub1{e}C\sub2{é}C\sub3	& 
        {te}C\sub1{e}C\sub2C\sub2{é} 
        \tabularnewline
        V & 
        {i}C\sub1C\sub2{e}C\sub3C\sub3{é} & 
        {i}C\sub1C\sub1{e}C\sub2C\sub2{é} 
        \tabularnewline
        VI & 
        {ta}C\sub1C\sub2{e}C\sub3C\sub3{é}	& 
        {ta}C\sub1C\sub1{e}C\sub2C\sub2{é} 
        \tabularnewline
        VII & 
        {sa}C\sub1{e}C\sub2C\sub2{é}C\sub3	& 
        {sa}C\sub1{e}C\sub2C\sub2{é} 
        \tabularnewline
        \hline
      \end{tabular}
    }\\
    \subfloat[Specific nominalisation]{
      \begin{tabular}{|>{\bfseries}fc|-c|-c|}
        \hline
        \SetRowStyle{\bfseries} Root Form & \multicolumn{2}{-c|}{Pattern} \tabularnewline
        \cline{2-3}
        \SetRowStyle{\bfseries} & Triliteral & Biliteral \tabularnewline
        \hline
        I & 
        C\sub1{é}C\sub2{ü}C\sub3 & 
        C\sub1{é}C\sub2{ü} 
        \tabularnewline
        II & 
        C\sub1{é}C\sub2C\sub2{ü}C\sub3 &
        C\sub1{é}C\sub2C\sub2{ü} 
        \tabularnewline
        III & 
        {ja}C\sub1C\sub2{é}C\sub3{ü} & 
        {ja}C\sub1C\sub2{é}C\sub2{ü} 
        \tabularnewline
        IV & 
        {te}C\sub1{é}C\sub2{ü}C\sub3	& 
        {te}C\sub1{é}C\sub2C\sub2{ü} 
        \tabularnewline
        V & 
        {i}C\sub1C\sub2{é}C\sub3C\sub3{ü} & 
        {i}C\sub1C\sub1{é}C\sub2C\sub2{ü} 
        \tabularnewline
        VI & 
        {ta}C\sub1C\sub2{é}C\sub3C\sub3{ü}	& 
        {ta}C\sub1C\sub1{é}C\sub2C\sub2{ü} 
        \tabularnewline
        VII & 
        {sa}C\sub1{é}C\sub2C\sub2{ü}C\sub3	& 
        {sa}C\sub1{é}C\sub2C\sub2{ü} 
        \tabularnewline
        \hline
      \end{tabular}
    }
    \caption{Generic and specific noun forms\label{tab:dev_generic_specific}}
  \end{table}

  \begin{exe}
    \ex \emph{EXAMPLES}
  \end{exe}

  % \subsection{Abstract Nouns}
  % \label{ssec:dev_abstract_nouns}

  % \ToBeWritten

  % \subsection{Collective Nouns, Mass Nouns, and Unit Nouns}
  % \label{ssec:dev_collective_mass_unit}

  % \ToBeWritten

  % \subsection{Dimunitive Nouns}
  % \label{ssec:dev_dimunitive_nouns}

  % \ToBeWritten

  % \subsection{Proper Nouns}
  % \label{ssec:dev_proper_nouns}

  % \ToBeWritten

\end{document}

