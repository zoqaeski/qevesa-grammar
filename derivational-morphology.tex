\documentclass[grammar]{subfiles}
\begin{document}
  \chapter{Derivational Morphology}
  \label{ch:derivational-morphology}

  As a highly synthetic language, derivation plays a major role in the formation of words in Qevesa. Due to its triliteral roots, the majority of words are in fact derived by productive transfixes, suffixes, and prefixes, as well as compounding operations.

  \section{Verb Root Forms}
  \label{sec:dev_verb_root_forms}

  The initial position P\sub0 is normally used for marking verbal root forms; that is, marking subtle variations on a root such as causatives and reflexives. There are 10 forms in total, although not every root can be marked for each form. The consonant root patterns for each form are given in Table~\ref{tab:dev_root_forms}.

  % Note that where forms concatenate consonants to form clusters, in all subsequent derivational and inflectional forms they are treated as a single consonant.

  \begin{table}[htpb]\small\capstart
    \begin{center}
      \begin{tabular}{|>{\bfseries}fc|-c|-c|}
        \hline
        \SetRowStyle{\bfseries} Root Form & \multicolumn{2}{-c|}{Pattern} \tabularnewline
        \cline{2-3}
        \SetRowStyle{\bfseries} & Triliteral & Biliteral \tabularnewline
        \hline
        I & 
        C\sub1\textbf{u}C\sub2\textbf{u}C\sub3 & 
        C\sub1\textbf{u}C\sub2\textbf{u} 
        \tabularnewline
        II & 
        C\sub1\textbf{u}C\sub2C\sub2\textbf{u}C\sub3 &
        C\sub1\textbf{u}C\sub2C\sub2\textbf{u} 
        \tabularnewline
        III & 
        \textbf{ja}C\sub1C\sub2\textbf{u}C\sub3\textbf{u} & 
        \textbf{ja}C\sub1C\sub2\textbf{u}C\sub2\textbf{u} 
        \tabularnewline
        IV & 
        \textbf{ta}C\sub1C\sub1\textbf{u}C\sub2\textbf{u}C\sub3	& 
        \textbf{ta}C\sub1C\sub1\textbf{u}C\sub2\textbf{u} 
        \tabularnewline
        V & 
        \textbf{ina}C\sub1\textbf{u}C\sub2C\sub3\textbf{u} & 
        \textbf{ina}C\sub1\textbf{u}C\sub2C\sub2\textbf{u} 
        \tabularnewline
        VI & 
        \textbf{me}C\sub1\textbf{u}C\sub2C\sub2\textbf{u}C\sub3	& 
        \textbf{me}C\sub1\textbf{u}C\sub2C\sub2\textbf{u} 
        \tabularnewline
        VII & 
        \textbf{i}C\sub1C\sub2\textbf{u}C\sub3C\sub3\textbf{u} & 
        \textbf{is}C\sub1\textbf{u}C\sub2C\sub2\textbf{u} 
        \tabularnewline
        VIII & 
        C\sub1\textbf{u}C\sub2C\sub3\textbf{u} & 
        C\sub1\textbf{u}C\sub2C\sub2\textbf{u} 
        \tabularnewline
        IX & 
        \textbf{e}C\sub1C\sub2\textbf{u}C\sub3\textbf{u} & 
        \textbf{e}C\sub1\textbf{u}C\sub2\textbf{u} 
        \tabularnewline
        \hline
      \end{tabular}
      \caption{Verb root forms\label{tab:dev_root_forms}}
    \end{center}
  \end{table}
  \footnotetext{Triliteral Form IX roots do not exist; the pattern shown in the table is purely hypothetical.}

  \subsection{Form I}
  \label{ssec:dev_verb_form_i}

  Form I is the most basic consonantal root form, containing no preformative affixes or pairing of consonants as occurs in the other forms. It is the closest indicator to the lexical meaning of the root, and has no particular semantic function associated with it.

  \subsection{Form II}
  \label{ssec:dev_verb_form_ii}

  Form II is the “intensive” stem. It typically indicates an intensive or causative meaning, and may also be used to form transitive verbs from intransitive roots.

  It is constructed by geminating the second consonant; a limited number of verbs replace the gemination with two root consonants.

  \subsection{Form III}
  \label{ssec:dev_verb_form_iii}

  Form III is commonly known as the “causative” stem. Its most common function is causative, and it typically converts transitive verbs into ditransitive ones. It can also have a causative meaning on verbs whose Form I root is intransitive, and for some verbs, may convey an assistive or factitive meaning.

  It is formed by pairing the first and second consonants and prefixing \textit{ja-}; biliteral roots duplicate the second consonant in the normal position.

  \subsection{Form IV}
  \label{ssec:dev_verb_form_iv}

  Form IV is the “reciprocal” stem, and is primarily used to make transitive verbs intransitive by adding an implication of reciprocity. Due to its reciprocal nature, many verbs denoting social interactions are found with Form IV stems. 

  It is formed by geminating the first consonant and prefixing with \textit{ta-}.

  %verbs are constructed by prefixing the Form II stem with \textit{me-}. In most cases, Form V is the reflexive variant of Form II.

  \subsection{Form V}
  \label{ssec:dev_verb_form_v}

  Form V is the “reflexive” stem, so called for historical reasons as it also includes a number of other intransitive meanings. True reflexives account for only a portion of the verbs in this form. Its main functions are: 

  \begin{itemize*}
  \item Forming reflexives from transitive roots
  \item Forming verbs denoting accompaniment
  \item Forming causative reflexives
  \item Forming \textit{autoreflexive} verbs, that is, intransitive actions performed on one’s body
  \end{itemize*}

  The only functions which are still fully productive are the forming of reflexives from transitive roots and the verbs of accompaniment. The group of autoreflexives are a closed class, and the causative reflexives are handled in modern Qevesa by Form VI.

  Triliteral roots construct this form by pairing the second and third consonants; biliteral roots construct this form by geminating the second consonant. Both types of root prefix \textit{ina-}.

  \subsection{Form VI}
  \label{ssec:dev_verb_form_vi}

  Form VI is the “reflexive causative” stem. Its primary function is to serve as the reflexive counterpart to the causative form; however, it is prone to semantic and metaphorical drift.

  It is constructed by prefixing the Form II stem with \textit{me-}. 

  %	\subsection{Form VII}
  %	\label{sec:dev_verb_form_vi}
  %
  %	Form VI verbs consist of the Form III stem prefixed with \textit{ina-}. This form indicates the result or consequence of a Form I verb, and is always intransitive.

  \subsection{Form VII}
  \label{ssec:dev_verb_form_vii}

  Form VII is the “attributive” stem, indicating attributes, physical traits, or colours, and is alway intransitive. It is often used as the base form from which adjectives may be derived.

  Triliteral roots construct this form by pairing the first and second consonants, geminating the third consonant, and prefixing with \textit{i-}. Biliteral roots construct this form by geminating the second consonant and prefixing with \textit{is-}.

  \subsection{Other Verb Forms}
  \label{ssec:dev_other_verb_forms}

  There are a small number of irregular stem forms, mainly used to construct auxiliary verbs, such as those indicating evidentiality or polarity. They are effectively a closed class, and are generally not productive.

  \subsubsection{Form VIII}
  \label{sssec:dev_verb_form_viii}

  Form VIII are constructed by pairing the second and third consonants. Few biliteral roots possess this form; those that do geminate the second consonant.

  \subsubsection{Form IX}
  \label{sssec:dev_verb_form_ix}

  Form IX are constructed by prefixing the Form I stem with \textit{e-}. Triliteral roots pair the first and second consonants; biliteral roots are otherwise unchanged.

  \section{Nominalisation}
  \label{sec:dev_nominalisation}

  Most Qevesa nouns are derived from biliteral, triliteral or quadriliteral lexical roots, and all nouns derived from a particular root are listed in a dictionary under that root entry. Some nouns, however, have solid stems, unanalysable into roots and patterns, although their consonants may be adapted into roots for derivation of new terms. Derived nouns are formed through application of particular morphological patterns; the use of patterns interlocking with root phonemes allows the formation of actual words or stems. The nominal patterns themselves carry meaning, such as “place where action is performed,” “person who performs action,” “name of action,” or “instrument used to carry out action.” The most frequently occurring noun patterns are listed in the following sections.

  It is important to note that not all root forms have all nominalisation patterns.

  \subsection{Verbal Nouns}
  \label{ssec:dev_verbal_nouns}

  Verbal nouns are systematically related to verb forms. The verbal noun names the action denoted by its corresponding verb; they are often abstract in meaning, but some of them have specific, concrete reference. Verbal nouns may also express infinitive forms. Typically, the infinitive verbal noun is the citation form of a root.

  The verbal noun pattern is generally formed in a variety of ways, depending on the verbal root form. The most common paradigms are given in Table~\ref{tab:dev_verbal_nouns}; most forms suffix \textit{-a} and many forms may prefix \textit{ja-}.

  \begin{table}[htpb]\small\capstart
    \begin{center}
      \begin{tabular}{|>{\bfseries}fc|-c|-c|}
        \hline
        \SetRowStyle{\bfseries} Root Form & \multicolumn{2}{-c|}{Pattern} \tabularnewline
        \cline{2-3}
        \SetRowStyle{\bfseries} & Triliteral & Biliteral \tabularnewline
        \hline
        I & 
        C\sub1\textbf{u}C\sub2\textbf{u}C\sub3\textbf{a} & 
        C\sub1\textbf{u}C\sub2\textbf{a} 
        \tabularnewline
        II & 
        C\sub1\textbf{u}C\sub2C\sub2\textbf{u}C\sub3\textbf{a} &
        C\sub1\textbf{u}C\sub2C\sub2\textbf{ua} 
        \tabularnewline
        III & 
        \textbf{ja}C\sub1C\sub2\textbf{u}C\sub3\textbf{a} & 
        \textbf{ja}C\sub1C\sub2\textbf{u}C\sub2\textbf{a} 
        \tabularnewline
        IV & 
        \textbf{ta}C\sub1C\sub1\textbf{u}C\sub2\textbf{u}C\sub3\textbf{a}	& 
        \textbf{ta}C\sub1C\sub1\textbf{u}C\sub2\textbf{a}
        \tabularnewline
        V & 
        \textbf{ina}C\sub1\textbf{u}C\sub2C\sub3\textbf{a} & 
        \textbf{ina}C\sub1\textbf{u}C\sub2C\sub2\textbf{a} 
        \tabularnewline
        VI & 
        \textbf{me}C\sub1\textbf{u}C\sub2C\sub2\textbf{u}C\sub3\textbf{a}	& 
        \textbf{me}C\sub1\textbf{u}C\sub2C\sub2\textbf{a} 
        \tabularnewline
        VII & 
        \textbf{i}C\sub1C\sub2\textbf{u}C\sub3C\sub3\textbf{a} & 
        \textbf{is}C\sub1\textbf{u}C\sub2C\sub2\textbf{a} 
        \tabularnewline
        VIII & 
        C\sub1\textbf{u}C\sub2C\sub3\textbf{a} & 
        C\sub1\textbf{u}C\sub2C\sub2\textbf{a} 
        \tabularnewline
        \hline
      \end{tabular}
      \caption{Verbal noun paradigms\label{tab:dev_verbal_nouns}}
    \end{center}
  \end{table}

  \begin{exe}
    \ex \emph{EXAMPLES}
  \end{exe}

  \subsection{Active and Passive Participles}
  \label{ssec:dev_active_passive_participles}

  Participles are descriptive terms derived from verbs. The active participle describes the doer or the agent of the action, and the passive participle describes or refers to the object or patient of the action. Both participles are predictably derived according to the verbal root forms; the most common patterns are listed in Table~\ref{tab:dev_nominal_participles}.

  \begin{table}[htpb]\small\capstart
    \begin{center}
      \subfloat[Active participles]{
        \begin{tabular}{|>{\bfseries}fc|-c|-c|}
          \hline
          \SetRowStyle{\bfseries} Root Form & \multicolumn{2}{-c|}{Pattern} \tabularnewline
          \cline{2-3}
          \SetRowStyle{\bfseries} & Triliteral & Biliteral \tabularnewline
          \hline
          I & 
          C\sub1\textbf{a}C\sub2\textbf{oi}C\sub3 & 
          C\sub1\textbf{a}C\sub2\textbf{oi} 
          \tabularnewline
          II & 
          C\sub1\textbf{a}C\sub2C\sub2\textbf{oi}C\sub3 &
          C\sub1\textbf{a}C\sub2C\sub2\textbf{oi} 
          \tabularnewline
          III & 
          \textbf{ja}C\sub1C\sub2\textbf{a}C\sub3\textbf{oi} & 
          \textbf{ja}C\sub1C\sub2\textbf{a}C\sub2\textbf{oi} 
          \tabularnewline
          IV & 
          \textbf{ta}C\sub1C\sub1\textbf{a}C\sub2\textbf{oi}C\sub3	& 
          \textbf{ta}C\sub1C\sub1\textbf{a}C\sub2\textbf{oi}
          \tabularnewline
          V & 
          \textbf{ina}C\sub1\textbf{a}C\sub2C\sub3\textbf{oi} & 
          \textbf{ina}C\sub1\textbf{a}C\sub2C\sub2\textbf{oi} 
          \tabularnewline
          VI & 
          \textbf{me}C\sub1\textbf{a}C\sub2C\sub2\textbf{oi}C\sub3	& 
          \textbf{me}C\sub1\textbf{a}C\sub2C\sub2\textbf{oi} 
          \tabularnewline
          VII & 
          \textbf{i}C\sub1C\sub2\textbf{a}C\sub3C\sub3\textbf{oi} & 
          \textbf{is}C\sub1\textbf{a}C\sub2C\sub2\textbf{oi} 
          \tabularnewline
          VIII & 
          C\sub1\textbf{a}C\sub2C\sub3\textbf{oi} & 
          C\sub1\textbf{a}C\sub2C\sub2\textbf{oi} 
          \tabularnewline
          \hline
        \end{tabular}}\\
      \subfloat[Passive participles]{
        \begin{tabular}{|>{\bfseries}fc|-c|-c|}
          \hline
          \SetRowStyle{\bfseries} Root Form & \multicolumn{2}{-c|}{Pattern} \tabularnewline
          \cline{2-3}
          \SetRowStyle{\bfseries} & Triliteral & Biliteral \tabularnewline
          \hline
          I & 
          C\sub1\textbf{o}C\sub2\textbf{i}C\sub3 & 
          C\sub1\textbf{o}C\sub2\textbf{i} 
          \tabularnewline
          II & 
          C\sub1\textbf{o}C\sub2C\sub2\textbf{i}C\sub3 &
          C\sub1\textbf{o}C\sub2C\sub2\textbf{i} 
          \tabularnewline
          III & 
          \textbf{ja}C\sub1C\sub2\textbf{o}C\sub3\textbf{i} & 
          \textbf{ja}C\sub1C\sub2\textbf{o}C\sub2\textbf{i} 
          \tabularnewline
          IV & 
          \textbf{ta}C\sub1C\sub1\textbf{o}C\sub2\textbf{i}C\sub3	& 
          \textbf{ta}C\sub1C\sub1\textbf{o}C\sub2\textbf{i}
          \tabularnewline
          V & 
          \textbf{ina}C\sub1\textbf{o}C\sub2C\sub3\textbf{i} & 
          \textbf{ina}C\sub1\textbf{o}C\sub2C\sub2\textbf{i} 
          \tabularnewline
          VI & 
          \textbf{me}C\sub1\textbf{o}C\sub2C\sub2\textbf{i}C\sub3	& 
          \textbf{me}C\sub1\textbf{o}C\sub2C\sub2\textbf{i} 
          \tabularnewline
          VII & 
          \textbf{i}C\sub1C\sub2\textbf{o}C\sub3C\sub3\textbf{i} & 
          \textbf{is}C\sub1\textbf{o}C\sub2C\sub2\textbf{i} 
          \tabularnewline
          VIII & 
          C\sub1\textbf{o}C\sub2C\sub3\textbf{i} & 
          C\sub1\textbf{o}C\sub2C\sub2\textbf{i} 
          \tabularnewline
          \hline
        \end{tabular}}
      \caption{Nominal participles\label{tab:dev_nominal_participles}}
    \end{center}
  \end{table}

  %\subsubsection{Active Participle}

  %\ToBeWritten

  \subsection{Location}
  \label{ssec:dev_nouns_location}

  Another noun pattern specifies the location in which an action is performed. Only Forms I–VI have locative patterns, and of these, not all are productive or even valid. The patterns for location are given in Table~\ref{tab:dev_nominal_location}.

  \begin{table}[htpb]\small\capstart
    \begin{center}
      \begin{tabular}{|>{\bfseries}fc|-c|-c|}
        \hline
        \SetRowStyle{\bfseries} Root Form & \multicolumn{2}{-c|}{Pattern} \tabularnewline
        \cline{2-3}
        \SetRowStyle{\bfseries} & Triliteral & Biliteral \tabularnewline
        \hline
        I & 
        \textbf{a}C\sub1C\sub2\textbf{e}C\sub3 & 
        \textbf{a}C\sub1\textbf{e}C\sub2
        \tabularnewline
        II & 
        C\sub1\textbf{a}C\sub2C\sub2\textbf{e}C\sub3 &
        C\sub1\textbf{a}C\sub2C\sub2\textbf{e} 
        \tabularnewline
        III & 
        \textbf{ja}C\sub1C\sub2\textbf{é}C\sub3 & 
        \textbf{ja}C\sub1C\sub2\textbf{é}C\sub2 
        \tabularnewline
        IV & 
        \textbf{ta}C\sub1C\sub1\textbf{a}C\sub2\textbf{e}C\sub3	& 
        \textbf{ta}C\sub1C\sub1\textbf{a}C\sub2\textbf{e}
        \tabularnewline
        V & 
        \textbf{ina}C\sub1\textbf{a}C\sub2C\sub3\textbf{e} & 
        \textbf{ina}C\sub1\textbf{a}C\sub2C\sub2\textbf{e} 
        \tabularnewline
        VI & 
        \textbf{me}C\sub1\textbf{a}C\sub2C\sub2\textbf{e}C\sub3	& 
        \textbf{me}C\sub1\textbf{a}C\sub2C\sub2\textbf{e} 
        \tabularnewline
        \hline
      \end{tabular}
      \caption{Nouns of location\label{tab:dev_nominal_location}}
    \end{center}
  \end{table}

  Some examples:

  \begin{exe}
    \ex \emph{EXAMPLES}
  \end{exe}

  \subsection{Instrument}
  \label{ssec:dev_nouns_instrument}

  A specific derivational pattern is used to indicate nouns of instrument; that is, nouns that denote items used in accomplishing a particular action. These patterns are only used with Forms I–V, and are listed in Table~\ref{tab:dev_nominal_instrument}.

  \begin{table}[htpb]\small\capstart
    \begin{center}
      \begin{tabular}{|>{\bfseries}fc|-c|-c|}
        \hline
        \SetRowStyle{\bfseries} Root Form & \multicolumn{2}{-c|}{Pattern} \tabularnewline
        \cline{2-3}
        \SetRowStyle{\bfseries} & Triliteral & Biliteral \tabularnewline
        \hline
        I & 
        C\sub1\textbf{i}C\sub2\textbf{á}C\sub3 & 
        C\sub1\textbf{i}C\sub2\textbf{ait} 
        \tabularnewline
        II & 
        C\sub1\textbf{i}C\sub2C\sub2\textbf{á}C\sub3 &
        C\sub1\textbf{i}C\sub2C\sub2\textbf{ait} 
        \tabularnewline
        III & 
        \textbf{ja}C\sub1C\sub2\textbf{i}C\sub3\textbf{ait} & 
        \textbf{ja}C\sub1C\sub2\textbf{i}C\sub2\textbf{ait} 
        \tabularnewline
        IV & 
        \textbf{ta}C\sub1C\sub1\textbf{i}C\sub2\textbf{á}C\sub3	& 
        \textbf{ta}C\sub1C\sub1\textbf{i}C\sub2\textbf{ait} 
        \tabularnewline
        V & 
        \textbf{ina}C\sub1\textbf{i}C\sub2C\sub3\textbf{ait} & 
        \textbf{ina}C\sub1\textbf{i}C\sub2C\sub2\textbf{ait} 
        \tabularnewline
        \hline
      \end{tabular}
      \caption{Nouns of instrument\label{tab:dev_nominal_instrument}}
    \end{center}
  \end{table}

  Some examples:

  \begin{exe}
    \ex \emph{EXAMPLES}
  \end{exe}

  \subsection{Intensity, Repetition, Profession}
  \label{ssec:dev_nouns_intensity_repetition_profession}

  A noun pattern exists to denote intensity or repeated actions; it also often denotes professions. The basic patterns are \textit{C\sub1oC\sub2C\sub2áC\sub3} and \textit{C\sub1oC\sub2C\sub2á}:

  \begin{exe}
    \ex \emph{EXAMPLES}
  \end{exe}

  The abstract noun denoting the name of a profession is often given by the patterns \textit{C\sub1oiC\sub2C\sub2áC\sub3} and \textit{C\sub1oiC\sub2C\sub2á}:

  \begin{exe}
    \ex \emph{EXAMPLES}
  \end{exe}

  \subsection{Common Nouns}
  \label{ssec:dev_common_nouns}

  \ToBeWritten

  \subsection{Generic and Specific Nouns}
  \label{ssec:dev_generic_nouns}

  The generic noun is a general nominalisation which represents the concept, process, activity or ability denoted by the root. This contrasts with the pattern that denotes a specific instance of the generic concept. Both patterns are related, and in many cases, the specific pattern is itself a derivation of the generic pattern. The patterns are listed in Table~\ref{tab:dev_generic_specific}.

  \begin{table}
    \begin{center}
      \subfloat[Generic nominalisation]{
        \begin{tabular}{|>{\bfseries}fc|-c|-c|}
          \hline
          \SetRowStyle{\bfseries} Root Form & \multicolumn{2}{-c|}{Pattern} \tabularnewline
          \cline{2-3}
          \SetRowStyle{\bfseries} & Triliteral & Biliteral \tabularnewline
          \hline
          I & 
          C\sub1\textbf{e}C\sub2\textbf{é}C\sub3 & 
          C\sub1\textbf{e}C\sub2\textbf{é} 
          \tabularnewline
          II & 
          C\sub1\textbf{e}C\sub2C\sub2\textbf{é}C\sub3 &
          C\sub1\textbf{e}C\sub2C\sub2\textbf{é} 
          \tabularnewline
          III & 
          \textbf{ja}C\sub1C\sub2\textbf{e}C\sub3\textbf{é} & 
          \textbf{ja}C\sub1C\sub2\textbf{e}C\sub2\textbf{é} 
          \tabularnewline
          IV & 
          \textbf{ta}C\sub1C\sub1\textbf{e}C\sub2\textbf{é}C\sub3	& 
          \textbf{ta}C\sub1C\sub1\textbf{e}C\sub2\textbf{é} 
          \tabularnewline
          V & 
          \textbf{ina}C\sub1\textbf{e}C\sub2C\sub3\textbf{é} & 
          \textbf{ina}C\sub1\textbf{e}C\sub2C\sub2\textbf{é} 
          \tabularnewline
          VI & 
          \textbf{me}C\sub1\textbf{e}C\sub2C\sub2\textbf{é}C\sub3	& 
          \textbf{me}C\sub1\textbf{e}C\sub2C\sub2\textbf{é} 
          \tabularnewline
          VII & 
          \textbf{i}C\sub1C\sub2\textbf{e}C\sub3C\sub3\textbf{é} & 
          \textbf{is}C\sub1\textbf{e}C\sub2C\sub2\textbf{é} 
          \tabularnewline
          VIII & 
          C\sub1\textbf{e}C\sub2C\sub3\textbf{é} & 
          C\sub1\textbf{e}C\sub2C\sub2\textbf{é} 
          \tabularnewline
          \hline
        \end{tabular}}\\
      \subfloat[Specific nominalisation]{
        \begin{tabular}{|>{\bfseries}fc|-c|-c|}
          \hline
          \SetRowStyle{\bfseries} Root Form & \multicolumn{2}{-c|}{Pattern} \tabularnewline
          \cline{2-3}
          \SetRowStyle{\bfseries} & Triliteral & Biliteral \tabularnewline
          I & 
          C\sub1\textbf{é}C\sub2\textbf{a}C\sub3 & 
          C\sub1\textbf{é}C\sub2\textbf{a} 
          \tabularnewline
          II & 
          C\sub1\textbf{é}C\sub2C\sub2\textbf{a}C\sub3 &
          C\sub1\textbf{é}C\sub2C\sub2\textbf{a} 
          \tabularnewline
          III & 
          \textbf{ja}C\sub1C\sub2\textbf{é}C\sub3\textbf{a} & 
          \textbf{ja}C\sub1C\sub2\textbf{é}C\sub2\textbf{a} 
          \tabularnewline
          IV & 
          \textbf{ta}C\sub1C\sub1\textbf{é}C\sub2\textbf{a}C\sub3	& 
          \textbf{ta}C\sub1C\sub1\textbf{é}C\sub2\textbf{a} 
          \tabularnewline
          V & 
          \textbf{ina}C\sub1\textbf{é}C\sub2C\sub3\textbf{a} & 
          \textbf{ina}C\sub1\textbf{é}C\sub2C\sub2\textbf{a} 
          \tabularnewline
          VI & 
          \textbf{me}C\sub1\textbf{é}C\sub2C\sub2\textbf{a}C\sub3	& 
          \textbf{me}C\sub1\textbf{é}C\sub2C\sub2\textbf{a} 
          \tabularnewline
          VII & 
          \textbf{i}C\sub1C\sub2\textbf{é}C\sub3C\sub3\textbf{a} & 
          \textbf{is}C\sub1\textbf{é}C\sub2C\sub2\textbf{a} 
          \tabularnewline
          VIII & 
          C\sub1\textbf{é}C\sub2C\sub3\textbf{a} & 
          C\sub1\textbf{é}C\sub2C\sub2\textbf{a} 
          \tabularnewline
          \hline
        \end{tabular}}
      \caption{Generic and specific noun forms\label{tab:dev_generic_specific}}
    \end{center}
  \end{table}

  \begin{exe}
    \ex \emph{EXAMPLES}
  \end{exe}

  % \subsection{Abstract Nouns}
  % \label{ssec:dev_abstract_nouns}

  % \ToBeWritten

  % \subsection{Collective Nouns, Mass Nouns, and Unit Nouns}
  % \label{ssec:dev_collective_mass_unit}

  % \ToBeWritten

  % \subsection{Dimunitive Nouns}
  % \label{ssec:dev_dimunitive_nouns}

  % \ToBeWritten

  % \subsection{Proper Nouns}
  % \label{ssec:dev_proper_nouns}

  % \ToBeWritten

\end{document}

