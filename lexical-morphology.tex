\documentclass[grammar]{subfiles}
\begin{document}
\chapter{Derivational Morphology}
\label{ch:lexical-morphology}

As a highly synthetic language, derivation plays a major role in the formation of words in Qevesa.  Due to its triliteral roots, the majority of words are in fact derived by productive transfixes, suffixes, and prefixes, as well as compounding operations.

\section{Nominalisation}
\label{sec:dev_nominalisation}

Most Qevesa nouns are derived from biliteral, triliteral or quadriliteral lexical roots, and all nouns derived from a particular root are listed in a dictionary under that root entry.  Some nouns, however, have solid stems, unanalysable into roots and patterns, although their consonants may be adapted into roots for derivation of new terms.  Derived nouns are formed through application of particular morphological patterns; the use of patterns interlocking with root phonemes allows the formation of actual words or stems.  The nominal patterns themselves carry meaning, such as “place where action is performed,” “person who performs action,” “name of action,” or “instrument used to carry out action.” The most frequently occurring noun patterns are listed in the following sections.

\subsection{Verbal Nouns}
\label{ssec:dev_verbal_nouns}

\tbw

\subsection{Active and Passive Participles}
\label{ssec:dev_active_passive_participles}

\tbw

\subsection{Instrument}
\label{ssec:dev_nouns_instrument}

\tbw

\subsection{Intensity, Repetition, Profession}
\label{ssec:dev_nouns_intensity_repetition_profession}

\tbw

\subsection{Common Nouns}
\label{ssec:dev_common_nouns}

\tbw

\subsection{Generic Nouns}
\label{ssec:dev_generic_nouns}

\tbw

\subsection{Dimunitive Nouns}
\label{ssec:dev_dimunitive_nouns}

\tbw

\subsection{Abstract Nouns}
\label{ssec:dev_abstract_nouns}

\tbw

\subsection{Collective Nouns, Mass Nouns, and Unit Nouns}
\label{ssec:dev_collective_mass_unit}

\tbw

\subsection{Proper Nouns}
\label{ssec:dev_proper_nouns}

\tbw

\end{document}

