\documentclass[grammar]{subfiles}
\begin{document}
	\chapter{Verbal Morphology}
	\label{ch:verbal_morphology}

	\section{Features}
	\label{sec:vm_features}

	The consonantal root patterns in Qevesa are used to form basic morphological paradigms. Qevesa verbs are highly inflected, indicating aspect by transfix patterns; topical agreement and modality are marked by agglutinative suffixes. All other constructions, including tense, voice, polarity and evidentiality, are indicated by periphrasis or syntax.

	The stem consists of the root and zero or more derivational affixes conjugated to a particular aspect. 

	\section{The Infinitive}
	\label{sec:vm_infinitive}

	The infinitive of the verb is typically used as the citation form of the root. It is formed by inserting a \emph{-u-} into P\sub{12} and P\sub{23}, resulting in \emph{C\sub1uC\sub2uC\sub3}. Unlike the other verb constructions, it never conjugates. However, it is used as the root for a number of additional constructions, such as some honorific registers.

	\section{Conjugation}
	\label{sec:vm_conjugation}

	Qevesa is a highly synthetic language, and verbs are conjugated to indicate aspect, topical agreement, and mood. The conjugated form of the verb is as follows:

	\begin{exe}
		\ex\label{exe:vm_conjugation} \textit{stem}\textsc{.aspect-topic-mood}
	\end{exe}

	\subsection{Aspect}
	\label{ssec:vm_aspect}

	Aspect is possibly the most important grammatical category marked on the verb. Instead of tense, aspect is used to mark the temporal flow (or lack thereof) of verbs. Qevesa distinguishes between imperfective aspects (those that are ongoing, habitual, repeated or generally containing internal structure) and perfective aspects (those that are viewed as a single whole). As a result, there are two primary transfix patterns that correspond to the imperfective and perfective aspects, and a number of secondary transfix patterns which indicate various subtle (mainly semantic) differences.	
	
	There are ten different aspects in total, five imperfective and five perfective.

	\subsubsection{The Imperfective Aspects}
	\label{sssec:vm_imperfective}

	The imperfective aspects are used to indicate:

	\begin{itemize*}
		\item actions in progress or ongoing states and activities, with significant course (in opinion of the speaker);
		\item activities posing the background for other (perfective) activities;
		\item simultaneous activities;
		\item durative activities, lasting through some time;
		\item multiple (iterative or frequentative) activities;
		\item habitual activities;
		\item motions without a strict aim;
		\item continuous states.
	\end{itemize*}

	The triliteral root patterns for the imperfective aspects are given in Table~\ref{tab:vm_imperfective_aspects}.

	\begin{table}[htpb]\small\capstart
		\begin{center}
			\subfloat[][Triliteral roots]{
			\begin{tabular}{|>{\bfseries}fc|-c|-c|-c|-c|-c|}
				\hline
				\SetRowStyle{\bfseries} Form & Imperfective & Stative & Durative & Frequentative & Habitual \tabularnewline
				\cline{2-6}
				\SetRowStyle{\scshape} & ipfv & stat & dur;pfv & freq & hab \tabularnewline
				\hline
				I & 
				C\sub1\textbf{u}C\sub2\textbf{i}C\sub3 & 
				C\sub1\textbf{ui}C\sub2\textbf{e}C\sub3 & 
				C\sub1\textbf{u}C\sub2\textbf{ú}C\sub3 & 
				C\sub1\textbf{u}C\sub2\textbf{o}C\sub3 & 
				C\sub1\textbf{u}C\sub2\textbf{a}C\sub3
				\tabularnewline
				II & 
				C\sub1\textbf{u}C\sub2C\sub2\textbf{i}C\sub3 & 
				C\sub1\textbf{ui}C\sub2C\sub2\textbf{e}C\sub3 & 
				C\sub1\textbf{u}C\sub2C\sub2\textbf{ú}C\sub3 & 
				C\sub1\textbf{u}C\sub2C\sub2\textbf{o}C\sub3 & 
				C\sub1\textbf{u}C\sub2C\sub2\textbf{a}C\sub3
				\tabularnewline
				III & 
				\textbf{ja}C\sub1C\sub2\textbf{u}C\sub3\textbf{i} & 
				\textbf{ja}C\sub1C\sub2\textbf{ui}C\sub3\textbf{e} & 
				\textbf{ja}C\sub1C\sub2\textbf{u}C\sub3\textbf{ú} & 
				\textbf{ja}C\sub1C\sub2\textbf{u}C\sub3\textbf{o} & 
				\textbf{ja}C\sub1C\sub2\textbf{u}C\sub3\textbf{a}
				\tabularnewline
				IV & 
				\textbf{me}C\sub1\textbf{u}C\sub2C\sub2\textbf{i}C\sub3 & 
				\textbf{me}C\sub1\textbf{ui}C\sub2C\sub2\textbf{e}C\sub3	& 
				\textbf{me}C\sub1\textbf{u}C\sub2C\sub2\textbf{ú}C\sub3 & 
				\textbf{me}C\sub1\textbf{u}C\sub2C\sub2\textbf{o}C\sub3 & 
				\textbf{me}C\sub1\textbf{u}C\sub2C\sub2\textbf{a}C\sub3  
				\tabularnewline
				V & 
				\textbf{te}C\sub1C\sub1\textbf{u}C\sub2\textbf{i}C\sub3 & 
				\textbf{te}C\sub1C\sub1\textbf{ui}C\sub2\textbf{e}C\sub3 & 
				\textbf{te}C\sub1C\sub1\textbf{u}C\sub2\textbf{ú}C\sub3 & 
				\textbf{te}C\sub1C\sub1\textbf{u}C\sub2\textbf{o}C\sub3 & 
				\textbf{te}C\sub1C\sub1\textbf{u}C\sub2\textbf{a}C\sub3
				\tabularnewline
				VI & 
				\textbf{ina}C\sub1C\sub2\textbf{u}C\sub3\textbf{i} & 
				\textbf{ina}C\sub1C\sub2\textbf{ui}C\sub3\textbf{e} & 
				\textbf{ina}C\sub1C\sub2\textbf{u}C\sub3\textbf{ú} & 
				\textbf{ina}C\sub1C\sub2\textbf{u}C\sub3\textbf{o} & 
				\textbf{ina}C\sub1C\sub2\textbf{u}C\sub3\textbf{a}
				\tabularnewline
				VII & 
				\textbf{i}C\sub1C\sub2\textbf{u}C\sub3C\sub3\textbf{i} & 
				\textbf{i}C\sub1C\sub2\textbf{ui}C\sub3C\sub3\textbf{e} & 
				\textbf{i}C\sub1C\sub2\textbf{u}C\sub3C\sub3\textbf{ú} & 
				\textbf{i}C\sub1C\sub2\textbf{u}C\sub3C\sub3\textbf{o} & 
				\textbf{i}C\sub1C\sub2\textbf{u}C\sub3C\sub3\textbf{a}
				\tabularnewline
				VIII & 
				C\sub1\textbf{u}C\sub2C\sub3\textbf{i} & 
				C\sub1\textbf{ui}C\sub2C\sub3\textbf{e} & 
				C\sub1\textbf{u}C\sub2C\sub3\textbf{ú} & 
				C\sub1\textbf{u}C\sub2C\sub3\textbf{o} & 
				C\sub1\textbf{u}C\sub2C\sub3\textbf{a}
				\tabularnewline
				\hline
			\end{tabular}}\\
		\subfloat[][Biliteral roots]{
			\begin{tabular}{|>{\bfseries}fc|-c|-c|-c|-c|-c|}
				\hline
				\SetRowStyle{\bfseries} Form & Imperfective & Stative & Durative & Frequentative & Habitual \tabularnewline
				\cline{2-6}
				\SetRowStyle{\scshape} & ipfv & stat & dur;pfv & freq & hab \tabularnewline
				\hline
				I & 
				C\sub1\textbf{u}C\sub2\textbf{i} & 
				C\sub1\textbf{ui}C\sub2\textbf{e} & 
				C\sub1\textbf{u}C\sub2\textbf{ú} & 
				C\sub1\textbf{u}C\sub2\textbf{o} & 
				C\sub1\textbf{u}C\sub2\textbf{a}
				\tabularnewline
				II & 
				C\sub1\textbf{u}C\sub2C\sub2\textbf{i} & 
				C\sub1\textbf{ui}C\sub2C\sub2\textbf{e} & 
				C\sub1\textbf{u}C\sub2C\sub2\textbf{ú} & 
				C\sub1\textbf{u}C\sub2C\sub2\textbf{o} & 
				C\sub1\textbf{u}C\sub2C\sub2\textbf{a}
				\tabularnewline
				III & 
				\textbf{ja}C\sub1C\sub2\textbf{u}C\sub2\textbf{i} & 
				\textbf{ja}C\sub1C\sub2\textbf{ui}C\sub2\textbf{e} & 
				\textbf{ja}C\sub1C\sub2\textbf{u}C\sub2\textbf{ú} & 
				\textbf{ja}C\sub1C\sub2\textbf{u}C\sub2\textbf{o} & 
				\textbf{ja}C\sub1C\sub2\textbf{u}C\sub2\textbf{a}
				\tabularnewline
				IV & 
				\textbf{me}C\sub1\textbf{u}C\sub2C\sub2\textbf{i} & 
				\textbf{me}C\sub1\textbf{ui}C\sub2C\sub2\textbf{e}	& 
				\textbf{me}C\sub1\textbf{u}C\sub2C\sub2\textbf{ú} & 
				\textbf{me}C\sub1\textbf{u}C\sub2C\sub2\textbf{o} & 
				\textbf{me}C\sub1\textbf{u}C\sub2C\sub2\textbf{a}  
				\tabularnewline
				V & 
				\textbf{te}C\sub1C\sub1\textbf{u}C\sub2\textbf{i} & 
				\textbf{te}C\sub1C\sub1\textbf{ui}C\sub2\textbf{e} & 
				\textbf{te}C\sub1C\sub1\textbf{u}C\sub2\textbf{ú} & 
				\textbf{te}C\sub1C\sub1\textbf{u}C\sub2\textbf{o} & 
				\textbf{te}C\sub1C\sub1\textbf{u}C\sub2\textbf{a}
				\tabularnewline
				VI & 
				\textbf{ina}C\sub1C\sub2\textbf{u}C\sub2\textbf{i} & 
				\textbf{ina}C\sub1C\sub2\textbf{ui}C\sub2\textbf{e} & 
				\textbf{ina}C\sub1C\sub2\textbf{u}C\sub2\textbf{ú} & 
				\textbf{ina}C\sub1C\sub2\textbf{u}C\sub2\textbf{o} & 
				\textbf{ina}C\sub1C\sub2\textbf{u}C\sub2\textbf{a}
				\tabularnewline
				VII & 
				\textbf{i}C\sub1\textbf{u}C\sub2C\sub2\textbf{i} & 
				\textbf{i}C\sub1\textbf{ui}C\sub2C\sub2\textbf{e} & 
				\textbf{i}C\sub1\textbf{u}C\sub2C\sub2\textbf{ú} & 
				\textbf{i}C\sub1\textbf{u}C\sub2C\sub2\textbf{o} & 
				\textbf{i}C\sub1\textbf{u}C\sub2C\sub2\textbf{a}
				\tabularnewline
				VIII & 
				C\sub1\textbf{u}C\sub2C\sub2\textbf{i} & 
				C\sub1\textbf{ui}C\sub2C\sub2\textbf{e} & 
				C\sub1\textbf{u}C\sub2C\sub2\textbf{ú} & 
				C\sub1\textbf{u}C\sub2C\sub2\textbf{o} & 
				C\sub1\textbf{u}C\sub2C\sub2\textbf{a}
				\tabularnewline
				IX & 
				\textbf{i}C\sub1\textbf{u}C\sub2\textbf{i} & 
				\textbf{i}C\sub1\textbf{ui}C\sub2\textbf{e} & 
				\textbf{i}C\sub1\textbf{u}C\sub2\textbf{ú} & 
				\textbf{i}C\sub1\textbf{u}C\sub2\textbf{o} & 
				\textbf{i}C\sub1\textbf{u}C\sub2\textbf{a}
				\tabularnewline
				\hline
			\end{tabular}}
			\caption{Imperfective aspectual patterns\label{tab:vm_imperfective_aspects}}
		\end{center}
	\end{table}

	% \newpage
	\subsubsection{The Perfective Aspects}
	\label{sssec:vm_perfective}

	The perfective aspects generally indicate activities that have distinct beginnings and ends which are relevant to the speaker. This implies past or future activities, but not present activities—an activity which is presently occurring cannot be ended, so it cannot be perfective. The perfective indicates the following:

	\begin{itemize*}
		\item states and activities which were ended or which will be ended, with insignificant course, or treated as a whole by the speaker;
		\item single-time activities;
		% \item actions whose goals have already been achieved;
		% \item reasons for the state;
		\item the beginning of the activity or the state;
		\item the end of the activity or the state;
		\item activities executed in many places, on many objects or by many subjects at the same time;
		\item actions or states which last some time
	\end{itemize*}

	The triliteral root patterns for the perfective aspects are given in Table~\ref{tab:vm_perfective_aspects}.

	\begin{table}[htpb]\small\capstart
		\begin{center}
			\subfloat[][Triliteral roots]{
			\begin{tabular}{|>{\bfseries}fc|-c|-c|-c|-c|-c|}
				\hline
				\SetRowStyle{\bfseries} Form & Perfective & Inchoative & Cessative & Durative & Momentane \tabularnewline
				\cline{2-6}
				\SetRowStyle{\scshape} & pfv & inch & cess & dur;pfv & momt \tabularnewline
				\hline
				I & 
				C\sub1\textbf{i}C\sub2\textbf{o}C\sub3\textbf{a} & 
				C\sub1\textbf{i}C\sub2\textbf{u}C\sub3\textbf{o} & 
				C\sub1\textbf{i}C\sub2\textbf{a}C\sub3\textbf{a} & 
				C\sub1\textbf{i}C\sub2\textbf{a}C\sub3\textbf{u} & 
				C\sub1\textbf{i}C\sub2\textbf{u}C\sub3\textbf{a}
				\tabularnewline
				II & 
				C\sub1\textbf{i}C\sub2C\sub2\textbf{o}C\sub3\textbf{a} & 
				C\sub1\textbf{i}C\sub2C\sub2\textbf{u}C\sub3\textbf{o} & 
				C\sub1\textbf{i}C\sub2C\sub2\textbf{a}C\sub3\textbf{a} & 
				C\sub1\textbf{i}C\sub2C\sub2\textbf{a}C\sub3\textbf{u} & 
				C\sub1\textbf{i}C\sub2C\sub2\textbf{u}C\sub3\textbf{a}
				\tabularnewline
				III & 
				\textbf{ja}C\sub1C\sub2\textbf{io}C\sub3\textbf{a} & 
				\textbf{ja}C\sub1C\sub2\textbf{iu}C\sub3\textbf{o} & 
				\textbf{ja}C\sub1C\sub2\textbf{í}C\sub3\textbf{a} & 
				\textbf{ja}C\sub1C\sub2\textbf{ia}C\sub3\textbf{u} & 
				\textbf{ja}C\sub1C\sub2\textbf{iu}C\sub3\textbf{a}
				\tabularnewline
				IV & 
				\textbf{me}C\sub1\textbf{i}C\sub2C\sub2\textbf{o}C\sub3\textbf{a}	& 
				\textbf{me}C\sub1\textbf{i}C\sub2C\sub2\textbf{u}C\sub3\textbf{o}	& 
				\textbf{me}C\sub1\textbf{i}C\sub2C\sub2\textbf{a}C\sub3\textbf{a}	& 
				\textbf{me}C\sub1\textbf{i}C\sub2C\sub2\textbf{a}C\sub3\textbf{u}	& 
				\textbf{me}C\sub1\textbf{i}C\sub2C\sub2\textbf{u}C\sub3\textbf{a}	 
				\tabularnewline
				V & 
				\textbf{te}C\sub1C\sub1\textbf{i}C\sub2\textbf{o}C\sub3\textbf{a} & 
				\textbf{te}C\sub1C\sub1\textbf{i}C\sub2\textbf{u}C\sub3\textbf{o} & 
				\textbf{te}C\sub1C\sub1\textbf{i}C\sub2\textbf{a}C\sub3\textbf{a} & 
				\textbf{te}C\sub1C\sub1\textbf{i}C\sub2\textbf{a}C\sub3\textbf{u} & 
				\textbf{te}C\sub1C\sub1\textbf{i}C\sub2\textbf{u}C\sub3\textbf{a}
				\tabularnewline
				VI & 
				\textbf{ina}C\sub1C\sub2\textbf{io}C\sub3\textbf{a} & 
				\textbf{ina}C\sub1C\sub2\textbf{iu}C\sub3\textbf{o} & 
				\textbf{ina}C\sub1C\sub2\textbf{í}C\sub3\textbf{a} & 
				\textbf{ina}C\sub1C\sub2\textbf{ia}C\sub3\textbf{u} & 
				\textbf{ina}C\sub1C\sub2\textbf{iu}C\sub3\textbf{a}
				\tabularnewline
				VII & 
				\textbf{i}C\sub1C\sub2\textbf{io}C\sub3C\sub3\textbf{a} & 
				\textbf{i}C\sub1C\sub2\textbf{iu}C\sub3C\sub3\textbf{o} & 
				\textbf{i}C\sub1C\sub2\textbf{í}C\sub3C\sub3\textbf{a} & 
				\textbf{i}C\sub1C\sub2\textbf{ia}C\sub3C\sub3\textbf{u} & 
				\textbf{i}C\sub1C\sub2\textbf{iu}C\sub3C\sub3\textbf{a}
				\tabularnewline
				VIII & 
				C\sub1\textbf{io}C\sub2C\sub3\textbf{a} & 
				C\sub1\textbf{iu}C\sub2C\sub3\textbf{o} & 
				C\sub1\textbf{í}C\sub2C\sub3\textbf{a} & 
				C\sub1\textbf{ia}C\sub2C\sub3\textbf{u} & 
				C\sub1\textbf{iu}C\sub2C\sub3\textbf{a}
				\tabularnewline
				\hline
			\end{tabular}}\\
		\subfloat[][Biliteral roots]{
			\begin{tabular}{|>{\bfseries}fc|-c|-c|-c|-c|-c|}
				\hline
				\SetRowStyle{\bfseries} Form & Perfective & Inchoative & Cessative & Durative & Momentane \tabularnewline
				\cline{2-6}
				\SetRowStyle{\scshape} & pfv & inch & cess & dur;pfv & momt \tabularnewline
				\hline
				I & 
				C\sub1\textbf{io}C\sub2\textbf{a} & 
				C\sub1\textbf{iu}C\sub2\textbf{o} & 
				C\sub1\textbf{í}C\sub2\textbf{a} & 
				C\sub1\textbf{ia}C\sub2\textbf{u} & 
				C\sub1\textbf{iu}C\sub2\textbf{a}
				\tabularnewline
				II & 
				C\sub1\textbf{io}C\sub2C\sub2\textbf{a} & 
				C\sub1\textbf{iu}C\sub2C\sub2\textbf{o} & 
				C\sub1\textbf{í}C\sub2C\sub2\textbf{a} & 
				C\sub1\textbf{ia}C\sub2C\sub2\textbf{u} & 
				C\sub1\textbf{iu}C\sub2C\sub2\textbf{a}
				\tabularnewline
				III & 
				\textbf{ja}C\sub1C\sub2\textbf{io}C\sub2\textbf{a} & 
				\textbf{ja}C\sub1C\sub2\textbf{iu}C\sub2\textbf{o} & 
				\textbf{ja}C\sub1C\sub2\textbf{í}C\sub2\textbf{a} & 
				\textbf{ja}C\sub1C\sub2\textbf{ia}C\sub2\textbf{u} & 
				\textbf{ja}C\sub1C\sub2\textbf{iu}C\sub2\textbf{a}
				\tabularnewline
				V & 
				\textbf{me}C\sub1\textbf{io}C\sub2C\sub2\textbf{a}	& 
				\textbf{me}C\sub1\textbf{iu}C\sub2C\sub2\textbf{o}	& 
				\textbf{me}C\sub1\textbf{í}C\sub2C\sub2\textbf{a}	& 
				\textbf{me}C\sub1\textbf{ia}C\sub2C\sub2\textbf{u}	& 
				\textbf{me}C\sub1\textbf{iu}C\sub2C\sub2\textbf{a}	 
				\tabularnewline
				V & 
				\textbf{te}C\sub1C\sub1\textbf{io}C\sub2\textbf{a} & 
				\textbf{te}C\sub1C\sub1\textbf{iu}C\sub2\textbf{o} & 
				\textbf{te}C\sub1C\sub1\textbf{í}C\sub2\textbf{a} & 
				\textbf{te}C\sub1C\sub1\textbf{ia}C\sub2\textbf{u} & 
				\textbf{te}C\sub1C\sub1\textbf{iu}C\sub2\textbf{a}
				\tabularnewline
				VI & 
				\textbf{ina}C\sub1C\sub2\textbf{io}C\sub2\textbf{a} & 
				\textbf{ina}C\sub1C\sub2\textbf{iu}C\sub2\textbf{o} & 
				\textbf{ina}C\sub1C\sub2\textbf{í}C\sub2\textbf{a} & 
				\textbf{ina}C\sub1C\sub2\textbf{ia}C\sub2\textbf{u} & 
				\textbf{ina}C\sub1C\sub2\textbf{iu}C\sub2\textbf{a}
				\tabularnewline
				VII & 
				\textbf{i}C\sub1\textbf{io}C\sub2C\sub2\textbf{a} & 
				\textbf{i}C\sub1\textbf{iu}C\sub2C\sub2\textbf{o} & 
				\textbf{i}C\sub1\textbf{í}C\sub2C\sub2\textbf{a} & 
				\textbf{i}C\sub1\textbf{ia}C\sub2C\sub2\textbf{u} & 
				\textbf{i}C\sub1\textbf{iu}C\sub2C\sub2\textbf{a}
				\tabularnewline
				VIII & 
				C\sub1\textbf{io}C\sub2C\sub2\textbf{a} & 
				C\sub1\textbf{iu}C\sub2C\sub2\textbf{o} & 
				C\sub1\textbf{í}C\sub2C\sub2\textbf{a} & 
				C\sub1\textbf{ia}C\sub2C\sub2\textbf{u} & 
				C\sub1\textbf{iu}C\sub2C\sub2\textbf{a}
				\tabularnewline
				IX & 
				\textbf{i}C\sub1\textbf{io}C\sub2\textbf{a} & 
				\textbf{i}C\sub1\textbf{iu}C\sub2\textbf{o} & 
				\textbf{i}C\sub1\textbf{í}C\sub2\textbf{a} & 
				\textbf{i}C\sub1\textbf{ia}C\sub2\textbf{u} & 
				\textbf{i}C\sub1\textbf{iu}C\sub2\textbf{a}
				\tabularnewline
				\hline
			\end{tabular}}
			\caption{Perfective aspectual patterns\label{tab:vm_perfective_aspects}}
		\end{center}
	\end{table}

	\newpage
	\subsection{Topical Agreement}
	\label{ssec:vm_topical_agreement}

	Qevesa is a topic-prominent language that tends towards a split-S active dechticaetiative morphosyntactic alignment. As a result, verbs are marked for agreement with the topic of the sentence, rather than the subject or agent. The topic of the sentence is the noun phrase in the nominative case. 

	\subsubsection{Primary Agreement}
	\label{sssec:vm_primary_agreement}

	The topic of the verb primarily indicates its experiencer, agent/donor, patient/recipient, or theme. If the topical noun phrase is a pronoun, it may be omitted. The suffixes for topical agreement are given in Table~\ref{tab:vm_primary_agreement}.

	\begin{table}[htpb]\small\capstart
		\begin{center}
			\begin{tabular}{|>{\scshape}fc|-c|-c|-c|}
				\hline
				\SetRowStyle{\bfseries} & Ergative & Accusative & Secundative \tabularnewline
				\cline{2-4}
				\SetRowStyle{\scshape} & erg & acc & sdt \tabularnewline
				\hline
				1sg			 & -ěm-/-jem- & -ěş-/-jeş- & -ět-/-jet- \tabularnewline
				2sg			 & -tam-	    & -taş-      & -tot- \tabularnewline
				3sg			 & -(a)m-     & -(a)ş-     & -(a)t- \tabularnewline
				1du;inc  & -jévam-    & -jévaş-    & -jévot- \tabularnewline
				1du;exc  & -čévam-    & -čévaş-    & -čévot- \tabularnewline
				2du			 & -távam-    & -távaş-    & -távot- \tabularnewline
				3du			 & -vam-      & -vaş-      & -vot- \tabularnewline
				1pl;inc  & -jésam-    & -jésaş-    & -jésot- \tabularnewline
				1pl;exc  & -čésam-    & -čésaş-    & -čésot- \tabularnewline
				2pl			 & -tásam-    & -tásaş-    & -tásot- \tabularnewline
				3pl			 & -sam-      & -saş-      & -sot- \tabularnewline
				3;inanim & -ňom-      & -ňoş-      & -ňot- \tabularnewline
				\hline
			\end{tabular}
			\caption{Primary topical agreement\label{tab:vm_primary_agreement}}
		\end{center}
	\end{table}

	The first person singular uses \emph{-ě-} after a consonant, and \emph{-je-} after a vowel; pronunciation is unaltered. The third-person singular suffixes insert an epenthetic \emph{-a-} when the suffix follows a consonant. The use of the singular, dual, and plural numbers is described in Section~\ref{ssec:nm_number}.

	\paragraph{Ergative Topic}
	\label{par:vm_erg_topic}

	An ergative topic indicates that the noun phrase in the nominative case is the voluntary experiencer of an intransitive verb; the agent of a monotransitive verb; and the donor of a ditransitive verb.

	\paragraph{Accusative Topic}
	\label{par:vm_acc_topic}

	An accusative topic indicates that the noun phrase in the nominative case is the involuntary experiencer of an intransitive verb; the patient of a monotransitive verb; and the recipient of a ditransitive verb.

	\paragraph{Secundative Topic}
	\label{par:vm_sdt_topic}

	A secundative topic indicates that the noun phrase in the nominative case is the theme of a ditransitive verb.

	\subsubsection{Secondary Topical Agreement}
	\label{sssec:vm_topic_secondary}

	If the topic of the phrase is not the experiencer, agent/donor, patient/recipient or theme, the verb can be marked with a suffix that corresponds to the role of the noun phrase in the nominative case. Unlike the primary cases, there are no combined pronoun suffixes, so pronouns must not be omitted. These suffixes are described in Table~\ref{tab:vm_secondary_agreement}.

	\begin{table}[htpb]\small\capstart
		\begin{center}
			\begin{tabular}{|>{\bfseries}fc->{\scshape}c|-c|}
				\hline
				\multicolumn{2}{|fc|}{\SetRowStyle{\bfseries}Case} & Suffix \tabularnewline
				\hline
				%Genitive & gen & -karu- \tabularnewline
				Essive			& ess & -(a)ll \tabularnewline
				Instrumental (Comitative) & ins & -(a)tt \tabularnewline
				Inessive		& ine & -(a)ss \tabularnewline
				Adessive		& ade & -(a)d \tabularnewline
				Illative		& ill & -(a)st \tabularnewline
				Allative		& all & -(a)ft \tabularnewline
				Elative			& ela & -(a)sp \tabularnewline
				Ablative		& abl & -(a)sk \tabularnewline
				Comparative & comp & -(a)nn \tabularnewline
				\hline
			\end{tabular}
			\caption{Secondary topical agreement\label{tab:vm_secondary_agreement}}
		\end{center}
	\end{table}

	An epenthetic \emph{-a-} is inserted when the suffix follows a consonant.

	\subsection{Mood}
	\label{ssec:vm_mood}

	Mood is another important category marked on the Qevesa verb. There are eight primary moods: indicative, admirative, irrealis, alethic, necessitative, precative, volitive, and hypothetical.

	The suffixes for mood are given in Table~\ref{tab:vm_modal_suffixes}.

	\begin{table}[htpb]\small\capstart
		\begin{center}
			\begin{tabular}{|>{\bfseries}fc->{\scshape}c|-c|}
				\hline
				\multicolumn{2}{|fc|}{\SetRowStyle{\bfseries}Mood} & Suffix \tabularnewline
				\hline
				Indicative		& ind & -o \tabularnewline
				Admirative		& mir & -óra \tabularnewline
				Irrealis			& irr & -il \tabularnewline
				Alethic				& ale & -is \tabularnewline
				Necessitative	& nec & -ic \tabularnewline
				Precative			& prec & -ła \tabularnewline
				Volitive	    & vol & -ir \tabularnewline
				Hypothetical	& hyp & -ot \tabularnewline
				\hline
			\end{tabular}
			\caption{Verbal mood suffixes\label{tab:vm_modal_suffixes}}
		\end{center}
	\end{table}

	\subsubsection{Indicative Mood}
	\label{sssec:vm_indicative}

	The indicative mood is the default mood. It is essentially a realis mood, indicating the factual nature of the statement.

	\subsubsection{Admirative Mood}
	\label{sssec:vm_admirative}

	The admirative mood is also a realis mood, that indicates new or unexpected information.

	\begin{exe}
		\ex \emph{EXAMPLE}
	\end{exe}

	\subsubsection{Irrealis Mood}
	\label{sssec:vm_irrealis}

	The irrealis mood denotes a counterfactual or non-actual sense.

	\begin{exe}
		\ex \emph{EXAMPLE}
	\end{exe}

	\subsubsection{Alethic Mood}
	\label{sssec:vm_alethic}

	The alethic mood denotes the logical necessity of the statement.

	\begin{exe}
		\ex \emph{EXAMPLE}
	\end{exe}

	\subsubsection{Necessitative Mood}
	\label{sssec:vm_necessitative}

	The necessitative mood denotes that the action must occur. It differs from the alethic mood in that the reason for necessity is not a logical conclusion.

	\begin{exe}
		\ex \emph{EXAMPLE}
	\end{exe}

	\subsubsection{Precative Mood}
	\label{sssec:vm_precative}

	The precative mood indicates that the action is a request or order.

	\begin{exe}
		\ex \emph{EXAMPLE}
	\end{exe}

	\subsubsection{Volitive Mood}
	\label{sssec:vm_optative}

	The volitive mood indicates a hope, desire, or wishes that the action denoted by the verb should come about.

	\begin{exe}
		\ex \emph{EXAMPLE}
	\end{exe}

	\subsubsection{Hypothetical Mood}
	\label{sssec:vm_hypothetical}

	The hypothetical mood indicates that the statement is counterfactual but possible.

	\begin{exe}
		\ex \emph{EXAMPLE}
	\end{exe}


	\section{Auxiliary Verbs}
	\label{sec:vm_auxiliary}
	
	Periphrastic constructions, such as polarity and evidentiality, are indicated with a series of auxiliary verbs. These conjugate similarly to ordinary verbs, but use a slightly different set of conjugations and affixes that are generally identical to the forms for attributive verbs\footnote{See Section~\ref{ssec:am_adjectival_verbs}, page~\pageref{ssec:am_adjectival_verbs}}. 
	
	Syntactically, auxiliary verbs occupy the same position as adverbs or modifiers, as described in Section~\ref{vs_auxiliary}. Morphologically, however, they are more akin to verbs, and tend to agree with their head verb in aspect, but this is not mandatory.

	\subsection{Polarity}
	\label{ssec:vm_polarity}

	The most commonly-used auxiliary verbs are those that indicate polarity. The affirmative verb, \emph{şuru}, meaning ‘to do’, is rarely used, except in situations when an explicitly positive statement is to be made. The negative verb, \emph{dumu}, is more commonly used, and shares the same root as the word for ‘zero’ or ‘none’.

	Both of these verbs conjugate to aspect as a Form IX root, prefixed with an initial \emph{e-}, as shown in Table~\ref{tab:vm_polarity_auxiliary_aspect}.

	\begin{table}[htpb]\small\capstart
		\begin{center}
			\begin{tabular}{|>{\bfseries}fc->{\scshape}c|-c|-c|}
				\hline
				\SetRowStyle{\bfseries} & & \multicolumn{2}{-c|}{Polarity} \tabularnewline
				\cline{3-4}
				\SetRowStyle{\scshape} & & aff & neg \tabularnewline
				\hline
				Imperfective	& ipfv			& eşuri  & edumi \tabularnewline
				Stative				& stat			& eşuire & eduime \tabularnewline
				Durative			& dur;ipfv	& eşurú  & edumú \tabularnewline
				Frequentative & freq			& eşuro  & edumo \tabularnewline
				Habitual			& hab				& eşura  & eduma \tabularnewline
				\hline\hline
				Perfective		& pfv				& eşiora & edioma \tabularnewline
				Inchoative		& inch			& eşiuro & ediumo \tabularnewline
				Cessative			& cess			& eşíra  & edíma \tabularnewline
				Durative			& dur;pfv		& eşiaru & ediamu \tabularnewline
				Momentane			& momt			& eşiura & ediuma \tabularnewline
				\hline
			\end{tabular}
			\caption{Polar verb aspectual conjugation\label{tab:vm_polarity_auxiliary_aspect}}
		\end{center}
	\end{table}

	\newpage
	Often the polar auxiliaries will be conjugated to a different aspect than their head verb, especially to indicate semantic nuances, for example:

	\begin{exe}
		\ex \textit{Ł’isátka soşima jem edíma antulúşo.}
		\glll Ł’=isá-tka-∅ soşim-a jem edíma antulú-ş-o\\
		\textsc{def=dist-hum-nom} girl\textsc{-nom} \textsc{1sg.erg} \textsc{neg.cess} think\textsc{\bs ass.dur;ipfv-3sg;acc-ind}\\
		{that} {girl} {I} {not stop} {thinking about her}\\
		\glt I cannot stop thinking about that girl.
	\end{exe}

	\subsection{Evidentiality}
	\label{ssec:vm_evidentiality}

	Evidentiality may also be expressed by means of auxiliary verbs. Qevesa possesses a set of auxiliary verbs which distinguish four degrees of evidentiality: witness, reportative, inferential, and assumptive. 
	
	All of the roots of the evidential auxiliaries are also verbs in their own right. However, they conjugate as Form VIII verbs, with some slightly irregular pattern forms. Their conjugation is given in Table~\ref{tab:vm_evidentiality_conjugation}.

	\begin{table}[htpb]\small\capstart
		\begin{center}
			\begin{tabular}{|>{\bfseries}fc->{\scshape}c|-c|-c|-c|-c|}
				\hline
				\SetRowStyle{\bfseries} & & \multicolumn{4}{-c|}{Evidentiality} \tabularnewline
				\cline{3-6}
				\SetRowStyle{\scshape} &  & exp   & rep    & infr   & asm		 \tabularnewline
				\hline
				Imperfective	& ipfv			& murri  & łukşi	& kučti  & quspi  \tabularnewline
				Stative				& stat			& muirre & łuikşe & kuičte & quispe \tabularnewline
				Durative			& dur;ipfv	& murrú  & łukşú  & kučtú  & quspú \tabularnewline
				Frequentative & freq			& murro  & łukşo	& kučto  & quspo  \tabularnewline
				Habitual			& hab				& murra  & łukşa  & kučta  & quspa \tabularnewline
				\hline\hline                                               
				Perfective		& pfv				& miorra & łiokşa & kiočta & qiospa \tabularnewline
				Inchoative		& inch			& miurro & łiukşo & kiučto & qiuspo \tabularnewline
				Cessative			& cess			& mírra  & łíkşa	& kíčta  & qíspa  \tabularnewline
				Durative			& dur;pfv		& miarru & łiakşu & kiačtu & qiaspu \tabularnewline
				Momentane			& momt			& miurra & łiukşa & kiučta & qiuspa \tabularnewline
				\hline
			\end{tabular}
			\caption{Conjugation of the evidential verbs \label{tab:vm_evidentiality_conjugation}}
		\end{center}
	\end{table}

	As with all auxiliary constructions, use of the evidential auxiliaries is not mandatory; rather, they are used to provide additional information. 

	\subsubsection{Witness}
	\label{sssec:vm_evd_witness}

	The witness degree of evidentiality is denoted by the verb \emph{murru}, meaning ‘to see’. It is used when the speaker was a witness to the event.

	\subsubsection{Reportative}
	\label{sssec:vm_evd_reportative}

	The reportative degree of evidentiality is denoted by the verb \emph{łukşu}, which has the same consonantal root as the verb \emph{łukuş} ‘to hear (speech)’.

	\subsubsection{Inferential}
	\label{sssec:vm_evd_inferential}

	The inferential degree of evidentiality is denoted by the verb \emph{kučtu}. It is used when the speaker infers that the event occurred but was not a witness.

	\subsubsection{Assumptive}
	\label{sssec:vm_evd_assumption}

	The assumption degree of evidentiality is denoted by the verb \emph{quspu}. It is used when the speaker is making an assumption about the occurrence of the event.

	\section{Irregular Verbs}
	\label{sec:vm_irregular}

	Qevesa verbal morphology is highly regular, with most irregularities occurring due to consonant groupings. %Roots that contain a /h/ frequently possess irregular forms, mainly because the /h/ will be elided or reduced to a pre-aspiration of the following consonant and the previous vowel lengthened. This may be represented in writing as well as speech.
	However, a number of common roots do possess irregular forms, and these are outlined in the following sections.

	\subsection{The Copulae}
	\label{ssec:vm_copulae}

	The most frequently-used irregular verb in Qevesa is the copula \emph{tesí}. It is one of a number of verbs which do not possess a regular infinitive of the form \emph{C\sub1uC\sub2u}; it also possesses a negative form (\emph{dumí}\footnotemark{}), unlike most other verbs. The basic conjugated forms of \emph{tesí} are given in Table~\ref{tab:vm_copulae_aspectual_conjugation}.
	\footnotetext{This is also the same consonantal root as the negative verb \emph{dumu} and associated forms which translate as ‘zero’ or ‘none’.}

	\begin{table}[htpb]\small\capstart
		\begin{center}
			\begin{tabular}{|>{\bfseries}fc->{\scshape}c|-c|-c|}
				\hline
				\SetRowStyle{\bfseries} & & Non-negative & Negative \tabularnewline
				\cline{3-4}
				\SetRowStyle{\scshape} & & cop & neg \tabularnewline
				\hline
				Infinitive  	& inf 			& tesí  & dumí \tabularnewline
				\hline\hline
				Imperfective	& ipfv			& tuší  & dumí \tabularnewline
				Stative				& stat			& tuiše & duimě \tabularnewline
				Durative			& dur;ipfv	& tušú  & dumjú \tabularnewline
				Frequentative & freq			& tušo  & dumjo \tabularnewline
				Habitual			& hab				& tuša & dumja \tabularnewline
				\hline\hline
				Perfective		& pfv				& tioša & diomja \tabularnewline
				Inchoative		& inch			& tiušo & diumjo \tabularnewline
				Cessative			& cess			& tíša  & dímja  \tabularnewline
				Durative			& dur;pfv		& tiašu & diamju \tabularnewline
				Momentane			& momt			& tiuša & diumja \tabularnewline
				\hline
			\end{tabular}
			\caption{Aspectual conjugation of the copulae \emph{tesí} and \emph{dumí}\label{tab:vm_copulae_aspectual_conjugation}}
		\end{center}
	\end{table}

	As well as irregular aspectual conjugations, some of the copular forms, especially the negative stative copula \emph{duimě}, also possess a number of irregular affixes to indicate personal or topical agreement. The personal conjuation of \emph{duimě} are listed in Table~\ref{tab:vm_neg_copula_person}.

	\begin{table}[htpb]\small\capstart
		\begin{center}
			\begin{tabular}{|>{\scshape}fc|-c|-c|-c|}
				\hline
				\SetRowStyle{\bfseries} & Ergative & Accusative & Secundative \tabularnewline
				\cline{2-4}
				\SetRowStyle{\scshape} & erg & acc & sdt \tabularnewline
				\hline
				1sg			 & duiměm     & duiměş     & duimět     \tabularnewline
				2sg			 & duimětam   & duimětaş   & duimětot   \tabularnewline
				3sg			 & duimjam    & duimjaş    & duimat     \tabularnewline
				1du;inc  & duimjévam  & duimjévaş  & duimjévot  \tabularnewline
				1du;exc  & duimečévam & duimečévaş & duimečévot \tabularnewline
				2du			 & duimětávam & duimětávaş & duimětávot \tabularnewline
				3du			 & duiměvam   & duiměvaş   & duiměvot   \tabularnewline
				1pl;inc  & duimjésam  & duimjésaş  & duimjésot  \tabularnewline
				1pl;exc  & duimečésam & duimčésaş  & duimečésot \tabularnewline
				2pl			 & duimětásam & duimětásaş & duimětásot \tabularnewline
				3pl			 & duiměsam   & duiměsaş   & duiměsot   \tabularnewline
				3;inanim & duimeňom   & duimeňoş   & duimeňot   \tabularnewline
				\hline
			\end{tabular}
			\caption{Personal conjugations of the negative stative copula \emph{duimě}\label{tab:vm_neg_copula_person}}
		\end{center}
	\end{table}

	The copulae can also be used in an existential sense, but only in nominal phrases and never with stative verbs. They play a major role in honorific registers, as described in Chapter~\ref{ch:registers}.

\end{document}
