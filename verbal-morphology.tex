\documentclass[grammar]{subfiles}
\begin{document}
\chapter{Verbal Morphology}
\label{ch:verbal_morphology}

\section{Features}
\label{sec:vm_features}

The consonantal root patterns in Qevesa are used to form basic morphological
paradigms.  Qevesa verbs are highly inflected, indicating aspect by transfix
patterns, and suffixes for the other marked elements.

\section{Verb Root Forms}
\label{sec:vm_root_forms}

Although the arrangement of consonants in a root is generally fixed, there are
regular processes to derive subtle semantic variations on the meaning of the
root, such as causatives and reflexives.  These root variants are called forms,
or \qevesa{???} (“constructions”), from the root \qevesa{mudat} (“build,
construct”).  There are five primary forms, numbered I–V; these are listed in
\cref{tab:vm_root_forms}.

\begin{table}[h!]\small\capstart
  \begin{tabular}{BFl -l}
    \toprule
    \SetRowStyle{\bfseries} Form & Pattern \\
    \midrule
    I   & C\sub1{u}C\sub2{a}C\sub3 \\
    II  & C\sub1{u}C\sub2C\sub2{a}C\sub3 \\
    III & C\sub1{u}C\sub2C\sub3{a} \\
    IV  & {mi}C\sub1C\sub2{u}C\sub3{a} \\
    V   & {ta}C\sub1{u}C\sub2C\sub3{a} \\
    VI* & C\sub1{e}C\sub2{u}C\sub3C\sub3{a} \\
    \bottomrule
  \end{tabular}
  \caption{Verb root forms\label{tab:vm_root_forms}}
\end{table}

\subsection{Form I}
\label{ssec:vm_verb_form_i}

Form I is the most common consonantal root form, containing no preformative
affixes or pairing of consonants as occurs in the other forms.  It is typically
the closest indicator to the lexical meaning of the root, and although it has
no particular semantic function associated with it, verbs in Form I are often
transitive.


\subsection{Form II: Intensive}
\label{ssec:vm_form_ii}

Form II is the intensive stem.  It typically indicates an intensive,
frequentative or causative meaning, and may also be used to form transitive
verbs from intransitive roots.

%   Form II roots include:
%   - CuC:aC
%   - puhhat “shout”
%   - murrá (*murrah) “watch”
%   - thuvvar “shatter”

\subsection{Form III: Causative}
\label{ssec:vm_form_iii}

Form III is commonly known as the causative stem.  Its most common function is
causative; it may also convert transitive verbs into ditransitive ones.  It can
also have a causative meaning on verbs whose Form 1 root is intransitive, and
for some verbs, may convey an assistive or factitive meaning.

%   Form III roots include:
%   - CuCCa 
%   - kupša “feed”
%   - rukta “dictate”
%   - murra “show”
%   - thuvra “cause to break”
%   
\subsection{Form IV: Reciprocal}
\label{ssec:vm_form_iv}

Form IV is commonly known as the reciprocal stem.  It commonly conveys meanings
of a reciprocal or reflexive nature, and is often used to create verbs denoting
social interactions. 

%   Form VI roots include:
%   - miCCuCa
%   - miphuta “converse”
%   - mirkuta “correspond”
%   
\subsection{Form V: Reciprocal Causative}
\label{ssec:vm_form_v}

Form V is the reciprocal causative stem, so called for historical reasons as it
also includes a number of other intransitive meanings.  It is subject to much
unpredictable metaphorical and semantic and drift, so actual meanings may vary
quite a lot from the Form 1 verb.  True reflexives account for only a portion
of the verbs in this form.  Its main functions are:

\begin{itemize}
  \item Forming reflexives from transitive roots 
  \item Forming verbs denoting accompaniment 
  \item Forming autoreflexive verbs, that is, intransitive actions performed on one’s body
\end{itemize}

%   Form V roots include: 
%   - taCuCCa

\subsection{Form VI: Adjectival}
\label{ssec:vm_form_vi}

Form VI is the adjectival stem, used to form predicative adjectives.  It 

%  Form VI roots include: 
%  - CeCuC:a
%  - hevurra “good” ← huvur
%  - temussa “tall” ← tumus (grow)

\section{The Infinitive}
\label{sec:vm_infinitive}

The infinitive verb is the citation form of the verb, as well as the non-finite
form used in constructions involving an auxiliary verb.  It is marked by the
patterns \qevesa{C\sub1{u}C\sub2{a}C\sub3}.

\ToBeWritten

\section{Participles}


\section{Conjugation}
\label{sec:vm_conjugation}

Qevesa is a highly synthetic language, and verbs are conjugated to indicate
aspect, mood, and personal agreement (trigger).  The conjugated form of the
verb is as follows:

\begin{exe}
  \ex\label{exe:vm_conjugation} \textit{stem}\bs\textsc{aspect-mood-trigger}
\end{exe}

\subsection{Aspect and Tense}
\label{ssec:vm_aspect_tense}

% Qevesa verbal morphology is structured around a two-by-three contrast of two
% aspects, perfective and imperfective, and three tenses, present, past and
% future.  There are also two imperatives, one for each aspect, which are not
% marked for tense.  These are marked by a series of ten transfix patterns, as
% shown in \cref{tab:vm_tense-aspect_relations}. 

Qevesa verbal morphology primarily indicates aspect rather than tense.  There
are seven aspectual paradigms, each marked with a transfix pattern.  These are
given in \cref{tab:vm_aspect_patterns}.

\begin{table}[h!]\small\capstart
  \begin{tabular}{BFl Sl -l -l -l -l -l}
    \toprule
    \SetRowStyle{\bfseries} Aspect & & I & II & III & IV & V \\
    \midrule
    Perfective & 
    \acs{perf} &
    C\sub1{i}C\sub2{u}C\sub3 &
    C\sub1{i}C\sub2C\sub2{u}C\sub3 &
    C\sub1{i}C\sub2C\sub3{u} &
    {mi}C\sub1C\sub2{i}C\sub3{u} &
    {ta}C\sub1{i}C\sub2C\sub3{u} \\
    %
    Momentane & 
    \acs{momt} &
    C\sub1{i}C\sub2{a}C\sub3 &
    C\sub1{i}C\sub2C\sub2{a}C\sub3 &
    C\sub1{i}C\sub2C\sub3{a} &
    {mi}C\sub1C\sub2{i}C\sub3{a} &
    {ta}C\sub1{i}C\sub2C\sub3{a} \\
    %
    Progressive & 
    \acs{prog} &
    C\sub1{a}C\sub2{u}C\sub3 &
    C\sub1{a}C\sub2C\sub2{u}C\sub3 &
    C\sub1{a}C\sub2C\sub3{u} &
    {mi}C\sub1C\sub2{a}C\sub3{u} &
    {ta}C\sub1{a}C\sub2C\sub3{u} \\
    %
    Durative & 
    \acs{dur} &
    C\sub1{a}C\sub2{i}C\sub3 &
    C\sub1{a}C\sub2C\sub2{i}C\sub3 &
    C\sub1{a}C\sub2C\sub3{i} &
    {mi}C\sub1C\sub2{a}C\sub3{i} &
    {ta}C\sub1{a}C\sub2C\sub3{i} \\
    %
    Habitual & 
    \acs{hab} &
    C\sub1{o}C\sub2{u}C\sub3 &
    C\sub1{o}C\sub2C\sub2{u}C\sub3 &
    C\sub1{o}C\sub2C\sub3{u} &
    {mi}C\sub1C\sub2{o}C\sub3{u} &
    {ta}C\sub1{o}C\sub2C\sub3{u} \\
    %
    Inchoative & 
    \acs{inch} &
    C\sub1{o}C\sub2{a}C\sub3 &
    C\sub1{o}C\sub2C\sub2{a}C\sub3 &
    C\sub1{o}C\sub2C\sub3{a} &
    {mi}C\sub1C\sub2{o}C\sub3{a} &
    {ta}C\sub1{o}C\sub2C\sub3{a} \\
    %
    Cessative & 
    \acs{cess} &
    C\sub1{o}C\sub2{i}C\sub3 &
    C\sub1{o}C\sub2C\sub2{i}C\sub3 &
    C\sub1{o}C\sub2C\sub3{i} &
    {mi}C\sub1C\sub2{o}C\sub3{i} &
    {ta}C\sub1{o}C\sub2C\sub3{i} \\
    %
    \bottomrule
  \end{tabular}
  \caption{Aspectual transfix patterns\label{tab:vm_aspect_patterns}}
\end{table}


\subsubsection{Perfective}
\label{vm:sssec_perfective}

The perfective aspect indicate activities viewed as a single whole.  It is
typically used to speak of singular events completed in the past, but may also
be used to speak of actions without internal structure.

% I wrote / I have written

\subsubsection{Momentane}
\label{vm:sssec_momentane}

The momentane aspect indicates brief single-time activities or states.

% A bolt of lighting struck the tree.
% The mouse squeaked.

\subsubsection{Progressive}
\label{vm:sssec_progressive}

The progressive aspect indicates ongoing actions with a change of state.  It
may also be used to describe intermittent actions.

% I am putting on clothes.
% I am hanging the painting on the wall.
% It is raining in Kirua:  Kiruazi pašurak.

\subsubsection{Durative}
\label{vm:sssec_durative}

The durative aspect indicates ongoing actions without a change of state, or
actions which last some time.

% I am wearing clothes.
% The picture is hanging on the wall.

\subsubsection{Habitual}
\label{vm:sssec_habitual}

The habitual aspect indicates actions that occur habitually.  Like the
progressive, it may also describe intermittent actions, but in a general sense.
It can also be used as a general imperfective aspect, without the implication
on continuous actions or states like the progressive and durative aspects. 

% I walk to work (every day).
% It is very windy on the plains.

\subsubsection{Inchoative}
\label{vm:sssec_inchoative}

The inchoative aspect emphasises the beginning of an activity or state.

% She started walking.

\subsubsection{Cessative}
\label{vm:sssec_cessative}

The cessative aspect emphasises the ending of an activity or state.

% He stopped writing.

%\newpage

\subsection{Modality}
\label{ssec:vm_modality}

Qevesa predominantly indicates modality by means of suffixes.  There are six
synthetic moods: indicative, imperative, mirative, conditional, optative and
potential.  These are listed in \cref{tab:vm_modal_suffixes}; the left column
indicates suffixes that follow a consonant, and the right column suffixes that
follow a vowel.

\begin{table}[h!]\small\capstart
  \begin{tabular}{BFl-Sl -l}
    \toprule
    \multicolumn{2}{fc}{\SetRowStyle{\bfseries}Mood} & Suffix \\
    \midrule
    Indicative  & \acs{ind}  & -∅   \\
    Imperative  & \acs{imp}  & -j   \\
    Mirative    & \acs{mir}  & -eni \\
    Conditional & \acs{cond} & -esi \\
    Optative    & \acs{opt}  & -eti \\
    Potential   & \acs{pot}  & -er \\
    \bottomrule
  \end{tabular}
  \caption{Verbal mood suffixes\label{tab:vm_modal_suffixes}}
\end{table}

The \emph{indicative} mood is used for factual statements and positive beliefs,
and as such is the default mood.  It is marked with a null morpheme. 

The \emph{imperative} mood is used for commands and requests. 

The \emph{mirative} mood is used to express surprise and also doubt, irony,
sarcasm, etc.  It is used to express statements contrary to the speaker’s
expectations or state of mind.

The \emph{conditional} mood is used to speak of an event whose realization is
dependent upon another condition. 

The \emph{optative} mood is used to express hopes, wishes and desires.

The \emph{potential} mood indicates that, in the opinion of the speaker, the
action or occurrence is considered likely.  It can also be used to express that
one has the ability to do something.

\subsection{Person Marking}
\label{ssec:vm_person_marking}

Person marking in Qevesa is somewhat complicated by the unusual morphosyntactic
alignment.  It broadly functions as a trigger system, in which the thematic
role (agent, patient, or oblique) of the noun marked by the direct case is
encoded in the verb.  

% Animate nouns can be marked as agent, patient or oblique, but inanimate nouns
% can only be marked as patient or oblique.  Intransitive verbs with animate
% subjects display a \emph{Split-S} alignment where the split indicates
% volition.  Intransitive verbs with an inanimate subject always display an
% ergative alignment.

The suffixes for person marking are listed in \cref{tab:vm_person_marking}.
Whilst the full set of pronouns is represented, the distinction between dual
and plural forms is lost for the first and third person.  The left columns give
suffixes that follow a consonant, and the right columns suffixes that follow a
vowel. 

\begin{table}[h!]\small\capstart
  \begin{tabular}{SFl -l -l -l -l -l -l}
    \toprule
    \SetRowStyle{\bfseries} & \multicolumn{2}{-c}{Agent Trigger} & \multicolumn{2}{-c}{Patient Trigger}& \multicolumn{2}{-c}{Oblique Trigger} \\
    & \multicolumn{2}{-c}{\acs{agt}} & \multicolumn{2}{-c}{\acs{pat}}& \multicolumn{2}{-c}{\acs{obl}} \\
    \midrule
    \acs{1p}\acs{sg}           & -ain  & -(j)en & -aiš  & -(j)ec & -aik  & -(j)ek  \\
    \acs{2p}\acs{sg}           & -tan  & -tan   & -tac  & -tac   & -tak  & -tak    \\
    \acs{3p}\acs{sg}           & -an   & -n     & -ac   & -c     & -ak   & -k      \\
    \acs{1p}\acs{du};\acs{inc} & -iun  & -jun   & -iuc  & -juc   & -iuk  & -juk    \\
    \acs{1p}\acs{du};\acs{exc} & -čen  & -čen   & -čec  & -čec   & -ček  & -ček    \\
    \acs{2p}\acs{du}           & -tun  & -tun   & -tuc  & -tuc   & -tuk  & -tuk    \\
    \acs{3p}\acs{du}           & -umin & -min   & -umic & -mic   & -umik & -mik    \\
    \acs{1p}\acs{pl};\acs{inc} & -isán & -sán   & -isác & -sác   & -isák & -sák    \\
    \acs{1p}\acs{pl};\acs{exc} & -čen  & -čen   & -čec  & -čec   & -ček  & -ček    \\
    \acs{2p}\acs{pl}           & -tán  & -tán   & -tác  & -tác   & -ták  & -ták    \\
    \acs{3p}\acs{pl}           & -amin & -min   & -amic & -mic   & -amik & -mik    \\
    \midrule
    \acs{inanim};\acs{sg}      &      &       & -oc   & -c   & -ok   & -k \\
    \bottomrule
  \end{tabular}
  \caption{Person marking suffixes\label{tab:vm_person_marking}}
\end{table}

% kupaš - to eat                 puhat - to say              huka - to go
% kipušain    kipušenien         pihutain   pihutesien       hikuen   hikuetien
% kipuštan    kipušenitan        pihuttan   pihutesitan      hikutan  hikuetitan
% kipušin     kipušenin          pihutin    pihutesin        hikun    hikuetin
% kipušiun    kipušenijun        pihutiun   pihutesijun      hikujun  hikuetijun
% kipuščen    kipušeničen        pihutčen   pihutesičen      hikučen  hikuetičen
% kipuštun    kipušenitun        pihuttun   pihutesitun      hikutun  hikuetitun
% kipušumin   kipušenimin        pihutumin  pihutesimin      hikumin  hikuetimin
% kipušisán   kipušenisán        pihutisán  pihutesisán      hikusán  hikuetisán
% kipuštán    kipušenitán        pihuttán   pihutesitán      hikután  hikuetitán
% kipušamin   kipušenimin        pihutamin  pihutesimin      hikumin  hikuetimin

% mištuja - to buy               rukat - to write
% mištijuen   mištijuesien       rikutain   rikuterain
% mištijutan  mištijuesitan      rikuttan   rikutertan
% mištijun    mištijuesin        rikutin    rikuterin
% mištijujun  mištijuesijun      rikutiun   rikuteriun
% mištijučen  mištijuesičen      rikutčen   rikuterčen
% mištijutun  mištijuesitun      rikuttun   rikutertun
% mištijumin  mištijuesimin      rikutumin  rikuterumin
% mištijusán  mištijuesisán      rikutisán  rikuterisán
% mištijután  mištijuesitán      rikuttán   rikutertán
% mištijumin  mištijuesimin      rikutamin  rikuteramin

\subsubsection{Agent Trigger}
\label{sssec:vm_agt_trigger}

The agent trigger indicates that the noun phrase in the direct case is the
voluntary experiencer of an intransitive verb or the agent of a transitive
verb.

Generally only animate nouns may be agents; to describe an action involving an
inanimate noun as agent, a construction using the oblique trigger and the
instrumental case is used instead. 

\subsubsection{Patient Trigger}
\label{sssec:vm_pat_trigger}

The patient trigger indicates that the noun phrase in the direct case is the
involuntary experiencer of an intransitive verb; the patient of a transitive
verb; and the recipient of a ditransitive verb.  

Only animate nouns may be voluntary agents of intransitive verbs; inanimate
nouns are always marked as involuntary experiencers of intransitive verbs.
Furthermore, some intransitive verbs are always involuntary, regardless of
animacy. 

\subsubsection{Oblique Trigger}
\label{sssec:vm_obl_trigger}

The oblique trigger indicates that the noun phrase in the direct case is
something other than the agent or patient of a transitive verb.  For
ditransitive verbs it normally indicates the theme or direct object.

Another common use of the oblique trigger is to express an inanimate agent, the
noun marked with both the instrumental case and the direct case.

\newpage
\section{Additional Suffixes}
\label{sec:vm_additional_suffixes}

In addition to the mandatory modal and personal suffixes, there are a number of
additional final suffixes that can be appended to the verb. 

\begin{itemize}
  \item Relativising \qevesa{-i}
  \item Interrogative \qevesa{-ko}
  \item \ToBeWritten
\end{itemize}

\newpage
\section{Auxiliary Verbs}
\label{sec:vm_auxiliary}

Auxiliary verbs are used to form periphrastic constructions not covered by the
synthetic forms described above.  The auxiliary verb takes the conjugations of
the main verb, which precedes it in the infinitive.  

\subsection{The Copula}
\label{ssec:vm_copula}

The most commonly used auxiliary verb is the copula, which is used to form a
variety of constructions.  It is unique in that it is the only verb that does
not consist of a multi-consonant root, though it conjugates similarly.  The
conjugated forms of the copula are listed in \cref{tab:vm_copula}.

\begin{table}[h!]\small\capstart
  \begin{tabular}{BFl Sl -l -l -l -l -l}
    \toprule
    \multicolumn{2}{Fc}{\SetRowStyle{\bfseries}Aspect} & \multicolumn{5}{-c}{Mood} \\
    \SetRowStyle{\scshape} & & ind & mir & cond & opt & pot \\
    \midrule
    Perfective  & \acs{perf} & izu & izueni & izuesi & izueti & izuer \\
    Momentane   & \acs{momt} & iza & izaeni & izaesi & izaeti & izaer \\
    Progressive & \acs{prog} & azu & azueni & azuesi & azueti & azuer \\
    Durative    & \acs{dur}  & azi & azieni & aziesi & azieti & azier \\
    Habitual    & \acs{hab}  & ozu & ozueni & ozuesi & ozueti & ozuer \\
    Inchoative  & \acs{inch} & oza & ozaeni & ozaesi & ozaeti & ozaer \\
    Cessative   & \acs{cess} & ozi & ozieni & oziesi & ozieti & ozier \\
    \bottomrule
  \end{tabular}
  \caption{Conjugation of the copula \label{tab:vm_copula}}
\end{table}

The modal suffixes on the copula are slightly different, but the suffixes for
person marking (see \cref{ssec:vm_person_marking}) are the same.  By itself,
the copula functions as an existential verb.

%  \begin{exe}
%    \ex \qevesa{A thauka sošima jem en nusat ozierik.}
%    \glll A= thauka sošim-a jem en nusat ozi-er-ik\\
%    \textsc{def=} \textsc{dist.anim} girl\textsc{-foc} \textsc{1sg.nom} \textsc{neg} think\bs\textsc{inf} \textsc{cop\bs cess-pot-obl}\\
%    {the} {that} {girl} {I} {not} {think} {can stop}\\
%    \glt I cannot stop thinking about that girl.
%  \end{exe}

\subsection{Negation}
\label{ssec:vm_negation}

Verbs in Qevesa are negated by using the negative particle \qevesa{en} and the
copula.  The main verb appears in the infinitive, with the copula taking its
inflections, as in a standard auxiliary construction.  If the verb is already
part of an auxiliary construction, the negation particle precedes this. 


% Peter-FOC doctor-ABS COP\HAB.
% Peter is a doctor.

% Peter-a doctor-el COP\HAB-OPT.
% Peter wants to become a doctor.
% 

%  \subsection{Evidentiality}
%  \label{ssec:vm_evidentiality}

%  Evidentiality may also be expressed by means of auxiliary verbs.  Qevesa
%  possesses a set of auxiliary verbs which distinguish four degrees of
%  evidentiality: witness, reportative, inferential, and assumptive. 

%  All of the roots of the evidential auxiliaries are also verbs in their own
%  right.  However, they conjugate as Form VIII verbs, with some slightly
%  irregular pattern forms.  Their conjugation is given in
%  \cref{tab:vm_evidentiality_conjugation}.

%  \subsubsection{Witness}
%  \label{sssec:vm_evd_witness}

%  The witness degree of evidentiality is denoted by the verb \qevesa{murru},
%  meaning ‘to see’.  It is used when the speaker was a witness to the event.

%  \subsubsection{Reportative}
%  \label{sssec:vm_evd_reportative}

%  The reportative degree of evidentiality is denoted by the verb
%  \qevesa{łukšu}, which has the same consonantal root as the verb
%  \qevesa{łukuš} ‘to hear (speech)’.

%  \subsubsection{Inferential}
%  \label{sssec:vm_evd_inferential}

%  The inferential degree of evidentiality is denoted by the verb
%  \qevesa{kučtu}.  It is used when the speaker infers that the event occurred
%  but was not a witness.

%  \subsubsection{Assumptive}
%  \label{sssec:vm_evd_assumption}

%  The assumption degree of evidentiality is denoted by the verb
%  \qevesa{quspu}.  It is used when the speaker is making an assumption about
%  the occurrence of the event.

\newpage
\section{Irregular Verbs}
\label{sec:vm_irregular}

Qevesa verbal morphology is in general highly regular.  However, due to sound
changes from Therasa, a number of formerly regular roots have developed
irregular conjugations, outlined in the sections below.  In the tables, the
following convensions apply: 

\begin{itemize}
  \item C = consonant
  \item P = plosive consonant
  \item H = aspirated plosive
  \item F = fricative, corresponding to the aspirated plosives
  \item K = other consonant
  \item A = vowel
  \item : = length marker
  \item lowercase letters indicate specific phonemes, given in \textsc{ipa}
  \item letters with subscripts refer to root consonants
\end{itemize}

%uppercase letters indicate phoneme types (C =
%consonant, P = plosive, H = *aspirated plosive, F = non-sibilant fricative, S
%= other fricative, A = vowel); and lowercase letters indicate specific phonemes.

\subsection{Soft Roots}
\label{ssec:vm_soft_roots}

Soft roots are those roots which have /h/ in one or more positions.  This
causes the following sound changes:

\begin{itemize}
  \item A word-final /h/ induces lengthening of the previous vowel.  Suffixes
    that follow are usually vowel-final.
  \item A /h/ following an unvoiced plosive causes it to become a geminate
    aspirated plosive, which are pronounced in Modern Qevesa as fricatives.
  \item Roots that have /h/ in more than one position follow the rules of both
    positions.  These are exceedlingly rare.
\end{itemize}

The patterns for soft roots are given in \cref{tab:vm_soft_roots}.

\begin{table}[h!]\small\capstart
  \tc{5pt}
  \begin{tabular}{BFl -l -l -l -l -l -l -l}
    \toprule
    \SetRowStyle{\bfseries} & First-soft && \multicolumn{2}{-c}{Second-soft} && \multicolumn{2}{-c}{Third-soft} \\
    \cmidrule{2-2} \cmidrule{4-5} \cmidrule{7-8}
    \SetRowStyle{\bfseries} & h C C && P h C & H h C && C P h & C K h \\
    \midrule
    I   & h\sub1AC\sub2AC\sub3       &  & P\sub1Ah\sub2AC\sub3       & F\sub1Ah\sub2AC\sub3  &  & C\sub1AP\sub2Aː        & C\sub1AK\sub2Aː    \\
    II  & h\sub1AC\sub2C\sub2AC\sub3 &  & P\sub1Ah\sub2h\sub2AC\sub3 & F\sub1Ah\sub2h\sub2AC &  & C\sub1AP\sub2P\sub2Aː  & C\sub1AK\sub2K\sub2Aː    \\
    III & h\sub1AC\sub2C\sub3A       &  & P\sub1AːC\sub3A            & F\sub1AːC\sub3A       &  & C\sub1AF\sub2ːA        & C\sub1AK\sub2ːA    \\
    IV  & miːC\sub2AC\sub3A          &  & meF\sub1ːAC\sub3A          & meF\sub1ːAC\sub3A     &  & miC\sub1P\sub2Ah\sub2A & miC\sub1K\sub2Ah\sub3A  \\
    V   & tah\sub1AC\sub2C\sub3A     &  & taP\sub1AːC\sub3A          & taF\sub1AːC\sub3A     &  & taC\sub1AF\sub3ːA      & taC\sub1AK\sub2ːA  \\
    \bottomrule
  \end{tabular}
  \caption{Soft root patterns\label{tab:vm_soft_roots}}
\end{table}

Soft roots include \qevesa{puhat} (“speak”) (H2) and \qevesa{mura} (“see”)
(H3).  Third-soft roots are typically written without the lengthened final
vowel in the infinitive to distinguish them from G-final roots. 

%  \newpage
\subsection{Weak Roots}
\label{ssec:vm_weak_roots}

Weak roots had /ɡ/ or /ɟ/ in one or more positions. 

G-roots (roots with /ɡ/) induced the most extensive changes: when initial, it
elided; when following a vowel, it lengthened that vowel; when following a
consonant, it lengthened the consonant; and when between two vowels, it
disappeared, causing adjacent transfix patterns to rearrange around the
remaining consonants.  These roots are thus the most irregular root forms,
often with unpredictable patterns.

J-roots (roots with /ɟ/) tend to be less irregular, as all occurrences of /ɟ/
weakened to the approximant /j/.  A syllable-final /j/ further weakened to the
vowel /i/, often resulting in the appearance of /-i/ offglide diphthongs.

The patterns for weak roots are given in \cref{tab:vm_weak_roots}.

\begin{table}[h!]\small\capstart
  \begin{tabular}{BFl c -l -l -l c -l -l -l}
    \toprule
    \SetRowStyle{\bfseries} && \multicolumn{3}{-c}{G-roots} && \multicolumn{3}{-c}{J-roots} \\
    \cmidrule{3-5} \cmidrule{7-9}
    \SetRowStyle{\bfseries} && g C C & C g C & C C g && ɟ C C & C ɟ C & C C ɟ \\
    \midrule
    I   &  & AC\sub2AC\sub3       & C\sub1AC\sub3A       & C\sub1AC\sub2Aː        &  & j\sub1AC\sub2AC\sub3       & C\sub1Aj\sub2AC\sub3   & C\sub1AC\sub2Ai   \\
    II  &  & AC\sub2C\sub2AC\sub3 & C\sub1AC\sub3C\sub3A & C\sub1AC\sub2C\sub2Aː  &  & j\sub1AC\sub2C\sub2AC\sub3 & C\sub1Aij\sub2AC\sub3  & C\sub1AC\sub2C\sub2Ai   \\
    III &  & AC\sub2C\sub3A       & C\sub1AːC\sub3A      & C\sub1AC\sub2C\sub2A   &  & j\sub1AC\sub2C\sub3A       & C\sub1AiC\sub3A        & C\sub1AC\sub2iA \\
    IV  &  & míC\sub2AC\sub3A     & miC\sub1ːAC\sub3A    & miC\sub1AC\sub2A       &  & míC\sub2AC\sub3A           & miC\sub1iAC\sub3A      & miC\sub1C\sub2Aj\sub3A \\
    V   &  & taAC\sub2C\sub3A     & taC\sub1AːC\sub3A    & taC\sub1AC\sub2C\sub2A &  & taj\sub1AC\sub2C\sub3A     & taC\sub1AiC\sub3A      & taC\sub1AC\sub2iA \\
    \bottomrule
  \end{tabular}
  \caption{Weak root patterns\label{tab:vm_weak_roots}}
\end{table}

Weak roots include \qevesa{unav} (“steal”) (G1), \qevesa{čuta} (“open”) (G2),
\qevesa{lukaj} (“trick, deceive”) (J3) and \qevesa{kujar} (“read, call,
invite”) (J2).  The root \qevesa{juta} (“know”) is both weak and soft (J1/H3).

\subsection{Biliteral Roots}
\label{ssec:vm_biliteral_roots}

Whilst the overwhelming majority of roots in Qevesa are triliteral, there is a
small closed class of true biliteral roots as opposed to the apparently
biliteral patterns that \emph{soft} and \emph{weak} roots display.  These are
ususally distinguishable in that they lack long vowels that were formed on
\emph{soft} and \emph{weak} roots from the elision of consonants, though there
are a handful of roots with apparent homonymy in some forms and conjugations.

\subsection{Quadriliteral Roots}
\label{ssec:vm_quadriliteral_roots}

Quadriliteral roots also exist.

\ToBeWritten


\end{document}
