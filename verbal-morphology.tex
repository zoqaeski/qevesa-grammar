\documentclass[grammar]{subfiles}
\begin{document}
  \chapter{Verbal Morphology}
  \label{ch:verbal_morphology}

  \section{Features}
  \label{sec:vm_features}

  The consonantal root patterns in Qevesa are used to form basic morphological paradigms.  Qevesa verbs are highly inflected, indicating tense and aspect by transfix patterns; topical agreement and modality are marked by agglutinative suffixes.  All other constructions, are indicated by periphrasis or syntax.

  The stem consists of the root and zero or more derivational affixes conjugated to a particular aspect. 


  \section{The Infinitive}
  \label{sec:vm_infinitive}

  The infinitive verb is the citation form of the verb, as well as the non-finite form used in constructions involving an auxiliary verb. 
  It is marked by the patterns \qevesa{C\sub1{u}C\sub2{u}C\sub3} and \qevesa{C\sub1{u}C\sub2{u}} . 

  \ToBeWritten


  \section{Conjugation}
  \label{sec:vm_conjugation}

  Qevesa is a highly synthetic language, and verbs are conjugated to indicate aspect, tense, topical agreement, and mood.  The conjugated form of the verb is as follows:

  \begin{exe}
    \ex\label{exe:vm_conjugation} \textit{stem}\bs\textsc{aspect;tense-topic-mood}
  \end{exe}

  \subsection{Aspect and Tense}
  \label{ssec:vm_aspect_tense}
  
  Qevesa verbal morphology is structured around a three-by-three contrast of three aspects, perfective, imperfective and perfect, and three tenses, present, past and future.  There are also two imperatives, one for each aspect, which are not marked for tense.  These are marked by a series of ten transfix patterns, as shown in Table~\ref{tab:vm_tense-aspect_relations}. 

  \begin{table}[htpb]\small\capstart
    \begin{tabular}{|>{\bfseries}fc|-c|-c|-c|-c|}
      \hline
      \SetRowStyle{\bfseries} & Present & Past & Future & Imperative \tnl
      \hline
      Aorist series       & —               & Aorist     & Future perfective   & Perfective imperative \tnl
      Imperfective series & Present         & Imperfect  & Future imperfective & Imperfective imperative \tnl
      Perfect series      & Present perfect & Pluperfect & Future perfect      & — \tnl
      \hline
    \end{tabular}
    \caption{Tense-Aspect relations\label{tab:vm_tense-aspect_relations}}
  \end{table}

  \subsubsection{The Aorist Series}
  \label{sssec:vm_aorist_series}

  The aorist series is generally used to indicate the perfective aspect; it views the action described by the verb as a single point in time, as an event, not a process.  

  \begin{itemize*}
    \item The \textbf{aorist} is used to express a single completed action that occurred in the past. 
    \item The \textbf{future perfective}, is used to express a completed action in the future. 
  \end{itemize*}
  
  The triliteral root patterns for the aorist series are given in Table~\ref{tab:vm_aorist_series}. 

  \begin{table}[htpb]\small\capstart
    \subfloat[Triliteral roots]{
      \begin{tabular}{|>{\bfseries}fc|-c|-c|}
        \hline
        \SetRowStyle{\bfseries} Form & Aorist & Future perfective \tnl
        \cline{2-3}
        \SetRowStyle{\scshape} & \acs{aor} & \acs{fut};\acs{pfv} \tnl
        \hline
        1 & 
        C\sub1{iu}C\sub2{o}C\sub3 & 
        C\sub1{iu}C\sub2{a}C\sub3
        \tnl
        2 & 
        C\sub1{iu}C\sub2C\sub2{o}C\sub3 & 
        C\sub1{iu}C\sub2C\sub2{a}C\sub3
        \tnl
        3 & 
        C\sub1{iu}C\sub2C\sub3{o} & 
        C\sub1{iu}C\sub2C\sub3{a}
        \tnl
        4 & 
        {i}C\sub1C\sub2{iu}C\sub3{o} & 
        {i}C\sub1C\sub2{iu}C\sub3{a}
        \tnl
        5 & 
        {me}C\sub1{iu}C\sub2{o}C\sub3 & 
        {me}C\sub1{iu}C\sub2{a}C\sub3  
        \tnl
        6 & 
        {ta}C\sub1C\sub2{iu}C\sub3{o} & 
        {ta}C\sub1C\sub2{iu}C\sub3{a}
        \tnl
        7 & 
        C\sub1{ë}C\sub2{iu}C\sub3{o} & 
        C\sub1{ë}C\sub2{iu}C\sub3{a}
        \tnl
        \hline
      \end{tabular}
    }
    \subfloat[Biliteral roots]{
      \begin{tabular}{|>{\bfseries}fc|-c|-c|}
        \hline
        \SetRowStyle{\bfseries} Form & Aorist & Future Perfective \tnl
        \cline{2-3}
        \SetRowStyle{\scshape} & \acs{aor} & \acs{fut};\acs{pfv} \tnl
        \hline
        1 & 
        C\sub1{iu}C\sub2{o} & 
        C\sub1{iu}C\sub2{a}
        \tnl
        2 & 
        {ju}C\sub1C\sub1{o}C\sub2 & 
        {ju}C\sub1C\sub1{a}C\sub2
        \tnl
        3 & 
        C\sub1{iu}C\sub2C\sub2{o} & 
        C\sub1{iu}C\sub2C\sub2{a}
        \tnl
        4 & 
        {i}C\sub1C\sub1{iu}C\sub2{o} & 
        {i}C\sub1C\sub1{iu}C\sub2{a}
        \tnl
        5 & 
        {me}C\sub1{iu}C\sub2{o} & 
        {me}C\sub1{iu}C\sub2{a}  
        \tnl
        6 & 
        {ta}C\sub1{iu}C\sub2C\sub2{o} & 
        {ta}C\sub1{iu}C\sub2C\sub2{a}
        \tnl
        7 & 
        {ë}C\sub1{iu}C\sub2{o} & 
        {ë}C\sub1{iu}C\sub2{a}
        \tnl
        \hline
    \end{tabular}}
    \caption{Aorist series transfix patterns\label{tab:vm_aorist_series}}
  \end{table}

  \subsubsection{The Imperfective Series}
  \label{sssec:vm_imperfective_series}

  The imperfective series is used to mark events actions in progress, with significant course to the speaker. 

  \begin{itemize*}
    \item The \textbf{present} is used to express events that are occurring at the time of speaking, or events that happen habitually. 
    \item The \textbf{imperfect} is used to express incomplete or continuous events in the past, or habitual past actions. 
    \item The \textbf{future imperfective} is used to express an event that will occur in the future. 
  \end{itemize*}

  The transfix patterns for this series are listed in Table~\ref{tab:vm_imperfective_series}. 

  \begin{table}[htpb]\small\capstart
      \subfloat[Triliteral roots]{
        \begin{tabular}{|>{\bfseries}fc|-c|-c|-c|}
          \hline
          \SetRowStyle{\bfseries} Form & Present & Imperfect & Future imperfective \tnl
          \cline{2-4}
          \SetRowStyle{\scshape} & \acs{prs} & \acs{ipf} & \acs{fut};\acs{ipfv} \tnl
          \hline
          1 & 
          C\sub1{u}C\sub2{i}C\sub3 & 
          C\sub1{u}C\sub2{o}C\sub3 & 
          C\sub1{u}C\sub2{a}C\sub3
          \tnl
          2 & 
          C\sub1{u}C\sub2C\sub2{i}C\sub3 & 
          C\sub1{u}C\sub2C\sub2{o}C\sub3 & 
          C\sub1{u}C\sub2C\sub2{a}C\sub3
          \tnl
          3 & 
          C\sub1{u}C\sub2C\sub3{i} & 
          C\sub1{u}C\sub2C\sub3{o} & 
          C\sub1{u}C\sub2C\sub3{a}
          \tnl
          4 & 
          {i}C\sub1C\sub2{u}C\sub3{i} & 
          {i}C\sub1C\sub2{u}C\sub3{o} & 
          {i}C\sub1C\sub2{u}C\sub3{a}
          \tnl
          5 & 
          {me}C\sub1{u}C\sub2{i}C\sub3 & 
          {me}C\sub1{u}C\sub2{o}C\sub3 & 
          {me}C\sub1{u}C\sub2{a}C\sub3  
          \tnl
          6 & 
          {ta}C\sub1C\sub2{u}C\sub3{i} & 
          {ta}C\sub1C\sub2{u}C\sub3{o} & 
          {ta}C\sub1C\sub2{u}C\sub3{a}
          \tnl
          7 & 
          C\sub1{ë}C\sub2{u}C\sub3{i} & 
          C\sub1{ë}C\sub2{u}C\sub3{o} & 
          C\sub1{ë}C\sub2{u}C\sub3{a}
          \tnl
          \hline
        \end{tabular}
      }\\
      \subfloat[Biliteral roots]{
        \begin{tabular}{|>{\bfseries}fc|-c|-c|-c|}
          \hline
          \SetRowStyle{\bfseries} Form & Present & Imperfect & Future imperfective \tnl
          \cline{2-4}
          \SetRowStyle{\scshape} & \acs{prs} & \acs{ipf} & \acs{fut};\acs{ipfv} \tnl
          \hline
          1 & 
          C\sub1{u}C\sub2{i} & 
          C\sub1{u}C\sub2{o} & 
          C\sub1{u}C\sub2{a}
          \tnl
          2 & 
          {u}C\sub1C\sub1{i}C\sub2 & 
          {u}C\sub1C\sub1{o}C\sub2 & 
          {u}C\sub1C\sub1{a}C\sub2
          \tnl
          3 & 
          C\sub1{u}C\sub2C\sub2{i} & 
          C\sub1{u}C\sub2C\sub2{o} & 
          C\sub1{u}C\sub2C\sub2{a}
          \tnl
          4 & 
          {i}C\sub1C\sub1{u}C\sub2{i} & 
          {i}C\sub1C\sub1{u}C\sub2{o} & 
          {i}C\sub1C\sub1{u}C\sub2{a}
          \tnl
          5 & 
          {me}C\sub1{u}C\sub2{i} & 
          {me}C\sub1{u}C\sub2{o} & 
          {me}C\sub1{u}C\sub2{a}  
          \tnl
          6 & 
          {ta}C\sub1{u}C\sub2C\sub2{i} & 
          {ta}C\sub1{u}C\sub2C\sub2{o} & 
          {ta}C\sub1{u}C\sub2C\sub2{a}
          \tnl
          7 & 
          {ë}C\sub1{u}C\sub2{i} & 
          {ë}C\sub1{u}C\sub2{o} & 
          {ë}C\sub1{u}C\sub2{a}
          \tnl
          \hline
        \end{tabular}}
      \caption{Imperfective series transfix patterns\label{tab:vm_imperfective_series}}
  \end{table}

  \subsubsection{The Perfect Series}
  \label{sssec:vm_perfect_series}

  The perfect series do not show as strong a distinction in aspect as the other three series. 
  %though it tends towards indicating the perfective over the imperfective. 
  Instead of distinguishing perfective from imperfective, this series indicates actions in the past with relevence to present or other past events.   
  
  \begin{itemize*}
    \item The \textbf{perfect} indicates actions begun in the past that are relevent in the present.  It may also convey an inferential meaning.
    \item The \textbf{pluperfect} indicates actions or events in the past that were completed prior to some other event. 
    \item The \textbf{future perfect} describes a future state that will result from a finished action.
  \end{itemize*}

  The transfix patterns for this series are listed in Table~\ref{tab:vm_perfect_series}. 

  \begin{table}[htpb]\small\capstart
    \subfloat[Triliteral roots]{
      \begin{tabular}{|>{\bfseries}fc|-c|-c|-c|}
        \hline
        \SetRowStyle{\bfseries} Form & Present Perfect & Pluperfect & Future Perfect \tnl
        \cline{2-4}
        \SetRowStyle{\scshape} & \acs{prs};\acs{perf} & \acs{plup} & \acs{fut};\acs{perf} \tnl
        \hline
        1 & 
        C\sub1{e}C\sub2{i}C\sub3 & 
        C\sub1{e}C\sub2{o}C\sub3 & 
        C\sub1{e}C\sub2{a}C\sub3
        \tnl
        2 & 
        C\sub1{e}C\sub2C\sub2{i}C\sub3 & 
        C\sub1{e}C\sub2C\sub2{o}C\sub3 & 
        C\sub1{e}C\sub2C\sub2{a}C\sub3
        \tnl
        3 & 
        C\sub1{e}C\sub2C\sub3{i} & 
        C\sub1{e}C\sub2C\sub3{o} & 
        C\sub1{e}C\sub2C\sub3{a}
        \tnl
        4 & 
        {i}C\sub1C\sub2{e}C\sub3{i} & 
        {i}C\sub1C\sub2{e}C\sub3{o} & 
        {i}C\sub1C\sub2{e}C\sub3{a}
        \tnl
        5 & 
        {me}C\sub1{e}C\sub2{i}C\sub3 & 
        {me}C\sub1{e}C\sub2{o}C\sub3 & 
        {me}C\sub1{e}C\sub2{a}C\sub3  
        \tnl
        6 & 
        {ta}C\sub1C\sub2{e}C\sub3{i} & 
        {ta}C\sub1C\sub2{e}C\sub3{o} & 
        {ta}C\sub1C\sub2{e}C\sub3{a}
        \tnl
        7 & 
        C\sub1{ë}C\sub2{e}C\sub3{i} & 
        C\sub1{ë}C\sub2{e}C\sub3{o} & 
        C\sub1{ë}C\sub2{e}C\sub3{a}
        \tnl
        \hline
      \end{tabular}
    }\\
    \subfloat[Biliteral roots]{
      \begin{tabular}{|>{\bfseries}fc|-c|-c|-c|}
        \hline
        \SetRowStyle{\bfseries} Form & Present Perfect & Pluperfect & Future Perfect \tnl
        \cline{2-4}
        \SetRowStyle{\scshape} & \acs{prs};\acs{perf} & \acs{plup} & \acs{fut};\acs{perf} \tnl
        \hline
        1 & 
        C\sub1{e}C\sub2{i} & 
        C\sub1{e}C\sub2{o} & 
        C\sub1{e}C\sub2{a}
        \tnl
        2 & 
        {ë}C\sub1C\sub1{i}C\sub2 & 
        {ë}C\sub1C\sub1{o}C\sub2 & 
        {ë}C\sub1C\sub1{a}C\sub2
        \tnl
        3 & 
        C\sub1{e}C\sub2C\sub2{i} & 
        C\sub1{e}C\sub2C\sub2{o} & 
        C\sub1{e}C\sub2C\sub2{a}
        \tnl
        4 & 
        {i}C\sub1C\sub1{e}C\sub2{i} & 
        {i}C\sub1C\sub1{e}C\sub2{o} & 
        {i}C\sub1C\sub1{e}C\sub2{a}
        \tnl
        5 & 
        {me}C\sub1{e}C\sub2{i} & 
        {me}C\sub1{e}C\sub2{o} & 
        {me}C\sub1{e}C\sub2{a}  
        \tnl
        6 & 
        {ta}C\sub1{e}C\sub2C\sub2{i} & 
        {ta}C\sub1{e}C\sub2C\sub2{o} & 
        {ta}C\sub1{e}C\sub2C\sub2{a}
        \tnl
        7 & 
        {ë}C\sub1{e}C\sub2{i} & 
        {ë}C\sub1{e}C\sub2{o} & 
        {ë}C\sub1{e}C\sub2{a}
        \tnl
        \hline
    \end{tabular}}
    \caption{Perfect series transfix patterns\label{tab:vm_perfect_series}}
  \end{table}

  \newpage
  \subsubsection{The Imperatives}
  \label{sssec:vm_imperatives}
  
  Qevesa possesses two imperatives, one for each aspect.  The Form 7 verb roots do not possess an imperative. 

  \begin{itemize*}
    \item The \textbf{perfective} is used for single complete actions. 
    \item The \textbf{imperfective} is used for continuous or otherwise incomplete actions. 
  \end{itemize*}

  The transfix patterns for this series are listed in Table~\ref{tab:vm_imperative_series}. 

  \begin{table}[htpb]\small\capstart
      \subfloat[Triliteral roots]{
        \begin{tabular}{|>{\bfseries}fc|-c|-c|}
          \hline
          \SetRowStyle{\bfseries} Form & Perfective Imperative & Imperfective Imperative \tnl
          \cline{2-3}
          \SetRowStyle{\scshape} & \acs{pfv};\acs{imp} & \acs{ipfv};\acs{imp} \tnl
          \hline
          1 & 
          C\sub1{ia}C\sub2{u}C\sub3 & 
          C\sub1{á}C\sub2{o}C\sub3
          \tnl
          2 & 
          C\sub1{ia}C\sub2C\sub2{u}C\sub3 & 
          C\sub1{á}C\sub2C\sub2{o}C\sub3
          \tnl
          3 & 
          C\sub1{ia}C\sub2C\sub3{u} & 
          C\sub1{á}C\sub2C\sub3{o}
          \tnl
          4 & 
          {i}C\sub1C\sub2{ia}C\sub3{u} & 
          {i}C\sub1C\sub2{á}C\sub3{o}
          \tnl
          5 & 
          {me}C\sub1{ia}C\sub2{u}C\sub3 & 
          {me}C\sub1{á}C\sub2{o}C\sub3  
          \tnl
          6 & 
          {ta}C\sub1C\sub2{ia}C\sub3{u} & 
          {ta}C\sub1C\sub2{á}C\sub3{o}
          \tnl
          \hline
        \end{tabular}
      }\\
      \subfloat[Biliteral roots]{
        \begin{tabular}{|>{\bfseries}fc|-c|-c|}
          \hline
          \SetRowStyle{\bfseries} Form & Perfective Imperative & Imperfective Imperative \tnl
          \cline{2-3}
          \SetRowStyle{\scshape} & \acs{pfv};\acs{imp} & \acs{ipfv};\acs{imp} \tnl
          \hline
          1 & 
          C\sub1{ia}C\sub2{u} & 
          C\sub1{á}C\sub2{o}
          \tnl
          2 & 
          {ja}C\sub1C\sub1{u}C\sub2 & 
          {á}C\sub1C\sub1{o}C\sub2
          \tnl
          3 & 
          C\sub1{ia}C\sub2C\sub2{u} & 
          C\sub1{á}C\sub2C\sub2{o}
          \tnl
          4 & 
          {i}C\sub1C\sub1{ia}C\sub2{u} & 
          {i}C\sub1C\sub1{á}C\sub2{o}
          \tnl
          5 & 
          {me}C\sub1{ia}C\sub2{u} & 
          {me}C\sub1{á}C\sub2{o}  
          \tnl
          6 & 
          {ta}C\sub1{ia}C\sub2C\sub2{u} & 
          {ta}C\sub1{á}C\sub2C\sub2{o}
          \tnl
          \hline
        \end{tabular}}
      \caption{Imperative series transfix patterns\label{tab:vm_imperative_series}}
  \end{table}

  %\newpage
  \subsection{Topical Agreement}
  \label{ssec:vm_topical_agreement}

  Qevesa is a topic-prominent language that tends towards a split-S active dechticaetiative morphosyntactic alignment.  As a result, verbs are marked for agreement with the topic of the sentence, rather than the subject or agent.  The topic of the sentence is the noun phrase in the focal case. 

  %\subsubsection{Primary Agreement}
  %\label{sssec:vm_primary_agreement}

  The topic of the verb primarily indicates its experiencer, agent/donor, patient/recipient, or theme. 
  It agrees with the topical noun phrase in animacy and number.
  %If the topical noun phrase is a pronoun, it may be omitted. 
  The suffixes for topical agreement are given in Table~\ref{tab:vm_primary_agreement}.

  \begin{table}[htpb]\small\capstart
    \begin{tabular}{|>{\scshape}fc>{\scshape}c|-c|-c|-c|}
      \hline
      \multicolumn{2}{|-c|}{\SetRowStyle{\bfseries}} & Nominative & Absolutive & Secundative \tnl
      \cline{3-5}
      \SetRowStyle{\scshape} & & \acs{nom} & \acs{abs} & \acs{sdt} \tnl
      \hline
      \acs{anim};\acs{sg}   & \acs{asg} & -(a)m & -(a)š & -(a)t  \tnl
      \acs{anim};\acs{du}   & \acs{adu} & -vám  & -váš  & -vát  \tnl
      \acs{anim};\acs{pl}   & \acs{apl} & -sám  & -sáš  & -sát  \tnl
      \acs{inanim};\acs{sg} & \acs{isg} & -nom  & -noš  & -not  \tnl
      \acs{inanim};\acs{du} & \acs{idu} & -vom  & -voš  & -vot  \tnl
      \acs{inanim};\acs{pl} & \acs{ipl} & -som  & -soš  & -nost  \tnl
      %1sg      & -(ě/e/i/je)m & -(ě/e/i/je)š & -(ě/e/i/je)t \tnl
      %2sg      & -tam         & -taš         & -tat  \tnl
      %1du;inc  & -jévm        & -jévš        & -jévt  \tnl
      %1du;exc  & -čévm        & -čévš        & -čévt  \tnl
      %2du      & -távm        & -távš        & -távt  \tnl
      %1pl;inc  & -jésm        & -jéšš        & -jést  \tnl
      %1pl;exc  & -čésm        & -čéšš        & -čést  \tnl
      %2pl      & -tásm        & -tášš        & -tást  \tnl
      %\hline
      \hline
    \end{tabular}
    \caption{Primary topical agreement\label{tab:vm_primary_agreement}}
  \end{table}

  %The first person singular uses \qevesa{-ě} after a consonant;  \qevesa{-e} after \qevesa{e-} (which merge and become \qevesa{-é-}); \qevesa{-i} after \qevesa{a-} and \qevesa{o-}; and \qevesa{-je} after all other vowels, including long vowels.  The third-person singular suffixes insert an epenthetic \qevesa{-a-} when the suffix follows a consonant.  The use of the singular, dual, and plural numbers is described in Section~\ref{ssec:nm_number}.

  %The animate endings are used when the topic of the verb is an animate noun, 

  %The third person inanimate ending is used when the topic of the verb is an inanimate noun, and is not marked for number.

  \subsubsection{Nominative Topic}
  \label{sssec:vm_nom_topic}

  An nominative topic indicates that the noun phrase in the focal case is the voluntary experiencer of an intransitive verb; the agent of a transitive verb; and the donor of a ditransitive verb.

  \subsubsection{Absolutive Topic}
  \label{sssec:vm_abs_topic}

  An absolutive topic indicates that the noun phrase in the focal case is the involuntary experiencer of an intransitive verb; the patient of a transitive verb; and the recipient of a ditransitive verb. 

  \subsubsection{Secundative Topic}
  \label{sssec:vm_sdt_topic}

  A secundative topic indicates that the noun phrase in the focal case is the theme of a ditransitive verb.  The secundative topic suffix is also used in cases when the topic is instrumental, locative or adverbial.
  
  \subsection{Modality}
  \label{ssec:vm_modality}

  Qevesa predominantly indicates modality by means of suffixes, with the exception of the imperatives described in Section~\ref{sssec:vm_imperatives}. 
  
  \ToBeWritten

  \section{Auxiliary Verbs}
  \label{sec:vm_auxiliary}

  Periphrastic constructions, such as polarity, are indicated with a series of auxiliary verbs. 

  The auxiliary verb is inflected, taking the conjugated form of the main verb, which precedes it in the infinitive.

  \begin{exe}
    \ex\label{exe:vm_auxiliary_conjugation} \textit{stem}\bs\acs{inf} \textit{auxiliary}\bs\textsc{aspect;tense;mood-topic(-mood)}
  \end{exe}

  \subsection{Polarity}
  \label{ssec:vm_polarity}

  The most commonly-used auxiliary verbs are those that indicate polarity.  The affirmative verb, \qevesa{zuru}, is generally only used in situations when an explicitly positive statement is to be made.  The negative verb, \qevesa{nuku}, is more commonly used, and shares the same root as the word for ‘zero’ or ‘none’.

  \begin{exe}
    \ex \qevesa{Misa turum niukasám.}
    \glll Misa turum niuka-sám\\
    \acs{3p}\acs{pl}.\acs{foc} write\bs\acs{inf} \acs{neg}\bs\acs{fut};\acs{pfv}-\acs{apl};\acs{nom}\\
    {They} {write} {will not}\\
    \glt They will not write.
  \end{exe}

  %Mood is another important category marked on the Qevesa verb.  There are eight primary moods: indicative, admirative, irrealis, alethic, necessitative, precative, volitive, and hypothetical.

  %The suffixes for mood are given in Table~\ref{tab:vm_modal_suffixes}.

%  \begin{table}[htpb]\small\capstart
%      \begin{tabular}{|>{\bfseries}fc->{\scshape}c|-c|}
%        \hline
%        \multicolumn{2}{|fc|}{\SetRowStyle{\bfseries}Mood} & Suffix \tnl
%        \hline
%        Indicative & \acs{ind} & -u   \tnl
%        Admirative & \acs{mir} & -óra \tnl
%        Irrealis   & \acs{irr} & -il  \tnl
%        Alethic    & \acs{ale} & -en \tnl
%        Commissive & \acs{com} & -ec  \tnl
%        Directive  & \acs{dir} & -ła  \tnl
%        Volitive   & \acs{vol} & -ir  \tnl
%        \hline
%      \end{tabular}
%      \caption{Verbal mood suffixes\label{tab:vm_modal_suffixes}}
%  \end{table}
%
%  \subsubsection{Indicative Mood}
%  \label{sssec:vm_indicative}
%
%  The indicative mood is the default mood.  It is essentially a realis mood, indicating the factual nature of the statement.
%
%  \subsubsection{Admirative Mood}
%  \label{sssec:vm_admirative}
%
%  The admirative mood is also a realis mood, that indicates new or unexpected information.
%
%  \begin{exe}
%    \ex \emph{EXAMPLE}
%  \end{exe}
%
%  \subsubsection{Irrealis Mood}
%  \label{sssec:vm_irrealis}
%
%  The irrealis mood denotes a counterfactual or non-actual sense.
%
%  \begin{exe}
%    \ex \emph{EXAMPLE}
%  \end{exe}
%
%  \subsubsection{Alethic Mood}
%  \label{sssec:vm_alethic}
%
%  The alethic mood denotes the logical necessity of the statement.
%
%  \begin{exe}
%    \ex \emph{EXAMPLE}
%  \end{exe}
%
%  \subsubsection{Commissive Mood}
%  \label{sssec:vm_commissive}
%
%  The commissive mood indicates a commitment or promise to do something.
%
%  \begin{exe}
%    \ex \emph{EXAMPLE}
%  \end{exe}
%
%  \subsubsection{Directive Mood}
%  \label{sssec:vm_directive}
%
%  The directive mood indicates that the action is a request or order.
%
%  \begin{exe}
%    \ex \emph{EXAMPLE}
%  \end{exe}
%
%  \subsubsection{Volitive Mood}
%  \label{sssec:vm_optative}
%
%  The volitive mood indicates a hope, desire, or wishes that the action denoted by the verb should come about.
%
%  \begin{exe}
%    \ex \emph{EXAMPLE}
%  \end{exe}
%
%



  %Both of these verbs conjugate to aspect as shown in Table~\ref{tab:vm_polarity_auxiliary_aspect}. 

%  \begin{table}[htpb]\small\capstart
%      \begin{tabular}{|>{\bfseries}fc->{\scshape}c|-c|-c|}
%        \hline
%        \SetRowStyle{\bfseries} & & \multicolumn{2}{-c|}{Polarity} \tnl
%        \cline{3-4}
%        \SetRowStyle{\scshape} & & \acs{aff} & \acs{neg} \tnl
%        \hline
%        Imperfective  & \acs{ipfv}           & rusi  & nuki \tnl
%        Stative       & \acs{stat}           & ruise & nuike \tnl
%        Durative      & \acs{dur};\acs{ipfv} & rusú  & nukú \tnl
%        Frequentative & \acs{freq}           & ruso  & nuko \tnl
%        Habitual      & \acs{hab}            & rusa  & nuka \tnl
%        \hline\hline
%        Perfective    & \acs{pfv}            & riosa & nioka \tnl
%        Inchoative    & \acs{inch}           & riuso & niuko \tnl
%        Cessative     & \acs{cess}           & rísa  & níka \tnl
%        Durative      & \acs{dur};\acs{pfv}  & riasu & niaku \tnl
%        Momentane     & \acs{momt}           & riusa & niuka \tnl
%        \hline
%      \end{tabular}
%      \caption{Polar verb aspectual conjugation\label{tab:vm_polarity_auxiliary_aspect}}
%  \end{table}


  % Peter-FOC doctor-ABS rusi-m-u
  % Peter is a doctor.
  % doctor-FOC Peter-NOM rusi-š-u
  % A doctor Peter is.

  % We are not hiding a drunk Verdurian in the room!
  % Potmi Verdurija akkaperossa čésam rafuk zumišóra!
  % drunk Verdurija-ø ak~kaper-ossa čés-am rafuk zummi-š-óra
  % drunk Verdurian-FOC DEF~room-INE 1PL;EXC-NOM hide\INF not\IPFV-ASG;ABS-MIR
  % drunk Verdurian in the room we hide are not!

  % putum - drink, alcohol
  % potmi = drunk
  % kupur = room
  % rufuk = hide, conceal, guard
  % tupun = die
  % itpunu = kill
  % šanam = mouse
  % nuvum = ???
  % navoim = cat
  % nulup = hunger, desire ??

  % You killed all the mice, so the cat is hungry.
  % Ta a mün aššanamesaš itpionamu, annavoima nuilpešen.
  % /taʔamyn aɕːamesaʂ itpʲionamu anːavoima nuilpeʂen/
  % Ta a=mün aš~šanam-es-aš itpiona-m-u, an~navoim-a nuilpe-š-en.
  % 2SG.FOC DEF=all DEF~mouse-PL-ABS kill\PFV-ASG;NOM-IND, DEF~cat-FOC hungry\STAT-ASG;ABS-ALE
  % You the all mice killed, the cat is hungry therefore

  % All the mice were killed by you, so the cat is hungry.
  % A mün aššanamesa tam itpionašu, annavoima nuilpešen.
  % /a myn aɕːanamesa tam itpʲionaʂu, anːavoima nuilpeʂen/
  % A=mün aš~šanam-es-a tam itpiona-š-u, an~navoim-a nuilpe-š-en.
  % DEF=all DEF~mouse-PL-FOC 2SG.NOM kill\PFV-ASG;ABS-IND, DEF~cat-FOC hungry\STAT-ASG;ABS-ALE
  % The all mice you killed, the cat is hungry therefore

  %\newpage
%  Sometimes the polar auxiliaries will be conjugated to a different aspect than their head verb, especially to indicate semantic nuances, for example:
%
%  \begin{exe}
%    \ex \qevesa{Assošima jem zíma nuttúlašu.}
%    \glll As-sošim-a jem zímma nuttúl-aš-u\\
%    \textsc{def-}girl\textsc{-foc} \textsc{1sg.nom} \textsc{neg\bs cess} think\textsc{\bs F2.dur;ipfv-asg;abs-ind}\\
%    {the girl} {I} {not stop} {thinking about her}\\
%    \glt I cannot stop thinking about that girl.
%  \end{exe}

% Using a determiner, such as isátka, strictly refers to location so is not necessary.
% e.g.  Asisátka assošima jem zímma nuttúlašu.

%  \subsection{Evidentiality}
%  \label{ssec:vm_evidentiality}
%
%  Evidentiality may also be expressed by means of auxiliary verbs.  Qevesa possesses a set of auxiliary verbs which distinguish four degrees of evidentiality: witness, reportative, inferential, and assumptive. 
%
%  All of the roots of the evidential auxiliaries are also verbs in their own right.  However, they conjugate as Form VIII verbs, with some slightly irregular pattern forms.  Their conjugation is given in Table~\ref{tab:vm_evidentiality_conjugation}.
%
%  \begin{table}[htpb]\small\capstart
%      \begin{tabular}{|>{\bfseries}fc->{\scshape}c|-c|-c|-c|-c|}
%        \hline
%        \SetRowStyle{\bfseries} & & \multicolumn{4}{-c|}{Evidentiality} \tnl
%        \cline{3-6}
%        \SetRowStyle{\scshape} &  & exp   & rep    & infr   & asm		 \tnl
%        \hline
%        Imperfective  & ipfv     & murri  & łukši  & kučti  & quspi  \tnl
%        Stative       & stat     & muirre & łuikše & kuičte & quispe \tnl
%        Durative      & dur;ipfv & murrú  & łukšú  & kučtú  & quspú  \tnl
%        Frequentative & freq     & murro  & łukšo  & kučto  & quspo  \tnl
%        Habitual      & hab      & murra  & łukša  & kučta  & quspa  \tnl
%        \hline\hline
%        Perfective    & pfv      & miorra & łiokša & kiočta & qiospa \tnl
%        Inchoative    & inch     & miurro & łiukšo & kiučto & qiuspo \tnl
%        Cessative     & cess     & mírra  & łíkša  & kíčta  & qíspa  \tnl
%        Durative      & dur;pfv  & miarru & łiakšu & kiačtu & qiaspu \tnl
%        Momentane     & momt     & miurra & łiukša & kiučta & qiuspa \tnl
%        \hline
%      \end{tabular}
%      \caption{Conjugation of the evidential verbs \label{tab:vm_evidentiality_conjugation}}
%  \end{table}
%
%  As with all auxiliary constructions, use of the evidential auxiliaries is not mandatory; rather, they are used to provide additional information. 
%
%  \subsubsection{Witness}
%  \label{sssec:vm_evd_witness}
%
%  The witness degree of evidentiality is denoted by the verb \qevesa{murru}, meaning ‘to see’.  It is used when the speaker was a witness to the event.
%
%  \subsubsection{Reportative}
%  \label{sssec:vm_evd_reportative}
%
%  The reportative degree of evidentiality is denoted by the verb \qevesa{łukšu}, which has the same consonantal root as the verb \qevesa{łukuš} ‘to hear (speech)’.
%
%  \subsubsection{Inferential}
%  \label{sssec:vm_evd_inferential}
%
%  The inferential degree of evidentiality is denoted by the verb \qevesa{kučtu}.  It is used when the speaker infers that the event occurred but was not a witness.
%
%  \subsubsection{Assumptive}
%  \label{sssec:vm_evd_assumption}
%
%  The assumption degree of evidentiality is denoted by the verb \qevesa{quspu}.  It is used when the speaker is making an assumption about the occurrence of the event.
%

  \section{Irregular Verbs}
  \label{sec:vm_irregular}

  %Qevesa verbal morphology is highly regular, with most irregularities occurring due to consonant groupings.  %Roots that contain a /h/ frequently possess irregular forms, mainly because the /h/ will be elided or reduced to a pre-aspiration of the following consonant and the previous vowel lengthened.  This may be represented in writing as well as speech.
  %However, a number of common roots do possess irregular forms, and these are outlined in the following sections.
  
  \ToBeWritten

%  \subsection{The Copulae}
%  \label{ssec:vm_copulae}
%
%  The most frequently-used irregular verb in Qevesa is the copula \qevesa{teši}.  It is one of a number of verbs which do not possess a regular infinitive of the form \qevesa{C\sub1{u}C\sub2{u}}; it also possesses a negative form (\qevesa{zemi}\footnotemark{}), unlike most other verbs.  The basic conjugated forms of \qevesa{teši} are given in Table~\ref{tab:vm_copulae_aspectual_conjugation}.
%  \footnotetext{This is also the same consonantal root as the negative verb \qevesa{zumu} and associated forms which translate as ‘zero’ or ‘none’.}
%
%  \begin{table}[htpb]\small\capstart
%      \begin{tabular}{|>{\bfseries}fc->{\scshape}c|-c|-c|}
%        \hline
%        \SetRowStyle{\bfseries} & & Non-negative & Negative \tnl
%        \cline{3-4}
%        \SetRowStyle{\scshape} & & cop & neg \tnl
%        \hline
%        Infinitive    & inf      & teši   & zemi   \tnl
%        \hline\hline
%        Imperfective  & ipfv     & tušši  & zummi  \tnl
%        Stative       & stat     & tuišše & zuimme \tnl
%        Durative      & dur;ipfv & tuššú  & zummú  \tnl
%        Frequentative & freq     & tuššo  & zummo  \tnl
%        Habitual      & hab      & tušša  & zumma  \tnl
%        \hline\hline
%        Perfective    & pfv      & tiošša & ziomma \tnl
%        Inchoative    & inch     & tiuššo & ziummo \tnl
%        Cessative     & cess     & tíšša  & zímma  \tnl
%        Durative      & dur;pfv  & tiaššu & ziammu \tnl
%        Momentane     & momt     & tiušša & ziumma \tnl
%        \hline
%      \end{tabular}
%      \caption{Aspectual conjugation of the copulae \qevesa{teši} and \qevesa{zemi}\label{tab:vm_copulae_aspectual_conjugation}}
%  \end{table}
%
%
%  The copulae can also be used in an existential sense, but only in nominal phrases and never with stative verbs.  They play a major role in honorific registers, as described in Chapter~\ref{ch:registers}.

\end{document}
