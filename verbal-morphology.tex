\documentclass[grammar]{subfiles}
\begin{document}
  \chapter{Verbal Morphology}
  \label{ch:verbal_morphology}

  \section{Features}
  \label{sec:vm_features}

  The consonantal root patterns in Qevesa are used to form basic morphological paradigms. Qevesa verbs are highly inflected, indicating aspect by transfix patterns; topical agreement and modality are marked by agglutinative suffixes. All other constructions, including tense, voice, polarity and evidentiality, are indicated by periphrasis or syntax.

  The stem consists of the root and zero or more derivational affixes conjugated to a particular aspect. 

  \section{The Infinitive}
  \label{sec:vm_infinitive}

  The infinitive of the verb is the citation form of the root. It is formed by inserting a \qevesa{-u-} into P\sub{12} and P\sub{23}, resulting in \qevesa{C\sub1{u}C\sub2{u}C\sub3}. 
  %The most common use of the infinitive is in auxiliary constructions, as described in Sections~\ref{sec:vm_auxiliary} and \ref{sec:vs_auxiliary}.

  \section{Conjugation}
  \label{sec:vm_conjugation}

  Qevesa is a highly synthetic language, and verbs are conjugated to indicate aspect, topical agreement, and mood. The conjugated form of the verb is as follows:

  \begin{exe}
    \ex\label{exe:vm_conjugation} \textit{stem}\textsc{.aspect-topic-mood}
  \end{exe}

  \subsection{Aspect}
  \label{ssec:vm_aspect}

  Aspect is possibly the most important grammatical category marked on the verb. 
  Instead of tense, aspect is used to mark the temporal flow (or lack thereof) of verbs. 
  Qevesa distinguishes between imperfective aspects (those that are ongoing, habitual, repeated or generally containing internal structure) and perfective aspects (those that are viewed as a single whole). 
  As a result, there are two primary transfix patterns that correspond to the imperfective and perfective aspects, and a number of secondary transfix patterns which indicate various subtle (mainly semantic) differences.	

  There are ten different aspects in total, five imperfective and five perfective.

  \subsubsection{The Imperfective Aspects}
  \label{sssec:vm_imperfective}

  The imperfective aspects are used to indicate:

  \begin{itemize*}
    \item actions in progress or ongoing states and activities, with significant course (in opinion of the speaker);
    \item activities posing the background for other (perfective) activities;
    \item simultaneous activities;
    \item durative activities, lasting through some time;
    \item multiple (iterative or frequentative) activities;
    \item habitual activities;
    \item motions without a strict aim;
    \item continuous states.
  \end{itemize*}

  The triliteral root patterns for the imperfective aspects are given in Table~\ref{tab:vm_imperfective_aspects}.

  \begin{table}[htpb]\small\capstart
      \subfloat[Triliteral roots]{
        \begin{tabular}{|>{\bfseries}fc|-c|-c|-c|-c|-c|}
          \hline
          \SetRowStyle{\bfseries} Form & Imperfective & Stative & Durative & Frequentative & Habitual \tabularnewline
          \cline{2-6}
          \SetRowStyle{\scshape} & ipfv & stat & dur;pfv & freq & hab \tabularnewline
          \hline
          I & 
          C\sub1{u}C\sub2{i}C\sub3 & 
          C\sub1{ui}C\sub2{e}C\sub3 & 
          C\sub1{u}C\sub2{ú}C\sub3 & 
          C\sub1{u}C\sub2{o}C\sub3 & 
          C\sub1{u}C\sub2{a}C\sub3
          \tabularnewline
          II & 
          C\sub1{u}C\sub2C\sub2{i}C\sub3 & 
          C\sub1{ui}C\sub2C\sub2{e}C\sub3 & 
          C\sub1{u}C\sub2C\sub2{ú}C\sub3 & 
          C\sub1{u}C\sub2C\sub2{o}C\sub3 & 
          C\sub1{u}C\sub2C\sub2{a}C\sub3
          \tabularnewline
          III & 
          {ja}C\sub1{u}C\sub2C\sub3{i} & 
          {ja}C\sub1{ui}C\sub2C\sub3{e} & 
          {ja}C\sub1{u}C\sub2C\sub3{ú} & 
          {ja}C\sub1{u}C\sub2C\sub3{o} & 
          {ja}C\sub1{u}C\sub2C\sub3{a}
          \tabularnewline
          IV & 
          {te}C\sub1{u}C\sub2{i}C\sub3 & 
          {te}C\sub1{ui}C\sub2{e}C\sub3	& 
          {te}C\sub1{u}C\sub2{ú}C\sub3 & 
          {te}C\sub1{u}C\sub2{o}C\sub3 & 
          {te}C\sub1{u}C\sub2{a}C\sub3  
          \tabularnewline
          V & 
          {i}C\sub1C\sub2{u}C\sub3C\sub3{i} & 
          {i}C\sub1C\sub2{ui}C\sub3C\sub3{e} & 
          {i}C\sub1C\sub2{u}C\sub3C\sub3{ú} & 
          {i}C\sub1C\sub2{u}C\sub3C\sub3{o} & 
          {i}C\sub1C\sub2{u}C\sub3C\sub3{a}
          \tabularnewline
          VI & 
          {ta}C\sub1C\sub2{u}C\sub3C\sub3{i} & 
          {ta}C\sub1C\sub2{ui}C\sub3C\sub3{e} & 
          {ta}C\sub1C\sub2{u}C\sub3C\sub3{ú} & 
          {ta}C\sub1C\sub2{u}C\sub3C\sub3{o} & 
          {ta}C\sub1C\sub2{u}C\sub3C\sub3{a}
          \tabularnewline
          VII & 
          {sa}C\sub1{u}C\sub2C\sub2{i}C\sub3 & 
          {sa}C\sub1{ui}C\sub2C\sub2{e}C\sub3 & 
          {sa}C\sub1{u}C\sub2C\sub2{ú}C\sub3 & 
          {sa}C\sub1{u}C\sub2C\sub2{o}C\sub3 & 
          {sa}C\sub1{u}C\sub2C\sub2{a}C\sub3
          \tabularnewline
          VIII & 
          C\sub1{u}C\sub2C\sub3{i} & 
          C\sub1{ui}C\sub2C\sub3{e} & 
          C\sub1{u}C\sub2C\sub3{ú} & 
          C\sub1{u}C\sub2C\sub3{o} & 
          C\sub1{u}C\sub2C\sub3{a}
          \tabularnewline
          IX & 
          {ě}C\sub1C\sub2{u}C\sub2C\sub3{i} & 
          {ě}C\sub1C\sub2{ui}C\sub2C\sub3{e} & 
          {ě}C\sub1C\sub2{u}C\sub2C\sub3{ú} & 
          {ě}C\sub1C\sub2{u}C\sub2C\sub3{o} & 
          {ě}C\sub1C\sub2{u}C\sub2C\sub3{a}
          \tabularnewline
          \hline
        \end{tabular}
      }\\
      \subfloat[Biliteral roots]{
        \begin{tabular}{|>{\bfseries}fc|-c|-c|-c|-c|-c|}
          \hline
          \SetRowStyle{\bfseries} Form & Imperfective & Stative & Durative & Frequentative & Habitual \tabularnewline
          \cline{2-6}
          \SetRowStyle{\scshape} & ipfv & stat & dur;pfv & freq & hab \tabularnewline
          \hline
          I & 
          C\sub1{u}C\sub2{i} & 
          C\sub1{ui}C\sub2{e} & 
          C\sub1{u}C\sub2{ú} & 
          C\sub1{u}C\sub2{o} & 
          C\sub1{u}C\sub2{a}
          \tabularnewline
          II & 
          C\sub1{u}C\sub2C\sub2{i} & 
          C\sub1{ui}C\sub2C\sub2{e} & 
          C\sub1{u}C\sub2C\sub2{ú} & 
          C\sub1{u}C\sub2C\sub2{o} & 
          C\sub1{u}C\sub2C\sub2{a}
          \tabularnewline
          III & 
          {ja}C\sub1{u}C\sub2\sub2{i} & 
          {ja}C\sub1{ui}C\sub2C\sub2{e} & 
          {ja}C\sub1{u}C\sub2\sub2{ú} & 
          {ja}C\sub1{u}C\sub2\sub2{o} & 
          {ja}C\sub1{u}C\sub2\sub2{a}
          \tabularnewline
          IV & 
          {te}C\sub1{u}C\sub2{i} & 
          {te}C\sub1{ui}C\sub2{e}	& 
          {te}C\sub1{u}C\sub2{ú} & 
          {te}C\sub1{u}C\sub2{o} & 
          {te}C\sub1{u}C\sub2{a}  
          \tabularnewline
          V & 
          {i}C\sub1C\sub1{u}C\sub2C\sub2{i} & 
          {i}C\sub1C\sub1{ui}C\sub2C\sub2{e} & 
          {i}C\sub1C\sub1{u}C\sub2C\sub2{ú} & 
          {i}C\sub1C\sub1{u}C\sub2C\sub2{o} & 
          {i}C\sub1C\sub1{u}C\sub2C\sub2{a}
          \tabularnewline
          VI & 
          {ta}C\sub1C\sub1{u}C\sub2C\sub2{i} & 
          {ta}C\sub1C\sub1{ui}C\sub2C\sub2{e} & 
          {ta}C\sub1C\sub1{u}C\sub2C\sub2{ú} & 
          {ta}C\sub1C\sub1{u}C\sub2C\sub2{o} & 
          {ta}C\sub1C\sub1{u}C\sub2C\sub2{a}
          \tabularnewline
          VII & 
          {sa}C\sub1{u}C\sub2C\sub2{i} & 
          {sa}C\sub1{ui}C\sub2C\sub2{e} & 
          {sa}C\sub1{u}C\sub2C\sub2{ú} & 
          {sa}C\sub1{u}C\sub2C\sub2{o} & 
          {sa}C\sub1{u}C\sub2C\sub2{a}
          \tabularnewline
          VIII & 
          C\sub1{u}C\sub2C\sub2{i} & 
          C\sub1{ui}C\sub2C\sub2{e} & 
          C\sub1{u}C\sub2C\sub2{ú} & 
          C\sub1{u}C\sub2C\sub2{o} & 
          C\sub1{u}C\sub2C\sub2{a}
          \tabularnewline
          IX & 
          {ě}C\sub1{u}C\sub2C\sub2{i} & 
          {ě}C\sub1{ui}C\sub2C\sub2{e} & 
          {ě}C\sub1{u}C\sub2C\sub2{ú} & 
          {ě}C\sub1{u}C\sub2C\sub2{o} & 
          {ě}C\sub1{u}C\sub2C\sub2{a}
          \tabularnewline
          \hline
        \end{tabular}}
      \caption{Imperfective aspectual patterns\label{tab:vm_imperfective_aspects}}
  \end{table}

  % \newpage
  \subsubsection{The Perfective Aspects}
  \label{sssec:vm_perfective}

  The perfective aspects generally indicate activities that have distinct beginnings and ends which are relevant to the speaker. This implies past or future activities, but not present activities—an activity which is presently occurring cannot be ended, so it cannot be perfective. The perfective indicates the following:

  \begin{itemize*}
    \item states and activities which were ended or which will be ended, with insignificant course, or treated as a whole by the speaker;
    \item single-time activities;
    %\item actions whose goals have already been achieved;
    %\item reasons for the state;
    \item the beginning of the activity or the state;
    \item the end of the activity or the state;
    \item activities executed in many places, on many objects or by many subjects at the same time;
    \item actions or states which last some time
  \end{itemize*}

  The triliteral root patterns for the perfective aspects are given in Table~\ref{tab:vm_perfective_aspects}.

  \begin{table}[htpb]\small\capstart
      \subfloat[Triliteral roots]{
        \begin{tabular}{|>{\bfseries}fc|-c|-c|-c|-c|-c|}
          \hline
          \SetRowStyle{\bfseries} Form & Perfective & Inchoative & Cessative & Durative & Momentane \tabularnewline
          \cline{2-6}
          \SetRowStyle{\scshape} & pfv & inch & cess & dur;pfv & momt \tabularnewline
          \hline
          I & 
          C\sub1{i}C\sub2{o}C\sub3{a} & 
          C\sub1{i}C\sub2{u}C\sub3{o} & 
          C\sub1{i}C\sub2{a}C\sub3{a} & 
          C\sub1{i}C\sub2{a}C\sub3{u} & 
          C\sub1{i}C\sub2{u}C\sub3{a}
          \tabularnewline
          II & 
          C\sub1{i}C\sub2C\sub2{o}C\sub3{a} & 
          C\sub1{i}C\sub2C\sub2{u}C\sub3{o} & 
          C\sub1{i}C\sub2C\sub2{a}C\sub3{a} & 
          C\sub1{i}C\sub2C\sub2{a}C\sub3{u} & 
          C\sub1{i}C\sub2C\sub2{u}C\sub3{a}
          \tabularnewline
          III & 
          {ja}C\sub1{io}C\sub2C\sub3{a} & 
          {ja}C\sub1{iu}C\sub2C\sub3{o} & 
          {ja}C\sub1{í}C\sub2C\sub3{a} & 
          {ja}C\sub1{ia}C\sub2C\sub3{u} & 
          {ja}C\sub1{iu}C\sub2C\sub3{a}
          \tabularnewline
          IV & 
          {te}C\sub1{i}C\sub2{o}C\sub3{a}	& 
          {te}C\sub1{i}C\sub2{u}C\sub3{o}	& 
          {te}C\sub1{i}C\sub2{a}C\sub3{a}	& 
          {te}C\sub1{i}C\sub2{a}C\sub3{u}	& 
          {te}C\sub1{i}C\sub2{u}C\sub3{a}	 
          \tabularnewline
          V & 
          {i}C\sub1C\sub2{io}C\sub3C\sub3{a} & 
          {i}C\sub1C\sub2{iu}C\sub3C\sub3{o} & 
          {i}C\sub1C\sub2{í}C\sub3C\sub3{a} & 
          {i}C\sub1C\sub2{ia}C\sub3C\sub3{u} & 
          {i}C\sub1C\sub2{iu}C\sub3C\sub3{a}
          \tabularnewline
          VI & 
          {ta}C\sub1C\sub2{io}C\sub3C\sub3{a} & 
          {ta}C\sub1C\sub2{iu}C\sub3C\sub3{o} & 
          {ta}C\sub1C\sub2{í}C\sub3C\sub3{a} & 
          {ta}C\sub1C\sub2{ia}C\sub3C\sub3{u} & 
          {ta}C\sub1C\sub2{iu}C\sub3C\sub3{a}
          \tabularnewline
          VII & 
          {sa}C\sub1{i}C\sub2C\sub2{o}C\sub3{a} & 
          {sa}C\sub1{i}C\sub2C\sub2{u}C\sub3{o} & 
          {sa}C\sub1{i}C\sub2C\sub2{a}C\sub3{a} & 
          {sa}C\sub1{i}C\sub2C\sub2{a}C\sub3{u} & 
          {sa}C\sub1{i}C\sub2C\sub2{u}C\sub3{a}
          \tabularnewline
          VIII & 
          C\sub1{io}C\sub2C\sub3{a} & 
          C\sub1{iu}C\sub2C\sub3{o} & 
          C\sub1{í}C\sub2C\sub3{a} & 
          C\sub1{ia}C\sub2C\sub3{u} & 
          C\sub1{iu}C\sub2C\sub3{a}
          \tabularnewline
          IX & 
          {ě}C\sub1C\sub2{io}C\sub2C\sub3{a} & 
          {ě}C\sub1C\sub2{iu}C\sub2C\sub3{o} & 
          {ě}C\sub1C\sub2{í}C\sub2C\sub3{a} & 
          {ě}C\sub1C\sub2{ia}C\sub2C\sub3{u} & 
          {ě}C\sub1C\sub2{iu}C\sub2C\sub3{a}
          \tabularnewline
          \hline
        \end{tabular}
      }\\
      \subfloat[Biliteral roots]{
        \begin{tabular}{|>{\bfseries}fc|-c|-c|-c|-c|-c|}
          \hline
          \SetRowStyle{\bfseries} Form & Perfective & Inchoative & Cessative & Durative & Momentane \tabularnewline
          \cline{2-6}
          \SetRowStyle{\scshape} & pfv & inch & cess & dur;pfv & momt \tabularnewline
          \hline
          I & 
          C\sub1{io}C\sub2{a} & 
          C\sub1{iu}C\sub2{o} & 
          C\sub1{í}C\sub2{a} & 
          C\sub1{ia}C\sub2{u} & 
          C\sub1{iu}C\sub2{a}
          \tabularnewline
          II & 
          C\sub1{io}C\sub2C\sub2{a} & 
          C\sub1{iu}C\sub2C\sub2{o} & 
          C\sub1{í}C\sub2C\sub2{a} & 
          C\sub1{ia}C\sub2C\sub2{u} & 
          C\sub1{iu}C\sub2C\sub2{a}
          \tabularnewline
          III & 
          {ja}C\sub1{io}C\sub2C\sub2{a} & 
          {ja}C\sub1{iu}C\sub2C\sub2{o} & 
          {ja}C\sub1{í}C\sub2C\sub2{a} & 
          {ja}C\sub1{ia}C\sub2C\sub2{u} & 
          {ja}C\sub1{iu}C\sub2C\sub2{a}
          \tabularnewline
          IV & 
          {te}C\sub1{io}C\sub2{a} & 
          {te}C\sub1{iu}C\sub2{o} & 
          {te}C\sub1{í}C\sub2{a}	 & 
          {te}C\sub1{ia}C\sub2{u} & 
          {te}C\sub1{iu}C\sub2{a}	 
          \tabularnewline
          V & 
          {i}C\sub1C\sub1{io}C\sub2C\sub2{a} & 
          {i}C\sub1C\sub1{iu}C\sub2C\sub2{o} & 
          {i}C\sub1C\sub1{í}C\sub2C\sub2{a} & 
          {i}C\sub1C\sub1{ia}C\sub2C\sub2{u} & 
          {i}C\sub1C\sub1{iu}C\sub2C\sub2{a}
          \tabularnewline
          VI & 
          {ta}C\sub1C\sub1{io}C\sub2C\sub2{a} & 
          {ta}C\sub1C\sub1{iu}C\sub2C\sub2{o} & 
          {ta}C\sub1C\sub1{í}C\sub2C\sub2{a} & 
          {ta}C\sub1C\sub1{ia}C\sub2C\sub2{u} & 
          {ta}C\sub1C\sub1{iu}C\sub2C\sub2{a}
          \tabularnewline
          VII & 
          {sa}C\sub1{io}C\sub2C\sub2{a} & 
          {sa}C\sub1{iu}C\sub2C\sub2{o} & 
          {sa}C\sub1{í}C\sub2C\sub2{a} & 
          {sa}C\sub1{ia}C\sub2C\sub2{u} & 
          {sa}C\sub1{iu}C\sub2C\sub2{a}
          \tabularnewline
          VIII & 
          C\sub1{io}C\sub2C\sub2{a} & 
          C\sub1{iu}C\sub2C\sub2{o} & 
          C\sub1{í}C\sub2C\sub2{a} & 
          C\sub1{ia}C\sub2C\sub2{u} & 
          C\sub1{iu}C\sub2C\sub2{a}
          \tabularnewline
          IX & 
          {ě}C\sub1{io}C\sub2C\sub2{a} & 
          {ě}C\sub1{iu}C\sub2C\sub2{o} & 
          {ě}C\sub1{í}C\sub2C\sub2{a} & 
          {ě}C\sub1{ia}C\sub2C\sub2{u} & 
          {ě}C\sub1{iu}C\sub2C\sub2{a}
          \tabularnewline
          \hline
        \end{tabular}}
      \caption{Perfective aspectual patterns\label{tab:vm_perfective_aspects}}
  \end{table}

  \newpage
  \subsection{Topical Agreement}
  \label{ssec:vm_topical_agreement}

  Qevesa is a topic-prominent language that tends towards a split-S active dechticaetiative morphosyntactic alignment. As a result, verbs are marked for agreement with the topic of the sentence, rather than the subject or agent. The topic of the sentence is the noun phrase in the focal case. 

  %\subsubsection{Primary Agreement}
  %\label{sssec:vm_primary_agreement}

  The topic of the verb primarily indicates its experiencer, agent/donor, patient/recipient, or theme. 
  It agrees with the topical noun phrase in animacy and number.
  %If the topical noun phrase is a pronoun, it may be omitted. 
  The suffixes for topical agreement are given in Table~\ref{tab:vm_primary_agreement}.

  \begin{table}[htpb]\small\capstart
    \begin{tabular}{|>{\scshape}fc>{\scshape}c|-c|-c|-c|}
      \hline
      \multicolumn{2}{|-c|}{\SetRowStyle{\bfseries}} & Nominative & Absolutive & Secundative \tabularnewline
      \cline{3-5}
      \SetRowStyle{\scshape} & & nom & abs & sdt \tabularnewline
      \hline
      anim;sg   & asg & -(a)m & -(a)ş & -(a)t  \tabularnewline
      anim;du   & adu & -vám  & -váş  & -vát  \tabularnewline
      anim;pl   & apl & -sám  & -sáş  & -sát  \tabularnewline
      inanim;sg & isg & -nom  & -noş  & -not  \tabularnewline
      inanim;du & idu & -vom  & -voş  & -vot  \tabularnewline
      inanim;pl & ipl & -som  & -soş  & -nost  \tabularnewline
      %1sg      & -(ě/e/i/je)m & -(ě/e/i/je)ş & -(ě/e/i/je)t \tabularnewline
      %2sg      & -tam         & -taş         & -tat  \tabularnewline
      %1du;inc  & -jévm        & -jévş        & -jévt  \tabularnewline
      %1du;exc  & -čévm        & -čévş        & -čévt  \tabularnewline
      %2du      & -távm        & -távş        & -távt  \tabularnewline
      %1pl;inc  & -jésm        & -jéşş        & -jést  \tabularnewline
      %1pl;exc  & -čésm        & -čéşş        & -čést  \tabularnewline
      %2pl      & -tásm        & -táşş        & -tást  \tabularnewline
      %\hline
      \hline
    \end{tabular}
    \caption{Primary topical agreement\label{tab:vm_primary_agreement}}
  \end{table}

  %The first person singular uses \qevesa{-ě} after a consonant;  \qevesa{-e} after \qevesa{e-} (which merge and become \qevesa{-é-}); \qevesa{-i} after \qevesa{a-} and \qevesa{o-}; and \qevesa{-je} after all other vowels, including long vowels. The third-person singular suffixes insert an epenthetic \qevesa{-a-} when the suffix follows a consonant. The use of the singular, dual, and plural numbers is described in Section~\ref{ssec:nm_number}.

  %The animate endings are used when the topic of the verb is an animate noun, 

  %The third person inanimate ending is used when the topic of the verb is an inanimate noun, and is not marked for number.

  \subsubsection{Nominative Topic}
  \label{sssec:vm_nom_topic}

  An nominative topic indicates that the noun phrase in the focal case is the voluntary experiencer of an intransitive verb; the agent of a transitive verb; and the donor of a ditransitive verb.

  \subsubsection{Absolutive Topic}
  \label{sssec:vm_abs_topic}

  An absolutive topic indicates that the noun phrase in the focal case is the involuntary experiencer of an intransitive verb; the patient of a transitive verb; and the recipient of a ditransitive verb. 

  \subsubsection{Secundative Topic}
  \label{sssec:vm_sdt_topic}

  A secundative topic indicates that the noun phrase in the focal case is the theme of a ditransitive verb. The secundative topic suffix is also used in cases when the topic is instrumental, locative or adverbial.
  
%  \subsubsection{Secondary Topical Agreement}
%  \label{sssec:vm_topic_secondary}
%
%  If the topic of the phrase is not the experiencer, agent/donor, patient/recipient or theme, the verb can be marked with a suffix that corresponds to the role of the noun phrase in the focal case. Unlike the primary cases, there are no combined pronoun suffixes, so pronouns must not be omitted. These suffixes are described in Table~\ref{tab:vm_secondary_agreement}.
%
%  \begin{table}[htpb]\small\capstart
%      \begin{tabular}{|>{\bfseries}fc->{\scshape}c|-c|}
%        \hline
%        \multicolumn{2}{|fc|}{\SetRowStyle{\bfseries}Case} & Suffix \tabularnewline
%        \hline
%        %Genitive                 & gen  & -karu- \tabularnewline
%        Essive                    & ess  & -(a)ll \tabularnewline
%        Instrumental (Comitative) & ins  & -(a)tt \tabularnewline
%        Inessive                  & ine  & -(a)ss \tabularnewline
%        Adessive                  & ade  & -(a)d  \tabularnewline
%        Illative                  & ill  & -(a)st \tabularnewline
%        Allative                  & all  & -(a)ft \tabularnewline
%        Elative                   & ela  & -(a)sp \tabularnewline
%        Ablative                  & abl  & -(a)sk \tabularnewline
%        Comparative               & comp & -(a)nn \tabularnewline
%        \hline
%      \end{tabular}
%      \caption{Secondary topical agreement\label{tab:vm_secondary_agreement}}
%  \end{table}
%
%  An epenthetic \qevesa{-a-} is inserted when the suffix follows a consonant.

  \newpage

  \subsection{Mood}
  \label{ssec:vm_mood}

  \ToBeWritten

  Mood is another important category marked on the Qevesa verb. There are eight primary moods: indicative, admirative, irrealis, alethic, necessitative, precative, volitive, and hypothetical.

  The suffixes for mood are given in Table~\ref{tab:vm_modal_suffixes}.

  \begin{table}[htpb]\small\capstart
      \begin{tabular}{|>{\bfseries}fc->{\scshape}c|-c|}
        \hline
        \multicolumn{2}{|fc|}{\SetRowStyle{\bfseries}Mood} & Suffix \tabularnewline
        \hline
        Indicative & ind & -u   \tabularnewline
        Admirative & mir & -óra \tabularnewline
        Irrealis   & irr & -il  \tabularnewline
        Alethic    & ale & -en \tabularnewline
        Commissive & com & -ec  \tabularnewline
        Directive  & dir & -ła  \tabularnewline
        Volitive   & vol & -ir  \tabularnewline
        \hline
      \end{tabular}
      \caption{Verbal mood suffixes\label{tab:vm_modal_suffixes}}
  \end{table}

  \subsubsection{Indicative Mood}
  \label{sssec:vm_indicative}

  The indicative mood is the default mood. It is essentially a realis mood, indicating the factual nature of the statement.

  \subsubsection{Admirative Mood}
  \label{sssec:vm_admirative}

  The admirative mood is also a realis mood, that indicates new or unexpected information.

  \begin{exe}
    \ex \emph{EXAMPLE}
  \end{exe}

  \subsubsection{Irrealis Mood}
  \label{sssec:vm_irrealis}

  The irrealis mood denotes a counterfactual or non-actual sense.

  \begin{exe}
    \ex \emph{EXAMPLE}
  \end{exe}

  \subsubsection{Alethic Mood}
  \label{sssec:vm_alethic}

  The alethic mood denotes the logical necessity of the statement.

  \begin{exe}
    \ex \emph{EXAMPLE}
  \end{exe}

  \subsubsection{Commissive Mood}
  \label{sssec:vm_commissive}

  The commissive mood indicates a commitment or promise to do something.

  \begin{exe}
    \ex \emph{EXAMPLE}
  \end{exe}

  \subsubsection{Directive Mood}
  \label{sssec:vm_directive}

  The directive mood indicates that the action is a request or order.

  \begin{exe}
    \ex \emph{EXAMPLE}
  \end{exe}

  \subsubsection{Volitive Mood}
  \label{sssec:vm_optative}

  The volitive mood indicates a hope, desire, or wishes that the action denoted by the verb should come about.

  \begin{exe}
    \ex \emph{EXAMPLE}
  \end{exe}


  \section{Auxiliary Verbs}
  \label{sec:vm_auxiliary}

  Periphrastic constructions, such as polarity, are indicated with a series of auxiliary verbs. These conjugate similarly to ordinary verbs, but use a slightly different set of conjugations and affixes that are generally identical to the forms for attributive verbs\footnote{See Section~\ref{ssec:am_adjectival_verbs}, page~\pageref{ssec:am_adjectival_verbs}}. 

  The auxiliary verb is inflected and follows the formerly main verb, which occurs in the verbal noun infinitive.

  \subsection{Polarity}
  \label{ssec:vm_polarity}

  The most commonly-used auxiliary verbs are those that indicate polarity. The affirmative verb, \qevesa{rusu}, is generally only used in situations when an explicitly positive statement is to be made. The negative verb, \qevesa{zumu}, is more commonly used, and shares the same root as the word for ‘zero’ or ‘none’.

  Both of these verbs conjugate to aspect as a Form VII root, as shown in Table~\ref{tab:vm_polarity_auxiliary_aspect}.

  \begin{table}[htpb]\small\capstart
      \begin{tabular}{|>{\bfseries}fc->{\scshape}c|-c|-c|}
        \hline
        \SetRowStyle{\bfseries} & & \multicolumn{2}{-c|}{Polarity} \tabularnewline
        \cline{3-4}
        \SetRowStyle{\scshape} & & aff & neg \tabularnewline
        \hline
        Imperfective	& ipfv			& russi  & zummi \tabularnewline
        Stative				& stat			& ruisse & zuimme \tabularnewline
        Durative			& dur;ipfv	& russú  & zummú \tabularnewline
        Frequentative & freq			& russo  & zummo \tabularnewline
        Habitual			& hab				& russa  & zumma \tabularnewline
        \hline\hline
        Perfective		& pfv				& riossa & ziomma \tabularnewline
        Inchoative		& inch			& riusso & ziummo \tabularnewline
        Cessative			& cess			& ríssa  & zímma \tabularnewline
        Durative			& dur;pfv		& riassu & ziammu \tabularnewline
        Momentane			& momt			& riussa & ziumma \tabularnewline
        \hline
      \end{tabular}
      \caption{Polar verb aspectual conjugation\label{tab:vm_polarity_auxiliary_aspect}}
  \end{table}

  \begin{exe}
    \ex \qevesa{Mi tarum ziommamu.}
    \glll Mi tarum ziomma-m-u\\
    \textsc{3sg.foc} write\textsc{\bs inf} not\textsc{\bs pfv-asg;nom-ind}\\
    {He} {write} {not}\\
    \glt He will not write.
  \end{exe}

  % We are not hiding a drunk Verdurian in the room!
  % Potmi Verdurija akkaperossa čésam rafuk zummişóra!
  % drunk Verdurija-ø ak~kaper-ossa čés-am rafuk zummi-ş-óra
  % drunk Verdurian-FOC DEF~room-INE 1PL;EXC-NOM hide\INF not\IPFV-ASG;ABS-MIR
  % drunk Verdurian in the room we hide are not!

  % putum - drink, alcohol
  % potmi = drunk
  % kupur = room
  % rufuk = hide, conceal, guard
  % tupun = die
  % metuppun = kill
  % šanam = mouse
  % nuvum = ???
  % navoim = cat
  % nulup = hunger, desire ??

  % You killed all the mice, so the cat is hungry.
  % Ta a mün aššanamesaş metipponamu, annavoima nuilpeşen.
  % /taʔamyn aɕːamesaʂ metipːonamu anːaʋoima nuilpeʂen/
  % Ta a=mün aš~šanam-es-aş metippona-m-u, an~navoim-a nuilpe-ş-en.
  % 2SG.FOC DEF=all DEF~mouse-PL-ABS kill\PFV-ASG;NOM-IND, DEF~cat-FOC hungry\STAT-ASG;ABS-ALE
  % You the all mice killed, the cat is hungry therefore

  % All the mice were killed by you, so the cat is hungry.
  % A mün aššanamesa tam metipponaşu, annavoima nuilpeşen.
  % A=mün aš~šanam-es-a tam metippona-ş-u, an~navoim-a nuilpe-ş-en.
  % DEF=all DEF~mouse-PL-FOC 2SG.NOM kill\PFV-ASG;ABS-IND, DEF~cat-FOC hungry\STAT-ASG;ABS-ALE
  % The all mice you killed, the cat is hungry therefore

  %\newpage
%  Sometimes the polar auxiliaries will be conjugated to a different aspect than their head verb, especially to indicate semantic nuances, for example:
%
%  \begin{exe}
%    \ex \qevesa{Assoşima jem zímma nuttúlaşu.}
%    \glll As-soşim-a jem zímma nuttúl-aş-u\\
%    \textsc{def-}girl\textsc{-foc} \textsc{1sg.nom} \textsc{neg\bs cess} think\textsc{\bs F2.dur;ipfv-asg;abs-ind}\\
%    {the girl} {I} {not stop} {thinking about her}\\
%    \glt I cannot stop thinking about that girl.
%  \end{exe}

% Using a determiner, such as isátka, strictly refers to location so is not necessary.
% e.g. Asisátka assoşima jem zímma nuttúlaşu.

%  \subsection{Evidentiality}
%  \label{ssec:vm_evidentiality}
%
%  Evidentiality may also be expressed by means of auxiliary verbs. Qevesa possesses a set of auxiliary verbs which distinguish four degrees of evidentiality: witness, reportative, inferential, and assumptive. 
%
%  All of the roots of the evidential auxiliaries are also verbs in their own right. However, they conjugate as Form VIII verbs, with some slightly irregular pattern forms. Their conjugation is given in Table~\ref{tab:vm_evidentiality_conjugation}.
%
%  \begin{table}[htpb]\small\capstart
%      \begin{tabular}{|>{\bfseries}fc->{\scshape}c|-c|-c|-c|-c|}
%        \hline
%        \SetRowStyle{\bfseries} & & \multicolumn{4}{-c|}{Evidentiality} \tabularnewline
%        \cline{3-6}
%        \SetRowStyle{\scshape} &  & exp   & rep    & infr   & asm		 \tabularnewline
%        \hline
%        Imperfective  & ipfv     & murri  & łukşi  & kučti  & quspi  \tabularnewline
%        Stative       & stat     & muirre & łuikşe & kuičte & quispe \tabularnewline
%        Durative      & dur;ipfv & murrú  & łukşú  & kučtú  & quspú  \tabularnewline
%        Frequentative & freq     & murro  & łukşo  & kučto  & quspo  \tabularnewline
%        Habitual      & hab      & murra  & łukşa  & kučta  & quspa  \tabularnewline
%        \hline\hline
%        Perfective    & pfv      & miorra & łiokşa & kiočta & qiospa \tabularnewline
%        Inchoative    & inch     & miurro & łiukşo & kiučto & qiuspo \tabularnewline
%        Cessative     & cess     & mírra  & łíkşa  & kíčta  & qíspa  \tabularnewline
%        Durative      & dur;pfv  & miarru & łiakşu & kiačtu & qiaspu \tabularnewline
%        Momentane     & momt     & miurra & łiukşa & kiučta & qiuspa \tabularnewline
%        \hline
%      \end{tabular}
%      \caption{Conjugation of the evidential verbs \label{tab:vm_evidentiality_conjugation}}
%  \end{table}
%
%  As with all auxiliary constructions, use of the evidential auxiliaries is not mandatory; rather, they are used to provide additional information. 
%
%  \subsubsection{Witness}
%  \label{sssec:vm_evd_witness}
%
%  The witness degree of evidentiality is denoted by the verb \qevesa{murru}, meaning ‘to see’. It is used when the speaker was a witness to the event.
%
%  \subsubsection{Reportative}
%  \label{sssec:vm_evd_reportative}
%
%  The reportative degree of evidentiality is denoted by the verb \qevesa{łukşu}, which has the same consonantal root as the verb \qevesa{łukuş} ‘to hear (speech)’.
%
%  \subsubsection{Inferential}
%  \label{sssec:vm_evd_inferential}
%
%  The inferential degree of evidentiality is denoted by the verb \qevesa{kučtu}. It is used when the speaker infers that the event occurred but was not a witness.
%
%  \subsubsection{Assumptive}
%  \label{sssec:vm_evd_assumption}
%
%  The assumption degree of evidentiality is denoted by the verb \qevesa{quspu}. It is used when the speaker is making an assumption about the occurrence of the event.
%
  \section{Irregular Verbs}
  \label{sec:vm_irregular}

  Qevesa verbal morphology is highly regular, with most irregularities occurring due to consonant groupings. %Roots that contain a /h/ frequently possess irregular forms, mainly because the /h/ will be elided or reduced to a pre-aspiration of the following consonant and the previous vowel lengthened. This may be represented in writing as well as speech.
  However, a number of common roots do possess irregular forms, and these are outlined in the following sections.
  
  \ToBeWritten

%  \subsection{The Copulae}
%  \label{ssec:vm_copulae}
%
%  The most frequently-used irregular verb in Qevesa is the copula \qevesa{teši}. It is one of a number of verbs which do not possess a regular infinitive of the form \qevesa{C\sub1{u}C\sub2{u}}; it also possesses a negative form (\qevesa{zemi}\footnotemark{}), unlike most other verbs. The basic conjugated forms of \qevesa{teši} are given in Table~\ref{tab:vm_copulae_aspectual_conjugation}.
%  \footnotetext{This is also the same consonantal root as the negative verb \qevesa{zumu} and associated forms which translate as ‘zero’ or ‘none’.}
%
%  \begin{table}[htpb]\small\capstart
%      \begin{tabular}{|>{\bfseries}fc->{\scshape}c|-c|-c|}
%        \hline
%        \SetRowStyle{\bfseries} & & Non-negative & Negative \tabularnewline
%        \cline{3-4}
%        \SetRowStyle{\scshape} & & cop & neg \tabularnewline
%        \hline
%        Infinitive    & inf      & teši   & zemi   \tabularnewline
%        \hline\hline
%        Imperfective  & ipfv     & tušši  & zummi  \tabularnewline
%        Stative       & stat     & tuišše & zuimme \tabularnewline
%        Durative      & dur;ipfv & tuššú  & zummú  \tabularnewline
%        Frequentative & freq     & tuššo  & zummo  \tabularnewline
%        Habitual      & hab      & tušša  & zumma  \tabularnewline
%        \hline\hline
%        Perfective    & pfv      & tiošša & ziomma \tabularnewline
%        Inchoative    & inch     & tiuššo & ziummo \tabularnewline
%        Cessative     & cess     & tíšša  & zímma  \tabularnewline
%        Durative      & dur;pfv  & tiaššu & ziammu \tabularnewline
%        Momentane     & momt     & tiušša & ziumma \tabularnewline
%        \hline
%      \end{tabular}
%      \caption{Aspectual conjugation of the copulae \qevesa{teši} and \qevesa{zemi}\label{tab:vm_copulae_aspectual_conjugation}}
%  \end{table}
%
%
%  The copulae can also be used in an existential sense, but only in nominal phrases and never with stative verbs. They play a major role in honorific registers, as described in Chapter~\ref{ch:registers}.

\end{document}
