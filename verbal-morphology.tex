\documentclass[grammar]{subfiles}
\begin{document}
\chapter{Verbal Morphology}
\label{ch:verbal_morphology}


\section{Features}
\label{sec:vm_features}

The Teranean language family use a \emph{triliteral root system}, not unlike
the Semitic languages of Earth, in which verb roots consist of an abstract
pattern of three consonants, with actual verb forms created by inserting
various vowel patterns between these consonants and adding various prefixes and
suffixes.  This discontinuous system is uses to form not only conjugated verbs,
but also nominal and adjectival derivations, to the extent that the majority of
the vocabulary consists of such constructions.

The Proto-Teranean language had a number of different types of verb roots, some
of which contained inherent vowels.  These various types of root were preserved
in the modern Teranean languages to varying degrees, with some becoming
prevalent and others gradually disappearing.  The eastern Teranean languages,
which includes Qevesa, developed a triliteral system as described above, but
all the languages retain traces of each of these subclasses of root in some
form or another.  Qevesa possesses four types of Proto-Teranean roots. 

The first and most common type of verb root is the true \emph{triliteral root},
which consists of three consonants and an inherent vowel between C₁ and C₂.
This vowel may be /a/, /e/ or /o/, with a strong tendency for /e/ to occur in
roots with a stative meaning. The citation form of these roots is
\conlang{*C₁VC₂uC₃}.  Throughout this text, the V listed in transfix patterns
will represent the inherent root vowel.

The second most frequent type is the \emph{biliteral root}, which consists of
two consonants and an inherent vowel in between them, which is typically /oː/
or /eː/, but may be any long vowel. There are a large number of apparently
biliteral roots that exist solely due to sound changes in which a consonant
elided in most positions.  Other biliteral roots are often augmented with
another consonant either before or between the two consonants, and it's
believed that the triliteral system evolved from biliteral origins. 

The third type is the \emph{quadriliteral root}, which consists of four
consonants with no inherent vowel.  The majority of these are reduplicated,
with the form *C₁C₂C₁C₂, and are often ononmatopoaeic.  Those quadriliteral
roots with four different consonants are almost always derived roots of foreign
origin, or extended roots formed by treating a set of four consonants as an
independent root.  The citation form of quadriliteral roots is
\conlang{*C₁aC₂C₃eC₄}.

The final and rarest type of root is the \emph{geminate root}, which consists
of two consonants, the second of which is geminated, and an inherent vowel /e/.
These roots conjugate triliterally in some forms and biliterally in others.  As
with the biliteral roots, there are some irregular triliteral roots which
appear to be geminates due to sound changes; these are distinguished by their
inherent vowel.  The citation form of geminate roots is \conlang{*C₁eC₂C₂}.

% The maximum possible number of verb conjugations derivable from a root—not
% counting participles, verbal nouns and nominalisations—is over eight thousand:
% ten root forms, seven aspects, six moods, and twenty-nine personal suffixes.
% There are also a number of non-mandatory suffixes that can be appended to these that
% convey additional grammatical information.

% \section{Inflectional Categories and Conjugation}
% \label{sec:vm_conjugation}
% 
% The system of verb conjugations in Qevesa is quite complicated, and is formed
% along two axes.  One axis, known as the \emph{scale}, is used to specify
% grammatical concepts such as causative, intensive, reciprocal, passive or
% reflexive, and involves varying the stem. Historically, verb patterns were
% derived from prefixes and infixes, but sound change and analogy have resulted
% in a regular system that is highly productive.  The other axis consists of the various suffixes that are appended to the stem, 
% 
% Qevesa is a highly synthetic language, and verbs are conjugated to indicate
% aspect, mood, and personal agreement (trigger).  The conjugated form of the

% ā ē ī ō ū ȳ
% á é í ó ú ý

\subsection{The Verb Structure}
\label{ssec:vm_structure}

The structure of the Qevesa verb involves a number of prefixes, suffixes, and discontinuous affixes, the order of which is important.

\begin{exe}
  \ex\label{ex:vm_structure} \textsc{pronomial marker-preverb-}\textit{stem}\bs
  \textsc{pattern.aspect}\textsc{-modal marker-trigger marker}
\end{exe}


\section{The Verbal Patterns}
\label{sec:vm_patterns}

Qevesa has a set of six \emph{verbal patterns}, also known as constructions
(\conlang{memódits}\footnotemark). These patterns are sets of verbal
conjugations with an associated grammatical function.  Each pattern contains a
full set of paradigms designating the various aspects; a root conjugated the
patterns has its meaning crossed with the pattern's grammatical function.  Not
all roots can be conjugated into all patterns, and some patterns are prone to
semantic drift.  The nine patterns are numbered from I–VII and are listed in
\cref{tab:vm_root_patterns}. 

\footnotetext{Derived abstract noun from \conlangt{modut}{build, construct}}

\begin{table}[h!]\small\capstart
  \begin{tabular}{BFc -c}
    \toprule
    \rowstyle{\bfseries} Pattern & Description \\
    \midrule
    I    & Base \\
    % II & Intensive \\
    II   & Causative \\
    III  & Reflexive \\
    IV   & Reciprocal \\
    V    & Causative Reflexive \\
    VI   & Passive Reflexive \\
    % VII & Stative \\
    \bottomrule
  \end{tabular}
  \caption{Verb root patterns\label{tab:vm_root_patterns}}
\end{table}

% I    rokut     | base
% II   saroktu   | causative
% III  narkotu   | reflexive, passive
% V    ratoktu   | reciprocal
% V    istorkut  | causative reflexive
% VI   morkut    | passive reflexive

% Perfective | Reciprocal
% KoPuŠ      → KatoPŠu
% → KoPPuŠ   → KatoPPuŠ   (Transitive)
% → saKoPŠu  → satoKPuŠ   (Causative)
% → naKPoŠu  → natoKPuŠ   (Passive/Reflexive)
%        

% Imperfective | Reciprocal
% aKPúŠo       → aKatúPŠo
% → aKaPPúŠo   → aKatPúŠo   (Transitive)
% → asaKPúŠo   → astaKúPŠo  (Causative)
% → anaKúPŠo   → antaKúPŠo  (Passive/Reflexive)
%        


Each pattern will be described in full in the following sections.  Within each
pattern is a conjugational paradigm that allows the verb to conjugate for
aspect and mood; personal suffixes are appended to these stems.


\subsection{Conjugation Stems}
\label{sssec:vm_conjugation}

There are six aspects formed by using a root and vowel template, divided into
three perfective aspects (\emph{perfective}, \emph{experiential}, and
\emph{momentane}) and three imperfective aspects (\emph{progressive},
\emph{durative}, and \emph{habitual}).  Each aspect has an indicative stem,
used to mark the indicative mood, and a modal stem to which modal suffixes are
appended.  If both the indicative and modal stems are the same, as occurs for
some patterns and conjugations, only the infinitive stem is listed in the
table. 

Each verbal pattern also has up to three other non-finite stems: the
\emph{infinitive}, an \emph{active participle}, and a \emph{passive
participle}.  


\subsection{Defective Triliteral Roots}
\label{ssec:vm_defective_roots}

Within the set of triliteral roots there are a number of subtypes caused by the
presence of certain consonants.  These are predictable from the root, but
significantly affect the vowel templates the root uses to conjugate, and in
some cases cause consonants to alternate between methods of articulation.
Although irregular, these \emph{defective roots} are almost entirely due to
historical sound changes. 

% Note that defective roots are not the only type of irregular root; other
% irregular roots also exist, almost all of which developed from regular sound
% changes.  For example, roots that contain the sequence \conlang{t-l}
% consistently replace the cluster \conlang{tl} with \conlang{č} or
% \conlang{čč}.


\subsubsection{Aspirate Roots}
\label{sssec:vm_aspirate_roots}

Aspirate roots, or H-roots, are those roots which have /h/ in one or more
positions, which results the following sound changes:

\begin{itemize}
  \item A syllable-final /h/ induces lengthening of the previous vowel.  Suffixes
    that follow are usually vowel-final.
  \item A /h/ following an unvoiced plosive caused it to become a geminate
    aspirated plosive, which are pronounced in Modern Qevesa as fricatives.
  \item Roots that have /h/ in more than one position follow the rules of both
    positions.  These are exceedlingly rare.
\end{itemize}


\subsubsection{Soft roots}
\label{sssec:vm_soft_roots}

Soft roots, or J-roots are also quite irregular in their conjugations. They are
characterised by having had /ɟ/ in one or more positions, and induced the
following changes to the conjugated forms: 

\begin{itemize}
  \item a syllable-initial /ɟ/ becomes /j/; 
  \item a syllable-final /ɟ/ tends to become /ʒ/ before stops, affricates and
    nasals, and /j/ before fricatives and liquids; and
  \item a geminate /ɟ/ becomes /iʒ/. 
\end{itemize}

These sound changes create a number of homonymic conjugated stems. 


\clearpage
\section{Pattern I}
\label{sec:vm_verb_pattern_i}

Pattern I is the most common literal root form, containing no preformative
affixes.  It is typically the closest indicator to the lexical meaning of the
root, and has no particular semantic function associated with it, so it
includes a wide variety of verbs, including transitive, intransitive, stative
and inchoative


\subsection{Triliteral Roots}
\label{ssec:vm_i_triliteral}
%
%
%\subsubsection{Aspect}
%\label{sssec:vm_i_triliteral_aspect}

The perfective indicative is the citation form of the Pattern I verb, and uses
a stem of the form \conlang{*C₁VC₂uC₃}.  The experiential aspect uses the
pattern \conlang{*C₁VC₂aC₃}, and the momentane aspect uses the pattern
\conlang{*C₁VC₂iC₃}.  The modal stems take the form \conlang{*C₁V₁C₂C₃V₂},
where V₁ and V₂ are the first and second vowels of the indicative stems. 

The imperfective aspects (progressive, durative and  habitual) use the stem
\conlang{aC₁C₂V₂ːC₃V₁}, where ‘V₁’ is the inherent root vowel and ‘V₂’ is
\conlang{-ú-} for the progressive aspect, \conlang{-á-} for the durative, and
\conlang{-í-} for the habitual.  The modal stem replaces the final vowel with
\conlang{-e}. 

In general, regardless of the root pattern, perfective aspects will always
contain the inherent vowel as the first vowel, and imperfective aspects are
always formed by switching the last two vowels, and prefixing with
\conlang{a-}. 

Example triliteral conjugations are given in \cref{tab:vm_i_triliteral_aspect_stems}. 

\begin{table}[h!]\small\capstart
    \begin{tabular}{BFl Kl -c -cw -c -c}
      \toprule
      & & \multicolumn{2}{c}{\conlangt{rokut}{write}} & \multicolumn{2}{c}{\conlangt{vesuk}{lay down}} \\
      \rowstyle{\bfseries} Aspect & & Indicative & Modal & Indicative & Modal \\
      \midrule
      Perfective   & \acs{perf} & rokut  & roktu  & vesuk  & vesku \\
      Experiential & \acs{exp}  & rokat  & rokta  & vesak  & veska \\
      Momentane    & \acs{momt} & rokit  & rokti  & vesik  & veski \\
      Progressive  & \acs{prog} & arkúto & arkúte & avsúke & avsúke \\
      Durative     & \acs{dur}  & arkáto & arkáte & avsáke & avsáke \\
      Habitual     & \acs{hab}  & arkíto & arkíte & avsíke & avsíke \\
      \bottomrule
    \end{tabular}
  \caption{Pattern I triliteral aspectual stems\label{tab:vm_i_triliteral_aspect_stems}}
\end{table}

\begin{table}[h!]\small\capstart
  \begin{tabular}{KFl -c -c -c}
    \toprule
    \rowstyle{\bfseries} & Perfective & Imperfective & Subjunctive  \\
    \rowstyle{\scshape}  & \acs{perf} & \acs{ipfv}   & \acs{subj}  \\
    \midrule
    \acs{1sg}            & marokuta   & markotí      & maroktú    \\
    \acs{2sg}            & turokuta   & tarkotí      & turoktú    \\
    \acs{3sg}            & rokuta     & arkotí       & roktú      \\
    \acs{1du}            & virokuti   & varkotí      & viroktú    \\
    \acs{2du}            & terokuta   & tarkoté      & teroktó    \\
    \acs{3du}            & jirokuta   & jarkoté      & jiroktó    \\
    \acs{1pl};\acs{inc}  & serokutis  & sarkotís     & seroktús   \\
    \acs{1pl};\acs{exc}  & zirokutis  & zerkotís     & ziroktús  \\
    \acs{2pl}            & cherokutas & cherkotés    & cheroktós   \\
    \acs{3pl}            & jarokutas  & jarkotés     & jaroktós   \\
    \midrule
    \acs{inanim}         & rokutu     & arkotýs      & aroktús   \\
    \bottomrule
  \end{tabular}
  \caption{Pattern I triliteral aspectual stems\label{tab:vm_i_triliteral_aspect_stems_p}}
\end{table}
%
%
%\subsubsection{Non-finite stems}
%\label{sssec:vm_i_triliteral_nonfinite}

The non-finite stems are the infinitive and the active and passive participles.
The infinitive is formed with the pattern \conlang{C₁uC₂eC₃e}; the active
participle with the pattern \conlang{eC₁áC₂iC₃}; and the passive participle
with the pattern \conlang{šeC₁C₂éC₃y}.  

\cref{tab:vm_i_triliteral_nonfinite_stems} lists the
non-finite stems of \conlangt{rokut}{write}.

\begin{table}[h!]\small\capstart
  \begin{tabulary}{0.75\textwidth}{BFl -C -C -C}
    \toprule
    \rowstyle{\bfseries} & Infinitive & Active Participle & Passive Participle \\
    \midrule
    Stem \rowstyle{\itshape} & rukete & erákit  & šerkéty \\
    Meaning                  & write  & writing & written \\
    \bottomrule
  \end{tabulary}
  \caption{Pattern I triliteral non-finite stems \label{tab:vm_i_triliteral_nonfinite_stems}}
\end{table}


\subsection{Biliteral Roots}
\label{ssec:vm_i_biliteral}

Biliteral roots lack distinct modal stems.  The perfective indicative is formed
by the pattern \conlang{*C₁VːC₂u}, and the experiential and momentane aspects
replace the final \conlang{-u} with \conlang{-a} or \conlang{-i}.

The imperfective aspects prefix \conlang{a-} and switch the final two vowels;
that is, they take the form \conlang{*aC₁V₂ːC₂V₁}, where V₁ is the short
inherent vowel and V₂ one of \conlang{-ú-} (progressive), \conlang{-á-}
(durative), or \conlang{-í} (habitual).  Biliteral roots whose inherent vowel
is \conlang{*í} or \conlang{*ú} have \conlang{y} as V₁, and biliteral roots
whose inherent vowel is \conlang{*á} have \conlang{e} as V₁.

The infinitive is marked by the suffix \conlang{-e}, the active participle by
the pattern \conlang{*eC₁áC₂i}, and the passive participle by the pattern
\conlang{*šeC₁VːC₂y}.

\cref{tab:vm_i_biliteral_stems} lists some example biliteral conjugations. 

\begin{table}[h!]\small\capstart
  \centering
  \subfloat[Aspect stems]{%
    \begin{tabulary}{0.75\textwidth}{BFl Kl -Cw -Cw -C}
      \toprule
      & & \conlang{mór} “see” & \conlang{šél} “love” & \conlangt{chív}{be cold} \\
      \rowstyle{\bfseries} Aspect & & Stem & Stem & Stem \\
      \midrule
      Perfective   & \acs{perf} & móru  & šélu  & chívu  \\
      Experiential & \acs{exp}  & móra  & šéla  & chíva  \\
      Momentane    & \acs{momt} & móri  & šéli  & chívi  \\
      Progressive  & \acs{prog} & amúro & ašúle & achúvy \\
      Durative     & \acs{dur}  & amáro & ašále & achávy \\
      Habitual     & \acs{hab}  & amíro & ašíle & achívy \\
      \bottomrule
    \end{tabulary}
  }\\
  \subfloat[Non-finite stems]{%
    \begin{tabulary}{0.75\textwidth}{BFl -C -C -C}
      \toprule
      \rowstyle{\bfseries} & Infinitive & Active Participle & Passive Participle \\
      \midrule
      Stem \rowstyle{\itshape} & kéte & ekáti & šekéty \\
      Meaning                  & go   & going & gone \\
      \bottomrule
    \end{tabulary}
  }
  \caption{Pattern I biliteral stems \label{tab:vm_i_biliteral_stems}}
\end{table}

\subsection{Geminate roots}
\label{ssec:vm_i_geminate_roots}

Geminate roots behave like biliteral roots in Pattern I, with the geminate
consonants remaining together in the perfective stems and being split in the
perfective stems.  They lack distinct modal stems. 

The perfective indicative is formed by the pattern \conlang{*C₁eC₂C₂u}, and the
experiential and momentane aspects replace the final \conlang{-u} with
\conlang{-a} or \conlang{-i}.

The imperfective aspects prefix \conlang{a-} and switch the final two vowels;
that is, they take the form \conlang{*aC₁C₂VːC₂e}, where V is one of
\conlang{-ú-} (progressive), \conlang{-á-} (durative), or \conlang{-í}
(habitual).


The non-finite stems of geminate roots in Pattern I are formed by splitting the
geminate consonant and treating them as two single consonants.  They use the
same patterns as triliteral roots:  \conlang{*C₁uC₂eC₂e} (infinitive),
\conlang{*eC₁áC₂iC₂} (active participle) and \conlang{*šeC₁C₂éC₂y} (passive
participle). 

Example conjugations of geminate roots are given in
\cref{tab:vm_i_geminate_stems}.


\begin{table}[h!]\small\capstart
  \centering
  \subfloat[Aspectual stems]{%
    \begin{tabular}{BFl Kl -cw -c}
      \toprule
      & & \conlangt{sepp}{turn} & \conlangt{temm}{finish} \\
      \rowstyle{\bfseries} Aspect & & Stem & Stem \\
      \midrule
      Perfective   & \acs{perf} & seppu  & temmu  \\
      Experiential & \acs{exp}  & seppa  & temma  \\
      Momentane    & \acs{momt} & seppi  & temmi  \\
      Progressive  & \acs{prog} & aspúpe & atmúme \\
      Durative     & \acs{dur}  & aspápe & atmáme \\
      Habitual     & \acs{hab}  & aspípe & atmíme \\
      \bottomrule
    \end{tabular}
  }\\
  \subfloat[Non-finite stems]{%
    \begin{tabulary}{0.75\textwidth}{BFl -C -C -C}
      \toprule
      \rowstyle{\bfseries} & Infinitive & Active Participle & Passive Participle \\
      \midrule
      Stem \rowstyle{\itshape} & supepe & esápip  & šespépy \\
      Meaning                  & turn   & turning & turned \\
      \bottomrule
    \end{tabulary}
  }
  \caption{Pattern I geminate stems \label{tab:vm_i_geminate_stems}}
\end{table}


\subsection{Defective Roots}
\label{ssec:vm_i_defective}

Defective roots generally follow the patterns outlined above, taking into
account the phonological changes listed in \cref{ssec:vm_defective_roots}.
Despite being irregular by nature, a lot of the irregularities of defective
roots are in fact fairly regular and predictable. 


\subsubsection{Aspirate Roots}
\label{sssec:vm_i_aspirate}

Aspirate roots (those with \conlang{*H} as a root consonant) have fairly predictable
irregularities.  First-aspirate roots begin with \conlang{á-} in the imperfective
aspects, and the second vowel is short.  Second-aspirate roots behave mostly like
regular triliteral roots, though the modal perfective stems have the pattern
\conlang{C₁VːC₃} to which the aspect suffixes \conlang{-u}, \conlang{-a} or \conlang{-i} are appended.
Third-aspirate roots always lengthen the vowel that would otherwise precede C₃.  

The non-finite stems are also mostly predictable: syllable-final /h/ lengthens the
preceding vowel; /h/ following a plosive causes it to assimilate to the
corresponding geminate fricative; /h/ following any other consonant causes it
to geminate. 

Examples of aspirate root conjugations are listed in
\cref{tab:vm_i_aspirate_stems}.  They can be distinguished from biliteral roots
by the form of the imperfective aspects. 

\begin{table}[h!]\small\capstart
  \centering
  \subfloat[Aspect stems]{%
    \begin{tabular}{BFl Kl -c -cw -c -cw -c -c}
      \toprule
      & & \multicolumn{2}{c}{\conlangt{hevur}{be good}} & \multicolumn{2}{c}{\conlangt{pohut}{speak}} & \multicolumn{2}{c}{\conlang{zokú} “tie, bind”} \\
      \rowstyle{\bfseries} Aspect & & Indicative & Modal & Indicative & Modal & Indicative & Modal \\
      \midrule
      Perfective   & \acs{perf} & hevur & hevru & pohut  & pótu   & zotú   & zotthu \\
      Experiential & \acs{exp}  & hevar & hevra & pohat  & póta   & zotá   & zottha \\
      Momentane    & \acs{momt} & hevir & hevri & pohit  & póti   & zotí   & zotthi \\
      Progressive  & \acs{prog} & ávure & ávure & affúto & affúte & aztúho & aztúhe \\
      Durative     & \acs{dur}  & ávare & ávare & affáto & affáte & aztáho & aztáhe \\
      Habitual     & \acs{hab}  & ávire & ávire & affíto & affíte & aztího & aztíhe \\
      \bottomrule
    \end{tabular}
  }\\
  \subfloat[Non-finite stems]{%
  \begin{tabulary}{0.75\textwidth}{BFl -C -C -C}
    \toprule
    \rowstyle{\bfseries} & Infinitive & Active Participle & Passive Participle \\
    \midrule
    Stem \rowstyle{\itshape} & huvere & ehávir       & šévery \\
    Meaning                  & good   & (being) good & (been) good \\
    \midrule
    Stem \rowstyle{\itshape} & puhete & epáhit       & šefféty \\
    Meaning                  & speak  & speaking     & spoken \\
    \midrule
    Stem \rowstyle{\itshape} & zuté   & ezátí        & šeztéhy \\
    Meaning                  & bind   & binding      & bound \\
    \bottomrule
  \end{tabulary}
  }
  \caption{Pattern I aspirate defective roots\label{tab:vm_i_aspirate_stems}}
\end{table}


\subsubsection{Soft Roots}
\label{sssec:vm_i_soft}

Soft roots (those with \conlang{*J} as a root consonant) are also fairly regular; 
most of them involve assimilation of \conlang{*j} to surrounding consonants. The most 
common assimilations are:

\begin{itemize}
  \item All occurrences of \conlang{*-j-} before a consonant become \conlang{-ž-} if the consonant is a
    stop or nasal, and \conlang{-i-} if the consonant is a fricative or liquid.  
  \item All occurrences of \conlang{*-j} after a fricative, afficate, or \conlang{n-} assimilate to the
    geminate palatalised equivalent; that is \conlang{*sj} → \conlang{šš}, \conlang{*zj} →
    \conlang{žž}, \conlang{*cj} → \conlang{čč}, and \conlang{*nj} → \conlang{ňň}.
  \item All occurrences of \conlang{*-ji-} and \conlang{*-ij-} become \conlang{-í-} except if they are
    preceded or followed by a different vowel, and word-final \conlang{*-Vj} becomes the
    rising diphthongs \conlang{-Vi}. 
\end{itemize}

Examples of soft root conjugations are listed in \cref{tab:vm_i_soft_stems}.
Note that the verb \conlang{jotuh} is also a third-aspirate root, which makes it
doubly defective.  There are only a very small number of such verbs.  

\begin{table}[h!]\small\capstart
  \centering
  \subfloat[Aspect stems]{%
    \begin{tabular}{BFl Kl -c -cw -c -cw -c -c}
      \toprule
      & & \multicolumn{2}{c}{\conlang{jotú} “know”} & \multicolumn{2}{c}{\conlangt{kojur}{read}} & \multicolumn{2}{c}{\conlang{voluj} “rise (sun, moon)”} \\
      \rowstyle{\bfseries} Aspect & & Indicative & Modal & Indicative & Modal & Indicative & Modal \\
      \midrule
      Perfective   & \acs{perf} & jotú   & jotthu & kojur  & koiru  & voluj  & volju \\
      Experiential & \acs{exp}  & jotá   & jottha & kojar  & koira  & volaj  & volja \\
      Momentane    & \acs{momt} & jotí   & jotthi & kojir  & koiri  & volí   & volí \\
      Progressive  & \acs{prog} & ažtúho & ažtúhe & akjúro & akjúre & avlújo & avlúje \\
      Durative     & \acs{dur}  & ažtáho & ažtáhe & akjáro & akjáre & avlájo & avláje \\
      Habitual     & \acs{hab}  & ažtího & ažtíhe & akíro  & akíre  & avlíjo & avlíje \\
      \bottomrule
    \end{tabular}
  }\\
  \subfloat[Non-finite stems]{%
  \begin{tabulary}{0.75\textwidth}{BFl -C -C -C}
    \toprule
    \rowstyle{\bfseries} & Infinitive & Active Participle & Passive Participle \\
    \midrule
    Stem \rowstyle{\itshape} & juté   & ejátí   & šežtéhy \\
    Meaning                  & know   & knowing & known \\
    \midrule
    Stem \rowstyle{\itshape} & kujere & ekájir  & šekjéry \\
    Meaning                  & read   & reading & read \\
    \midrule
    Stem \rowstyle{\itshape} & vuleje & eválí   & ševléjy \\
    Meaning                  & rise   & rising  & raised \\
    \bottomrule
  \end{tabulary}
  }
  \caption{Pattern I soft defective stems\label{tab:vm_i_soft_stems}}
\end{table}

\subsubsection{Quadriliteral Roots}
\label{sssec:vm_i_quadriliteral}

Quadriliteral roots are all irregular in Pattern I, mainly on account of their
rarity.  They generally conjugate is similarly to triliteral roots, albeit with
a short \conlang{-a-} after C₂.

The perfective indicative aspect takes the form \conlang{*C₁eC₂aC₃C₄u}. The
experiential and momentane aspects replace the \conlang{-u-} with \conlang{-a-} or \conlang{-i-}.

The imperfective aspects use the pattern \conlang{*aC₁C₂eC₃VːC₄y}, where V is
\conlang{-ú-}, \conlang{-á-} or \conlang{-í-} for the progressive, durative and habitual aspects.

The non-finite stems are formed similarly to those for triliteral roots.  The
infinitive is formed with the pattern \conlang{*C₁uC₂eC₃C₄e}; the active participle
with the pattern \conlang{*eC₁C₂áC₃iC₄}; and the passive participle with the pattern
\conlang{*šeC₁C₂éC₃C₄y}.

An example conjugation using the verb \conlangt{zanzen}{annoy} is given in
\cref{tab:vm_i_quadriliteral_aspect_stems}. 

\begin{table}[h!]\small\capstart
  \centering
  \subfloat[Aspectual stems]{%
    \begin{tabulary}{0.6\textwidth}{BFl Kl -C -C}
      \toprule
      & & \conlangt{zanzen}{annoy} & \conlangt{parzem}{translate, interpret} \\
      \rowstyle{\bfseries} Aspect & & Stem & Stem \\
      \midrule
      Perfective   & \acs{perf} & zenaznu   & perazmu  \\
      Experiential & \acs{exp}  & zenazna   & perazma  \\
      Momentane    & \acs{momt} & zenazni   & perazmi  \\
      Progressive  & \acs{prog} & azenzúna  & aperzúma \\
      Durative     & \acs{dur}  & azenzána  & aperzáma \\
      Habitual     & \acs{hab}  & azenzína  & aperzíma \\
      \bottomrule
    \end{tabulary}
  }\\
  \subfloat[Non-finite stems]{%
    \begin{tabulary}{0.75\textwidth}{BFl -C -C -C}
      \toprule
      \rowstyle{\bfseries} & Infinitive & Active Participle & Passive Participle \\
      \midrule
      Stem \rowstyle{\itshape} & zunezne & ezánzin  & šeznézny \\
      Meaning                  & annoy   & annoying & annoyed \\
      \bottomrule
    \end{tabulary}
  }
  \caption{Pattern I quadriliteral stems\label{tab:vm_i_quadriliteral_aspect_stems}}
\end{table}


% \subsection{Irregular Biliteral Roots}
% \label{ssec:vm_i_irregular_biliteral}
% 
% There are a small number of irregular biliteral roots that have entirely different
% conjugation stems, including \conlangt{šym}{name, call} and the copula \conlang{vár}.
% 
% % Biliteral roots lack distinct modal stems.  The perfective indicative is formed
% % by the pattern \conlang{*C₁VːC₂u}, and the experiential and momentane aspects replace
% % the final \conlang{-u} with \conlang{-a} or \conlang{-i}.
% % 
% % The imperfective aspects prefix \conlang{a-} and switch the final two vowels; that
% % is, they take the form \conlang{*aC₁V₂ːC₂V₁}, where V₁ is the short inherent vowel and
% % V₂ one of \conlang{-ú-} (progressive), \conlang{-á-} (durative), or \conlang{-í} (habitual).
% % 
% % The infinitive is marked by the suffix \conlang{-e}, the active participle by the
% % pattern \conlang{*eC₁áC₂i}, and the passive participle by the pattern \conlang{šeC₁VːC₂y}.
% 
% \cref{tab:vm_i_biliteral_stems} lists some example biliteral conjugations. 
% 
% \begin{table}[h!]\small\capstart
%   \centering
%   \subfloat[Aspect stems]{%
%     \begin{tabular}{BFl Kl -cw -c}
%       \toprule
%       & & \conlang{šym} “call, name” & \conlang{šél} “love” \\
%       \rowstyle{\bfseries} Aspect & & Stem & Stem \\
%       \midrule
%       Perfective   & \acs{perf} & šymu  & šélu  \\
%       Experiential & \acs{exp}  & šyme  & šéla  \\
%       Momentane    & \acs{momt} & šymi  & šéli  \\
%       Progressive  & \acs{prog} & ašúmy & ašúle \\
%       Durative     & \acs{dur}  & ašémy & ašále \\
%       Habitual     & \acs{hab}  & ašímy & ašíle \\
%       \bottomrule
%     \end{tabular}
%   }\\
%   \subfloat[Non-finite stems]{%
%     \begin{tabulary}{0.75\textwidth}{BFl -C -C -C}
%       \toprule
%       \rowstyle{\bfseries} & Infinitive & Active Participle & Passive Participle \\
%       \midrule
%       Stem \rowstyle{\itshape} & kéte & etáki & šekéty \\
%       Meaning                  & go   & going & gone \\
%       \bottomrule
%     \end{tabulary}
%   }
%   \caption{Pattern I biliteral stems \label{tab:vm_i_biliteral_stems}}
% \end{table}

\clearpage
\section{Pattern II: Causative}
\label{sec:vm_pattern_ii}

Pattern II is commonly known as the \emph{causative} stem.  Its most common
function is causative; it may also convert transitive verbs into ditransitive
ones.  It can also have a causative meaning on verbs whose Pattern I root is
intransitive, and for some verbs, may convey an assistive or factitive meaning.
Roots in this pattern include: 

\begin{itemize}
  \item \conlangt{kopuš}{eat} → \conlangt{sakopšu}{feed}
  \item \conlangt{rokut}{write} → \conlangt{saroktu}{dictate}
  \item \conlangt{losut}{learn} → \conlangt{salostu}{teach}
  \item \conlangt{pesuk}{fall} → \conlangt{sapesku}{fell sth (e.g. a tree)}
  \item \conlangt{mór}{see} → \conlangt{samóru}{show}
  \item \conlangt{két}{go} → \conlangt{sakétu}{send}
\end{itemize}


The basic form of Pattern II verbs is prefixing \conlang{sa-} onto the root
\conlang{C₁VC₂C₃}, and as a result this pattern is also referred to as the
\emph{S-stem}.  Some examples of Pattern II verbs include:

\subsection{Triliteral Roots}
\label{ssec:vm_ii_triliteral}

The perfective indicative uses a stem of the form \conlang{*saC₁VC₂C₃u}. The
experiential and momentane aspects replace the final \conlang{-u} with \conlang{-a} or
\conlang{-i}.  Pattern II verbs lack distinct modal stems in the perfective aspects.  

The imperfective aspects (progressive, durative and  habitual) use the stem
\conlang{*asaC₁C₂V₂ːC₃V₁}, where V₁ is the inherent vowel and V₂ is \conlang{-ú-} for the
progressive aspect, \conlang{-á-} for the durative, and \conlang{-í-} for the habitual.
The modal stem replaces the final vowel with \conlang{-e}. 

The infinitive is formed with the pattern \conlang{*saC₁eC₂C₃e}; the active participle
with the pattern \conlang{*esC₁áC₂iC₃}; and the passive participle with the pattern
\conlang{*šesC₁éC₂C₃y}.

Example triliteral conjugations are given in \cref{tab:vm_ii_triliteral_stems}. 

\begin{table}[h!]\small\capstart
  \centering
  \subfloat[Aspectual stems]{%
    \begin{tabular}{BFl Kl -c -cw -c -c}
      \toprule
      & & \multicolumn{2}{c}{\conlang{sakopšu} “feed”} & \multicolumn{2}{c}{\conlangt{salostu}{teach}} \\
      \rowstyle{\bfseries} Aspect & & Indicative & Modal & Indicative & Modal \\
      \midrule
      Perfective   & \acs{perf} & sakopšu  & sakopšu  & salostu  & salostu \\
      Experiential & \acs{exp}  & sakopša  & sakopša  & salosta  & salosta \\
      Momentane    & \acs{momt} & sakopši  & sakopši  & salosti  & salosti \\
      Progressive  & \acs{prog} & asakpúšo & asakpúše & asalsúto & asalsúte \\
      Durative     & \acs{dur}  & asakpášo & asakpáše & asalsáto & asalsáte \\
      Habitual     & \acs{hab}  & asakpíšo & asakpíše & asalsíto & asalsíte \\
      \bottomrule
    \end{tabular}
  }\\
  \subfloat[Non-finite stems]{%
    \begin{tabulary}{0.75\textwidth}{BFl -C -C -C}
      \toprule
      \rowstyle{\bfseries} & Infinitive & Active Participle & Passive Participle \\
      \midrule
      Stem \rowstyle{\itshape} & sakepše & eskápiš & šesképšy \\
      Meaning                  & feed    & feeding & fed \\
      \bottomrule
    \end{tabulary}
  }
  \caption{Pattern II triliteral stems \label{tab:vm_ii_triliteral_stems}}
\end{table}


\subsection{Biliteral Roots}
\label{ssec:vm_ii_biliteral}

Biliteral roots in Pattern II have similar conjugations to Pattern I, with the
addition of the prefix \conlang{sa-} or the infix \conlang{-s-} that is inserted immediately before
C₁.  The infix assimilates to the point of articulation of a following
fricative, effectively causing it to geminate. 

The infinitive is marked by the pattern \conlang{*saC₁VːC₂e}, the active participle by the
pattern \conlang{*esC₁VːC₂i}, and the passive participle by the pattern \conlang{šesC₁éC₂y}.  

Some examples are listed in \cref{tab:vm_ii_biliteral_stems}. 

\begin{table}[h!]\small\capstart
  \centering
  \subfloat[Aspectual stems]{%
    \begin{tabulary}{0.6\textwidth}{BFl Kl -C -C}
      \toprule
      & & \conlangt{sakétu}{send} & \conlangt{samóru}{show} \\
      \rowstyle{\bfseries} Aspect & & Stem  \\
      \midrule
      Perfective   & \acs{perf} & sakétu  & samóru \\
      Experiential & \acs{exp}  & sakéta  & samóra \\
      Momentane    & \acs{momt} & sakéti  & samóri \\
      Progressive  & \acs{prog} & astúke  & asmúro \\
      Durative     & \acs{dur}  & astáke  & asmáro \\
      Habitual     & \acs{hab}  & astíke  & asmíro \\
      \bottomrule
    \end{tabulary}
  }\\
  \subfloat[Non-finite stems]{%
    \begin{tabulary}{0.75\textwidth}{BFl -C -C -C}
      \toprule
      \rowstyle{\bfseries} & Infinitive & Active Participle & Passive Participle \\
      \midrule
      Stem \rowstyle{\itshape} & sakéte & estáki  & šeskéty \\
      Meaning                  & send   & sending & sent \\
      \bottomrule
    \end{tabulary}
  }
  \caption{Pattern II biliteral stems \label{tab:vm_ii_biliteral_stems}}
\end{table}


\subsection{Quadriliteral roots}
\label{ssec:vm_ii_quadriliteral_roots}

Quadriliteral roots form Pattern II similarly to Pattern I. The prefix
\conlang{sa-} or the infix \conlang{-s-} is inserted immediately before C₁, the infix
assimilating to a geminate C₁ if that consonant is a fricative. 

The infinitive is marked by the pattern \conlang{*saC₁uC₂eC₃C₄e}, the active participle
by the pattern \conlang{*esaC₁C₂áC₃iC₄}, and the passive participle by the pattern
\conlang{*šesaC₁C₂éC₃C₄y}.

The conjugation of quadriliteral roots in Pattern II is given in \cref{tab:vm_ii_quadriliteral_stems}.

\begin{table}[h!]\small\capstart
  \centering
  \subfloat[Aspectual stems]{%
    \begin{tabular}{BFl Kl -c -c}
      \toprule
      & & \conlang{sazarkel} “(cause to be) centred” & \conlang{saparzem} “(cause to be) translated” \\
      \rowstyle{\bfseries} Aspect & & Stem & Stem \\
      \midrule
      Perfective   & \acs{perf} & sazeraklu   & saperazmu  \\
      Experiential & \acs{exp}  & sazerakla   & saperazma  \\
      Momentane    & \acs{momt} & sazerakli   & saperazmi  \\
      Progressive  & \acs{prog} & asazrekúla  & asaprezúma \\
      Durative     & \acs{dur}  & asazrekála  & asaprezáma \\
      Habitual     & \acs{hab}  & asazrekíla  & asaprezíma \\
      \bottomrule
    \end{tabular}
  }\\
  \subfloat[Non-finite stems]{%
    \begin{tabulary}{0.75\textwidth}{BFl -C -C -C}
      \toprule
      \rowstyle{\bfseries} & Infinitive & Active Participle & Passive Participle \\
      \midrule
      Stem \rowstyle{\itshape} & sazurekle & esazrákil & šesazrékly \\
      Meaning                  & centre    & centring  & be centred \\
      \bottomrule
    \end{tabulary}
  }
  \caption{Pattern II quadriliteral stems\label{tab:vm_ii_quadriliteral_stems}}
\end{table}


\subsection{Geminate roots}
\label{ssec:vm_ii_geminate_roots}

Geminate roots form Pattern II similarly to biliteral roots, with a geminate
second consonant.  The perfective aspects are formed with the pattern
\conlang{*saC₁V₁C₂C₂V₂}, where V₁ is the inherent vowel and V₂ is one of \conlang{-u-},
\conlang{-a-} or \conlang{-i-} for the perfective, experiential, and momentane aspects.  

The imperfective aspects use the pattern \conlang{*asaC₁V₂ːC₂C₂V₁}, where V₁ is the
inherent vowel and V₂ is \conlang{-ú-} for the progressive aspect, \conlang{-á-} for the
durative, and \conlang{-í-} for the habitual. As the inherent vowel for geminate
roots is almost always \conlang{e}, these roots lack a distinct modal form.

The infinitive is formed with the pattern \conlang{*saC₁uC₂C₂e}, the active participle
with \conlang{*esaC₁áC₂C₂i}, and the participle with \conlang{*šesC₁éC₂C₂y}.  

Example geminate stems are listed in \cref{tab:vm_ii_geminate_stems}.

\begin{table}[h!]\small\capstart
  \centering
  \subfloat[Aspectual stems]{%
    \begin{tabular}{BFl Kl -c}
      \toprule
      & & {\conlangt{sasyppu}{cause, bring about}} \\
      \rowstyle{\bfseries} Aspect & & Stem \\
      \midrule
      Perfective   & \acs{perf} & saseppu   \\
      Experiential & \acs{exp}  & saseppa   \\
      Momentane    & \acs{momt} & saseppi   \\
      Progressive  & \acs{prog} & asasúppe  \\
      Durative     & \acs{dur}  & asasáppe  \\
      Habitual     & \acs{hab}  & asasíppe  \\
      \bottomrule
    \end{tabular}
  }\\
  \subfloat[Non-finite stems]{%
    \begin{tabulary}{0.75\textwidth}{BFl -C -C -C}
      \toprule
      \rowstyle{\bfseries} & Infinitive & Active Participle & Passive Participle \\
      \midrule
      Stem \rowstyle{\itshape} & sasuppe & esasáppi & šesséppy \\
      Meaning                  & cause   & causing  & caused \\
      \bottomrule
    \end{tabulary}
  }
  \caption{Pattern II geminate stems \label{tab:vm_ii_geminate_stems}}
\end{table}


\subsection{Defective Roots}
\label{ssec:vm_ii_defective_roots}

Defective roots in Pattern II follow the same phonological assimilation rules
as have previously described.  This results in a number of predictable
irregularities, the most apparent being that the active and passive participles
assimilate \conlang{*-sj-} to \conlang{-šš-} and \conlang{*-sh-} to \conlang{-ss-}.

An example of a defective triliteral conjugation is given in \cref{tab:vm_ii_defective_stems}. 

\begin{table}[h!]\small\capstart
  \centering
  \subfloat[Aspectual stems]{%
    \begin{tabulary}{0.6\textwidth}{BFl Kl -C -C}
      \toprule
      & & \multicolumn{2}{c}{\conlangt{sajotthu}{inform}} \\
      \rowstyle{\bfseries} Aspect & & Indicative & Modal \\
      \midrule
      Perfective   & \acs{perf} & sajotthu & sajotthu \\
      Experiential & \acs{exp}  & sajottha & sajottha \\
      Momentane    & \acs{momt} & sajotthi & sajotthi \\
      Progressive  & \acs{prog} & asažtúho & asažtúhe \\
      Durative     & \acs{dur}  & asažtáho & asažtáhe \\
      Habitual     & \acs{hab}  & asažtího & asažtíhe \\
      \bottomrule
    \end{tabulary}
  }\\
  \subfloat[Non-finite stems]{%
    \begin{tabulary}{0.75\textwidth}{BFl -C -C -C}
      \toprule
      \rowstyle{\bfseries} & Infinitive & Active Participle & Passive Participle \\
      \midrule
      Stem \rowstyle{\itshape} & sajetthe & eššátí    & šeššétthy \\
      Meaning                  & inform   & informing & informed \\
      \bottomrule
    \end{tabulary}
  }
  \caption{Pattern II defective stems \label{tab:vm_ii_defective_stems}}
\end{table}


\clearpage
\section{Pattern III: Reflexive}
\label{sec:vm_pattern_iii}

Pattern III is commonly known as the \emph{reflexive} stem, though this is
something of a misnomer as true reflexives only account for a portion of the
verbs in this pattern.  Verbs in Pattern III are subject to a large amount of
semantic drift, and some roots lack base forms in Patterns I or II.  The main
functions of this pattern are: 

\begin{itemize}
  \item Forming reflexives from transitive roots: \conlang{šomú} “shave” → \conlang{našmohu}
    “shave oneself”
  \item Forming causative reflexives from stative roots: \conlangt{vorun}{wear} →
    \conlang{navronu} “dress oneself (cause oneself to wear)”
  \item Forming so-called autoreflexive verbs that denote (often involuntary)
    actions performed on one’s body: \conlang{nášoru} “sneeze”
  \item Forming verbs with unpredictable semantics: \conlang{narkotu} “copy (sth)”,
    \conlangt{nakjoru}{read aloud, recite}, \conlang{namóru} “look inwards, introspect”
\end{itemize}

Of the functions listed, the only fully productive class is the reflexives from
transitive roots.  The verbs with unpredictable semantics are generally
admitting of new forms, but the causative reflexives are mostly handled by
Pattern V in modern Qevesa, and the autoreflexives are a closed class.

The basic form of Pattern III roots is by prefixing \conlang{na-} onto the root, and as
a result, this pattern is also known as the \emph{N-stem}.


\subsection{Triliteral Roots}
\label{ssec:vm_iii_triliteral_roots}

Triliteral roots form the perfective aspects with the pattern \conlang{*naC₁C₂V₁C₃V₂},
where V₁ is the inherent root vowel and V₂ is one of \conlang{-u}, \conlang{-a} or \conlang{-i}
for the various subtypes.  

The imperfective aspects are formed with the pattern \conlang{*anaC₁V₂ːC₂C₃V₁},
where V₁ is the inherent root vowel, and V₂ is \conlang{-ú-} for the progressive
aspect, \conlang{-á-} for the durative aspect, and \conlang{-í-} for the habitual aspect.
Perfective aspects lack a distinct modal form in Pattern III, but imperfective
aspects form it by replacing the final vowel with \conlang{-e}. 

The infinitive is formed with the pattern \conlang{*nuC₁C₂eC₃e}; the active participle
with the pattern \conlang{*enáC₁C₂iC₃} and the passive participle with the pattern
\conlang{*šenC₁iC₂C₃u}. 

Examples of triliteral stems in Pattern III are given in \cref{tab:vm_iii_triliteral_stems}. 

\begin{table}[h!]\small\capstart
  \centering
  \subfloat[Aspectual stems]{%
    \begin{tabular}{BFl Kl -c -cw -c -c}
      \toprule
      & & \multicolumn{2}{c}{\conlang{narkotu} “copy (sth)”} & \multicolumn{2}{c}{\conlangt{navronu}{dress oneself}} \\
      \rowstyle{\bfseries} Aspect & & Indicative & Modal & Indicative & Modal \\
      \midrule
      Perfective   & \acs{perf} & narkotu  & narkotu  & navronu  & navronu  \\
      Experiential & \acs{exp}  & narkota  & narkota  & navrona  & navrona  \\
      Momentane    & \acs{momt} & narkoti  & narkoti  & navroni  & navroni  \\
      Progressive  & \acs{prog} & anarúkto & anarúkte & anavúrno & anavúrne \\
      Durative     & \acs{dur}  & anarákto & anarákte & anavárno & anavárne \\
      Habitual     & \acs{hab}  & anaríkto & anaríkte & anavírno & anavírne \\
      \bottomrule
    \end{tabular}
  }\\
  \subfloat[Non-finite stems]{%
    \begin{tabulary}{0.75\textwidth}{BFl -C -C -C}
      \toprule
      \rowstyle{\bfseries} & Infinitive & Active Participle & Passive Participle \\
      \midrule
      Stem \rowstyle{\itshape} & nurkete & enárkit & šenrikty \\
      Meaning                  & copy    & copying & copied \\
      \bottomrule
    \end{tabulary}
  }
  \caption{Pattern III triliteral stems \label{tab:vm_iii_triliteral_stems}}
\end{table}


\subsection{Biliteral Roots}
\label{ssec:vm_iii_biliteral_roots}

Biliteral roots form the perfective aspects by prefixing the Pattern I stem
with \conlang{na-}.  The imperfective stems are formed by inserting the prefix
\conlang{-n-} immediately before C₁.  Like their Pattern I counterparts, biliteral
roots in this pattern also lack distinct modal stems. 

The infinitive is formed with the pattern \conlang{*naC₁VːC₂e}; the active participle
with the pattern \conlang{*enC₁áC₂i} and the passive participle with the pattern
\conlang{*šenC₁VːC₂y}. 

Examples of biliteral stems are given in \cref{tab:vm_iii_biliteral_stems}. 

\begin{table}[h!]\small\capstart
  \centering
  \subfloat[Aspectual stems]{%
    \begin{tabular}{BFl Kl -cw -c}
      \toprule
      & & \conlang{namóru} “introspect” & \conlang{natévu} “sense, feel within” \\
      \rowstyle{\bfseries} Aspect & & Stem & Stem \\
      \midrule
      Perfective   & \acs{perf} & namóru & natévu \\
      Experiential & \acs{exp}  & namóra & natéva \\
      Momentane    & \acs{momt} & namóri & natévi \\
      Progressive  & \acs{prog} & anmúro & antúve \\
      Durative     & \acs{dur}  & anmáro & antáve \\
      Habitual     & \acs{hab}  & anmíro & antíve \\
      \bottomrule
    \end{tabular}
  }\\
  \subfloat[Non-finite stems]{%
    \begin{tabulary}{0.75\textwidth}{BFl -C -C -C}
      \toprule
      \rowstyle{\bfseries} & Infinitive & Active Participle & Passive Participle \\
      \midrule
      Stem \rowstyle{\itshape} & namóre     & enmóri        & šenmóry \\
      Meaning                  & introspect & introspecting & introspected \\
      \bottomrule
    \end{tabulary}
  }
  \caption{Pattern III biliteral stems \label{tab:vm_iii_biliteral_stems}}
\end{table}
    

\subsection{Quadriliteral roots}
\label{ssec:vm_iii_quadriliteral_roots}

Pattern III quadriliteral roots are rare. 

Quadriliteral roots form Pattern III similarly to Pattern II. The prefix
\conlang{na-} or the infix \conlang{-n-} is inserted immediately before C₁, the infix
assimilating to a geminate C₁ if that consonant is a fricative. 

The infinitive is marked by the pattern \conlang{*naC₁C₂eC₃C₄e}, the active participle
by the pattern \conlang{*enaC₁C₂áC₃iC₄}, and the passive participle by the pattern
\conlang{*šenaC₁C₂éC₃C₄y}.


\subsection{Geminate roots}
\label{ssec:vm_iii_geminate_roots}

Geminate roots form Pattern III similarly to Pattern II, except for the
perfective indicative aspects which split the geminate consonant C₂ into two
single consonants.   The perfective indicative aspects are formed with the
pattern \conlang{*naC₁V₁C₂V₂C₂}, where V₁ is the inherent vowel and V₂ is one of
\conlang{-u-}, \conlang{-a-} or \conlang{-i-}, and the modal perfective aspects use the pattern
\conlang{*naC₁V₁C₂C₂V₂}.  

The imperfective aspects use the pattern \conlang{*anC₁V₂ːC₂C₂V₁} in the indicative,
replacing the final vowel with \conlang{-e} to form the modal stem. 

The infinitive is formed with the pattern \conlang{*nuC₁C₂eC₂e}, the active participle
with \conlang{*enáC₁C₂iC₂}, and the participle with \conlang{*šenC₁éC₂C₂y}.  

% \cref{tab:vm_iii_geminate_stems} lists the Pattern III stems for
% the verb \conlang{nasypúp} “melt”.


\subsection{Defective Roots}
\label{ssec:vm_iii_defective_roots}

Defective roots in Pattern III follow the same phonological assimilation rules
as have previously described. 


\clearpage
\section{Pattern IV: Reciprocal}
\label{sec:vm_pattern_iv}

Pattern IV is the \emph{reciprocal} stem, whose primary purpose is to create
verbs that convey meanings of a reciprocal or reflexive nature.  It is often
used to create verbs denoting social interactions or accompaniment, or to form
transitive verbs from intransitive roots.  This pattern is also subject to some
semantic and metaphorical drift, though not as severe as in Pattern III. Some
examples include:

\begin{itemize}
  \item \conlangt{pohut}{speak} → \conlangt{patótu}{converse (with)}
  \item \conlangt{rokut}{write} → \conlangt{ratoktu}{correspond (with)}
  \item \conlangt{šopur}{buy} → \conlangt{šatopru}{buy (from)}
  \item \conlangt{kétu}{go} → \conlangt{katétu}{go together, go with} (accompaniment)
  \item \conlangt{kéru}{ask} → \conlangt{katéru}{ask for (sth)} (intransitive → transitive)
\end{itemize}

The general form of Pattern IV verbs is inserting the infix \conlang{-at-} immediately
after the first consonant, and as a result it may also be referred to as the \emph{T-stem}. 


\subsection{Triliteral Roots}
\label{ssec:vm_iv_triliteral_roots}

Triliteral roots form the perfective aspects with the pattern
\conlang{*C₁atV₁C₂C₃V₂}, where V₁ is the inherent root vowel and V₂ is one of \conlang{-u},
\conlang{-a} or \conlang{-i} for the various subtypes.  

The imperfective aspects are formed with the pattern \conlang{*aC₁atV₂ːC₂C₃V₁}, where
V₂ is the \conlang{-ú-} for the progressive aspect, \conlang{-á-} for the durative aspect,
and \conlang{-í-} for the habitual aspect.  Perfective aspects lack a distinct modal
form in Pattern V, but imperfective aspects form it by replacing the final
vowel with \conlang{-e}. 

The infinitive is formed with the pattern \conlang{*C₁atuC₂eC₃e}; the active
participle with the pattern \conlang{*aC₁átC₂iC₃} and the passive participle with the
pattern \conlang{*šeC₁atiC₂C₃y}. 

Examples of triliteral stems in Pattern IV are given in
\cref{tab:vm_iv_triliteral_stems}. 

\begin{table}[h!]\small\capstart
  \centering
  \subfloat[Aspectual stems]{%
    \begin{tabular}{BFl Kl -c -cw -c -c}
      \toprule
      & & \multicolumn{2}{c}{\conlangt{ratoktu}{correspond (with)}} & \multicolumn{2}{c}{\conlangt{šatopru}{buy (from)}} \\
      \rowstyle{\bfseries} Aspect & & Indicative & Modal & Indicative & Modal \\
      \midrule
      Perfective   & \acs{perf} & ratoktu  & ratoktu  & šatopru  & šatopru   \\
      Experiential & \acs{exp}  & ratokta  & ratokta  & šatopra  & šatopra   \\
      Momentane    & \acs{momt} & ratokti  & ratokti  & šatopri  & šatopri   \\
      Progressive  & \acs{prog} & aratúkto & aratúkte & ašatúpro & ašatúpre  \\
      Durative     & \acs{dur}  & aratákto & aratákte & ašatápro & ašatápre  \\
      Habitual     & \acs{hab}  & aratíkto & aratíkte & ašatípro & ašatípre  \\
      \bottomrule
    \end{tabular}
  }\\
  \subfloat[Non-finite stems]{%
    \begin{tabulary}{0.75\textwidth}{BFl -C -C -C}
      \toprule
      \rowstyle{\bfseries} & Infinitive & Active Participle & Passive Participle \\
      \midrule
      Stem \rowstyle{\itshape} & ratukete   & erátkit       & šeratikty \\
      Meaning                  & correspond & corresponding & corresponded \\
      \bottomrule
    \end{tabulary}
  }
  \caption{Pattern IV triliteral stems \label{tab:vm_iv_triliteral_stems}}
\end{table}


\subsection{Biliteral Roots}
\label{ssec:vm_iv_biliteral_roots}

Biliteral roots form the aspects by inserting the infix \conlang{-at-} immediately
after C₁ on the Pattern I stem.  Like their Pattern I counterparts, biliteral
roots in this pattern also lack distinct modal stems. 

The infinitive is formed with the pattern \conlang{*C₁atVːC₂e}; the active participle
with the pattern \conlang{*eC₁táC₂i} and the passive participle with the pattern
\conlang{*šeC₁atVːC₂y}. 

Examples of biliteral stems are given in \cref{tab:vm_iv_biliteral_stems}. 

\begin{table}[h!]\small\capstart
  \centering
  \subfloat[Aspectual stems]{%
    \begin{tabular}{BFl Kl -cw -c}
      \toprule
      & & \conlangt{katétu}{go together (with)} & \conlangt{katéru}{ask for (sth)} \\
      \rowstyle{\bfseries} Aspect & & Stem & Stem \\
      \midrule
      Perfective   & \acs{perf} & katétu  & katéru \\
      Experiential & \acs{exp}  & katéta  & katéra \\
      Momentane    & \acs{momt} & katéti  & katéri \\
      Progressive  & \acs{prog} & akatúte & akatúre \\
      Durative     & \acs{dur}  & akatáte & akatáre \\
      Habitual     & \acs{hab}  & akatíte & akatíre \\
      \bottomrule
    \end{tabular}
  }\\
  \subfloat[Non-finite stems]{%
    \begin{tabulary}{0.75\textwidth}{BFl -C -C -C}
      \toprule
      \rowstyle{\bfseries} & Infinitive & Active Participle & Passive Participle \\
      \midrule
      Stem \rowstyle{\itshape} & katére        & ektári       & šekatéry \\
      Meaning                  & ask for (sth) & asking (for) & asked (for) \\
      \bottomrule
    \end{tabulary}
  }
  \caption{Pattern IV biliteral stems \label{tab:vm_iv_biliteral_stems}}
\end{table}


\subsection{Quadriliteral roots}
\label{ssec:vm_iv_quadriliteral_roots}

Quadriliteral roots form Pattern IV similarly to Pattern II. The infix
\conlang{at-} is inserted immediately after C₁. 

The perfective indicative aspect takes the form \conlang{*C₁ateC₂aC₃C₄u}. The
experiential and momentane aspects replace the \conlang{-u-} with \conlang{-a-} or \conlang{-i-}.

The imperfective aspects use the pattern \conlang{*aC₁atC₂eC₃VːC₄y}, where V is
\conlang{-ú-}, \conlang{-á-} or \conlang{-í-} for the progressive, durative and habitual aspects.

The infinitive is marked by the pattern \conlang{*C₁atuC₂eC₃C₄e}, the active participle
by the pattern \conlang{*eC₁atC₂áC₃iC₄}, and the passive participle by the pattern
\conlang{*šeC₁atC₂éC₃C₄y}.


\subsection{Geminate roots}
\label{ssec:vm_iv_geminate_roots}

Geminate roots form Pattern IV similarly to Pattern II.   The perfective apects are formed with the pattern
\conlang{*C₁atV₁C₂C₂V₂}, where V₁ is the inherent vowel and V₂ is one of \conlang{-ú-},
\conlang{-á-} or \conlang{-í-}.

The imperfective aspects use the pattern \conlang{*aC₁atV₂ːC₂C₂V₁} in the indicative,
replacing the final vowel with \conlang{-e} to form the modal stem. 

The infinitive is formed with the pattern \conlang{*C₁atC₂uC₂e}, the active participle
with \conlang{*eC₁atáC₂iC₂}, and the participle with \conlang{*šeC₁atiC₂úC₂}.  


\subsection{Defective Roots}
\label{ssec:vm_iv_defective_roots}

Defective roots in Pattern IV follow the same phonological assimilation rules
as have previously described. 


\clearpage
\section{Pattern V: Causative Reflexive}
\label{sec:vm_pattern_v}

Pattern V is the \emph{causative reflexive} stem, and generally functions as
the reflexive counterpart to Patterns II and II.  However, it is often subject
to large amounts of unpredictable semantic and metaphorical drift.  Verbs in
this pattern often have an inchoative sense associated with them. Some examples
from this pattern include: 

\begin{itemize}
  \item \conlangt{satorkut}{not procrastinate} (literally “make oneself write”)
  \item \conlangt{satovrun}{deserve}
  \item \conlangt{satopphut}{make oneself speak}
  \item \conlangt{satolsut}{learn}
  \item \conlangt{satotuk}{curse oneself, curse own luck}
  \item \conlangt{satolkuj}{deceive oneself}
\end{itemize}

It is marked by the prefix \conlang{sut-} in all forms, leading to its referral as the \emph{ST-stem}. 

% Staktab (Active Scale V), also known as the “reflexive of causative”, is one
% of the trickiest Alashian verb classes as far as semantics are concerned. Its
% most fundamental function is to serve as the reflexive counterpart to 'aktēb,
% the causative verbal scale, such that a verb like στάκταβ staktab (from *ktāb
% “write”) literally means “make oneself write”. However, this basic meaning is
% often subject to large amounts of unpredictable metaphorical and semantic
% drift; in this case, the verb στάκταβ staktab is more commonly used to mean
% “not procrastinate, not put off” (whether or not actual writing is involved).
% In particular Active Scale V often has an inchoative sense.

\subsection{Triliteral Roots}
\label{ssec:vm_v_triliteral_roots}

Triliteral roots form the perfective indicative aspects with the pattern
\conlang{*sutV₁C₁C₂V₂C₃}, where V₁ is the inherent root vowel and V₂ is one of
\conlang{-u-}, \conlang{-a-} or \conlang{-i-} for the various subtypes.  The modal perfective
aspects append the suffix \conlang{-e}. 

The imperfective aspects are formed with the pattern \conlang{*astaC₁V₂ːC₂C₃V₁},
where V₁ is the inherent root vowel and V₂ is \conlang{-ú-} for the progressive
aspect, \conlang{-á-} for the durative aspect, and \conlang{-í-} for the habitual aspect.
The modal conjugations are formed by replacing the final vowel of the
indicative stems with \conlang{-e}. 

The infinitive is formed with the pattern \conlang{*istuC₁C₂eC₃e}; the active participle
with the pattern \conlang{*estáC₁C₂iC₃} and the passive participle with the pattern
\conlang{*šestiC₁C₂éC₃y}. 

Examples of triliteral stems in Pattern V are given in \cref{tab:vm_v_triliteral_stems}. 

\begin{table}[h!]\small\capstart
  \centering
  \subfloat[Aspectual stems]{%
    \begin{tabular}{BFl Kl -c -c}
      \toprule
      & & \multicolumn{2}{c}{\conlangt{sutolsut}{learn}} \\
      \rowstyle{\bfseries} Aspect & & Indicative & Modal \\
      \midrule
      Perfective   & \acs{perf} & sutolsut  & sutolsute \\
      Experiential & \acs{exp}  & sutolsat  & sutolsate \\
      Momentane    & \acs{momt} & sutolsit  & sutolsite \\
      Progressive  & \acs{prog} & astalústo & astalúste \\
      Durative     & \acs{dur}  & astalásto & astaláste \\
      Habitual     & \acs{hab}  & astalísto & astalíste \\
      \bottomrule
    \end{tabular}
  }\\
  \subfloat[Non-finite stems]{%
    \begin{tabulary}{0.75\textwidth}{BFl -C -C -C}
      \toprule
      \rowstyle{\bfseries} & Infinitive & Active Participle & Passive Participle \\
      \midrule
      Stem \rowstyle{\itshape} & istulsete & estálsit & šestilséty \\
      Meaning                  & learn     & learning & learned \\
      \bottomrule
    \end{tabulary}
  }
  \caption{Pattern V triliteral stems \label{tab:vm_v_triliteral_stems}}
\end{table}


\subsection{Biliteral Roots}
\label{ssec:vm_v_biliteral_roots}

Biliteral roots form the perfective aspects by the pattern \conlang{*satV₁C₁V₂C₂},
where V₁ is the short inherent vowel and V₂ is one of \conlang{-u-}, \conlang{-a-} or
\conlang{-i-}.  The imperfective stems use the pattern \conlang{*astV₂ːC₁V₁C₂}, again with
V₁ as the short inherent vowel and V₂ one of \conlang{-ú-}, \conlang{-á-} or \conlang{-í-}.  Both
aspects form the modal stem by suffixing with \conlang{-e}. 

The infinitive is formed with the pattern \conlang{*istaC₁VːC₂e}; the active
participle with the pattern \conlang{*estáC₁iC₂} and the passive participle with the
pattern \conlang{*šestiC₁éC₂y}. 

Examples of biliteral stems are given in \cref{tab:vm_v_biliteral_stems}. 

\begin{table}[h!]\small\capstart
  \centering
  \subfloat[Aspectual stems]{%
    \begin{tabular}{BFl Kl -c -c}
      \toprule
      & & \multicolumn{2}{c}{\conlangt{istamur}{reflect}} \\
      \rowstyle{\bfseries} Aspect & & Indicative & Modal \\
      \midrule
      Perfective   & \acs{perf} & satomur & satomure  \\
      Experiential & \acs{exp}  & satomar & satomare  \\
      Momentane    & \acs{momt} & satomir & satomire  \\
      Progressive  & \acs{prog} & astúmor & astúmore  \\
      Durative     & \acs{dur}  & astámor & astámore  \\
      Habitual     & \acs{hab}  & astímor & astímore  \\
      \bottomrule
    \end{tabular}
  }\\
  \subfloat[Non-finite stems]{%
    \begin{tabulary}{0.75\textwidth}{BFl -C -C -C}
      \toprule
      \rowstyle{\bfseries} & Infinitive & Active Participle & Passive Participle \\
      \midrule
      Stem \rowstyle{\itshape} & istamóre & estámir    & šestiméry \\
      Meaning                  & reflect  & reflecting & reflected \\
      \bottomrule
    \end{tabulary}
  }
  \caption{Pattern V biliteral stems \label{tab:vm_v_biliteral_stems}}
\end{table}
    

\subsection{Quadriliteral roots}
\label{ssec:vm_v_quadriliteral_roots}

Quadriliteral roots form Pattern V similarly to Pattern II. The prefix
\conlang{sat-} is inserted immediately before C₁. 
 
The infinitive is marked by the pattern \conlang{*istaC₁uC₂C₃eC₄}, the active participle
by the pattern \conlang{*istaC₁C₂VːC₃iC₄}, and the passive participle by the pattern
\conlang{*šestiC₁C₂C₃úC₄}.  
 
 
\subsection{Geminate roots}
\label{ssec:vm_v_geminate_roots}

Geminate roots form Pattern V similarly to biliteral roots, albeit with the
geminated final root consonant. The perfective aspects are formed with the
pattern \conlang{*istV₁C₁V₂C₂}, where V₁ is the short inherent vowel and V₂ is one
of \conlang{-u-}, \conlang{-a-} or \conlang{-i-}.  The imperfective stems use the pattern
\conlang{*astV₂ːC₁V₁C₂C₂}, again with V₁ as the short inherent vowel and V₂ one of
\conlang{-ú-}, \conlang{-á-} or \conlang{-í-}.  Both aspects form the modal stem by suffixing
with \conlang{-C₂e}. 

The infinitive is formed with the pattern \conlang{*istaC₁uC₂C₂e}; the active
participle with the pattern \conlang{*estáC₁C₂iC₂} and the passive participle with the
pattern \conlang{*šestiC₁éC₂C₂y}. 

Examples of geminate stems are given in \cref{tab:vm_v_geminate_stems}. 

\begin{table}[h!]\small\capstart
  \centering
  \subfloat[Aspectual stems]{%
    \begin{tabular}{BFl Kl -c -c}
      \toprule
      & & \multicolumn{2}{c}{\conlangt{istysup}{come into being, appear, turn up}} \\
      \rowstyle{\bfseries} Aspect & & Indicative & Modal \\
      \midrule
      Perfective   & \acs{perf} & satesupp & satesuppe  \\
      Experiential & \acs{exp}  & satesapp & satesappe  \\
      Momentane    & \acs{momt} & satesipp & satesippe  \\
      Progressive  & \acs{prog} & astúsepp & astúseppe  \\
      Durative     & \acs{dur}  & astásepp & astáseppe  \\
      Habitual     & \acs{hab}  & astísepp & astíseppe  \\
      \bottomrule
    \end{tabular}
  }\\
  \subfloat[Non-finite stems]{%
    \begin{tabulary}{0.75\textwidth}{BFl -C -C -C}
      \toprule
      \rowstyle{\bfseries} & Infinitive & Active Participle & Passive Participle \\
      \midrule
      Stem \rowstyle{\itshape} & istasuppe & estáspip  & šestiséppy \\
      Meaning                  & appear    & appearing & appeared \\
      \bottomrule
    \end{tabulary}
  }
  \caption{Pattern V geminate stems \label{tab:vm_v_geminate_stems}}
\end{table}


\subsection{Defective Roots}
\label{ssec:vm_v_defective_roots}

Defective roots in Pattern V follow the same phonological assimilation rules
as have previously described. 


\clearpage
\section{Pattern VI: Passive Reflexive}
\label{sec:vm_pattern_vi}

Pattern VI is the \emph{passive reflexive} stem, 
%
%%   Pattern V roots include: 
%%   - intVCCuC
%%   - inorkut “subscribed”  jantorúkta

It is marked by the infix \conlang{-nt-} in all forms, and may also be known as the \emph{NT-stem}.


\subsection{Triliteral Roots}
\label{ssec:vm_vi_triliteral_roots}

Triliteral roots form the perfective aspects with the pattern
\conlang{*nitV₁C₁C₂V₂C₃}, where V₁ is the inherent root vowel and V₂ is one of
\conlang{-u-}, \conlang{-a-} or \conlang{-i-} for the various subtypes.  The modal perfective
aspects append the suffix \conlang{-e}.

The imperfective aspects are formed with the pattern \conlang{*antaC₁V₂C₂C₃V₁},
where V₁ is the inherent root vowel, and V₂ is one of \conlang{-ú-}, \conlang{-a-} or
\conlang{-i-} for the progressive, durative or habitual aspects.  The modal
imperfective aspects replace the final \conlang{-a} with \conlang{-e}.

The infinitive is formed with the pattern \conlang{*intuC₁C₂eC₃e}; the active participle
with the pattern \conlang{*entáC₁C₂iC₃} and the passive participle with the pattern
\conlang{*šentiC₁C₂éC₃y}. 

Examples of triliteral stems in Pattern VI are given in \cref{tab:vm_vi_triliteral_stems}. 

\begin{table}[h!]\small\capstart
  \centering
  \subfloat[Aspectual stems]{%
    \begin{tabular}{BFl Kl -c -c}
      \toprule
      & & \multicolumn{2}{c}{\conlangt{intorkut}{subscribe}} \\
      \rowstyle{\bfseries} Aspect & & Indicative & Modal \\
      \midrule
      Perfective   & \acs{perf} & nitorkut  & nitorkute \\
      Experiential & \acs{exp}  & nitorkat  & nitorkate \\
      Momentane    & \acs{momt} & nitorkit  & nitorkite \\
      Progressive  & \acs{prog} & antarúkto & antarúkte \\
      Durative     & \acs{dur}  & antarákto & antarákte \\
      Habitual     & \acs{hab}  & antaríkto & antaríkte \\
      \bottomrule
    \end{tabular}
  }\\
  \subfloat[Non-finite stems]{%
    \begin{tabulary}{0.75\textwidth}{BFl -C -C -C}
      \toprule
      \rowstyle{\bfseries} & Infinitive & Active Participle & Passive Participle \\
      \midrule
      Stem \rowstyle{\itshape} & inturkete & entárkit    & šentirkéty \\
      Meaning                  & subscribe & subscribing & subscribed \\
      \bottomrule
    \end{tabulary}
  }
  \caption{Pattern V triliteral stems \label{tab:vm_vi_triliteral_stems}}
\end{table}


\subsection{Biliteral Roots}
\label{ssec:vm_vi_biliteral_roots}

\Tbw


\subsection{Quadriliteral Roots}
\label{ssec:vm_vi_quadriliteral_roots}

\Tbw


\subsection{Geminate Roots}
\label{ssec:vm_vi_geminate_roots}

\Tbw


% \clearpage
% \section{Pattern VII: Stative}
% \label{sec:vm_pattern_vii}
% 
% Pattern VII is the \emph{stative} stem, used to form stative verbs and verbs
% that describe attributes and qualities.  
% %Most adjective-like words are formed from this pattern, such as \conlang{ivlešu} “tall” (from \conlang{veluš} “grow”).
% 
% 
% \subsection{Triliteral Roots}
% \label{ssec:vm_vii_triliteral_roots}
% 
% Triliteral roots form the perfective aspects with the pattern \conlang{*iC₁C₂V₁C₃V₂},
% where V₁ is the inherent root vowel and V₂ is one of \conlang{-u}, \conlang{-a} or \conlang{-i}
% for the various subtypes.  There is no modal stem for this pattern. 
% 
% The imperfective aspects are formed with the pattern \conlang{*eC₁V₂ːC₂C₃a}, where
% V₂ is \conlang{-ú-} for the progressive aspect, \conlang{-á-} for the durative aspect, and
% \conlang{-í-} for the habitual aspect.  
% 
% The infinitive is formed with the pattern \conlang{*iC₁C₂eC₃e} and the passive
% participle with the pattern \conlang{*šeiC₁C₂uC₃}; Pattern VII verbs lack an active
% participle.  
% 
% Examples of triliteral stems in Pattern VII are given in \cref{tab:vm_vii_triliteral_stems}. 
% 
% \begin{table}[h!]\small\capstart
%   \centering
%   \subfloat[Aspectual stems]{%
%     \begin{tabular}{BFl Kl -c}
%       \toprule
%       & & \conlangt{iksetu}{be ready} \\
%       \rowstyle{\bfseries} Aspect & & Stem \\
%       \midrule
%       Perfective   & \acs{perf} & iksetu \\
%       Experiential & \acs{exp}  & ikseta \\
%       Momentane    & \acs{momt} & ikseti \\
%       Progressive  & \acs{prog} & ekústa \\
%       Durative     & \acs{dur}  & ekásta \\
%       Habitual     & \acs{hab}  & ekísta \\
%       \bottomrule
%     \end{tabular}
%   }\\
%   \subfloat[Non-finite stems]{%
%     \begin{tabular}{BFl -c -c}
%       \toprule
%       \rowstyle{\bfseries}     & Infinitive & Passive Participle \\
%       \midrule
%       Stem \rowstyle{\itshape} & iksete & šeiksyt \\
%       Meaning                  & ready  & readied \\
%       \bottomrule
%     \end{tabular}
%   }
%   \caption{Pattern VII triliteral stems \label{tab:vm_vii_triliteral_stems}}
% \end{table}
% 
% 
% \subsection{Biliteral Roots}
% \label{ssec:vm_vii_biliteral_roots}
% 
% Biliteral roots form the perfective aspects with the pattern \conlang{*C₁iC₂V₂},
% where V₂ is one of \conlang{-u}, \conlang{-a} or \conlang{-i} for the various subtypes.  There is
% no distinct modal stem for this pattern.
% 
% The imperfective aspects are formed with the pattern \conlang{*eC₁V₂ːC₂V₁}, where V₁
% is the inherent vowel and V₂ is \conlang{-ú-} for the progressive aspect, \conlang{-á-} for
% the durative aspect, and \conlang{-í-} for the habitual aspect.  
% 
% % The infinitive is formed with the pattern \conlang{*iC₁C₂eC₃} and the passive
% % participle with the pattern \conlang{*šeiC₁C₂uC₃}.  
% 
% % Examples of biliteral stems in Pattern VII are given in \cref{tab:vm_vii_biliteral_stems}. 
% 
% % \begin{table}[h!]\small\capstart
% %   \centering
% %   \subfloat[Aspectual stems]{%
% %     \begin{tabular}{BFl Kl -cw -c}
% %       \toprule
% %       & & \conlang{iprešu} “be red” & \conlangt{iksetu}{be ready} \\
% %       \rowstyle{\bfseries} Aspect & & Stem & Stem \\
% %       \midrule
% %       Perfective   & \acs{perf} & iprešu &  iksetu  \\
% %       Experiential & \acs{exp}  & ipreša &  ikseta  \\
% %       Momentane    & \acs{momt} & ipreši &  ikseti  \\
% %       Progressive  & \acs{prog} & epúrša &  ekústa  \\
% %       Durative     & \acs{dur}  & epárša &  ekásta  \\
% %       Habitual     & \acs{hab}  & epírša &  ekísta  \\
% %       \bottomrule
% %     \end{tabular}
% %   }\\
% %   \subfloat[Non-finite stems]{%
% %     \begin{tabular}{BFl -c -c}
% %       \toprule
% %       \rowstyle{\bfseries}     & Infinitive & Passive Participle \\
% %       \midrule
% %       Stem \rowstyle{\itshape} & ipreš & šeipruš \\
% %       Meaning                  & red   & red \\
% %       \bottomrule
% %     \end{tabular}
% %   }
% %   \caption{Pattern VII biliteral stems \label{tab:vm_vii_biliteral_stems}}
% % \end{table}
% 
% 
% \subsection{Geminate Roots}
% \label{ssec:vm_vii_geminate_roots}
% 
% Geminate roots behave like biliteral roots in Pattern VII.  The perfective
% aspects are formed with the pattern \conlang{*C₁iC₂C₂V₂}, where V₂ is one of \conlang{-u},
% \conlang{-a} or \conlang{-i} for the various subtypes.  
% 
% The imperfective aspects are formed with the pattern \conlang{*eC₁V₂ːC₂C₂i}, where
% V₁ is the inherent vowel and V₂ is \conlang{-ú-} for the progressive aspect, \conlang{-á-}
% for the durative aspect, and \conlang{-í-} for the habitual aspect.  There is no
% distinct modal form for this pattern. 
% 
% The infinitive is formed with the pattern \conlang{*iC₁eC₂C₂e} and the passive
% participle with the pattern \conlang{*šiC₁yC₂C₂}.  
% 
% \begin{table}[h!]\small\capstart
%   \centering
%   \subfloat[Aspectual stems]{%
%     \begin{tabular}{BFl Kl -c -c }
%       \toprule
%       & & \multicolumn{2}{c}{\conlangt{zillu}{green}}  \\
%       \rowstyle{\bfseries} Aspect & & Stem \\
%       \midrule
%       Perfective   & \acs{perf} & zillu    \\
%       Experiential & \acs{exp}  & zilla    \\
%       Momentane    & \acs{momt} & zilli    \\
%       Progressive  & \acs{prog} & ezúlli  \\
%       Durative     & \acs{dur}  & ezálli  \\
%       Habitual     & \acs{hab}  & ezílli  \\
%       \bottomrule
%     \end{tabular}
%   }\\
%   \subfloat[Non-finite stems]{%
%     \begin{tabular}{BFl -c -c}
%       \toprule
%       \rowstyle{\bfseries}     & Infinitive & Passive Participle \\
%       \midrule
%       Stem \rowstyle{\itshape} & izelle & šizyll \\
%       Meaning                  & green  & green \\
%       \bottomrule
%     \end{tabular}
%   }
%   \caption{Pattern VII geminate stems \label{tab:vm_vii_geminate_stems}}
% \end{table}
% 
% 
% \subsection{Defective Roots}
% \label{ssec:vm_vii_defective_roots}
% 
% Defective roots in Pattern V follow the same phonological assimilation rules
% as have been previiously described.  Some examples are listed in
% \cref{tab:vm_vii_defective_stems}. 
% 
% 
% \begin{table}[h!]\small\capstart
%   \centering
%   \subfloat[Aspectual stems]{%
%     \begin{tabular}{BFl Kl -c -c }
%       \toprule
%       & & \conlang{íveru} “be good” \\
%       \rowstyle{\bfseries} Aspect & & Stem \\
%       \midrule
%       Perfective   & \acs{perf} & íveru   \\
%       Experiential & \acs{exp}  & ívera   \\
%       Momentane    & \acs{momt} & íveri   \\
%       Progressive  & \acs{prog} & ehúvra  \\
%       Durative     & \acs{dur}  & ehávra  \\
%       Habitual     & \acs{hab}  & ehívra  \\
%       \bottomrule
%     \end{tabular}
%   }\\
%   \subfloat[Non-finite stems]{%
%     \begin{tabular}{BFl -c -c}
%       \toprule
%       \rowstyle{\bfseries}     & Infinitive & Passive Participle \\
%       \midrule
%       Stem \rowstyle{\itshape} & ívere & šévyr \\
%       Meaning                  & good  & good \\
%       \bottomrule
%     \end{tabular}
%   }
%   \caption{Pattern VII defective stems \label{tab:vm_vii_defective_stems}}
% \end{table}
% 
% 
% \clearpage
\section{Aspect}
\label{sec:vm_aspect}

Qevesa lacks a means to indicate tense, exclusively using aspectual stems
instead.  The six morphological aspects are the \emph{perfective},
\emph{experiential}, \emph{momentane}, \emph{progressive}, \emph{durative}, and
\emph{habitual}.


\subsection{Perfective}
\label{vp:ssec_perfective}

The perfective aspect indicate activities viewed as a single whole.  It is
typically used to speak of singular events completed in the past, but may also
be used to speak of actions without internal structure, or events that are
bounded temporally, spacially, or conceptually. 

\begin{exe}
  \ex \conlang{Kesselost veki zekétuns.}\\ % Kesselast veki zekétuns
  \gll Kessel-ost veki ze-két-u-ns\\
  Kessel-\acs{loc} to \acs{1pl};\acs{exc}-go-\acs{perf}-\acs{agt}.\acs{pl}\\
  \glt We went to Kessel.
  \ex \conlang{Ni peks lamiztivaš mórun.}\\
  \gll Ni-∅ peks lamizti-v-aš mór-u-n\\ 
  \acs{3sg}-\acs{dir} five ballgame-\acs{du}-\acs{acc} see-\acs{perf}-\acs{3sg}.\acs{agt}\\
  \glt He has watched five ballgames.
\end{exe}

The bounded nature of the perfective is often indicated by specifying a duration:

\begin{exe}
  \ex \conlang{Kori meséhitevid zevatesnuš.}\\
  \gll kori meséhit-ev-id ze-vatesn-u-š\\
  three hour-\acs{du}-\acs{ess} \acs{1du};\acs{exc}-sleep.together-\acs{perf}-\acs{pat}\\
  \glt We slept together for three hours.
\end{exe}

% I wrote / I have written
% I was an architect (and still am, depending on context).

When used with an object that has a partitive number, the perfective aspect
conveys an atelic sense:

\begin{exe}
  \ex \conlang{A rekátoš hakojurin.}\\
  \gll A rekát-oš ha-kojur-in\\
  \acs{def} book-\acs{acc} \acs{1sg}-read\bs\acs{perf}-\acs{agt}\\
  \glt I read the book.
  \ex \conlang{A rekátiniš hakojurin.}\\
  \gll A rekát-in-iš ha-kojur-in\\
  \acs{def} book-\acs{part}-\acs{acc} \acs{1sg}-read\bs\acs{perf}-\acs{agt}\\
  \glt I read [some of] the book [and have not finished it].
\end{exe}


\subsection{Experiential}
\label{vp:ssec_experiential}

The experiential aspect expresses that the situation has been experienced
before.  There is some overlap between the perfective and experiential aspects,
but the experiential carries connotations of ‘completeness’ that the perfective
does not.   

\begin{exe}
  \ex \conlang{Ni peks lamiztivaš móran.}\\
  \gll Ni-∅ peks lamizti-v-aš mór-a-n\\ 
  \acs{3sg}-\acs{dir} five ballgame-\acs{du}-\acs{acc} see-\acs{exp}-\acs{agt}\\
  \glt He has watched five ballgames [in his entire life].
  \ex \conlang{Velnapad a párisoš tumóran.}\\
  \gll Velnapa-d a páris-oš tu-mór-a-n\\
  tomorrow-\acs{ess} \acs{def} city-\acs{acc} \acs{2sg}-see-\acs{exp}-\acs{agt}\\
  \glt Tomorrow you will have seen [everything in] the city.
  \ex \conlang{A rekátoš hakojarin.}\\
  \gll A rekát-oš ha-kojar-in\\
  \acs{def} book-\acs{acc} \acs{1sg}-read\bs\acs{exp}-\acs{agt}\\
  \glt I read the book [and finished it].
\end{exe}


\subsection{Momentane}
\label{vp:ssec_momentane}

The momentane aspect indicates brief single-time activities or states.   

\begin{exe}
  \ex \conlang{A vurecen zóqin.}\\
  \gll A vurece-n zóq-i-n\\
  \acs{def} lightning-\acs{dir} flash-\acs{momt}-\acs{agt}\\
  \glt The lighting flashed. % todo as the rain beat down on the window.
  \ex \conlang{Nikost sehátost véra {mórin} še matokhrunésaš jotún.}\\
  \gll nik-ost sehát-ost véra mór-i-n še matokhr-u-nés-aš jotú-n\\ 
  \acs{3sg}.\acs{gen}-\acs{loc} watch-\acs{loc} towards see-\acs{momt}-\acs{agt} and late-\acs{perf}-\acs{subj}-\acs{pat} know\bs\acs{perf}-\acs{agt}\\
  \glt She glanced at her watch, and knew she would be late.
\end{exe}

% A bolt of lighting struck the tree.
% The mouse squeaked.

\subsection{Progressive and Durative}
\label{vp:ssec_progressive_durative}

The progressive aspect indicates ongoing actions with a change of state.  

\begin{exe}
  \ex\label{ex:vm_putting_on_clothes} \conlang{Veráninoš havrúnon.}\\
  \gll verán-in-oš h-avrúno-n\\
  clothes-\acs{part}-\acs{acc} \acs{1sg}-wear\bs\acs{prog}-\acs{agt}\\
  \glt I am putting on clothes.
\end{exe}

The durative aspect indicates ongoing actions without a change of state, or
actions which last some time.

\begin{exe}
  \ex\label{ex:vm_wearing_clothes} \conlang{Veráninoš havránon.}\\
  \gll verán-in-oš h-avráno-n\\
  clothes-\acs{part}-\acs{acc} \acs{1sg}-wear\bs\acs{dur}-\acs{agt}\\
  \glt I am wearing clothes.
\end{exe}

There are a number of verb patterns that imply either the progressive or the
durative as their imperfective aspect, or have subtly different meanings
depending on which is used.  Adjectival verbs use the progressive aspect to
indicate a change to the quality described by the adjective, and the durative
is used to indicate a more-or-less continuous state. 
%Verbs that imply a change of state (such as \conlangt{navronu}{dress oneself}) will
%use the progressive aspect 

%Some verbs which imply a change of state, such as \conlang{navronu} 

% I am putting on clothes.
% I am hanging the painting on the wall.
% It is raining in Kirua:  Kiruazi apšórak.


% I am wearing clothes.
% The picture is hanging on the wall.

% I can't stop sneezing
% sneeze-PASS-SUBJ stop-I

\subsection{Habitual}
\label{vp:ssec_habitual}

The habitual aspect describes actions that occur habitually or intermittently. 

% Like the
% progressive, it may also describe intermittent actions, but in a general sense.

% I walk to work (every day).



\section{Verb Mood}
\label{sec:vm_mood_affect}

Qevesa inflects verbs for five basic moods: \emph{indicative}, \emph{mirative},
\emph{conditional}, \emph{optative}, \emph{potential}, and \emph{imperative}.
The indicative mood is marked by separate stems described in the previous
section, and with the exception of the imperative mood, the others are marked
by suffixes appended to the modal stem of the verb.  
%These suffixes are listed in \cref{tab:vm_person_marking}.  

\begin{table}[h!]\small\capstart
  \begin{tabular}{BFl -Kl -l -l}
    \toprule
    \multicolumn{2}{Fc}{\rowstyle{\bfseries}Mood} & Suffix \\
    \midrule
    Subjunctive & \acs{subj} & -nés- \\
    Mirative    & \acs{mir}  & -lá-  \\
    Conditional & \acs{cond} & -zod-  \\
    Optative    & \acs{opt}  & -pe-  \\
    \bottomrule
  \end{tabular}
  \caption{Verbal mood suffixes\label{tab:vm_modal_suffixes}}
\end{table}

The imperative mood is marked on the infinitive verb stem rather than the modal
verb stem, using the suffixes listed in \cref{tab:vm_imperative_affixes}. The
final vowel of the infinitive is dropped before appending the suffix, although
diphthongs ending in \conlang{-i} replace that vowel with a \conlang{-j}.

\begin{table}[h!]\small\capstart
  \begin{tabular}{BFl -Kl -c -c}
    \toprule
    \multicolumn{2}{Fc}{\rowstyle{\bfseries}Aspect} & Prefix & Suffix \\
    \midrule
    Perfective   & \acs{perf}.\acs{imp} &    & -úm   \\
    Imperfective & \acs{ipfv}.\acs{imp} & a- & -ím   \\
    \bottomrule
  \end{tabular}
  \caption{Imperative affixes\label{tab:vm_imperative_affixes}}
\end{table}

\subsection{Indicative Mood}
\label{ssec:vm_indicative}

The indicative mood is used for factual statements and positive beliefs,
and as such is the default mood.  


\subsection{Mirative Mood}
\label{ssec:vm_mirative}

The mirative mood is used to express surprise and also doubt, irony,
sarcasm.  It is used to express statements contrary to the speaker’s
expectations or state of mind.


\subsection{Conditional Mood}
\label{ssec:vm_conditional}

The conditional mood is used to speak of an event whose realization is
dependent upon another condition. 


\subsection{Optative Mood}
\label{ssec:vm_optative}

The optative mood is used to express hopes, wishes and desires.


\subsection{Potential Mood}
\label{ssec:vm_potential}

The potential mood indicates that, in the opinion of the speaker, the
action or occurrence is considered likely.  It can also be used to express that
one has the ability to do something.


\subsection{Imperative Mood}
\label{ssec:vm_imperative}

The imperative mood is used for commands and requests. 

\section{Pronomial Markers}
\label{sec:vm_pronomial_markers}

The Qevesa verb uses a combination of prefixed pronomial markers and suffixed
trigger markers. Both prefixes and suffixes are accompanied by epenthetic vowels
that are inserted before or after a consonant.

%The pre-radical vowels, abbreviated PV and also known as version vowels, are ა-
%a-, ე- e-, ი- i-, and უ- u-, and occur immediately before the verb root or
%stem. They have a number of functions, the more common of which are summarized
%below. In some cases, however, no apparent function can be assigned to the
%pre-radical vowel. Note that the pre-radical vowel ა- a- should not be confused
%with the preverb of the same form.
% ა- a-
%    • forms Class 1 denominatives, e.g., აფაროებს a-parto-eb-s ('he widens it'
%        ← ფართო parto 'wide')
%    • forms causatives, e.g., აწერინებს a-ts'er-in-eb-s ('he causes him to
%    write it’ ← წერს ts'er-s ‘he writes it’)
%    • indicates that the action takes place on something (the ‘superessive
%    version’), e.g., ახატავს a-khat’-av-s ('he paints it on it' ← ხატავს
%    khat’-av-s ‘he paints it’)
% 
% ე- e-
%    • refers to indirect objects, mostly with Class 2 verbs, e.g., ემალება
%    e-mal-eb-a (‘he hides himself from him ← იმალება i-mal-eb-a ‘he hides
%    himself’)
%    • refers to pluperfect screeve subjects, e.g., გაგცეღო ga-gv-e-gh-o (‘we
%    opened it’)
% 
% ი- i-
%    • indicates first and second person indirect objects when the action takes
%    place for someone's benefit (the ‘benefactive version’), e.g.,
%    გიშენებთ g-i-shen-eb-t ('we build it for you' ← ვაშენებთ v-a-shen-eb-t
%    ‘we build it’),
%    • marks inverted subjects in the first and second persons, e.g., გაგიგია
%    ga-g-i-g-i-a ('you have heard it')
%    • indicates reflexivity, e.g., იბანს i-ban-s ('he washes himself' ← ბანს
%    ban-s ‘he washes him’)
%    • forms the future / aorist stem of Class 3 verbs, e.g., ითამაშებს
%    i-tamash-eb-s (‘he will play’ ← თამაშობს tamash-ob-s ‘he plays’),
%    იტირებს i-t’ir-eb-s ‘he will cry’ ← ტირის t’ir-i-s ‘he cries’)
% 
% უ- u-
%    • indicates an indirect object in the third person, e.g., გავუგზავნეთ
%    ga-v-u-gzavn-e-t (‘we sent it to him’ ← გავგზავნეთ ga-v-gzavn-e-t ‘we
%    sent it’)
%    • marks an inverted subject in the third person, e.g., დაულევია
%    da-u-lev-i-a (‘he drank it’)

 
\subsection{Agent Trigger}
\label{ssec:vm_agt_trigger}

The agent trigger indicates that the noun phrase in the direct case is the
voluntary experiencer of an intransitive verb or the agent of a transitive
verb.  This trigger is equivalent to the active voice in other languages, and
the prefixes and suffixes are given in \cref{tab:vm_pronomial_agent_marking}.

\begin{table}[h!]\small\capstart
  \begin{tabular}{KFl -c -c -c}
    \toprule
    \rowstyle{\bfseries} & \multicolumn{2}{-c}{Prefix} & Suffix \\
    \rowstyle{\scshape} & \acs{perf} & \acs{ipfv} & Suffix \\
    \midrule
    \acs{1sg}              & h(a)-      & h-         & -(i)n \\
    \acs{2sg}              & t(u)-      & t-         & -(u)n \\
    \acs{3sg}              & ∅-         & j-         & -(a)n \\
    \acs{1du};\acs{inc}    & v(i)-      & v-         & -(i)n \\
    \acs{1pl};\acs{exc}    & z(e)-      & z-         & -(i)n \\
    \acs{2du}              & t(e)-      & t-         & -(a)n \\
    \acs{3du}              & ∅-,        & j-         & -(a)n \\
    \acs{1pl};\acs{inc}    & s(e)-      & s-         & -(i)ns \\
    \acs{1pl};\acs{exc}    & z(e)-      & z-         & -(i)ns \\
    \acs{2pl}              & t(e)-      & t-         & -(a)ns \\
    \acs{3pl}              & ∅-,        & j-         & -(a)ns \\
%    \midrule
    \bottomrule
  \end{tabular}
  \caption{Pronomial agent marking patterns\label{tab:vm_pronomial_agent_marking}}
\end{table}

\begin{exe}
  \ex \conlang{Jaffúton.}\\
  \gll j-affúto-n\\
  \acs{3sg}-speak\bs\acs{prog}-\acs{3sg}.\acs{agt}\\
  \glt She is speaking.
  \ex \conlang{Rekátoš harokutin.}\\
  \gll rekát-oš h-arokut-in\\
  book-\acs{acc} \acs{1sg}-write\bs\acs{perf}-\acs{agt}\\
  \glt I wrote (a) book.
\end{exe}

Generally only animate nouns may be agents; to describe an action involving an
inanimate noun as agent, a construction using the oblique trigger and the
instrumental case is used instead. 


\subsection{Patient Trigger}
\label{ssec:vm_pat_trigger}

The patient trigger indicates that the noun phrase in the direct case is the
involuntary experiencer of an intransitive verb; the patient of a transitive
verb; and the recipient of a ditransitive verb.  This trigger is roughly
equivalent to the passive and mediopassive voices in other languages. 

Only animate nouns may be voluntary agents of intransitive verbs; inanimate
nouns are always marked as involuntary experiencers of intransitive verbs.
Furthermore, some intransitive verbs are always involuntary, regardless of
animacy. The prefixes and suffixes for the patient trigger are given in
\cref{tab:vm_pronomial_patient_marking}.

\begin{table}[h!]\small\capstart
  \begin{tabular}{KFl -c -c -c}
    \toprule
    \rowstyle{\bfseries} & \multicolumn{2}{-c}{Prefix} & Suffix \\
    \rowstyle{\scshape} & \acs{perf} & \acs{ipfv} & Suffix \\
    \midrule
    \acs{1sg}              & m(e)-      & m-         & -(i)š \\
    \acs{2sg}              & k(e)-      & k-         & -(u)š \\
    \acs{3sg}              & ∅-         & j-         & -(a)š \\
    \acs{1du};\acs{inc}    & v(i)-      & v-         & -(i)š \\
    \acs{1pl};\acs{exc}    & z(e)-      & z-         & -(i)š \\
    \acs{2du}              & k(e)-      & k-         & -(a)š \\
    \acs{3du}              & ∅-         & j-         & -(a)š \\
    \acs{1pl};\acs{inc}    & s(e)-      & s-         & -(i)št \\
    \acs{1pl};\acs{exc}    & z(e)-      & z-         & -(i)št \\
    \acs{2pl}              & k(e)-      & k-         & -(a)št \\
    \acs{3pl}              & ∅-         & j-         & -(a)št \\
    \midrule
    \acs{inanim};\acs{sg}  & ∅-         & ∅-         & -(o)šo \\
    % \acs{inanim};\acs{pl}  & ∅-         & ∅-         & -(o)ši \\
    \bottomrule
  \end{tabular}
  \caption{Pronomial patient marking patterns\label{tab:vm_pronomial_patient_marking}}
\end{table}

\begin{exe}
  \ex \conlang{Rekáta jem kojuroš.}\\
  \gll rekát-a jem kojur-oš\\
  book-\acs{dir} \acs{1sg}.\acs{erg} read\bs\acs{perf}-\acs{3sg};\acs{inanim}.\acs{pat}\\
  \glt A book was read by me.
  \ex \conlang{A sekátevi sanálušo.}\\
  \gll A sekát-ev-i sanál-u-šo\\
  \acs{def} letter-\acs{du}-\acs{dir} deliver\bs\acs{perf}-\acs{3sg};\acs{inanim}.\acs{pat}\\
  \glt [Both of] the letters were delivered.
  \ex \conlang{Ni nášoruš.}\\
  \gll Ni-∅ nášoru-š\\
  \acs{3sg}-\acs{dir} sneeze\bs\acs{perf}-\acs{3sg}.\acs{pat}\\
  \glt He sneezed.
\end{exe}

\subsection{Oblique Trigger}
\label{ssec:vm_obl_trigger}

The oblique trigger indicates that the noun phrase in the direct case is
something other than the agent or patient of a transitive verb.  For
ditransitive verbs it normally indicates the theme or direct object.

Another common use of the oblique trigger is to express an inanimate agent of a
verb. In this case, the noun will be double-marked with both the instrumental
case and the direct case.  The prefixes and suffixes for the patient trigger
are given in \cref{tab:vm_pronomial_oblique_marking}.

\begin{table}[h!]\small\capstart
  \begin{tabular}{KFl -c -c -c}
    \toprule
    \rowstyle{\bfseries} & \multicolumn{2}{-c}{Prefix} & Suffix \\
    \rowstyle{\scshape} & \acs{perf} & \acs{ipfv} & Suffix \\
    \midrule
    \acs{1sg}              & m(e)-      & m-         & -(i)k \\
    \acs{2sg}              & k(e)-      & k-         & -(u)k \\
    \acs{3sg}              & ∅-         & j-         & -(a)k \\
    \acs{1du};\acs{inc}    & v(i)-      & v-         & -(i)k \\
    \acs{1pl};\acs{exc}    & z(e)-      & z-         & -(i)k \\
    \acs{2du}              & k(e)-      & k-         & -(a)k \\
    \acs{3du}              & ∅-         & j-         & -(a)k \\
    \acs{1pl};\acs{inc}    & s(e)-      & s-         & -(i)ks \\
    \acs{1pl};\acs{exc}    & z(e)-      & z-         & -(i)ks \\
    \acs{2pl}              & k(e)-      & k-         & -(a)ks \\
    \acs{3pl}              & ∅-         & j-         & -(a)ks \\
    \midrule
    \acs{inanim};\acs{sg}  & ∅-         & ∅-         & -(o)ko \\
    % \acs{inanim};\acs{pl}  & ∅-         & ∅-         & -(o)ki \\
    \bottomrule
  \end{tabular}
  \caption{Pronomial oblique marking patterns\label{tab:vm_pronomial_oblique_marking}}
\end{table}


\section{Preverbal Markers}
\label{sec:vm_preverbs}

\Tbw


% Class 1-3 present and aorist series verbs are based on one of two stems: the 'present' and the 'future / aorist'.
%
% SERIES STEM
% present ROOT + PSF
% future PVB + ROOT
% aorist
%
% Class 1-3 stems
% The primary function of the preverb (abbreviated PVB) is to indicate direction
% when used with verbs of motion. It has the secondary functions of indicating
% that the action is viewed as completed (the ‘perfective aspect’), and of
% changing the basic meaning of a verb stem. The preverb has also some acquired
% additional functions which are not considered here.
%
% The more common preverbs with their directional meanings are listed in this
% table. (The variants in parentheses are older forms.)
%
% \begin{table}[h!]\small\capstart
%   \begin{tabular}{GFl -Il -l -Gl -Il -l}
%     \toprule
%     \multicolumn{3}{Fc}{\rowstyle{\bfseries}‘there’ (away from speaker)} & \multicolumn{3}{-c}{‘here’ (towards speaker)} \\
%     \midrule
%     ა(ღ)-    & a(gh)-   & up, upwards                  & ა(ღ)მო-   & a(gh)+mo-   & up, upwards \\
%     გა(ნ)-   & ga(n)-   & out                          & გამო-     & ga+mo-      & out \\
%     გა(რ)და- & ga(r)da- & over, across                 & გა(რ)დმო- & ga(r)d+mo-  & over, across \\
%     და-      & da-      & to and fro, about, around \\
%     მი-      & mi-      & there                        & მო-       & mo-         & here \\
%     შე-      & she-     & in                           & შემო-     & she+mo-     & in \\
%     ჩა-      & cha-     & down, downwards              & ჩამო-     & cha+mo-     & down,downwards \\
%     წა(რ)-   & ts’a(r)- & away, off                    & წა(რ)მო-  & ts’a(r)+mo- & away, off \\
%     \bottomrule
%   \end{tabular}
%   \caption{Directional preverbs\label{tab:vm_directional_preverbs}}
% \end{table}

\end{document}
