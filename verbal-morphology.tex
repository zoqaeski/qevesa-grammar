\documentclass[grammar]{subfiles}
\begin{document}
  \chapter{Verbal Morphology}
  \label{ch:verbal_morphology}

  \section{Features}
  \label{sec:vm_features}

  The consonantal root patterns in Qevesa are used to form basic morphological
  paradigms.  Qevesa verbs are highly inflected, indicating tense and aspect by
  transfix patterns; topical agreement and modality are marked by agglutinative
  suffixes.  All other constructions, are indicated by periphrasis or syntax.

  The stem consists of the root and zero or more derivational affixes conjugated to a particular aspect. 

  \section{Verb Root Forms}
  \label{sec:vm_root_forms}

  Although the arrangement of consonants in a root is generally fixed, there
  are regular processes to derive subtle semantic variations on the meaning of
  the root, such as causatives and reflexives.  These root variants are called
  forms, or \qevesa{???} (“constructions”), from the root \qevesa{mukut}
  (“build, construct”).  There are five primary forms, numbered I–V; these are
  listed in \cref{tab:vm_root_forms}.

  \begin{table}[htpb]\small\capstart
    \begin{tabular}{>{\bfseries}Fl -l}
      \toprule
      \SetRowStyle{\bfseries} Form & Pattern \\
      \midrule
      I   & C\sub1{u}C\sub2{u}C\sub3 \\
      II  & C\sub1{u}C\sub2C\sub3{u} \\
      III & {u}C\sub1C\sub2{u}C\sub3 \\
      IV  & {me}C\sub1{u}C\sub2C\sub3{u} \\
      V   & {ta}C\sub1C\sub2{u}C\sub3{u} \\
      \bottomrule
    \end{tabular}
    \caption{Verb root forms\label{tab:vm_root_forms}}
  \end{table}

  \subsection{Form I: Active}
  \label{ssec:vm_verb_form_i}

  Form I is the most common consonantal root form, containing no preformative
  affixes or pairing of consonants as occurs in the other forms.  It is
  typically the closest indicator to the lexical meaning of the root, and
  although it has no particular semantic function associated with it, verbs in
  Form I are often transitive.


  \subsection{Form II: Intensive}
  \label{ssec:vm_form_ii}

  Form II is the intensive stem. It typically indicates an intensive,
  frequentative or causative meaning, and may also be used to form transitive
  verbs from intransitive roots.
   
%   Form II roots include:
%   - [i]CuCCu[/i]
%   - [i]pútu[/i] “shout”
%   - [i]murru[/i] (*[i]murhu[/i]) “watch”
%   - [i]thuvru[/i] “shatter”
   
  \subsection{Form III: Causative}
  \label{ssec:vm_form_iii}

   Form III is commonly known as the causative stem. Its most common function
   is causative; it may also convert transitive verbs into ditransitive ones.
   It can also have a causative meaning on verbs whose Form 1 root is
   intransitive, and for some verbs, may convey an assistive or factitive
   meaning.
   
%   Form III roots include:
%   - [i]uCCuC[/i] 
%   - [i]ukpuš[/i] “feed”
%   - [i]utrum[/i] “dictate”
%   - [i]umrú[/i] “show”
%   - [i]uthvur[/i] “cause to break”
%   
  \subsection{Form VI: Reciprocal}
  \label{ssec:vm_form_vi}

   Form VI is commonly known as the reciprocal stem. It commonly conveys
   meanings of a reciprocal or reflexive nature, and is oen used to create
   verbs denoting social interactions. 
   
%   Form VI roots include:
%   - [i]meCuCCu[/i]
%   - [i]mepútu[/i] “converse”
%   - [i]meturmu[/i] “correspond”
%   
  \subsection{Form V: Reciprocal Causative}
  \label{ssec:vm_form_v}

   Form V is the reciprocal causative stem, so called for historical reasons as
   it also includes a number of other intransitive meanings. It is subject to
   much unpredictable metaphorical and semantic and dri, so actual meanings
   may vary quite a lot from the Form 1 verb. True reflexives account for only
   a portion of the verbs in this form. Its main functions are:
%   
%   - Forming reflexives from transitive roots - Forming verbs denoting
%   accompaniment - Forming autoreflexive verbs, that is, intransitive actions
%   performed on one’s body[/list]
%   
%   Form V roots include: - [i]taCCuCu[/i]

  \section{The Infinitive}
  \label{sec:vm_infinitive}

  The infinitive verb is the citation form of the verb, as well as the
  non-finite form used in constructions involving an auxiliary verb.  It is
  marked by the patterns \qevesa{C\sub1{u}C\sub2{u}C\sub3}.

  \ToBeWritten


  \section{Conjugation}
  \label{sec:vm_conjugation}

  Qevesa is a highly synthetic language, and verbs are conjugated to indicate
  aspect, tense, topical agreement, and mood.  The conjugated form of the verb
  is as follows:

  \begin{exe}
    \ex\label{exe:vm_conjugation} \textit{stem}\bs\textsc{aspect-mood-topic}
  \end{exe}

  \subsection{Aspect and Tense}
  \label{ssec:vm_aspect_tense}
  
%  Qevesa verbal morphology is structured around a two-by-three contrast of two
%  aspects, perfective and imperfective, and three tenses, present, past and
%  future.  There are also two imperatives, one for each aspect, which are not
%  marked for tense.  These are marked by a series of ten transfix patterns, as
%  shown in Table~\ref{tab:vm_tense-aspect_relations}. 
  
  Qevesa verbal morphology primarily indicates aspect rather than tense.  There
  are seven aspectual paradigms, each marked with a transfix pattern.  These
  are given in Table~\ref{tab:vm_aspect_patterns}.

  \begin{table}[htpb]\small\capstart
      \begin{tabular}{>{\bfseries}Fl >{\scshape}l -l -l -l -l -l}
        \toprule
        \SetRowStyle{\bfseries} Aspect & & I & II & III & IV & V \\
        \midrule
        Perfective & 
        \acs{perf} &
        C\sub1{u}C\sub2{o}C\sub3 & 
        C\sub1{u}C\sub2C\sub3{o} & 
        {u}C\sub1C\sub2{o}C\sub3 & 
        {me}C\sub1{u}C\sub2C\sub3{o} & 
        {ta}C\sub1C\sub2{u}C\sub3{o} \\
        %
        Momentane & 
        \acs{momt} &
        C\sub1{u}C\sub2{a}C\sub3 & 
        C\sub1{u}C\sub2C\sub3{a} & 
        {u}C\sub1C\sub2{a}C\sub3 & 
        {me}C\sub1{u}C\sub2C\sub3{a} & 
        {ta}C\sub1C\sub2{u}C\sub3{a} \\
        %
        Progressive & 
        \acs{prog} &
        C\sub1{i}C\sub2{u}C\sub3 & 
        C\sub1{i}C\sub2C\sub3{u} & 
        {i}C\sub1C\sub2{u}C\sub3 & 
        {me}C\sub1{i}C\sub2C\sub3{u} & 
        {ta}C\sub1C\sub2{i}C\sub3{u} \\
        %
        Durative & 
        \acs{dur} &
        C\sub1{a}C\sub2{u}C\sub3 & 
        C\sub1{a}C\sub2C\sub3{u} & 
        {a}C\sub1C\sub2{u}C\sub3 & 
        {me}C\sub1{a}C\sub2C\sub3{u} & 
        {ta}C\sub1C\sub2{a}C\sub3{u} \\
        %
        Habitual & 
        \acs{hab} &
        C\sub1{o}C\sub2{u}C\sub3 & 
        C\sub1{o}C\sub2C\sub3{u} & 
        {o}C\sub1C\sub2{u}C\sub3 & 
        {me}C\sub1{o}C\sub2C\sub3{u} & 
        {ta}C\sub1C\sub2{o}C\sub3{u} \\
        %
        Inchoative & 
        \acs{inch} &
        C\sub1{o}C\sub2{a}C\sub3 & 
        C\sub1{o}C\sub2C\sub3{a} & 
        {o}C\sub1C\sub2{a}C\sub3 & 
        {me}C\sub1{o}C\sub2C\sub3{a} & 
        {ta}C\sub1C\sub2{o}C\sub3{a} \\
        %
        Cessative & 
        \acs{cess} &
        C\sub1{o}C\sub2{o}C\sub3 & 
        C\sub1{o}C\sub2C\sub3{o} & 
        {o}C\sub1C\sub2{o}C\sub3 & 
        {me}C\sub1{o}C\sub2C\sub3{o} & 
        {ta}C\sub1C\sub2{o}C\sub3{o} \\
        %
        \bottomrule
      \end{tabular}
    \caption{Aspectual transfix patterns\label{tab:vm_aspect_patterns}}
  \end{table}


  \subsubsection{Perfective}
  \label{vm:sssec_perfective}

  The perfective aspect indicate activities viewed as a single whole.
  It is typically used to speak of singular events completed in the past.

  % I wrote / I have written

  \subsubsection{Momentane}
  \label{vm:sssec_momentane}

  The momentane aspect indicates brief single-time activities or states.

  % A bolt of lighting struck the tree.
  % The mouse squeaked.

  \subsubsection{Progressive}
  \label{vm:sssec_progressive}

  The progressive aspect indicates ongoing actions with a change of state.  It
  may also be used to describe intermittent actions.

  % I am putting on clothes.
  % I am hanging the painting on the wall.
  % It is raining in Kirua:  Kiruazia pišurat.

  \subsubsection{Durative}
  \label{vm:sssec_durative}

  The durative aspect inicates ongoing actions without a change of state, or actions which last some time.

  % I am wearing clothes.
  % The picture is hanging on the wall.

  \subsubsection{Habitual}
  \label{vm:sssec_habitual}

  The habitual aspect indicates actions that occur habitually.  Like the progressive, it may also describe intermittent actions, but in a general sense.   

  % I walk to work (every day).
  % It is very windy on the plains.

  \subsubsection{Inchoative}
  \label{vm:sssec_inchoative}

  The inchoative aspect emphasises the beginning of an activity or state.

  % She stated walking.
  
  \subsubsection{Cessative}
  \label{vm:sssec_cessative}

  The cessative aspect emphasises the ending of an activity or state.

  % He finished writing.
  
%  \newpage
%  \subsubsection{The Imperatives}
%  \label{sssec:vm_imperatives}
%  
%  Qevesa possesses two imperatives, one for each aspect.  The Form 7 verb roots do not possess an imperative. 
%
%  \begin{itemize*}
%    \item The \textbf{perfective} is used for single complete actions. 
%    \item The \textbf{imperfective} is used for continuous or otherwise incomplete actions. 
%  \end{itemize*}

%  The transfix patterns for this series are listed in Table~\ref{tab:vm_imperative_series}. 

%  \begin{table}[htpb]\small\capstart
%      \begin{tabularx}{0.625 \textwidth}{>{\bfseries}Fl *{2}{f}}
%          \toprule
%          \SetRowStyle{\bfseries} Form & Perfective Imperative & Imperfective Imperative \tnl
%          \SetRowStyle{\scshape} & \acs{pfv};\acs{imp} & \acs{ipfv};\acs{imp} \tnl
%          \midrule
%          I & 
%          C\sub1{iu}C\sub2{o}C\sub3 & 
%          C\sub1{uo}C\sub2{u}C\sub3 \\
%          II & 
%          C\sub1{ui}C\sub2C\sub3{o} & 
%          C\sub1{uo}C\sub2C\sub3{u} \\
%          III & 
%          {i}C\sub1C\sub2{o}C\sub3 & 
%          {o}C\sub1C\sub2{u}C\sub3 \\
%          IV & 
%          {me}C\sub1C\sub1{ui}C\sub2{o}C\sub3 & 
%          {me}C\sub1C\sub1{oi}C\sub2{u}C\sub3 \\
%          V & 
%          {ta}C\sub1C\sub2{ui}C\sub3{o} & 
%          {ta}C\sub1C\sub2{oi}C\sub3{u} \\
%          \bottomrule
%        \end{tabularx}
%      \caption{Imperative series transfix patterns\label{tab:vm_imperative_series}}
%  \end{table}

  %\newpage
  \subsection{Topical Agreement}
  \label{ssec:vm_topical_agreement}

  Many of the languages in the ??? family, which includes Therasa and its
  descendants, employ some variant on an active-stative morphosyntactic
  alignment.  Verbs in Qevesa and related languages are marked for topic, which
  may be the agent, patient or some oblique noun phrase, irrespective of
  valency of the verb.  This is hypothesised to be a remnant of a system of
  polypersonal agreement which collapsed into a single suffix that indicated
  the most important element in the clause. 

  Nouns are marked with a corresponding \emph{focal case}\footnotemark{} which
  serves to indicate the topic of the clause.  The topic markers on the verb
  therefore indicate the role of the topical noun.  Syntax also plays a role:
  nouns in the focal case are always the first element in a clause.
  \footnotetext{See Section~\ref{nm_focal_case} for more details}

  The suffixes for topical agreement are given in Table~\ref{tab:vm_topical_agreement}.

  \begin{table}[htpb]\small\capstart
    \begin{tabularx}{0.625 \textwidth}{>{\bfseries}Fl>{\scshape}l *{3}{f}}
      \toprule
      \multicolumn{2}{-c}{\SetRowStyle{\bfseries}} & Nominative & Absolutive & Oblique \tnl
      \SetRowStyle{\scshape} & & \acs{nom} & \acs{abs} & \acs{obl} \tnl
      \midrule
      Animate   & \acs{anim}   & -(a)m & -(a)š & -(a)t  \tnl
      Inanimate & \acs{inanim} & -nom  & -noš  & -not  \tnl
      \bottomrule
    \end{tabularx}
    \caption{Topical agreement\label{tab:vm_topical_agreement}}
  \end{table}

  \subsubsection{Nominative Topic}
  \label{sssec:vm_nom_topic}

  An nominative topic indicates that the noun phrase in the focal case is the
  voluntary experiencer of an intransitive verb or the agent of a transitive
  verb.

  \subsubsection{Absolutive Topic}
  \label{sssec:vm_abs_topic}

  An absolutive topic indicates that the noun phrase in the focal case is the
  involuntary experiencer of an intransitive verb; the patient of a transitive
  verb; and the recipient of a ditransitive verb.  Only animate nouns may be
  voluntary agents of intransitive verbs; inanimate nouns are always marked as
  involuntary experiencers of intransitive verbs.  Furthermore, some
  intransitive verbs are always involuntary. 

  \subsubsection{Oblique Topic}
  \label{sssec:vm_obl_topic}

  An oblique topic indicates that the noun phrase in the focal case is something
  other than the agent or patient of a transitive verb.  For ditransitive verbs
  it normally indicates the theme or direct object.  
  
  \subsection{Modality}
  \label{ssec:vm_modality}

  Qevesa predominantly indicates modality by means of suffixes, with the
  exception of the imperatives described in Section~\ref{sssec:vm_imperatives}.
  There are five synthetic moods: indicative, mirative, conditional, optative
  and potential.  These are listed in Table~\ref{tab:vm_modal_suffixes}; the
  left column indicates suffixes that follow a consonant, and the right column
  suffixes that follow a vowel.
  
  \begin{table}[htpb]\small\capstart
      \begin{tabular}{>{\bfseries}Fl->{\scshape}l -l -l}
        \toprule
        \multicolumn{2}{fc}{\SetRowStyle{\bfseries}Mood} & \multicolumn{2}{-c}{Suffix} \\
        \midrule
        Indicative  & \acs{ind}  & -∅   & -∅   \\
        Mirative    & \acs{mir}  & -ine & -nne \\
        Conditional & \acs{cond} & -isi & -ssi \\
        Optative    & \acs{opt}  & -iti & -tti \\
        Potential   & \acs{pot}  & -ill & -ll  \\
        \bottomrule
      \end{tabular}
      \caption{Verbal mood suffixes\label{tab:vm_modal_suffixes}}
  \end{table}

  The \emph{indicative} mood is used for factual statements and positive
  beliefs, and as such is the default mood.  It is marked with a null morpheme. 

  The \emph{mirative} mood is used to express surprise and also doubt, irony,
  sarcasm, etc.  It is used to express statements contrary to the speaker’s
  expectations or state of mind.

  The \emph{conditional} mood is used to speak of an event whose realization is dependent upon another condition. 

  The \emph{optative} mood is used to express hopes, wishes and desires.

  The \emph{potential} mood indicates that, in the opinion of the speaker, the
  action or occurrence is considered likely.  Some of its uses overlap with the
  conditional mood.

%  \section{Auxiliary Verbs}
%  \label{sec:vm_auxiliary}
%
%  Periphrastic constructions, such as polarity, are indicated with a series of auxiliary verbs. 
%
%  The auxiliary verb is inflected, taking the conjugated form of the main verb, which precedes it in the infinitive.
%
%  \begin{exe}
%    \ex\label{exe:vm_auxiliary_conjugation} \textit{stem}\bs\acs{inf} \textit{auxiliary}\bs\textsc{aspect;tense;mood-topic(-mood)}
%  \end{exe}
%
%  \subsection{Polarity}
%  \label{ssec:vm_polarity}
%
%  The most commonly-used auxiliary verbs are those that indicate polarity.  The affirmative verb, \qevesa{zuru}, is generally only used in situations when an explicitly positive statement is to be made.  The negative verb, \qevesa{nuku}, is more commonly used, and shares the same root as the word for ‘zero’ or ‘none’.
%
%  \begin{exe}
%    \ex \qevesa{Misa turum niukam.}
%    \glll Misa turum niuka-m\\
%    \acs{3p}\acs{pl}.\acs{foc} write\bs\acs{inf} \acs{neg}\bs\acs{fut};\acs{pfv}-\acs{anim};\acs{nom}\\
%    {They} {write} {will not}\\
%    \glt They will not write.
%  \end{exe}


%  \begin{table}[htpb]\small\capstart
%      \begin{tabular}{>{\bfseries}fc->{\scshape}c -c -c}
%        \toprule
%        \SetRowStyle{\bfseries} & & \multicolumn{2}{-c}{Polarity} \tnl
%        \cline{3-4}
%        \SetRowStyle{\scshape} & & \acs{aff} & \acs{neg} \tnl
%        \midrule
%        Imperfective  & \acs{ipfv}           & rusi  & nuki \tnl
%        Stative       & \acs{stat}           & ruise & nuike \tnl
%        Durative      & \acs{dur};\acs{ipfv} & rusú  & nukú \tnl
%        Frequentative & \acs{freq}           & ruso  & nuko \tnl
%        Habitual      & \acs{hab}            & rusa  & nuka \tnl
%        \midrule
%        Perfective    & \acs{pfv}            & riosa & nioka \tnl
%        Inchoative    & \acs{inch}           & riuso & niuko \tnl
%        Cessative     & \acs{cess}           & rísa  & níka \tnl
%        Durative      & \acs{dur};\acs{pfv}  & riasu & niaku \tnl
%        Momentane     & \acs{momt}           & riusa & niuka \tnl
%        \bottomrule
%      \end{tabular}
%      \caption{Polar verb aspectual conjugation\label{tab:vm_polarity_auxiliary_aspect}}
%  \end{table}


  % Peter-FOC doctor-ABS rusi-m-u
  % Peter is a doctor.
  % doctor-FOC Peter-NOM rusi-š-u
  % A doctor Peter is.

  % We are not hiding a drunk Verdurian in the room!
  % Potmi Verdurija akkaperossa čésam rafuk zumišóra!
  % drunk Verdurija-ø ak~kaper-ossa čés-am rafuk zummi-š-óra
  % drunk Verdurian-FOC DEF~room-INE 1PL;EXC-NOM hide\INF not\IPFV-ASG;ABS-MIR
  % drunk Verdurian in the room we hide are not!

  % putum - drink, alcohol
  % potmi = drunk
  % kupur = room
  % rufuk = hide, conceal, guard
  % tupun = die
  % itpunu = kill
  % šanam = mouse
  % nuvum = ???
  % navoim = cat
  % nulup = hunger, desire ??

  % You killed all the mice, so the cat is hungry.
  % Ta a mun aššanamesaš itpionamu, annavoima nuilpešen.
  % /taʔamyn aɕːamesaʂ itpʲionamu anːavoima nuilpeʂen/
  % Ta a=mun aš~šanam-es-aš itpiona-m-u, an~navoim-a nuilpe-š-en.
  % 2SG.FOC DEF=all DEF~mouse-PL-ABS kill\PFV-ASG;NOM-IND, DEF~cat-FOC hungry\STAT-ASG;ABS-ALE
  % You the all mice killed, the cat is hungry therefore

  % All the mice were killed by you, so the cat is hungry.
  % A mun aššanamesa tam itpionašu, annavoima nuilpešen.
  % /a myn aɕːanamesa tam itpʲionaʂu, anːavoima nuilpeʂen/
  % A=mun aš~šanam-es-a tam itpiona-š-u, an~navoim-a nuilpe-š-en.
  % DEF=all DEF~mouse-PL-FOC 2SG.NOM kill\PFV-ASG;ABS-IND, DEF~cat-FOC hungry\STAT-ASG;ABS-ALE
  % The all mice you killed, the cat is hungry therefore

  %\newpage
%  Sometimes the polar auxiliaries will be conjugated to a different aspect than their head verb, especially to indicate semantic nuances, for example:
%
%  \begin{exe}
%    \ex \qevesa{Assošima jem zíma nuttúlašu.}
%    \glll As-sošim-a jem zímma nuttúl-aš-u\\
%    \textsc{def-}girl\textsc{-foc} \textsc{1sg.nom} \textsc{neg\bs cess} think\textsc{\bs F2.dur;ipfv-asg;abs-ind}\\
%    {the girl} {I} {not stop} {thinking about her}\\
%    \glt I cannot stop thinking about that girl.
%  \end{exe}

% Using a determiner, such as isátka, strictly refers to location so is not necessary.
% e.g.  Asisátka assošima jem zímma nuttúlašu.

%  \subsection{Evidentiality}
%  \label{ssec:vm_evidentiality}
%
%  Evidentiality may also be expressed by means of auxiliary verbs.  Qevesa possesses a set of auxiliary verbs which distinguish four degrees of evidentiality: witness, reportative, inferential, and assumptive. 
%
%  All of the roots of the evidential auxiliaries are also verbs in their own right.  However, they conjugate as Form VIII verbs, with some slightly irregular pattern forms.  Their conjugation is given in Table~\ref{tab:vm_evidentiality_conjugation}.
%
%  \begin{table}[htpb]\small\capstart
%      \begin{tabular}{>{\bfseries}fc->{\scshape}c -c -c -c -c}
%        \toprule
%        \SetRowStyle{\bfseries} & & \multicolumn{4}{-c}{Evidentiality} \tnl
%        \cline{3-6}
%        \SetRowStyle{\scshape} &  & exp   & rep    & infr   & asm		 \tnl
%        \midrule
%        Imperfective  & ipfv     & murri  & łukši  & kučti  & quspi  \tnl
%        Stative       & stat     & muirre & łuikše & kuičte & quispe \tnl
%        Durative      & dur;ipfv & murrú  & łukšú  & kučtú  & quspú  \tnl
%        Frequentative & freq     & murro  & łukšo  & kučto  & quspo  \tnl
%        Habitual      & hab      & murra  & łukša  & kučta  & quspa  \tnl
%        \midrule
%        Perfective    & pfv      & miorra & łiokša & kiočta & qiospa \tnl
%        Inchoative    & inch     & miurro & łiukšo & kiučto & qiuspo \tnl
%        Cessative     & cess     & mírra  & łíkša  & kíčta  & qíspa  \tnl
%        Durative      & dur;pfv  & miarru & łiakšu & kiačtu & qiaspu \tnl
%        Momentane     & momt     & miurra & łiukša & kiučta & qiuspa \tnl
%        \bottomrule
%      \end{tabular}
%      \caption{Conjugation of the evidential verbs \label{tab:vm_evidentiality_conjugation}}
%  \end{table}
%
%  As with all auxiliary constructions, use of the evidential auxiliaries is not mandatory; rather, they are used to provide additional information. 
%
%  \subsubsection{Witness}
%  \label{sssec:vm_evd_witness}
%
%  The witness degree of evidentiality is denoted by the verb \qevesa{murru}, meaning ‘to see’.  It is used when the speaker was a witness to the event.
%
%  \subsubsection{Reportative}
%  \label{sssec:vm_evd_reportative}
%
%  The reportative degree of evidentiality is denoted by the verb \qevesa{łukšu}, which has the same consonantal root as the verb \qevesa{łukuš} ‘to hear (speech)’.
%
%  \subsubsection{Inferential}
%  \label{sssec:vm_evd_inferential}
%
%  The inferential degree of evidentiality is denoted by the verb \qevesa{kučtu}.  It is used when the speaker infers that the event occurred but was not a witness.
%
%  \subsubsection{Assumptive}
%  \label{sssec:vm_evd_assumption}
%
%  The assumption degree of evidentiality is denoted by the verb \qevesa{quspu}.  It is used when the speaker is making an assumption about the occurrence of the event.
%

  \section{Irregular Verbs}
  \label{sec:vm_irregular}

  %Qevesa verbal morphology is highly regular, with most irregularities occurring due to consonant groupings.  %Roots that contain a /h/ frequently possess irregular forms, mainly because the /h/ will be elided or reduced to a pre-aspiration of the following consonant and the previous vowel lengthened.  This may be represented in writing as well as speech.
  %However, a number of common roots do possess irregular forms, and these are outlined in the following sections.
  
  \ToBeWritten

%  \subsection{The Copulae}
%  \label{ssec:vm_copulae}
%
%  The most frequently-used irregular verb in Qevesa is the copula \qevesa{teši}.  It is one of a number of verbs which do not possess a regular infinitive of the form \qevesa{C\sub1{u}C\sub2{u}}; it also possesses a negative form (\qevesa{zemi}\footnotemark{}), unlike most other verbs.  The basic conjugated forms of \qevesa{teši} are given in Table~\ref{tab:vm_copulae_aspectual_conjugation}.
%  \footnotetext{This is also the same consonantal root as the negative verb \qevesa{zumu} and associated forms which translate as ‘zero’ or ‘none’.}
%
%  \begin{table}[htpb]\small\capstart
%      \begin{tabular}{>{\bfseries}fc->{\scshape}c -c -c}
%        \toprule
%        \SetRowStyle{\bfseries} & & Non-negative & Negative \tnl
%        \cline{3-4}
%        \SetRowStyle{\scshape} & & cop & neg \tnl
%        \midrule
%        Infinitive    & inf      & teši   & zemi   \tnl
%        \midrule
%        Imperfective  & ipfv     & tušši  & zummi  \tnl
%        Stative       & stat     & tuišše & zuimme \tnl
%        Durative      & dur;ipfv & tuššú  & zummú  \tnl
%        Frequentative & freq     & tuššo  & zummo  \tnl
%        Habitual      & hab      & tušša  & zumma  \tnl
%        \midrule
%        Perfective    & pfv      & tiošša & ziomma \tnl
%        Inchoative    & inch     & tiuššo & ziummo \tnl
%        Cessative     & cess     & tíšša  & zímma  \tnl
%        Durative      & dur;pfv  & tiaššu & ziammu \tnl
%        Momentane     & momt     & tiušša & ziumma \tnl
%        \bottomrule
%      \end{tabular}
%      \caption{Aspectual conjugation of the copulae \qevesa{teši} and \qevesa{zemi}\label{tab:vm_copulae_aspectual_conjugation}}
%  \end{table}
%
%
%  The copulae can also be used in an existential sense, but only in nominal phrases and never with stative verbs.  They play a major role in honorific registers, as described in Chapter~\ref{ch:registers}.

\end{document}
