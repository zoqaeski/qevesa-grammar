\documentclass[grammar]{subfiles}
\begin{document}
\chapter{Verbal Morphology}
\label{ch:verbal_morphology}


\section{Features}
\label{sec:vm_features}

Qevesa verbs are traditionally described in terms of a \emph{triliteral
  root system}, in which verb roots consist of an abstract pattern of three
consonants (e.g. \Q{R-K-T} “write”), with actual verb forms created by
inserting various vowel patterns between these consonants and adding various
prefixes and suffixes.  This discontinuous system is used to form not only
conjugated verbs, but also nominal and adjectival derivations, to the extent
that the majority of the vocabulary consists of such constructions. 

However, this is a very simplified interpretation.  The Proto-Teranean language
had a number of different types of verb roots, some of which contained inherent
vowels.  These various types of root were preserved in the modern Teranean
languages to varying degrees, with some becoming prevalent and others gradually
disappearing.  The eastern Teranean languages, which includes Qevesa, developed
a triliteral system as described above, but all the languages retain traces of
each of these subclasses of root in some form or another. 

Qevesa possesses four types of Proto-Teranean roots. 

% In Alashian, four different types of Proto-Semitic roots are clearly present.
% Therefore instead of presenting roots as abstract groups of three consonants,
% in this grammar verb roots will be presented in a form representing their
% unique structures, so in place of *K-T-B the root “write” will be given as
% *ktāb, reflecting the fact that the root does actually have an internal
% structure beyond the consonants themselves. This is not, however, to deny
% that Alashian verbal morphology is discontinuous; vowel patterns such as
% *C1aC2aC3 are useful in Alashian as well as in other Semitic languages, but
% the inner workings of the verbal system should not be simplified to a pure
% literal root + vowel template system.

The first and most common type of verb root is the true \emph{triliteral root},
which consists of three consonants and an inherent vowel between C₁ and C₂.
This vowel may be either /e/ or /o/, with a strong tendency for /e/ to occur in
roots with a stative meaning, and /o/ in all others. The citation form of these
roots is \Q{C₁VC₂uC₃}.  Throughout this text, the V listed in transfix patterns
will represent the inherent root vowel.

The second most frequent type is the \emph{biliteral root}, which consists of two
consonants and an inherent vowel in between them, which is typically /aː/ or
/eː/, but may be any long vowel. There are a large number of apparently
biliteral roots that exist solely due to sound changes in which a consonant
elided in most positions.  Other biliteral roots are often augmented with
another consonant either before or between the two consonants, and it's
believed that the triliteral system evolved from biliteral origins. 

The third type is the \emph{quadriliteral root}, which consists of four consonants
with no inherent vowel.  The majority of these are reduplicated, with the form
*C₁C₂C₁C₂, and are often ononmatopoaeic.  Those quadriliteral roots with four
different consonants are almost always derived roots of foreign origin, or
extended roots formed by treating a set of four consonants as an independent
root.  The citation form of quadriliteral roots is \Q{C₁aC₂C₃eC₄}.

The final and rarest type of root is the \emph{geminate root}, which consists of two
consonants, the second of which is geminated, and an inherent vowel /a/. These
roots conjugate triliterally in some forms and biliterally in others.
As with the biliteral roots, there are some irregular triliteral roots
which appear to be geminates due to sound changes; these are distinguished by
their inherent vowel.  The citation form of geminate roots is \Q{C₁yC₂C₂}.

% The maximum possible number of verb conjugations derivable from a root—not
% counting participles, verbal nouns and nominalisations—is over eight thousand:
% ten root forms, seven aspects, six moods, and twenty-nine personal suffixes.
% There are also a number of non-mandatory suffixes that can be appended to these that
% convey additional grammatical information.

% \section{Inflectional Categories and Conjugation}
% \label{sec:vm_conjugation}
% 
% The system of verb conjugations in Qevesa is quite complicated, and is formed
% along two axes.  One axis, known as the \emph{scale}, is used to specify
% grammatical concepts such as causative, intensive, reciprocal, passive or
% reflexive, and involves varying the stem. Historically, verb patterns were
% derived from prefixes and infixes, but sound change and analogy have resulted
% in a regular system that is highly productive.  The other axis consists of the various suffixes that are appended to the stem, 
% 
% Qevesa is a highly synthetic language, and verbs are conjugated to indicate
% aspect, mood, and personal agreement (trigger).  The conjugated form of the

% ā ē ī ō ū ȳ
% á é í ó ú ý


\section{The Verbal Patterns}
\label{sec:vm_patterns}

Qevesa has a set of eight \emph{verbal patterns}, also known as constructions
(\Q{mimdotes}\footnotemark). These patterns are sets of verbal conjugations
with an associated grammatical function.  Each pattern contains a full set of
paradigms designating the various aspects; a root conjugated the patterns has
its meaning crossed with the pattern's grammatical function.  Not all roots can
be conjugated into all patterns, and some patterns are prone to semantic drift.
The nine patterns are numbered from I–VIII and are listed in
\cref{tab:vm_root_patterns}. 

\footnotetext{From \Q{modut} “build, construct”}

\begin{table}[h!]\small\capstart
  \begin{tabular}{BFc -c}
    \toprule
    \SetRowStyle{\bfseries} Pattern & Description \\
    \midrule
    I    & Base \\
    II   & Intensive \\
    III  & Causative \\
    IV   & Reflexive \\
    V    & Reciprocal \\
    VI   & Causative Reflexive \\
    VII  & Passive Reflexive \\
    VIII & Stative \\
    \bottomrule
  \end{tabular}
  \caption{Verb root patterns\label{tab:vm_root_patterns}}
\end{table}

% I    rokut     | base
% II   rokkut    | intensive, transitive
% III  suroktu   | causative
% IV   narkotu   | reflexive, passive
% V    ratoktu   | reciprocal
% VI   istorkut  | causative reflexive
% VII  mitorkut  | passive reflexive
% VIII irkotu    | stative

Each pattern will be described in full in the following sections.  Within each
pattern is a conjugational paradigm that allows the verb to conjugate for
aspect and mood; personal suffixes are appended to these stems.


\subsection{Conjugation Stems}
\label{sssec:vm_conjugation}

There are six aspects formed by using a root and vowel template, divided into
three perfective aspects (\emph{perfective}, \emph{experiential}, and
\emph{momentane}) and three imperfective aspects (\emph{progressive},
\emph{durative}, and \emph{habitual}).  Each aspect has an indicative stem,
used to mark the indicative mood, and a modal stem to which modal suffixes are
appended.  If both the indicative and modal stems are the same, as occurs for
some patterns and conjugations, only the infinitive stem is listed in the
table. 

Each verbal pattern also has up to three other non-finite stems: the
\emph{infinitive}, an \emph{active participle}, and a \emph{passive participle}.  


\subsection{Defective Triliteral Roots}
\label{ssec:vm_defective_roots}

Within the set of triliteral roots there are a number of subtypes caused by
the presence of certain consonants.  These are predictable from the root, but
significantly affect the vowel templates the root uses to conjugate, and in
some cases cause consonants to alternate between methods of articulation.
Although irregular, these \emph{defective roots} are almost entirely due to
historical sound changes. 


\subsubsection{Aspirate Roots}
\label{sssec:vm_aspirate_roots}

Aspirate roots, or H-roots, are those roots which have /h/ in one or more positions, which
results the following sound changes:

\begin{itemize}
  \item A syllable-final /h/ induces lengthening of the previous vowel.  Suffixes
    that follow are usually vowel-final.
  \item A /h/ following an unvoiced plosive caused it to become a geminate
    aspirated plosive, which are pronounced in Modern Qevesa as fricatives.
  \item Roots that have /h/ in more than one position follow the rules of both
    positions.  These are exceedlingly rare.
\end{itemize}


\subsubsection{Soft roots}
\label{sssec:vm_soft_roots}

Soft roots, or J-roots are also quite irregular in their conjugations. They are
characterised by having had /ɟ/ in one or more positions, and induced the
following changes to the conjugated forms: 

\begin{itemize}
  \item a syllable-initial /ɟ/ becomes /j/; 
  \item a syllable-final /ɟ/ tends to become /ʒ/ before stops, affricates and
    nasals, and /j/ before fricatives and liquids; and
  \item a geminate /ɟ/ becomes /iʒ/. 
\end{itemize}

These sound changes create a number of homonymic conjugated stems. 


\section{Pattern I}
\label{sec:vm_verb_pattern_i}

Pattern I is the most common literal root form, containing no preformative
affixes.  It is typically the closest indicator to the lexical meaning of the
root, and has no particular semantic function associated with it, so it
includes a wide variety of verbs, including transitive, intransitive, stative
and inchoative


\subsection{Triliteral Roots}
\label{ssec:vm_i_triliteral}
%
%
%\subsubsection{Aspect}
%\label{sssec:vm_i_triliteral_aspect}

The perfective indicative is the citation form of the Pattern I verb, and uses
a stem of the form *\Q{C₁VC₂uC₃}.  The experiential aspect uses the pattern
*\Q{C₁VC₂aC₃}, and the momentane aspect uses the pattern *\Q{C₁VC₂iC₃}.  The
modal stems take the form *\Q{C₁V₁C₂C₃V₂}, where V₁ and V₂ are the first and
second vowels of the indicative stems. 

The imperfective aspects (progressive, durative and  habitual) use the stem
\Q{jaC₁C₂VːC₃}, where ‘V’ is \Q{-ú-} for the progressive aspect,
\Q{-á-} for the durative, and \Q{-í-} for the habitual.  The modal stem
appends an \Q{-e} to the indicative stem. 

In general, regardless of the root pattern, perfective aspects will always
contain the inherent vowel as the first vowel, and imperfective aspects are
always prefixed with \Q{ja-}.

Example triliteral conjugations are given in \cref{tab:vm_i_triliteral_aspect_stems}. 

\begin{table}[h!]\small\capstart
    \begin{tabular}{BFl Sl -c -c -c -c -c}
      \toprule
      & & \multicolumn{2}{c}{\Q{rokut} “write”} && \multicolumn{2}{c}{\Q{vesuk} “lay down”} \\
      \SetRowStyle{\bfseries} Aspect & & Indicative & Modal & & Indicative & Modal \\
      \midrule
      Perfective   & \acs{perf} & rokut  & roktu   &  & vesuk  & vesku \\
      Experiential & \acs{exp}  & rokat  & rokta   &  & vesak  & veska \\
      Momentane    & \acs{momt} & rokit  & rokti   &  & vesik  & veski \\
      Progressive  & \acs{prog} & jarkút & jarkúte &  & javsúk & javsúke \\
      Durative     & \acs{dur}  & jarkát & jarkáte &  & javsák & javsáke \\
      Habitual     & \acs{hab}  & jarkít & jarkíte &  & javsík & javsíke \\
      \bottomrule
    \end{tabular}
  \caption{Pattern I triliteral aspectual stems\label{tab:vm_i_triliteral_aspect_stems}}
\end{table}
%
%
%\subsubsection{Non-finite stems}
%\label{sssec:vm_i_triliteral_nonfinite}

The non-finite stems are the infinitive and the active and passive participles.
The infinitive is formed with the pattern \Q{C₁uC₂eC₃e}; the active participle
with the pattern \Q{eC₁áC₂iC₃}; and the passive participle with the pattern
\Q{šeC₁C₂ýC₃}.  

\cref{tab:vm_i_triliteral_nonfinite_stems} lists the
non-finite stems of \Q{rokut} “write”.

\begin{table}[h!]\small\capstart
  \begin{tabular}{BFl -c -c -c}
    \toprule
    \SetRowStyle{\bfseries} & Infinitive & Active Participle & Passive Participle \\
    \midrule
    Stem \SetRowStyle{\itshape} & rukete & erákit  & šerkýt \\
    Meaning                     & write & writing & written \\
    \bottomrule
  \end{tabular}
  \caption{Pattern I triliteral non-finite stems \label{tab:vm_i_triliteral_nonfinite_stems}}
\end{table}


\subsection{Biliteral Roots}
\label{ssec:vm_i_biliteral}

Biliteral roots lack distinct modal stems.  The perfective indicative is formed
by the pattern *\Q{C₁VːC₂u}, and the experiential and momentane aspects replace
the final \Q{-u} with \Q{-a} or \Q{-i}.

The imperfective aspects prefix \Q{ja-} and switch the final two vowels; that
is, they take the form *\Q{jaC₁V₂ːC₂V₁}, where V₁ is the short inherent vowel and
V₂ one of \Q{-ú-} (progressive), \Q{-á-} (durative), or \Q{-í} (habitual).

The infinitive is marked by the suffix \Q{-e}, the active participle by the
pattern *\Q{eC₁áC₂i}, and the passive participle by the prefix \Q{še-}.  

\cref{tab:vm_i_biliteral_stems} lists some example biliteral conjugations. 

\begin{table}[h!]\small\capstart
  \centering
  \subfloat[Aspect stems]{%
    \begin{tabular}{BFl Sl -c -c -c}
      \toprule
      & & \Q{máru} “see” & & \Q{šélu} “love” \\
      \SetRowStyle{\bfseries} Aspect & & Stem & & Stem \\
      \midrule
      Perfective   & \acs{perf} & máru   &  & šélu  \\
      Experiential & \acs{exp}  & mára   &  & šéla  \\
      Momentane    & \acs{momt} & mári   &  & šéli  \\
      Progressive  & \acs{prog} & jamúra &  & jašúle \\
      Durative     & \acs{dur}  & jamára &  & jašále \\
      Habitual     & \acs{hab}  & jamíra &  & jašíle \\
      \bottomrule
    \end{tabular}
  }\\
  \subfloat[Non-finite stems]{%
    \begin{tabular}{BFl -c -c -c}
      \toprule
      \SetRowStyle{\bfseries} & Infinitive & Active Participle & Passive Participle \\
      \midrule
      Stem \SetRowStyle{\itshape} & téke & etáki & šeték \\
      Meaning                     & go   & going & gone \\
      \bottomrule
    \end{tabular}
  }
  \caption{Pattern I biliteral stems \label{tab:vm_i_biliteral_stems}}
\end{table}


\subsection{Geminate roots}
\label{ssec:vm_i_geminate_roots}

Geminate roots behave like biliteral roots in Pattern I, with the geminate
consonants remaining together in the perfective stems and being split in the
perfective stems.  They lack distinct modal stems. 

The perfective indicative is formed by the pattern *\Q{C₁yC₂C₂u}, and the
experiential and momentane aspects replace the final \Q{-u} with \Q{-a} or
\Q{-i}.

The imperfective aspects prefix \Q{ja-} and switch the final two vowels; that
is, they take the form *\Q{jaC₁C₂VːC₂y}, where V is one of \Q{-ú-}
(progressive), \Q{-á-} (durative), or \Q{-í} (habitual).


The non-finite stems of geminate roots in Pattern I are formed by splitting the
geminate consonant and treating them as two single consonants.  They use the
same patterns as triliteral roots:  *\Q{C₁uC₂eC₂e} (infinitive), *\Q{eC₁áC₂iC₂}
(active participle) and *\Q{šeC₁C₂ýC₂} (passive participle). 

Example conjugations of geminate roots are given in
\cref{tab:vm_i_geminate_stems}.


\begin{table}[h!]\small\capstart
  \centering
  \subfloat[Aspectual stems]{%
    \begin{tabular}{BFl Sl -c -c -c}
      \toprule
      & & \Q{vass} “flow” & & \Q{tamm} “finish” \\
      \SetRowStyle{\bfseries} Aspect & & Stem & & Stem \\
      \midrule
      Perfective   & \acs{perf} & vyssu   &  & tymmu  \\
      Experiential & \acs{exp}  & vyssa   &  & tymma  \\
      Momentane    & \acs{momt} & vyssi   &  & tymmi  \\
      Progressive  & \acs{prog} & javsúsy &  & jatmúmy \\
      Durative     & \acs{dur}  & javsásy &  & jatmámy \\
      Habitual     & \acs{hab}  & javsísy &  & jatmímy \\
      \bottomrule
    \end{tabular}
  }\\
  \subfloat[Non-finite stems]{%
    \begin{tabular}{BFl -c -c -c}
      \toprule
      \SetRowStyle{\bfseries} & Infinitive & Active Participle & Passive Participle \\
      \midrule
      Stem \SetRowStyle{\itshape} & vusese & evásis & ševsýs \\
      Meaning                     & flow  & flowing & flowed \\
      \bottomrule
    \end{tabular}
  }
  \caption{Pattern I geminate stems \label{tab:vm_i_geminate_stems}}
\end{table}


\subsection{Defective Roots}
\label{ssec:vm_i_defective}

Defective roots generally follow the patterns outlined above, taking into
account the phonological changes listed in \cref{ssec:vm_defective_roots}.
Despite being irregular by nature, a lot of the irregularities of defective
roots are in fact fairly regular and predictable. 


\subsubsection{Aspirate Roots}
\label{sssec:vm_i_aspirate}

Aspirate roots (those with *\Q{H} as a root consonant) have fairly predictable
irregularities.  First-aspirate roots begin with \Q{á-} in the imperfective
aspects, and the second vowel is short.  Second-aspirate roots behave mostly like
regular triliteral roots, though the modal perfective stems have the pattern
\Q{C₁VːC₃} to which the aspect suffixes \Q{-u}, \Q{-a} or \Q{-i} are appended.
Third-aspirate roots always lengthen the vowel that would otherwise precede C₃.  

The non-finite stems are also mostly predictable: syllable-final /h/ lengthens the
preceding vowel; /h/ following a plosive causes it to assimilate to the
corresponding geminate fricative; /h/ following any other consonant causes it
to geminate. 

Examples of aspirate root conjugations are listed in
\cref{tab:vm_i_aspirate_stems}.  They can be distinguished from biliteral roots
by the form of the imperfective aspects. 

\begin{table}[h!]\small\capstart
  \centering
  \subfloat[Aspect stems]{%
    \begin{tabular}{BFl Sl -c -c -c -c -c -c -c -c}
      \toprule
      & & \multicolumn{2}{c}{\Q{hevur} “be good”} && \multicolumn{2}{c}{\Q{pohut} “speak”} & & \multicolumn{2}{c}{\Q{žoruh} “tie, bind”} \\
      \SetRowStyle{\bfseries} Aspect & & Indicative & Modal & & Indicative & Modal & & Indicative & Modal \\
      \midrule
      Perfective   & \acs{perf} & hevur & hevru  &  & pohut   & pótu     &  & žorú  & žorru \\
      Experiential & \acs{exp}  & hevar & hevra  &  & pohat   & póta     &  & žorá  & žorra \\
      Momentane    & \acs{momt} & hevir & hevri  &  & pohit   & póti     &  & žorí  & žorri \\
      Progressive  & \acs{prog} & jávur & jávure &  & japphút & japphúte &  & jažrú & jažrúhe \\
      Durative     & \acs{dur}  & jávar & jávare &  & japphát & jappháte &  & jažrá & jažráhe \\
      Habitual     & \acs{hab}  & jávir & jávire &  & japphít & japphíte &  & jažrí & jažríhe \\
      \bottomrule
    \end{tabular}
  }\\
  \subfloat[Non-finite stems]{%
  \begin{tabular}{BFl -c -c -c}
    \toprule
    \SetRowStyle{\bfseries} & Infinitive & Active Participle & Passive Participle \\
    \midrule
    Stem \SetRowStyle{\itshape} & humese & ehámis   & šémys \\
    Meaning                     & send   & sending  & sent \\
    \midrule
    Stem \SetRowStyle{\itshape} & puhete & epáhit   & šepphýt \\
    Meaning                     & speak  & speaking & spoken \\
    \midrule
    Stem \SetRowStyle{\itshape} & žuré   & ežárí    & šežrý \\
    Meaning                     & bind   & binding  & bound \\
    \bottomrule
  \end{tabular}
  }
  \caption{Pattern I aspirate defective roots\label{tab:vm_i_aspirate_stems}}
\end{table}


\subsubsection{Soft Roots}
\label{sssec:vm_i_soft}

Soft roots (those with *\Q{J} as a root consonant) are fairly regular.  All
occurrences of *\Q{-j-} before a consonant become \Q{-ž-} if the consonant is a
stop or nasal, and \Q{-i-} if the consonant is a fricative or liquid.  All
occurrences of *\Q{-ji-} and *\Q{-ij-} become \Q{-í-} except if they are
preceded or followed by a different vowel, and word-final *\Q{-Vj} becomes the
rising diphthongs \Q{-Vi}. 

Examples of soft root conjugations are listed in \cref{tab:vm_i_soft_stems}.
Note that the verb \Q{jotuh} is also a third-aspirate root, which makes it
doubly defective.  There are only a very small number of such verbs.  

\begin{table}[h!]\small\capstart
  \centering
  \subfloat[Aspect stems]{%
    \begin{tabular}{BFl Sl -c -c -c -c -c -c -c -c}
      \toprule
      & & \multicolumn{2}{c}{\Q{jotuh} “know”} && \multicolumn{2}{c}{\Q{kojur} “read”} & & \multicolumn{2}{c}{\Q{voluj} “rise (sun, moon)”} \\
      \SetRowStyle{\bfseries} Aspect & & Indicative & Modal & & Indicative & Modal & & Indicative & Modal \\
      \midrule
      Perfective   & \acs{perf} & jotú  & jotthu  &  & kojur  & koiru   &  & voluj  & volju \\
      Experiential & \acs{exp}  & jotá  & jottha  &  & kojar  & koira   &  & volaj  & volja \\
      Momentane    & \acs{momt} & jotí  & jotthi  &  & kojir  & koiri   &  & volí   & volí \\
      Progressive  & \acs{prog} & jažtú & jažtúhe &  & jakjúr & jakjúre &  & javlúj & javlúje \\
      Durative     & \acs{dur}  & jažtá & jažtáhe &  & jakjár & jakjáre &  & javláj & javláje \\
      Habitual     & \acs{hab}  & jažtí & jažtíhe &  & jakír  & jakíre  &  & javlí  & javlíje \\
      \bottomrule
    \end{tabular}
  }\\
  \subfloat[Non-finite stems]{%
  \begin{tabular}{BFl -c -c -c}
    \toprule
    \SetRowStyle{\bfseries} & Infinitive & Active Participle & Passive Participle \\
    \midrule
    Stem \SetRowStyle{\itshape} & juté  & ejátí   & šežtý \\
    Meaning                     & know  & knowing & known \\
    \midrule
    Stem \SetRowStyle{\itshape} & kujere & ekájir  & šekjýr \\
    Meaning                     & read  & reading & read \\
    \midrule
    Stem \SetRowStyle{\itshape} & vuleje & eválí   & ševlýj \\
    Meaning                     & rise  & rising  & raised \\
    \bottomrule
  \end{tabular}
  }
  \caption{Pattern I soft defective stems\label{tab:vm_i_soft_stems}}
\end{table}


\section{Pattern II: Intensive}
\label{sec:vm_pattern_ii}

Pattern II is commonly known as the \emph{intensive} or \emph{transitive} stem.
It is primarily used to mean a stronger or iterative form of the action, or to
form transitive verbs from intransitive and stative roots.  Adjectival roots
typically ascribe a causative meaning to Pattern II verbs.  This pattern is
also the base form of quadriliteral roots. 

%%   Pattern II roots include:
%%   - CuC:aC
%%   - puhhat “shout”
%%   - murrá (*murrah) “watch”
%%   - thuvvar “smash”
%

Adjectival roots often use Pattern II to form superlative roots; some examples
include \Q{hevvur} “best” (from \Q{hevur} “good”) and \Q{velluš} “tallest” (from
\Q{veluš} “tall”).  


\subsection{Triliteral Roots}
\label{ssec:vm_ii_triliteral}

The perfective indicative uses a stem of the form *\Q{C₁VC₂C₂uC₃}, where ‘V’ is
the inherent vowel.  The experiential and momentane aspects replace the \Q{-u-}
with \Q{-a-} or \Q{-i-}.  The modal stem appends a \Q{-y} to the indicative
stem. 

The imperfective aspects use the pattern *\Q{jaC₁C₁V₂ːC₂C₃V₁}, where V₁ is the
inherent vowel and V₂ is \Q{-ú-}, \Q{-á-} or \Q{-í-} for the progressive,
durative and habitual aspects. The modal stems replace the final vowel with
\Q{-e}; those roots whose inherent vowel is \Q{-e-} do not have a distinct
modal stem.  

The non-finite stems are formed similarly to those for Pattern I verbs, albeit
with a geminated second consonant.  The infinitive is formed with the pattern
*\Q{C₁uC₂C₂eC₃e}; the active participle with the pattern *\Q{eC₁áC₂C₂iC₃}; and
the passive participle with the pattern *\Q{šeC₁iC₂C₂úC₃}.

Some examples of Pattern II conjugations are listed in \cref{tab:vm_ii_triliteral_stems}.

\begin{table}[h!]\small\capstart
  \centering
  \subfloat[Aspectual stems]{%
    \begin{tabular}{BFl Sl -c -c -c -c -c}
      \toprule
      & & \multicolumn{2}{c}{\Q{sovvut} “remind, exhort”} & & \multicolumn{2}{c}{\Q{leccum} “shrink, reduce, make small”} \\
      \SetRowStyle{\bfseries} Aspect & & Indicative stem & Modal stem & & Indicative stem & Modal stem \\
      \midrule
      Perfective   & \acs{perf} & sopput   & soppute  &  & leccum   & leccume \\
      Experiential & \acs{exp}  & soppat   & soppate  &  & leccam   & leccame \\
      Momentane    & \acs{momt} & soppit   & soppite  &  & leccim   & leccime \\
      Progressive  & \acs{prog} & jassúpto & jassúpte &  & jallúcme & jallúcme \\
      Durative     & \acs{dur}  & jassápto & jassápte &  & jallácme & jallácme \\
      Habitual     & \acs{hab}  & jassípto & jassípte &  & jallícme & jallícme \\
      \bottomrule
    \end{tabular}
  }\\
  \subfloat[Non-finite stems]{%
    \begin{tabular}{BFl -c -c -c}
      \toprule
      \SetRowStyle{\bfseries} & Infinitive & Active Participle & Passive Participle \\
      \midrule
      Stem \SetRowStyle{\itshape} & suppete & esáppit   & šesippýt \\
      Meaning                     & remind & reminding & reminded \\
      \bottomrule
    \end{tabular}
  }
  \caption{Pattern II triliteral stems \label{tab:vm_ii_triliteral_stems}}
\end{table}


\subsection{Biliteral Roots}
\label{ssec:vm_ii_biliteral}

Biliteral roots in Pattern II, like Pattern I, also lack distinct modal stems.
The perfective indicative is formed by the pattern *\Q{C₁VːC₂C₂u}, and the
experiential and momentane aspects replace the final \Q{-u} with \Q{-a} or
\Q{-i}.

The imperfective aspects prefix \Q{ja-} and switch the final two vowels; that
is, they take the form *\Q{jaC₁V₂ːC₂C₂V₁}, where V₁ is the short inherent vowel
and V₂ one of \Q{-ú-} (progressive), \Q{-á-} (durative), or \Q{-í} (habitual).

The infinitive is marked by the pattern *\Q{C₁uC₂C₂éne}, the active participle
by the pattern *\Q{eC₁VC₂C₂í}, and the passive participle by the pattern
*\Q{šeC₁aC₂C₂ú}.  

\cref{tab:vm_ii_biliteral_stems} lists some example biliteral conjugations. 

\begin{table}[h!]\small\capstart
  \centering
  \subfloat[Aspectual stems]{%
    \begin{tabular}{BFl Sl -c -c -c}
      \toprule
      & & \Q{kérru} “request” & & \Q{máčču} “heat, make hot” \\
      \SetRowStyle{\bfseries} Aspect & & Stem & & Stem \\
      \midrule
      Perfective   & \acs{perf} & kérru   &  & máčču  \\
      Experiential & \acs{exp}  & kérra   &  & máčča  \\
      Momentane    & \acs{momt} & kérri   &  & máčči  \\
      Progressive  & \acs{prog} & jakúrre &  & jamúčča \\
      Durative     & \acs{dur}  & jakárre &  & jamáčča \\
      Habitual     & \acs{hab}  & jakírre &  & jamíčča \\
      \bottomrule
    \end{tabular}
  }\\
  \subfloat[Non-finite stems]{%
    \begin{tabular}{BFl -c -c -c}
      \toprule
      \SetRowStyle{\bfseries} & Infinitive & Active Participle & Passive Participle \\
      \midrule
      Stem \SetRowStyle{\itshape} & muččéne & emaččí & šemaččý \\
      Meaning                     & heat    & heating & heated \\
      \bottomrule
    \end{tabular}
  }
  \caption{Pattern II biliteral stems \label{tab:vm_ii_biliteral_stems}}
\end{table}


\subsection{Quadriliteral Roots}
\label{ssec:vm_ii_quadriliteral}

The base form of quadriliteral roots is Pattern II; they cannot conjugate into
Pattern I. 

The perfective indicative aspect takes the form *\Q{C₁aC₂C₃uC₄}. The
experiential and momentane aspects replace the \Q{-u-} with \Q{-a-} or \Q{-i-}.
The modal stem appends a \Q{-y} to the indicative stem. 

The imperfective aspects use the pattern *\Q{jaC₁eC₂C₃VːC₄y}, where V is
\Q{-ú-}, \Q{-á-} or \Q{-í-} for the progressive, durative and habitual aspects.
The modal stems replace the final \Q{-u} with \Q{-e}.  

The non-finite stems are formed similarly to those for triliteral roots, with
the geminated second consonant replaced with C₂C₃.  The infinitive is formed
with the pattern *\Q{C₁uC₂C₃eC₄e}; the active participle with the pattern
*\Q{eC₁áC₂C₃iC₄}; and the passive participle with the pattern
*\Q{šeC₁iC₂C₃úC₄}.

An example conjugation using the verb \Q{zanzen} “annoy” is given in
\cref{tab:vm_ii_quadriliteral_aspect_stems}. 

\begin{table}[h!]\small\capstart
  \centering
  \subfloat[Aspectual stems]{%
    \begin{tabular}{BFl Sl -c -c}
      \toprule
      & & \multicolumn{2}{c}{\Q{zanzen} “annoy”} \\
      \SetRowStyle{\bfseries} Aspect & & Indicative stem & Modal stem  \\
      \midrule
      Perfective   & \acs{perf} & zanzun    & zanzune \\
      Experiential & \acs{exp}  & zanzan    & zanzane \\
      Momentane    & \acs{momt} & zanzin    & zanzine \\
      Progressive  & \acs{prog} & jazenzúny & jazenzúne  \\
      Durative     & \acs{dur}  & jazenzány & jazenzáne  \\
      Habitual     & \acs{hab}  & jazenzíny & jazenzíne  \\
      \bottomrule
    \end{tabular}
  }\\
  \subfloat[Non-finite stems]{%
    \begin{tabular}{BFl -c -c -c}
      \toprule
      \SetRowStyle{\bfseries} & Infinitive & Active Participle & Passive Participle \\
      \midrule
      Stem \SetRowStyle{\itshape} & zunzene & ezánzin  & šezinzýn \\
      Meaning                     & annoy  & annoying & annoyed \\
      \bottomrule
    \end{tabular}
  }
  \caption{Pattern II quadriliteral stems\label{tab:vm_ii_quadriliteral_aspect_stems}}
\end{table}


\subsection{Geminate roots}
\label{ssec:vm_ii_geminate_roots}

Geminate roots conjugate as triliteral roots in Pattern II, with the geminate
consonant being split into two single consonants. Example conjugations of
geminate roots are given in \cref{tab:vm_ii_geminate_stems}. 

\begin{table}[h!]\small\capstart
  \centering
  \subfloat[Aspectual stems]{%
    \begin{tabular}{BFl Sl -c -c}
      \toprule
      & & \multicolumn{2}{c}{\Q{vassús} “flood”} \\
      \SetRowStyle{\bfseries} Aspect & & Indicative & Modal \\
      \midrule
      Perfective   & \acs{perf} & vyssus   & vyssuse  \\
      Experiential & \acs{exp}  & vyssas   & vyssase  \\
      Momentane    & \acs{momt} & vyssis   & vyssise  \\
      Progressive  & \acs{prog} & javsússy & javsússe \\
      Durative     & \acs{dur}  & javsássy & javsásse \\
      Habitual     & \acs{hab}  & javsíssy & javsísse \\
      \bottomrule
    \end{tabular}
  }\\
  \subfloat[Non-finite stems]{%
    \begin{tabular}{BFl -c -c -c}
      \toprule
      \SetRowStyle{\bfseries} & Infinitive & Active Participle & Passive Participle \\
      \midrule
      Stem \SetRowStyle{\itshape} & vusse & evássis   & ševissýs \\
      Meaning                     & flood  & flooding & flooded \\
      \bottomrule
    \end{tabular}
  }
  \caption{Pattern II geminate stems \label{tab:vm_ii_geminate_stems}}
\end{table}


\subsection{Defective Roots}
\label{ssec:vm_ii_defective}

Defective roots in Pattern II are fairly regular, with the only irregularities
being those introduced by the sound changes in \cref{ssec:vm_defective_roots}.
The most noticeable irregularities occur with third-defective roots, where
elision and vowel-lengthening alters patterns in a relatively predictable way.
Examples of defective conjugations are given in
\cref{tab:vm_ii_defective_stems}. 

\begin{table}[h!]\small\capstart
  \centering
  \subfloat[Aspect stems]{%
    \begin{tabular}{BFl Sl -c -c -c -c -c -c -c -c}
      \toprule
      & & \multicolumn{2}{c}{\Q{jonnur} “plunder”} && \multicolumn{2}{c}{\Q{volluj} “soar”} & & \multicolumn{2}{c}{\Q{žorrú} “fasten”} \\
      \SetRowStyle{\bfseries} Aspect & & Indicative & Modal & & Indicative & Modal & & Indicative & Modal \\
      \midrule
      Perfective   & \acs{perf} & jonnur   & jonnuru  &  & volluj   & volluju  &  & žorrú    & žorrú  \\
      Experiential & \acs{exp}  & jonnar   & jonnaru  &  & vollaj   & vollaju  &  & žorrá    & žorrá  \\
      Momentane    & \acs{momt} & jonnir   & jonniru  &  & vollí    & volliju  &  & žorrí    & žorrí  \\
      Progressive  & \acs{prog} & jaižúnro & jaižúnre &  & javvúljo & javvúlje &  & jažžúrro & jažžúrre \\
      Durative     & \acs{dur}  & jaižánro & jaižánre &  & javváljo & javválje &  & jažžárro & jažžárre \\
      Habitual     & \acs{hab}  & jaižínro & jaižínre &  & javvíljo & javvílje &  & jažžírro & jažžírre \\
      \bottomrule
    \end{tabular}
  }\\
  \subfloat[Non-finite stems]{%
  \begin{tabular}{BFl -c -c -c}
    \toprule
    \SetRowStyle{\bfseries} & Infinitive & Active Participle & Passive Participle \\
    \midrule
    Stem \SetRowStyle{\itshape} & junnese  & ejánnis    & šejinnys \\
    Meaning                     & plunder & plundering & plundered \\
    \midrule
    Stem \SetRowStyle{\itshape} & vullei  & evállí     & ševillý \\
    Meaning                     & soar    & soaring    & soared \\
    \midrule
    Stem \SetRowStyle{\itshape} & žurré   & ežárrí     & šežirrý \\
    Meaning                     & fasten  & fastening  & fastened \\
    \bottomrule
  \end{tabular}
  }
  \caption{Pattern II defective stems\label{tab:vm_ii_defective_stems}}
\end{table}


\clearpage
\section{Pattern III: Causative}
\label{sec:vm_pattern_iii}

Pattern III is commonly known as the \emph{causative} stem.  Its most common
function is causative; it may also convert transitive verbs into ditransitive
ones.  It can also have a causative meaning on verbs whose Pattern I root is
intransitive, and for some verbs, may convey an assistive or factitive meaning.
Roots in this pattern include \Q{sakopsu} “feed”, \Q{saroktu} “dictate”,
\Q{sadostu} “teach”, and \Q{sapesku} “fell sth (e.g. a tree)”.

The basic form of Pattern III verbs is prefixing \Q{sa-} onto the root
\Q{C₁VC₂C₃}, and as a result this pattern is also referred to as the
\emph{S-stem}. 

%%   Pattern III roots include:
%%   - šuCuCCa 
%%   - šukupša “feed”
%%   - šurukta “dictate”
%%   - šumurra “show”
%%   - šuthuvra “cause to break”


\subsection{Triliteral Roots}
\label{ssec:vm_iii_triliteral}

The perfective indicative uses a stem of the form *\Q{saC₁VC₂C₃u}. The
experiential and momentane aspects replace the final \Q{-u} with \Q{-a} or
\Q{-i}.  Pattern III verbs lack distinct modal stems in the perfective aspects.  

The imperfective aspects (progressive, durative and  habitual) use the stem
*\Q{jasaC₁C₂V₂ːC₃V₁}, where V₁ is the inherent vowel and V₂ is \Q{-ú-} for the
progressive aspect, \Q{-á-} for the durative, and \Q{-í-} for the habitual.
The modal stem replaces the final vowel with \Q{-e}. 

The infinitive is formed with the pattern *\Q{saC₁C₂uC₃e}; the active participle
with the pattern *\Q{esC₁áC₂iC₃}; and the passive participle with the pattern
*\Q{šesaC₁C₂úC₃}.

Example triliteral conjugations are given in \cref{tab:vm_iii_triliteral_stems}. 

\begin{table}[h!]\small\capstart
  \centering
  \subfloat[Aspectual stems]{%
    \begin{tabular}{BFl Sl -c -c -c -c -c}
      \toprule
      & & \multicolumn{2}{c}{\Q{sakospu} “feed”} && \multicolumn{2}{c}{\Q{sadostu} “teach”} \\
      \SetRowStyle{\bfseries} Aspect & & Indicative & Modal & & Indicative & Modal \\
      \midrule
      Perfective   & \acs{perf} & sakospu   & sakospu   &  & sadostu   & sadostu \\
      Experiential & \acs{exp}  & sakospa   & sakospa   &  & sadosta   & sadosta \\
      Momentane    & \acs{momt} & sakospi   & sakospi   &  & sadosti   & sadosti \\
      Progressive  & \acs{prog} & jasaksúpo & jasaksúpe &  & jasadsúto & jasadsúte \\
      Durative     & \acs{dur}  & jasaksápo & jasaksápe &  & jasadsáto & jasadsáte \\
      Habitual     & \acs{hab}  & jasaksípo & jasaksípe &  & jasadsíto & jasadsíte \\
      \bottomrule
    \end{tabular}
  }\\
  \subfloat[Non-finite stems]{%
    \begin{tabular}{BFl -c -c -c}
      \toprule
      \SetRowStyle{\bfseries} & Infinitive & Active Participle & Passive Participle \\
      \midrule
      Stem \SetRowStyle{\itshape} & saksupe & eskásip & šesaksýp \\
      Meaning                     & feed   & feeding & fed \\
      \bottomrule
    \end{tabular}
  }
  \caption{Pattern III triliteral stems \label{tab:vm_iii_triliteral_stems}}
\end{table}


\subsection{Biliteral Roots}
\label{ssec:vm_iii_biliteral}

Biliteral roots in Pattern III have similar conjugations to Pattern I, with the
addition of the prefix \Q{sa-} or the infix \Q{-s-} that is inserted immediately before
C₁.  The infix assimilates to the point of articulation of a following
fricative, effectively causing it to geminate. 

The infinitive is marked by the pattern *\Q{saC₁VːC₂e}, the active participle by the
pattern *\Q{esC₁VːC₂i}, and the passive participle by the prefix \Q{šes-}.  

Some examples are listed in \cref{tab:vm_iii_biliteral_stems}. 

\begin{table}[h!]\small\capstart
  \centering
  \subfloat[Aspectual stems]{%
    \begin{tabular}{BFl Sl -c}
      \toprule
      & & \Q{sutéku} “send”  \\
      \SetRowStyle{\bfseries} Aspect & & Stem  \\
      \midrule
      Perfective   & \acs{perf} & satéku  \\
      Experiential & \acs{exp}  & satéka  \\
      Momentane    & \acs{momt} & satéki  \\
      Progressive  & \acs{prog} & jastúke  \\
      Durative     & \acs{dur}  & jastáke  \\
      Habitual     & \acs{hab}  & jastíke  \\
      \bottomrule
    \end{tabular}
  }\\
  \subfloat[Non-finite stems]{%
    \begin{tabular}{BFl -c -c -c}
      \toprule
      \SetRowStyle{\bfseries} & Infinitive & Active Participle & Passive Participle \\
      \midrule
      Stem \SetRowStyle{\itshape} & satéke & estáki  & šesték \\
      Meaning                     & send   & sending & sent \\
      \bottomrule
    \end{tabular}
  }
  \caption{Pattern III biliteral stems \label{tab:vm_iii_biliteral_stems}}
\end{table}


\subsection{Quadriliteral roots}
\label{ssec:vm_iii_quadriliteral_roots}

Quadriliteral roots form Pattern III similarly to Pattern II. The prefix
\Q{sa-} or the infix \Q{-s-} is inserted immediately before C₁, the infix
assimilating to a geminate C₁ if that consonant is a fricative. 

The infinitive is marked by the pattern *\Q{saC₁C₂uC₃eC₄}, the active participle
by the pattern *\Q{esC₁VːC₂C₃iC₄}, and the passive participle by the pattern
*\Q{šesC₁iC₂C₃úC₄}.  


\subsection{Geminate roots}
\label{ssec:vm_iii_geminate_roots}

Geminate roots form Pattern III similarly to biliteral roots, with a geminate
second consonant.  The perfective aspects are formed with the pattern
*\Q{saC₁V₁C₂C₂V₂}, where V₁ is the inherent vowel and V₂ is one of \Q{-u-},
\Q{-a-} or \Q{-i-}.  

The imperfective aspects use the pattern *\Q{jasaC₁V₂ːC₂C₂V₂} in the
infinitive, replacing the final vowel with \Q{-e} to form the modal stem. 

The infinitive is formed with the pattern *\Q{saC₁uC₂C₂e}, the active participle
with *\Q{esaC₁áC₂C₂i}, and the participle with *\Q{šesC₁iC₂úC₂}.  

As with all Pattern III stems, the infix \Q{-s-} assimilates to an immediately
following fricative.  \cref{tab:vm_iii_geminate_stems} lists the Pattern III stems for
the verb \Q{savyssu} “melt”.

\begin{table}[h!]\small\capstart
  \centering
  \subfloat[Aspectual stems]{%
    \begin{tabular}{BFl Sl -c -c}
      \toprule
      & & \multicolumn{2}{c}{\Q{savyssu} “melt”} \\
      \SetRowStyle{\bfseries} Aspect & & Indicative & Modal \\
      \midrule
      Perfective   & \acs{perf} & savyssu   & savyssu  \\
      Experiential & \acs{exp}  & savyssa   & savyssa  \\
      Momentane    & \acs{momt} & savyssi   & savyssi  \\
      Progressive  & \acs{prog} & jasavússy & jasavússe \\
      Durative     & \acs{dur}  & jasavássy & jasavásse \\
      Habitual     & \acs{hab}  & jasavíssy & jasavísse \\
      \bottomrule
    \end{tabular}
  }\\
  \subfloat[Non-finite stems]{%
    \begin{tabular}{BFl -c -c -c}
      \toprule
      \SetRowStyle{\bfseries} & Infinitive & Active Participle & Passive Participle \\
      \midrule
      Stem \SetRowStyle{\itshape} & savusse & esavássi & ševvisýs \\
      Meaning                     & melt   & melting  & molten \\
      \bottomrule
    \end{tabular}
  }
  \caption{Pattern III geminate stems \label{tab:vm_iii_geminate_stems}}
\end{table}


\subsection{Defective Roots}
\label{ssec:vm_iii_defective_roots}

Defective roots in Pattern III follow the same phonological assimilation rules
as have previously described. 


\section{Pattern IV: Reflexive}
\label{sec:vm_pattern_iv}

Pattern IV is commonly known as the \emph{reflexive} stem, though this is
something of a misnomer as true reflexives only account for a portion of the
verbs in this pattern.  Verbs in Pattern IV are subject to a large amount of
semantic drift, and some roots lack base forms in Patterns I or II.  The main
functions of this pattern are: 

\begin{itemize}
  \item Forming reflexives from transitive roots: \Q{šomú} “shave” → \Q{našmohu}
    “shave oneself”
  \item Forming causative reflexives from stative roots: \Q{vorun} “wear” →
    \Q{navronu} “dress oneself (cause oneself to wear)”
  \item Forming so-called autoreflexive verbs that denote (often involuntary)
    actions performed on one’s body: \Q{náčoru} “sneeze”
  \item Forming verbs with unpredictable semantics: \Q{narkotu} “copy (sth)”,
    \Q{nakjoru} “read aloud, recite”, \Q{namáru} “look inwards, introspect”
\end{itemize}

Of the functions listed, the only fully productive class is the reflexives from
transitive roots.  The verbs with unpredictable semantics are generally
admitting of new forms, but the causative reflexives are mostly handled by
Pattern VI in modern Qevesa, and the autoreflexives are a closed class.


\subsection{Triliteral Roots}
\label{ssec:vm_iv_triliteral_roots}

Triliteral roots form the perfective aspects with the pattern *\Q{naC₁C₂V₁C₃V₂},
where V₁ is the inherent root vowel and V₂ is one of \Q{-u}, \Q{-a} or \Q{-i}
for the various subtypes.  

The imperfective aspects are formed with the pattern *\Q{janaC₁V₂ːC₂C₃V₁},
where V₁ is the inherent root vowel, and V₂ is \Q{-ú-} for the progressive
aspect, \Q{-á-} for the durative aspect, and \Q{-í-} for the habitual aspect.
Perfective aspects lack a distinct modal form in Pattern IV, but imperfective
aspects form it by replacing the final vowel with \Q{-e}. 

The infinitive is formed with the pattern *\Q{naC₁uC₂eC₃e}; the active participle
with the pattern *\Q{enC₁áC₂iC₃} and the passive participle with the pattern
*\Q{šenC₁iC₂C₃u}. 

Examples of triliteral stems in Pattern IV are given in \cref{tab:vm_iv_triliteral_stems}. 

\begin{table}[h!]\small\capstart
  \centering
  \subfloat[Aspectual stems]{%
    \begin{tabular}{BFl Sl -c -c -c -c -c}
      \toprule
      & & \multicolumn{2}{c}{\Q{narkotu} “copy (sth)”} && \multicolumn{2}{c}{\Q{navronu} “dress oneself”} \\
      \SetRowStyle{\bfseries} Aspect & & Indicative & Modal & & Indicative & Modal \\
      \midrule
      Perfective   & \acs{perf} & narkotu   & narkotu   &  & navronu   & navronu  \\
      Experiential & \acs{exp}  & narkota   & narkota   &  & navrona   & navrona  \\
      Momentane    & \acs{momt} & narkoti   & narkoti   &  & navroni   & navroni  \\
      Progressive  & \acs{prog} & janarúkto & janarúkte &  & janavúrno & janavúrne \\
      Durative     & \acs{dur}  & janarákto & janarákte &  & janavárno & janavárne \\
      Habitual     & \acs{hab}  & janaríkto & janaríkte &  & janavírno & janavírne \\
      \bottomrule
    \end{tabular}
  }\\
  \subfloat[Non-finite stems]{%
    \begin{tabular}{BFl -c -c -c}
      \toprule
      \SetRowStyle{\bfseries} & Infinitive & Active Participle & Passive Participle \\
      \midrule
      Stem \SetRowStyle{\itshape} & narukete & enrákit & šenrikty \\
      Meaning                     & copy    & copying & copied \\
      \bottomrule
    \end{tabular}
  }
  \caption{Pattern IV triliteral stems \label{tab:vm_iv_triliteral_stems}}
\end{table}


\subsection{Biliteral Roots}
\label{ssec:vm_iv_biliteral_roots}

Biliteral roots form the perfective aspects by prefixing the Pattern I stem
with \Q{na-}.  The imperfective stems are formed by inserting the prefix
\Q{-n-} immediately before C₁.  Like their Pattern I counterparts, biliteral
roots in this pattern also lack distinct modal stems. 

The infinitive is formed with the pattern *\Q{naC₁VːC₂e}; the active participle
with the pattern *\Q{enC₁áC₂i} and the passive participle with the pattern
*\Q{šenC₁VːC₂y}. 

Examples of biliteral stems are given in \cref{tab:vm_iv_biliteral_stems}. 

\begin{table}[h!]\small\capstart
  \centering
  \subfloat[Aspectual stems]{%
    \begin{tabular}{BFl Sl -c -c -c}
      \toprule
      & & \Q{namáru} “introspect” & & \Q{natévu} “sense, feel within” \\
      \SetRowStyle{\bfseries} Aspect & & Stem & & Stem \\
      \midrule
      Perfective   & \acs{perf} & namáru  &  & natévu \\
      Experiential & \acs{exp}  & namára  &  & natéva \\
      Momentane    & \acs{momt} & namári  &  & natévi \\
      Progressive  & \acs{prog} & janmúra &  & jantúve \\
      Durative     & \acs{dur}  & janmára &  & jantáve \\
      Habitual     & \acs{hab}  & janmíra &  & jantíve \\
      \bottomrule
    \end{tabular}
  }\\
  \subfloat[Non-finite stems]{%
    \begin{tabular}{BFl -c -c -c}
      \toprule
      \SetRowStyle{\bfseries} & Infinitive & Active Participle & Passive Participle \\
      \midrule
      Stem \SetRowStyle{\itshape} & namáre     & enmári        & šenmáry \\
      Meaning                     & introspect & introspecting & introspected \\
      \bottomrule
    \end{tabular}
  }
  \caption{Pattern IV biliteral stems \label{tab:vm_iv_biliteral_stems}}
\end{table}
    

\subsection{Quadriliteral roots}
\label{ssec:vm_iv_quadriliteral_roots}

Quadriliteral roots form Pattern IV similarly to Pattern II. The prefix
\Q{na-} or the infix \Q{-n-} is inserted immediately before C₁. 

The infinitive is marked by the pattern *\Q{naC₁uC₂C₃eC₄e}, the active
participle by the pattern *\Q{anC₁VːC₂C₃iC₄}, and the passive participle by the
pattern *\Q{šenC₁iC₂C₃úC₄}.  


\subsection{Geminate roots}
\label{ssec:vm_iv_geminate_roots}

Geminate roots form Pattern IV similarly to Pattern III, except for the
perfective indicative aspects which split the geminate consonant C₂ into two
single consonants.   The perfective indicative aspects are formed with the
pattern *\Q{naC₁V₁C₂V₂C₂}, where V₁ is the inherent vowel and V₂ is one of
\Q{-u-}, \Q{-a-} or \Q{-i-}, and the modal perfective aspects use the pattern
*\Q{naC₁V₁C₂C₂V₂}.  

The imperfective aspects use the pattern *\Q{janC₁V₂ːC₂C₂V₁} in the indicative,
replacing the final vowel with \Q{-e} to form the modal stem. 

The infinitive is formed with the pattern *\Q{naC₁C₂uC₂e}, the active participle
with *\Q{enC₁áC₂iC₂}, and the participle with *\Q{šenC₁iC₂úC₂}.  

% \cref{tab:vm_iii_geminate_stems} lists the Pattern IV stems for
% the verb \Q{navysús} “melt”.


\subsection{Defective Roots}
\label{ssec:vm_iv_defective_roots}

Defective roots in Pattern IV follow the same phonological assimilation rules
as have previously described. 


\section{Pattern V: Reciprocal}
\label{sec:vm_pattern_v}

Pattern V is the \emph{reciprocal} stem, whose primary purpose is to create verbs that
convey meanings of a reciprocal or reflexive nature.  It is often used to
create verbs denoting social interactions or accompaniment, or to form
transitive verbs from intransitive roots.  This pattern is also subject to some
semantic and metaphorical drift, though not as severe as in Pattern IV.

%   pohut “speak” → patótu “converse (with)”
%   rokut “write” → ratoktu “correspond (with)”
%   šopur “buy” → šatopru “buy (from)”
%   téku “go” → tatéku “go together, go with” (accompaniment)
%   kéru “ask” → katéru “ask for (sth)” (intransitive → transitive)

The general form of Pattern V verbs is inserting the infix \Q{-at-} immediately
after the first consonant.  


\subsection{Triliteral Roots}
\label{ssec:vm_v_triliteral_roots}

Triliteral roots form the perfective aspects with the pattern
*\Q{C₁atV₁C₂C₃V₂}, where V₁ is the inherent root vowel and V₂ is one of \Q{-u},
\Q{-a} or \Q{-i} for the various subtypes.  

The imperfective aspects are formed with the pattern *\Q{jaC₁atV₂ːC₂C₃a}, where
V₂ is the \Q{-ú-} for the progressive aspect, \Q{-á-} for the durative aspect,
and \Q{-í-} for the habitual aspect.  Perfective aspects lack a distinct modal
form in Pattern V, but imperfective aspects form it by replacing the final
\Q{-a} with \Q{-e}. 

The infinitive is formed with the pattern *\Q{C₁atuC₂eC₃e}; the active
participle with the pattern *\Q{aC₁átC₂iC₃} and the passive participle with the
pattern *\Q{šeC₁atiC₂C₃y}. 

Examples of triliteral stems in Pattern V are given in
\cref{tab:vm_v_triliteral_stems}. 

\begin{table}[h!]\small\capstart
  \centering
  \subfloat[Aspectual stems]{%
    \begin{tabular}{BFl Sl -c -c -c -c -c}
      \toprule
      & & \multicolumn{2}{c}{\Q{ratoktu} “correspond (with)”} && \multicolumn{2}{c}{\Q{šatopru} “buy (from)”} \\
      \SetRowStyle{\bfseries} Aspect & & Indicative & Modal & & Indicative & Modal \\
      \midrule
      Perfective   & \acs{perf} & ratoktu   & ratoktu   &  & šatopru   & šatopru   \\
      Experiential & \acs{exp}  & ratokta   & ratokta   &  & šatopra   & šatopra   \\
      Momentane    & \acs{momt} & ratokti   & ratokti   &  & šatopri   & šatopri   \\
      Progressive  & \acs{prog} & jaratúkta & jaratúkte &  & jašatúpra & jašatúpre  \\
      Durative     & \acs{dur}  & jaratákta & jaratákte &  & jašatápra & jašatápre  \\
      Habitual     & \acs{hab}  & jaratíkta & jaratíkte &  & jašatípra & jašatípre  \\
      \bottomrule
    \end{tabular}
  }\\
  \subfloat[Non-finite stems]{%
    \begin{tabular}{BFl -c -c -c}
      \toprule
      \SetRowStyle{\bfseries} & Infinitive & Active Participle & Passive Participle \\
      \midrule
      Stem \SetRowStyle{\itshape} & ratukete    & erátkit       & šeratikty \\
      Meaning                     & correspond & corresponding & corresponded \\
      \bottomrule
    \end{tabular}
  }
  \caption{Pattern V triliteral stems \label{tab:vm_v_triliteral_stems}}
\end{table}


\subsection{Biliteral Roots}
\label{ssec:vm_v_biliteral_roots}

Biliteral roots form the aspects by inserting the infix \Q{-at-} immediately
after C₁ on the Pattern I stem.  Like their Pattern I counterparts, biliteral
roots in this pattern also lack distinct modal stems. 

The infinitive is formed with the pattern *\Q{C₁atVːC₂e}; the active participle
with the pattern *\Q{eC₁táC₂i} and the passive participle with the pattern
*\Q{šeC₁atVːC₂y}. 

Examples of biliteral stems are given in \cref{tab:vm_v_biliteral_stems}. 

\begin{table}[h!]\small\capstart
  \centering
  \subfloat[Aspectual stems]{%
    \begin{tabular}{BFl Sl -c -c -c}
      \toprule
      & & \Q{tatéku} “go together (with)” & & \Q{katéru} “ask for (sth)” \\
      \SetRowStyle{\bfseries} Aspect & & Stem & & Stem \\
      \midrule
      Perfective   & \acs{perf} & tatéku   &  & katéru \\
      Experiential & \acs{exp}  & tatéka   &  & katéra \\
      Momentane    & \acs{momt} & tatéki   &  & katéri \\
      Progressive  & \acs{prog} & jatatúke &  & jakatúre \\
      Durative     & \acs{dur}  & jatatáke &  & jakatáre \\
      Habitual     & \acs{hab}  & jatatíke &  & jakatíre \\
      \bottomrule
    \end{tabular}
  }\\
  \subfloat[Non-finite stems]{%
    \begin{tabular}{BFl -c -c -c}
      \toprule
      \SetRowStyle{\bfseries} & Infinitive & Active Participle & Passive Participle \\
      \midrule
      Stem \SetRowStyle{\itshape} & katére        & ektári & šekatéry \\
      Meaning                     & ask for (sth) & asking & asked  \\
      \bottomrule
    \end{tabular}
  }
  \caption{Pattern V biliteral stems \label{tab:vm_v_biliteral_stems}}
\end{table}


\subsection{Quadriliteral roots}
\label{ssec:vm_v_quadriliteral_roots}

Quadriliteral roots form Pattern IV similarly to Pattern II. The infix
\Q{-at-} is inserted immediately after C₁. 

The infinitive is marked by the pattern *\Q{C₁atC₂uC₃eC₄e}, the active participle
by the pattern *\Q{eC₁atáC₂C₃iC₄}, and the passive participle by the pattern
*\Q{šeC₁atiC₂C₃úC₄}.  


\subsection{Geminate roots}
\label{ssec:vm_v_geminate_roots}

Geminate roots form Pattern V similarly to Pattern III.   The perfective apects are formed with the pattern
*\Q{C₁atV₁C₂C₂V₂}, where V₁ is the inherent vowel and V₂ is one of \Q{-ú-},
\Q{-á-} or \Q{-í-}.

The imperfective aspects use the pattern *\Q{jaC₁atV₂ːC₂C₂V₁} in the indicative,
replacing the final vowel with \Q{-e} to form the modal stem. 

The infinitive is formed with the pattern *\Q{C₁atC₂uC₂e}, the active participle
with *\Q{eC₁atáC₂iC₂}, and the participle with *\Q{šeC₁atiC₂úC₂}.  


\subsection{Defective Roots}
\label{ssec:vm_v_defective_roots}

Defective roots in Pattern V follow the same phonological assimilation rules
as have previously described. 


\section{Pattern VI: Causative Reflexive}
\label{sec:vm_pattern_vi}

Pattern VI is the \emph{causative reflexive} stem, and generally functions as the
reflexive counterpart to Patterns II and III.  However, it is often subject to
large amounts of unpredictable semantic and metaphorical drift.  Verbs in this
pattern often have an inchoative sense associated with them. 

It is marked by the infix \Q{-st-} in all forms. 

% Staktab (Active Scale VI), also known as the “reflexive of causative”, is one
% of the trickiest Alashian verb classes as far as semantics are concerned. Its
% most fundamental function is to serve as the reflexive counterpart to 'aktēb,
% the causative verbal scale, such that a verb like στάκταβ staktab (from *ktāb
% “write”) literally means “make oneself write”. However, this basic meaning is
% often subject to large amounts of unpredictable metaphorical and semantic
% drift; in this case, the verb στάκταβ staktab is more commonly used to mean
% “not procrastinate, not put off” (whether or not actual writing is involved).
% In particular Active Scale VI often has an inchoative sense.

%%   
%%   - istorkut “not procrastinate”
%%   - istovrun “deserve”
%%   - istopphut “make oneself speak”
%%   - istodsut “learn”
%%   - istótuk “curse oneself, curse own luck”
%%   - istolkui “deceive oneself”

\subsection{Triliteral Roots}
\label{ssec:vm_vi_triliteral_roots}

Triliteral roots form the perfective indicative aspects with the pattern
*\Q{istV₁C₁C₂V₂C₃}, where V₁ is the inherent root vowel and V₂ is one of
\Q{-u-}, \Q{-a-} or \Q{-i-} for the various subtypes.  The modal perfective
aspects append the suffix \Q{-e}. 

The imperfective aspects are formed with the pattern *\Q{jastV₁C₁V₂ːC₂C₃a},
where V₁ is the inherent root vowel and V₂ is \Q{-ú-} for the progressive
aspect, \Q{-á-} for the durative aspect, and \Q{-í-} for the habitual aspect.
The modal conjugations are formed by replacing the final \Q{-a} of the
indicative stems with \Q{-e}. 

The infinitive is formed with the pattern *\Q{istuC₁C₂eC₃e}; the active participle
with the pattern *\Q{estáC₁C₂iC₃} and the passive participle with the pattern
*\Q{šestiC₁C₂uC₃}. 

Examples of triliteral stems in Pattern VI are given in \cref{tab:vm_vi_triliteral_stems}. 

\begin{table}[h!]\small\capstart
  \centering
  \subfloat[Aspectual stems]{%
    \begin{tabular}{BFl Sl -c -c}
      \toprule
      & & \multicolumn{2}{c}{\Q{istodsut} “learn”} \\
      \SetRowStyle{\bfseries} Aspect & & Indicative & Modal \\
      \midrule
      Perfective   & \acs{perf} & istodsut   & istodsute \\
      Experiential & \acs{exp}  & istodsat   & istodsate \\
      Momentane    & \acs{momt} & istodsit   & istodsite \\
      Progressive  & \acs{prog} & jastodústa & jastodúste \\
      Durative     & \acs{dur}  & jastodásta & jastodáste \\
      Habitual     & \acs{hab}  & jastodísta & jastodíste \\
      \bottomrule
    \end{tabular}
  }\\
  \subfloat[Non-finite stems]{%
    \begin{tabular}{BFl -c -c -c}
      \toprule
      \SetRowStyle{\bfseries} & Infinitive & Active Participle & Passive Participle \\
      \midrule
      Stem \SetRowStyle{\itshape} & istudsete & estádsit & šestidsyt \\
      Meaning                     & learn    & learning & learned \\
      \bottomrule
    \end{tabular}
  }
  \caption{Pattern VI triliteral stems \label{tab:vm_vi_triliteral_stems}}
\end{table}


\subsection{Biliteral Roots}
\label{ssec:vm_vi_biliteral_roots}

Biliteral roots form the perfective aspects by the pattern *\Q{istV₁C₁V₂C₂},
where V₁ is the short inherent vowel and V₂ is one of \Q{-u-}, \Q{-a-} or
\Q{-i-}.  The imperfective stems use the pattern *\Q{jastV₂ːC₁V₁C₂}, again with
V₁ as the short inherent vowel and V₂ one of \Q{-ú-}, \Q{-á-} or \Q{-í-}.  Both
aspects form the modal stem by suffixing with \Q{-e}. 

The infinitive is formed with the pattern *\Q{istaC₁VːC₂e}; the active
participle with the pattern *\Q{estáC₁iC₂} and the passive participle with the
pattern *\Q{šestiC₁yC₂}. 

Examples of biliteral stems are given in \cref{tab:vm_vi_biliteral_stems}. 

\begin{table}[h!]\small\capstart
  \centering
  \subfloat[Aspectual stems]{%
    \begin{tabular}{BFl Sl -c -c}
      \toprule
      & & \multicolumn{2}{c}{\Q{istamur} “reflect”} \\
      \SetRowStyle{\bfseries} Aspect & & Indicative & Modal \\
      \midrule
      Perfective   & \acs{perf} & istamur  & istamure  \\
      Experiential & \acs{exp}  & istamar  & istamare  \\
      Momentane    & \acs{momt} & istamir  & istamire  \\
      Progressive  & \acs{prog} & jastúmar & jastúmare  \\
      Durative     & \acs{dur}  & jastámar & jastámare  \\
      Habitual     & \acs{hab}  & jastímar & jastímare  \\
      \bottomrule
    \end{tabular}
  }\\
  \subfloat[Non-finite stems]{%
    \begin{tabular}{BFl -c -c -c}
      \toprule
      \SetRowStyle{\bfseries} & Infinitive & Active Participle & Passive Participle \\
      \midrule
      Stem \SetRowStyle{\itshape} & istamáre & estámir    & šestimyr \\
      Meaning                     & reflect  & reflecting & reflected \\
      \bottomrule
    \end{tabular}
  }
  \caption{Pattern VI biliteral stems \label{tab:vm_vi_biliteral_stems}}
\end{table}
    

\subsection{Quadriliteral roots}
\label{ssec:vm_vi_quadriliteral_roots}

Quadriliteral roots form Pattern VI similarly to Pattern II. The prefix
\Q{ista-} is inserted immediately before C₁. 
 
The infinitive is marked by the pattern *\Q{istaC₁uC₂C₃eC₄}, the active participle
by the pattern *\Q{istaC₁C₂VːC₃iC₄}, and the passive participle by the pattern
*\Q{šestiC₁C₂C₃úC₄}.  
 
 
\subsection{Geminate roots}
\label{ssec:vm_vi_geminate_roots}

Geminate roots form Pattern VI similarly to biliteral roots, albeit with the
geminated final root consonant. The perfective aspects are formed with the
pattern *\Q{istV₁C₁V₂C₂C₂}, where V₁ is the short inherent vowel and V₂ is one
of \Q{-u-}, \Q{-a-} or \Q{-i-}.  The imperfective stems use the pattern
*\Q{jastV₂ːC₁V₁C₂C₂}, again with V₁ as the short inherent vowel and V₂ one of
\Q{-ú-}, \Q{-á-} or \Q{-í-}.  Both aspects form the modal stem by suffixing
with \Q{-e}. 

The infinitive is formed with the pattern *\Q{istaC₁uC₂C₂e}; the active
participle with the pattern *\Q{estáC₁C₂iC₂} and the passive participle with the
pattern *\Q{šestiC₁C₂yC₂}. 

Examples of biliteral stems are given in \cref{tab:vm_vi_geminate_stems}. 

\begin{table}[h!]\small\capstart
  \centering
  \subfloat[Aspectual stems]{%
    \begin{tabular}{BFl Sl -c -c}
      \toprule
      & & \multicolumn{2}{c}{\Q{istavsus} “(begin to) flow”} \\
      \SetRowStyle{\bfseries} Aspect & & Indicative & Modal \\
      \midrule
      Perfective   & \acs{perf} & istyvuss  & istyvusse  \\
      Experiential & \acs{exp}  & istyvass  & istyvasse  \\
      Momentane    & \acs{momt} & istyviss  & istyvisse  \\
      Progressive  & \acs{prog} & jastúvyss & jastúvysse  \\
      Durative     & \acs{dur}  & jastávyss & jastávysse  \\
      Habitual     & \acs{hab}  & jastívyss & jastívysse  \\
      \bottomrule
    \end{tabular}
  }\\
  \subfloat[Non-finite stems]{%
    \begin{tabular}{BFl -c -c -c}
      \toprule
      \SetRowStyle{\bfseries} & Infinitive & Active Participle & Passive Participle \\
      \midrule
      Stem \SetRowStyle{\itshape} & istavusse       & estávsis            & šestivsys \\
      Meaning                     & (begin to) flow & (beginning to) flow & (begun to) flow \\
      \bottomrule
    \end{tabular}
  }
  \caption{Pattern VI biliteral stems \label{tab:vm_vi_geminate_stems}}
\end{table}


\subsection{Defective Roots}
\label{ssec:vm_vi_defective_roots}

Defective roots in Pattern VI follow the same phonological assimilation rules
as have previously described. 


\section{Pattern VII: Passive Reflexive}
\label{sec:vm_pattern_vii}

Pattern VII is the \emph{passive reflexive} stem, and commonly used to form
anticausative verbs.  It also has a number of irregular uses \tbw\emph{such as…?}
%
%%   Pattern V roots include: 
%%   - intoCCuC
%%   - intorkut “subscribed”  jantorúkta

It is marked by the infix \Q{-nt-} in all forms.


\subsection{Triliteral Roots}
\label{ssec:vm_vii_triliteral_roots}

Triliteral roots form the perfective aspects with the pattern
*\Q{intV₁C₁C₂V₂C₃}, where V₁ is the inherent root vowel and V₂ is one of
\Q{-u-}, \Q{-a-} or \Q{-i-} for the various subtypes.  The modal perfective
aspects append the suffix \Q{-e}.

The imperfective aspects are formed with the pattern *\Q{jantV₁C₁V₂C₂C₃a},
where V₁ is the inherent root vowel, and V₂ is one of \Q{-ú-}, \Q{-a-} or
\Q{-i-} for the progressive, durative or habitual aspects.  The modal
imperfective aspects replace the final \Q{-a} with \Q{-e}.


\section{Pattern VIII: Stative} \label{sec:vm_pattern_viii}

Pattern VIII is the \emph{stative} stem, used to form stative verbs and verbs
that describe attributes and qualities.  Most adjective-like words are formed
from this pattern, such as \Q{ivlešu} “tall” (from \Q{veluš} “grow”).


\subsection{Triliteral Roots}
\label{ssec:vm_viii_triliteral_roots}

Triliteral roots form the perfective aspects with the pattern *\Q{iC₁C₂V₁C₃V₂},
where V₁ is the inherent root vowel and V₂ is one of \Q{-u}, \Q{-a} or \Q{-i}
for the various subtypes.  The perfective aspects lack a distinct modal stem. 

The imperfective aspects are formed with the pattern *\Q{jeC₁V₂ːC₂C₃a}, where
V₂ is \Q{-ú-} for the progressive aspect, \Q{-á-} for the durative aspect, and
\Q{-í-} for the habitual aspect.  The modal conjugations are formed by
replacing the final \Q{-a} of the indicative stems with \Q{-e}. 

The infinitive is formed with the pattern *\Q{iC₁C₂eC₃e} and the passive
participle with the pattern *\Q{šeiC₁C₂uC₃}; Pattern VIII verbs lack an active
participle.  

Examples of triliteral stems in Pattern VIII are given in \cref{tab:vm_viii_triliteral_stems}. 

\begin{table}[h!]\small\capstart
  \centering
  \subfloat[Aspectual stems]{%
    \begin{tabular}{BFl Sl -c -c}
      \toprule
      & & \multicolumn{2}{c}{\Q{iksetu} “be ready”} \\
      \SetRowStyle{\bfseries} Aspect & & Indicative & Modal \\
      \midrule
      Perfective   & \acs{perf} & iksetu  & iksetu \\
      Experiential & \acs{exp}  & ikseta  & ikseta \\
      Momentane    & \acs{momt} & ikseti  & ikseti \\
      Progressive  & \acs{prog} & jekústa & jekúste \\
      Durative     & \acs{dur}  & jekásta & jekáste \\
      Habitual     & \acs{hab}  & jekísta & jekíste \\
      \bottomrule
    \end{tabular}
  }\\
  \subfloat[Non-finite stems]{%
    \begin{tabular}{BFl -c -c}
      \toprule
      \SetRowStyle{\bfseries}     & Infinitive & Passive Participle \\
      \midrule
      Stem \SetRowStyle{\itshape} & iksete     & šeiksyt \\
      Meaning                     & ready      & readied \\
      \bottomrule
    \end{tabular}
  }
  \caption{Pattern VIII triliteral stems \label{tab:vm_viii_triliteral_stems}}
\end{table}


\subsection{Biliteral Roots}
\label{ssec:vm_viii_biliteral_roots}

Biliteral roots form the perfective aspects with the pattern *\Q{C₁iC₂V₂},
where V₂ is one of \Q{-u}, \Q{-a} or \Q{-i} for the various subtypes.  The
perfective aspects lack a distinct modal stem. 

The imperfective aspects are formed with the pattern *\Q{jeC₁V₂ːC₂V₁}, where V₁
is the inherent vowel and V₂ is \Q{-ú-} for the progressive aspect, \Q{-á-} for
the durative aspect, and \Q{-í-} for the habitual aspect.  The modal
conjugations are formed by replacing the final vowel of the indicative stems
with \Q{-e}. 

% The infinitive is formed with the pattern *\Q{iC₁C₂eC₃} and the passive
% participle with the pattern *\Q{šeiC₁C₂uC₃}.  

% Examples of biliteral stems in Pattern VIII are given in \cref{tab:vm_viii_biliteral_stems}. 

% \begin{table}[h!]\small\capstart
%   \centering
%   \subfloat[Aspectual stems]{%
%     \begin{tabular}{BFl Sl -c -c -c -c -c}
%       \toprule
%       & & \multicolumn{2}{c}{\Q{iprešu} “be red” & & \multicolumn{2}{c}{\Q{iksetu} “be ready”} \\
%       \SetRowStyle{\bfseries} Aspect & & Indicative & Modal & Indicative & Modal \\
%       \midrule
%       Perfective   & \acs{perf} & iprešu  & iprešu  & iksetu  & iksetu \\
%       Experiential & \acs{exp}  & ipreša  & ipreša  & ikseta  & ikseta \\
%       Momentane    & \acs{momt} & ipreši  & ipreši  & ikseti  & ikseti \\
%       Progressive  & \acs{prog} & jepúrša & jepúrše & jekústa & jekúste \\
%       Durative     & \acs{dur}  & jepárša & jepárše & jekásta & jekáste \\
%       Habitual     & \acs{hab}  & jepírša & jepírše & jekísta & jekíste \\
%       \bottomrule
%     \end{tabular}
%   }\\
%   \subfloat[Non-finite stems]{%
%     \begin{tabular}{BFl -c -c}
%       \toprule
%       \SetRowStyle{\bfseries}     & Infinitive & Passive Participle \\
%       \midrule
%       Stem \SetRowStyle{\itshape} & ipreš      & šeipruš \\
%       Meaning                     & red        & red \\
%       \bottomrule
%     \end{tabular}
%   }
%   \caption{Pattern VIII biliteral stems \label{tab:vm_viii_biliteral_stems}}
% \end{table}


\subsection{Geminate Roots}
\label{ssec:vm_viii_geminate_roots}

Geminate roots behave like biliteral roots in Pattern VIII.  The perfective
aspects are formed with the pattern *\Q{C₁iC₂C₂V₂}, where V₂ is one of \Q{-u},
\Q{-a} or \Q{-i} for the various subtypes.  

The imperfective aspects are formed with the pattern *\Q{jeC₁V₂ːC₂C₂i}, where
V₁ is the inherent vowel and V₂ is \Q{-ú-} for the progressive aspect, \Q{-á-}
for the durative aspect, and \Q{-í-} for the habitual aspect.  The modal
conjugations are formed by replacing the final vowel of the indicative stems
with \Q{-e}. 

The infinitive is formed with the pattern *\Q{iC₁eC₂C₂e} and the passive
participle with the pattern *\Q{šiC₁yC₂C₂}.  

\begin{table}[h!]\small\capstart
  \centering
  \subfloat[Aspectual stems]{%
    \begin{tabular}{BFl Sl -c -c }
      \toprule
      & & \multicolumn{2}{c}{\Q{zillu} “green”}  \\
      \SetRowStyle{\bfseries} Aspect & & Indicative & Modal \\
      \midrule
      Perfective   & \acs{perf} & zillu   & zillu  \\
      Experiential & \acs{exp}  & zilla   & zilla  \\
      Momentane    & \acs{momt} & zilli   & zilli  \\
      Progressive  & \acs{prog} & jezúlli & jezúlle \\
      Durative     & \acs{dur}  & jezálli & jezálle \\
      Habitual     & \acs{hab}  & jezílli & jezílle \\
      \bottomrule
    \end{tabular}
  }\\
  \subfloat[Non-finite stems]{%
    \begin{tabular}{BFl -c -c}
      \toprule
      \SetRowStyle{\bfseries}     & Infinitive & Passive Participle \\
      \midrule
      Stem \SetRowStyle{\itshape} & izelle & šizyll \\
      Meaning                     & green  & green \\
      \bottomrule
    \end{tabular}
  }
  \caption{Pattern VIII geminate stems \label{tab:vm_viii_geminate_stems}}
\end{table}


\subsection{Defective Roots}
\label{ssec:vm_viii_defective_roots}

Defective roots in Pattern VI follow the same phonological assimilation rules
as have been previously described.  Some examples are listed in
\cref{tab:vm_viii_defective_stems}. 


\begin{table}[h!]\small\capstart
  \centering
  \subfloat[Aspectual stems]{%
    \begin{tabular}{BFl Sl -c -c }
      \toprule
      & & \multicolumn{2}{c}{\Q{íveru} “be good”} \\
      \SetRowStyle{\bfseries} Aspect & & Indicative & Modal \\
      \midrule
      Perfective   & \acs{perf} & íveru  & íveru  \\
      Experiential & \acs{exp}  & ívera  & ívera  \\
      Momentane    & \acs{momt} & íveri  & íveri  \\
      Progressive  & \acs{prog} & jehúvra & jehúvre \\
      Durative     & \acs{dur}  & jehávra & jehávre \\
      Habitual     & \acs{hab}  & jehívra & jehívre \\
      \bottomrule
    \end{tabular}
  }\\
  \subfloat[Non-finite stems]{%
    \begin{tabular}{BFl -c -c}
      \toprule
      \SetRowStyle{\bfseries}     & Infinitive & Passive Participle \\
      \midrule
      Stem \SetRowStyle{\itshape} & ívere     & šévyr \\
      Meaning                     & good      & good \\
      \bottomrule
    \end{tabular}
  }
  \caption{Pattern VIII defective stems \label{tab:vm_viii_defective_stems}}
\end{table}


\clearpage
\section{Aspect}
\label{sec:vm_aspect}

Qevesa verbal morphology indicates aspect instead of tense, to the extent that
there is no means to indicate tense on the verb phrase; the closest
approximation is periphrastically by means of adverbial phrases referring to
time. 


\subsection{Perfective}
\label{vp:ssec_perfective}

The perfective aspect indicate activities viewed as a single whole.  It is
typically used to speak of singular events completed in the past, but may also
be used to speak of actions without internal structure.

\begin{exe}
  \ex \Q{Kesselanti tékujen}
  \gll Kessel-anti ték-u-jen\\
  Kessel\textsc{-all} go\textsc{-perf-1sg.agt}\\
  \glt I went to Kessel.
  \ex \Q{Mi kori lamiztivaš márun.}
  \gll Mi-∅ kori lamizti-v-aš már-u-n\\ 
  \textsc{3sg-dir} three ballgame\textsc{-du-abs} see\textsc{-perf-3sg.agt}\\
  \glt He has watched three ballgames.
\end{exe}

% I wrote / I have written
% I was an architect (and still am, depending on context).


\subsection{Experiential}
\label{vp:ssec_experiential}

The experiential aspect ascribes to a subject the property of having
experienced the event.  There is some overlap between the perfective and
experiential aspects, but the experiential carries connotations of
‘completeness’ that the perfective does not.   

\begin{exe}
  \ex \Q{Mi kori lamiztivaš máran.}
  \gll Mi-∅ kori lamizti-v-aš már-a-n\\ 
  \textsc{3sg-dir} three ballgame\textsc{-du-abs} see\textsc{-exp-3sg.agt}\\
  \glt He has watched three ballgames [in his entire life].
  \ex \Q{Kovelnapalli a póriš máratan.}
  \gll ko-velnapa-lli a póri-š már-a-tan\\
  \textsc{prox}-tomorrow-\textsc{ess} \textsc{def} city\textsc{-abs} see\textsc{-exp-2sg.agt}\\
  \glt Tomorrow you will have seen [everything in] the city.
\end{exe}

\subsection{Momentane}
\label{vp:ssec_momentane}

The momentane aspect indicates brief single-time activities or states. 

% A bolt of lighting struck the tree.
% The mouse squeaked.

\subsection{Progressive and Durative}
\label{vp:ssec_progressive_durative}

The progressive aspect indicates ongoing actions with a change of state.  

\begin{exe}
  \ex\label{ex:vm_putting_on_clothes} \Q{Veráninaš javrúnen.}
  \gll verán-in-aš javrún-en\\
  clothes\textsc{-part-abs} wear\textsc{\bs prog-1sg.agt}\\
  \glt I am putting on clothes.
\end{exe}

The durative aspect indicates ongoing actions without a change of state, or
actions which last some time.

\begin{exe}
  \ex\label{ex:vm_wearing_clothes} \Q{Veráninaš javránen.}
  \gll verán-in-aš javrán-en\\
  clothes\textsc{-part-abs} wear\textsc{-dur-1sg.agt}\\
  \glt I am wearing clothes.
\end{exe}

There are a number of verb patterns that imply either the progressive or the
durative as their imperfective aspect, or have subtly different meanings
depending on which is used.  Adjectival verbs use the progressive aspect to
indicate a change to the quality described by the adjective, and the durative
is used to indicate a more-or-less continuous state. 
%Verbs that imply a change of state (such as \Q{navronu} “dress oneself”) will
%use the progressive aspect 

%Some verbs which imply a change of state, such as \Q{navronu} 

% I am putting on clothes.
% I am hanging the painting on the wall.
% It is raining in Kirua:  Kiruazi apšórak.


% I am wearing clothes.
% The picture is hanging on the wall.

% I can't stop sneezing
% sneeze-PASS-SUBJ stop-I

\subsection{Habitual}
\label{vp:ssec_habitual}

The habitual aspect describes actions that occur habitually or intermittently  

% Like the
% progressive, it may also describe intermittent actions, but in a general sense.

% I walk to work (every day).



\section{Verb Mood}
\label{sec:vm_mood_affect}

Qevesa inflects verbs for five basic moods: \emph{indicative}, \emph{mirative},
\emph{conditional}, \emph{optative}, \emph{potential}, and \emph{imperative}.
The indicative mood is marked by separate stems described in the previous
section, and with the exception of the imperative mood, the others are marked
by suffixes appended to the modal stem of the verb.  These suffixes are listed
in \cref{tab:vm_person_marking}.  

\begin{table}[h!]\small\capstart
  \begin{tabular}{BFl-Sl -l -l}
    \toprule
    \multicolumn{2}{fc}{\SetRowStyle{\bfseries}Mood} & Suffix \\
    \midrule
    Mirative    & \acs{mir}  & -l-  \\
    Conditional & \acs{cond} & -z-  \\
    Optative    & \acs{opt}  & -t-  \\
    Potential   & \acs{pot}  & -r-  \\
    \bottomrule
  \end{tabular}
  \caption{Verbal mood suffixes\label{tab:vm_modal_suffixes}}
\end{table}

The imperative mood is marked on the infinitive verb stem rather than the modal
verb stem.  There is a perfective imperative and an imperfective imperative,
both marked with suffixes and prefixes listed in
\cref{tab:vm_imperative_affixes}.  The \Q{-t-} is epenthetic and inserted if
the infinitive stem ends with a vowel. If the infinitive stem begins with a
vowel, this vowel is dropped and the imperfective prefix becomes \Q{já-}. 

\begin{table}[h!]\small\capstart
  \begin{tabular}{BFl-Sl -c -c}
    \toprule
    \multicolumn{2}{fc}{\SetRowStyle{\bfseries}Aspect} & Prefix & Suffix \\
    \midrule
    Perfective   & \acs{perf}.\acs{imp} &     & -(t)um   \\
    Imperfective & \acs{ipfv}.\acs{imp} & ja- & -(t)om   \\
    \bottomrule
  \end{tabular}
  \caption{Imperative affixes\label{tab:vm_imperative_affixes}}
\end{table}

\subsection{Indicative Mood}
\label{ssec:vp_indicative}

The indicative mood is used for factual statements and positive beliefs,
and as such is the default mood.  


\subsection{Mirative Mood}
\label{ssec:vp_mirative}

The mirative mood is used to express surprise and also doubt, irony,
sarcasm.  It is used to express statements contrary to the speaker’s
expectations or state of mind.


\subsection{Conditional Mood}
\label{ssec:vp_conditional}

The conditional mood is used to speak of an event whose realization is
dependent upon another condition. 


\subsection{Optative Mood}
\label{ssec:vp_optative}

The optative mood is used to express hopes, wishes and desires.


\subsection{Potential Mood}
\label{ssec:vp_potential}

The potential mood indicates that, in the opinion of the speaker, the
action or occurrence is considered likely.  It can also be used to express that
one has the ability to do something.


\subsection{Imperative Mood}
\label{ssec:vp_imperative}

The imperative mood is used for commands and requests. 

\section{Person Marking}
\label{sec:vm_person_marking}

Person marking in Qevesa is somewhat complicated by the unusual morphosyntactic
alignment.  It broadly functions as a \emph{trigger system}, in which the thematic
role (agent, patient, or oblique) of the noun marked by the direct case is
encoded in the verb.  

The personal suffixes mark for first, second and third person in singular, dual
and plural numbers, with the first person plural also making a distinction
between inclusive and exclusive.  The inanimate suffixes do not indicate
number, nor are there inanimate suffixes for the agent trigger.  These suffixes
are listed in \cref{tab:vm_person_marking}; the left columns list suffixes that
follow a consonant, and the right columns those that follow a vowel. 

\begin{table}[h!]\small\capstart
  \begin{tabular}{SFl -l -l -l -l -l -l}
    \toprule
    \SetRowStyle{\bfseries} & \multicolumn{2}{-c}{Agent Trigger} & \multicolumn{2}{-c}{Patient Trigger}& \multicolumn{2}{-c}{Oblique Trigger} \\
    & \multicolumn{2}{-c}{\acs{agt}} & \multicolumn{2}{-c}{\acs{pat}}& \multicolumn{2}{-c}{\acs{obl}} \\
    \midrule
    \acs{1p}\acs{sg}           & -en    & -jen  & -eš    & -ješ  & -ek    & -jek  \\
    \acs{2p}\acs{sg}           & -tan   & -tan  & -taš   & -taš  & -tak   & -tak    \\
    \acs{3p}\acs{sg}           & -yn    & -n    & -yš    & -š    & -yk    & -k      \\
    \acs{1p}\acs{du}           & -evyn  & -vyn  & -evyš  & -vyš  & -evyk  & -vyk  \\
    \acs{2p}\acs{du}           & -avtin & -vtin & -avtiš & -vtiš & -avtik & -vtik  \\
    \acs{3p}\acs{du}           & -yván  & -ván  & -yváš  & -váš  & -yvák  & -vák    \\
    \acs{1p}\acs{pl};\acs{inc} & -isen  & -sen  & -iseš  & -seš  & -isek  & -sek    \\
    \acs{1p}\acs{pl};\acs{exc} & -ečen  & -čen  & -ečes  & -češ  & -eček  & -ček   \\
    \acs{2p}\acs{pl}           & -astin & -stin & -astiš & -stiš & -astik & -stik  \\
    \acs{3p}\acs{pl}           & -imsen & -msen & -imseš & -mseš & -imsek & -msek   \\
    \midrule
    \acs{inanim}               &        &       & -oš    & -š    & -ok    & -k \\
    \bottomrule
  \end{tabular}
  \caption{Person marking suffixes\label{tab:vm_person_marking}}
\end{table}


\subsection{Agent Trigger}
\label{ssec:vp_agt_trigger}

The agent trigger indicates that the noun phrase in the direct case is the
voluntary experiencer of an intransitive verb or the agent of a transitive
verb.  This trigger is equivalent to the active voice in other languages.

\begin{exe}
  \ex \Q{Japphútin.}
  \gll japphút-in\\
  speak\textsc{\bs prog-3sg.agt}\\
  \glt She is speaking.
  \ex \Q{Rekáteš jarkúten.}
  \gll rekát-e-š jarkút-en\\
  book\textsc{-indef-abs} write\textsc{\bs prog-1sg.agt}\\
  \glt I am writing a book.
\end{exe}

Generally only animate nouns may be agents; to describe an action involving an
inanimate noun as agent, a construction using the oblique trigger and the
instrumental case is used instead. 


\subsection{Patient Trigger}
\label{ssec:vp_pat_trigger}

The patient trigger indicates that the noun phrase in the direct case is the
involuntary experiencer of an intransitive verb; the patient of a transitive
verb; and the recipient of a ditransitive verb.  This trigger is roughly
equivalent to the passive and mediopassive voices in other languages. 

Only animate nouns may be voluntary agents of intransitive verbs; inanimate
nouns are always marked as involuntary experiencers of intransitive verbs.
Furthermore, some intransitive verbs are always involuntary, regardless of
animacy. 

\begin{exe}
  \ex \Q{Rekáte jem kojuroš.}
  \gll rekát-e-∅ jem kojur-oš\\
  book\textsc{-indef-dir} \textsc{1sg.erg} read\textsc{\bs perf-3sg;inanim.pat}\\
  \glt A book was read by me.
  \ex \Q{Rekáte kojuroš.}
  \gll rekát-e-∅ kojur-oš\\
  book\textsc{-indef-dir} read\textsc{\bs perf-3sg;inanim.pat}\\
  \glt A book was read.
  \ex \Q{Mi náčoruš.}
  \gll mi-∅ náčoru-š\\
  \textsc{3sg-dir} sneeze\textsc{\bs perf-3sg.pat}\\
  \glt He sneezed.
\end{exe}


\subsection{Oblique Trigger}
\label{ssec:vp_obl_trigger}

The oblique trigger indicates that the noun phrase in the direct case is
something other than the agent or patient of a transitive verb.  For
ditransitive verbs it normally indicates the theme or direct object.

Another common use of the oblique trigger is to express an inanimate agent of a
verb. In this case, the noun will be double-marked with both the instrumental
case and the direct case. 

\end{document}
