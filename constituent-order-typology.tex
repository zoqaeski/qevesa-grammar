\documentclass[grammar]{subfiles}
\begin{document}
  \chapter{Constituent Order Typology}
  \label{ch:constituent-order-typology}

  The preceding chapters dealt primarily with the morphology of Qevesa, with only occasional references to principles of usage. All major aspects of word formation have been covered. The focus of this document shifts to syntax: how the language assembles words into meaningful sentences.

  \section{Main Clauses}
  \label{sec:cot_main_clauses}

  Qevesa syntax is fairly fluid, and tends towards being largely left-branching or head-final. The only strict requirement of a sentence is that the verb must occur last, and that the topic, if present, must be first. All other elements may be freely ordered by importance. The general word order is thus \emph{\textsc{topic–comment–verb}}.

  \subsection{Topic Marking}
  \label{ssec:cot_topic_marking}

  Qevesa is a \emph{topic-prominent} language, which means that the topic is semantically the most important argument of the verb. The topic is indicated by the noun phrase in the nominative case, with the syntactic role marked on the verb. Any of the constituent phrases can be marked as the topic; it usually consists of the element that the speaker considers to be the most important.

  Qevesa verbs must agree in person and number with the topic of the sentence. Verbs are marked for the syntactic role of the topic; when this marking indicates a sufficient degree of information, such as a pronoun in the first or second person, the topical phrase may be omitted.

  \section{Verb Phrase}
  \label{sec:cot_verb_phrase}

  Transitive verb phrases in Qevesa typically consist of just a verb.
  \ToBeWritten

  \section{Noun Phrase}
  \label{sec:cot_noun_phrase}

  \section{Adpositional phrase}
  \label{sec:cot_adpositional_phrase}

  \section{Comparative constructions}
  \label{sec:cot_comparative_constructions}

  \section{Questions and interrogative constructions}
  \label{sec:cot_questions}

\end{document}

