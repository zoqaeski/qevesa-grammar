\documentclass[grammar]{subfiles}
\begin{document}
  \chapter{Numerals}
  \label{ch:numerals}

  Qevesa, in common with other Teralo languages, uses a duodecimal or base-12 number system for both integers and fractions.

  \section{Cardinals}
  \label{sec:num_cardinals}

  The base number words are the cardinal numerals. 
  With the exception of a \qevesa{nak} (“zero, none”), the stems for numerals cannot be composed into consonantal roots. 
  The cardinals from 0\dec\ to 21\dec\ are listed in Table~\ref{tab:num_cardinals}.

  \begin{table}[htpb]\small\capstart
        \begin{tabular}{fc -c -c -c}
          \hline
          \SetRowStyle{\bfseries} & Cardinal & Ordinal & Multiplicative & Collective \tnl
          \hline
          0  & nak       & nakto       & nakami	\tnl
          1  & sen       & sento       & senemi	\tnl
          2  & heti      & hetto       & hetimi		\tnl
          3  & koro      & koroto      & koromi	\tnl
          4  & qese      & qeseto      & qesemi	\tnl
          5  & neca      & necato      & necami	\tnl
          6  & zum       & zumto       & zummi	\tnl
          7  & ikuš      & ikušto      & ikušmi	\tnl
          8  & soppi     & soppito     & soppimi	\tnl
          9  & jokka     & jokkato     & jokkami	\tnl
          A  & mieri     & mierito     & mierimi \tnl
          B  & túre      & túreto      & túremi \tnl
          10 & ševa      & ševato      & ševami        \tnl
          11 & ševasen   & ševasento   & ševasenemi	  \tnl
          12 & ševaheti  & ševahetto   & ševahetimi	  \tnl
          13 & ševakoro  & ševakoroto  & ševakoromi	  \tnl
          14 & ševaqese  & ševaqeseto  & ševaqesemi	  \tnl
          15 & ševaneca  & ševanecato  & ševanecami	  \tnl
          16 & ševazum   & ševazumto   & ševazumumi	  \tnl
          17 & ševaikuš  & ševaikušto  & ševaikušumi	  \tnl
          28 & ševasoppi & ševasoppito  & ševasoppimi	  \tnl
          29 & ševajokka & ševajokkato & ševajokkami	  \tnl
          2A & ševamieri & ševamierito & ševamierimi	  \tnl
          2B & ševatúre  & ševatúreto  & ševatúremi	  \tnl
          \hline
        \end{tabular}
      \caption{Cardinal numerals from 0\dec\ to 23\dec\label{tab:num_cardinals}}
  \end{table}

  Numerals from 20\duo\ to B0\duo\ are suffixed with \qevesa{-ša}:

  \begin{exe}
    \ex
    \begin{tabular}[t]{r >{\itshape}l}
      20\duo & hetiša\\
      30\duo & koroša\\
      40\duo & qeseša\\
      50\duo & necaša\\
      70\duo & ikušša\\
      A0\duo & mieriša\\
      BB\duo & túreša-túre\\
    \end{tabular}
  \end{exe}

  Numerals from 100\duo\ to B00\duo\ are suffixed with \qevesa{-toc}:

  \begin{exe}
    \ex
    \begin{tabular}[t]{r >{\itshape}l}
      100\duo & sentoc \\
      200\duo & hettoc \\
      300\duo & korotoc \\
      409\duo & qesetoc-jokka \\
      752\duo & ikuštoc-necaša-heti \\
    \end{tabular}
  \end{exe}

  %\newpage
  Numerals from 1000\duo\ to B000\duo\  use the suffix \qevesa{-síva}:

  \begin{exe}
    \ex
    \begin{tabular}[t]{r >{\itshape}l}
      1000\duo    & sensíva\\
      2000\duo    & hetsíva\\
      4000\duo    & qesesíva\\
      8603\duo    & soppisíva-zumtoc-koro\\
      10,000\duo  & ševasíva\\
      17,029\duo  & ševaikušsíva-hetiša-jokka\\
      50,000\duo  & necašasíva\\
      93,487\duo  & jokkaša-korosíva qesetoc-soppiša-ikuš\\
      100,000\duo & sentocsíva\\
      582,196\duo & necatoc-soppiša-hetsíva sentoc-jokkaša-zum\\
    \end{tabular}
  \end{exe}

  %\newpage
  Numerals from 10\sup6\duo\ to 10\sup{12}\duo−1 are formed by the addition of the suffix \qevesa{-múl}:

  \begin{exe}
    \ex
    \begin{tabular}[t]{r >{\itshape}l}
      1·10\sup6\duo       & semmúl \textup{(*\emph{senmúl} )}\\
      2·10\sup6\duo       & hetimúl\\
      70·10\sup6\duo      & ikuššamúl\\
      300·10\sup6\duo     & korotocmúl\\
      419,203,52A\duo     & qesetoc-ševasoppimúl hettoc-korosíva necatoc-hetiša-mieri\\
      900,000,000,000\duo & jokkatocsívamúl\\
    \end{tabular}
  \end{exe}

  Using this system alone, it is possible to count up to 1BBB,BBB,BBB,BBB\duo, or 
17,832,200,896,511\dec\footnotemark.
  
  \footnotetext{In full, this is \qevesa{ševatúretoc-túreša-túresívamúl túretoc-túreša-túremúl túretoc-túreša-túresíva túretoc-túreša-túre}}

  \section{Ordinals}
  \label{sec:num_ordinals}

  The ordinal numerals are formed by appending the suffix \qevesa{-ik} to the number word.  For large numerals, the suffix is applied to the last word in the sequence.  The ordinals from *0\sup{th} to 23\dec\sup{st} are given in Table~\ref{tab:num_ordinals}.

  \section{Multiplicatives}
  \label{sec:num_multiplicatives}

  Numerals in Qevesa also have a special form for multiplicatives, formed by appending the suffix \qevesa{-mi}.  If the numeral stem ends in a consonant, an epenthetic vowel identical to the nucleus vowel of the previous syllable is inserted.  The multiplicative numbers from 0\dec\ to 23\dec\ are listed in Table~\ref{tab:num_multiplicatives}.



  \begin{exe}
    \ex \emph{EXAMPLES}
  \end{exe}

  \section{Fractions}
  \label{sec:num_fractions}

  Fractions are formed by appending the suffix \qevesa{-Vna} where \textit{V} is the nucleus vowel of the previous syllable.  The fractional numbers from 0\dec\ to 21\dec\ are listed in Table~\ref{tab:num_cardinals}.

  \begin{table}[htpb]\small\capstart
      \subfloat{
        \begin{tabular}{c c}
          \hline
          \multicolumn{2}{c}{\bfseries Fractional} \tnl
          \hline
          *\sup1⁄\sub{0} & *nakana	\tnl
          *\sup1⁄\sub{1} & *senna	\tnl
          \sup1⁄\sub{2} & hetina		\tnl
          \sup1⁄\sub{3} & korona	\tnl
          \sup1⁄\sub{4} & qesena	\tnl
          \sup1⁄\sub{5} & necana	\tnl
          \sup1⁄\sub{6} & zumuna	\tnl
          \sup1⁄\sub{7} & ikušuna	\tnl
          \sup1⁄\sub{8} & soppina	\tnl
          \sup1⁄\sub{9} & jokkana	\tnl
          \sup1⁄\sub{10} & mierina \tnl
          \sup1⁄\sub{11} & túrena \tnl
          \hline
        \end{tabular}}\qquad
      \subfloat{
        \begin{tabular}{c c}
          \hline
          \multicolumn{2}{c}{\bfseries Fractional} \tnl
          \hline
          \sup1⁄\sub{12} & ševana     \tnl
          \sup1⁄\sub{13} & ševasenna	\tnl
          \sup1⁄\sub{14} & ševahetina		\tnl
          \sup1⁄\sub{15} & ševakorona	\tnl
          \sup1⁄\sub{16} & ševaqesena	\tnl
          \sup1⁄\sub{18} & ševanecana	\tnl
          \sup1⁄\sub{17} & ševazumuna	\tnl
          \sup1⁄\sub{19} & ševaikušuna	\tnl
          \sup1⁄\sub{20} & ševasoppina	\tnl
          \sup1⁄\sub{21} & ševajokkana	\tnl
          \sup1⁄\sub{22} & ševamierina	\tnl
          \sup1⁄\sub{23} & ševatúrena	\tnl
          \hline
        \end{tabular}}
      \caption{Fractional numerals from 0\dec\ to 23\dec\label{tab:num_fractional}}
  \end{table}

  \newpage
  The numerator of a fraction precedes the denominator and is in the ordinal form:

  \begin{exe}
    \ex
    \begin{xlist}
      \ex \qevesa{ikušik ševana}
      \glll ikuš-ik ševa-na\\
      seven-\acs{ord} twelve-\acs{frac}\\
      seven twelfth\\
      \glt seven-twelfths
      \ex \qevesa{hetik korona litasevok}
      \glll het-ik koro-na litas-ev-ok\\
      two-\acs{ord} three-\acs{frac} bread-\acs{du}-\acs{gen}\\
      two third bread\\
      \glt two-thirds of bread
    \end{xlist}
  \end{exe}

  If the denominator of a fraction is a compound number, the fractional suffix is appended to the final word in the sequence:

  \begin{exe}
    \ex
    \begin{xlist}
      \ex \qevesa{zumšana}
      \glll zumša-na\\
      sixty-\acs{frac}\\
      sixtieth\\
      \glt (a) sixtieth
      \ex \qevesa{soppík hetišana}
      \glll soppi-ik heti-ša-na\\
      eight-\acs{ord} two-dozen-\acs{frac}\\
      eight twenty-fourths\\
      \glt eight twenty-fourths
    \end{xlist}
  \end{exe}


  More complex fractions {\em are yet to be written about… in particular, I need:
    \begin{itemize}
      \item Integer ± unit fraction
      \item Integer × unit fraction
    \end{itemize}
  }


\end{document}

