\documentclass[grammar]{subfiles}
\begin{document}

\chapter{Historical Phonology and Morphology}
\label{app:history}

\Tbw

\section{The Phonology of Proto-Teranean}
\label{sec:history:phonology}

Proto-Teranean is reconstructed to have had twenty-nine consonants, and three
vowels, which could be either short or long, listed in
\cref{tab:history:phonology}. 

\begin{table}[h!]\small\capstart
  \subfloat[Consonants]{%
  \begin{tabular}{BFc -c -c -c -c -c -c}
    \toprule
    \rowstyle{\bfseries}  & \multirow{2}{*}{Labial} & \multirow{2}{*}{Coronal} & \multicolumn{2}{-c}{Dorsal} & \multirow{2}{*}{Pharyngeal} & \multirow{2}{*}{Glottal} \\
    \rowstyle{\bfseries} &     &           & Palatal & Plain & \\
    \midrule
    Nasal               & *m  & *n        & *ń    & *ŋ   & \\
    Voiceless Plosive   & *p  & *t        & *ḱ    & *k   &       & *ʔ \\
    Glottalised Plosive & *pʼ & *tʼ       & *ḱʼ   & *kʼ  & \\
    Voiced Plosive      & *b  & *d        & *ǵ    & (*g) & \\
    Central Fricative   &     & *s        &       &      & *ħ *ʕ & *h \\
    Lateral  Fricative  &     & *ɬ *tɬ *ɮ & \\
    Liquid              &     & *r *l     & \\
    Approximant         & *w  &           & *j \\
    \bottomrule
  \end{tabular}
}\\
\subfloat[Vowels]{%
  \begin{tabular}{BFl -c -c -c}
    \toprule
    \rowstyle{\bfseries} & Front & Central & Back \\
    \midrule
    Close & *i̯     &        & *u̯ \\
    Mid   & *e *eː &        & *o *oː \\
    Open  &        & *a *aː \\
    \bottomrule
  \end{tabular}
}
  \caption{Proto-Teranean phonemic inventory\label{tab:history:phonology}}
\end{table}

The phonemes /*ǵ/ and /*g/ were in complimentary distribution, with /*ǵ/ only
occurring before front vowels and /*g/ only occurring before back vowels; in
all descendant languages these two phonemes merged into either /ɟ/ or /g/.  All
of the consonants, with the exception of the glottalised consonants and glottal
stop, could be geminate when intervocalic.  A geminate glottalised consonant
was realised as a sequence of plain stop followed by the glottal stop, and the
geminate glottal stop induced lengthening of the previous vowel.

The close vowels /*i̯/ and /*u̯/ appear to have been in free variation with the
semivowel phonemes /*j/ and /*w/.  There are six diphthongs consisting of the
short vowels and the semivowels: /*ej *ew *oj *ow *aj *aw/.

% To aid in consistency and readability, the phonetic representation will be
% modified slightly as follows:
%
% \begin{itemize}
%   \item The lateral fricatives /*ɬ/ and /*ɮ/ will be represented as *ṡ and *ż
%   \item The affricates /*ts/ and /*tɬ/ will be represented as *c and *ċ
%   \item The palatal approximant /*j/ will be represented as *y
% \end{itemize}

\subsection{Eastern Phonological Developments}
\label{ssec:history:eastern_developments}

The phonological history of the Teranean languages tends to be quite regular,
primarily due to the analogical pressures exerted by their morphology.
Conditional sound changes are quite rare, and those that do occur tend to be
levelled by analogy.

\subsubsection{Collapse of Dorsal Series}
\label{sssec:history:east:dorsal_consonants}

The two dorsal series of consonants were the among the first sounds to be lost
throughout the Teranean language family.  In every branch the two series
differentiated to adjacent points of articulation: in the Eastern grouping, of
which Qevesa is a member, the palatal series became true palatal consonants
(and subsequently alveolar-palatal affricates), and the plain series became
velar consonants; in the Western grouping, the plain dorsal consonants became
uvular, and the palatal series became plain velars.  In no descendent family
were the nasal pair /*ń~*ŋ/ maintained as separate phonemes, collapsing into
/ɲ/ in the Eastern branches and /ŋ/ in the Western branch.

\subsubsection{Loss of the Pharyngeals}
\label{sssec:history:east:pharyngeals}

The pharyngeal consonants had always been marginal in the Teranean languages;
no living descendant in either the Western or Eastern branches possesses them.
Their existence, suspected due to the colouring effect they had on adjacent
vowels, was only confirmed by the discovery of clay tablets \tbw

\Tbw what happened?

/ħ, ʔ/ > /h/ > /ː∅/

\subsubsection{Loss of Lateral Fricatives}
\label{sssec:history:east:lateral}

The lateral fricatives were the next sounds to be lost in the Eastern and
Highveld branches.  In both these branches, /*ɬ/ became the laminal retroflex
fricative /ʂ/; both distinguished this fricative from the affricated nature of
/*c/, which by this point had developed into /tɕ/ in the Highland branch.

% Around this time the first vowel shift occurred; some long vowels were lost,
% and a chain shift caused back vowels to rise and front vowels to lower:
%
% \begin{itemize}
%   \item /*ā/ > /*a/ > /o/
%   \item /*ō/ > /*o/ > /u/
%   \item /*ū/ > /*u/ > /*y/ > /i/
%   \item /*ē/ > /*e/ > /*ə/ > /a/
%   \item /*ī/ > /*i/ > /ı/ > /e/
% \end{itemize}
%
% This feature is specific to Highveld languages


\subsubsection{Loss of Glottalised Consonants}
\label{sssec:history:east:glottalised}

The glottalised plosives /*pʼ *tʼ *ḱʼ *kʼ/ are generally believed to have been
ejectives, as this realisation is found in the Western Teranean languages.  At
some point after the split between the Eastern and Western branches, a shift
occurred in the Eastern branch in which the glottalised plosives became
aspirated plosives:

\begin{itemize}
  \item /*pʼ/ > /*pʰ/  
  \item /*tʼ/ > /*tʰ/  
  \item /*cʼ/ > /*cʰ/  
  \item /*kʼ/ > /*kʰ/  
\end{itemize}

This loss of glottalised consonants is considered to be the main defining
feature of the non Western Teranean languages.  The Central Steppe branch was
particularly innovative in that the glottalised plosives became voiced
plosives, triggering a chain shift:

\begin{itemize}
  \item /*pʼ/ > /*b/ > /*p/ > /*pʰ/  
  \item /*tʼ/ > /*d/ > /*t/ > /*tʰ/  
  \item /*tɕʼ/ > /*dʑ/ > /*tɕ/ > /*tɕʰ/  
  \item /*kʼ/ > /*g/ > /*k/ > /*kʰ/  
\end{itemize}

The Highveld branch, closely related to the Eastern branch, also lost the
glottalised plosives via a chain shift, but the glottalised plosives became unaspirated
plosives instead of aspirated plosives as in the Eastern branch:

\begin{itemize}
  \item /*pʼ/ > /*p/ > /*pʰ/  
  \item /*tʼ/ > /*t/ > /*tʰ/  
  \item /*tɕʼ/ > /*tɕ/ > /*tɕʰ/  
  \item /*kʼ/ > /*k/ > /*kʰ/  
\end{itemize}

\subsubsection{First Palatalisation}
\label{sssec:history:east:first_palatalisation}

The next stage in the development of the Eastern Teranean languages was the
first palatalisation, though this is somewhat of a misnomer as the end result
was often the \emph{loss} of the palatal approximant when it followed another
consonant.  An unaspirated plosive followed by a palatal approximant saw the
approximant assimilate towards the point of articulation of the plosive,
occasionally followed by the reduction of the resulting affricate to a simple
fricative.

\begin{itemize}
  \item /*pjV/ > /*pçV/ > /psV/
  \item /*bjV/ > /*bʝV/ > /bzV/
  \item /*tjV/ > /*tsV/ > /sV/
  \item /*djV/ > /*dzV/ > /zV/
  \item /*cjV/ > /*cɕV/ > /ɕːV/
  \item /*ɟjV/ > /*ɟʝV/ > /ʑːV/
  \item /*kjV/ > /*cʰV/ > /tɕV/
  % \item /*ɡjV/ > /*ɟʝV/ > /dʑV/ only occurred in the Central Steppe languages
\end{itemize}

The associated vowels were also fronted, with /*a/ and /*o/ becoming /e/, and
the original /*e/ became /i/.  The original /*u/ was fronted to /y/.  This
affected all clusters involving a stop and a palatal approximant, regardless of
whether they occurred initially or medially.


\subsubsection{Second Plosive Shift}
\label{sssec:history:east:second_plosive_shift}

Old Qevesa had a three way distinction between voiced, unaspirated, and
aspirated plosives.  During the Middle Qevesa period, a second shift occurred
in which the aspirated plosives became unvoiced fricatives and the voiced
plosives became voiced fricatives.  

\begin{itemize}
  \item /*b/ > /v/, /*pʰ/ > /f/
  \item /*d/ > /ð/, /*tʰ/ > /θ/
  \item /*ɟ/ > /ʝ/ > /ʑ~j/, /*cʰ/ > /ɕ~tɕ/
  \item /*kʰ/ > /x/
\end{itemize}

The palatal series diverged slighly from this scheme, as intervocalic aspirated
plosives had become affricates instead of plain fricatives by this point.  As the entire Eastern branch 

% The voiced velar stop continued to shift, becoming a glottal fricative when
% syllable-initial and eliding when syllable-final, inducing lengthening of the
% previous vowel.

\subsubsection{Environment-Driven Vowel Shifts}
\label{sssec:history:east:vowel_shifts}

Short vowels in open syllables underwent the following shift:

\begin{itemize}
  \item /e/ > /i/
  \item /o/ > /u/
\end{itemize}

Short vowels in closed syllables were unaffected, and long vowels in both open
and closed syllables retained the original quality but lost the length
distinction.

Later reduction and simplification of the diphthongs inherited from
Proto-Teranean resulted in the  Long vowels reappearance of phonemic length
distinction in vowels with the monophthongisation of the original diphthongs.

\begin{itemize}
  \item /*aj/ > /*ai/ > /eː/
  \item /*aw/ > /*ɔu/ > /oː/
  \item /*ej/ > /*ei/ > /iː/
  \item /*ew/ > /*eu/ > /uː/
  \item /*oj/ > /*oi/ > /*ui/ > /*yi/ > /yː/
  \item /*ow/ > /*ou/ > /uː/
\end{itemize}

This vowel shift took place over the course of approximately two centuries, and was
largely complete by the time the first palatalisation began to occur.

\Tbw

\subsubsection{Glide Shifts}
\label{sssec:history:east:glide_shifts}

Word-initial \conlang{*w} has generally been unstable in the Eastern Teranean
languages.  In Qevesa, all initial /*w/ were converted to /v/, including
prefixes and clitics.

\Tbw
 
\subsection{Western Phonological Developments}
\label{ssec:history:western_developments}

The Western Teranean languages underwent a different series of shifts.  The
dorsal series in the Western branch underwent a backwards shift, in which the
plain dorsal consonants became uvular, and the palatal series became plain
velars: 

\begin{itemize}
  \item /*ḱ/ > /k/, /*k/ > /q/
  \item /*ḱʼ/ > /kʼ/, /*kʼ/ > /qʼ/
  \item /*ǵ~*g/ > /g/
\end{itemize}

In addition, the voiced uvular stop became a voiced uvular fricative when word
initial, and an unvoiced uvular fricative before unvoiced plosives.  The second
consonant of the root became geminated if it was intervocalic.

Further detail on the Western branch of the Teranean languages is outside the
scope of this document.

% The lateral consonants underwent a chain shift, in which the voiced lateral
% fricative became a lateral approximant, and the original alveolar lateral
% approximant shifted to a velarised lateral approximant before becoming a
% labiovelar approximant: 
%
% \begin{itemize}
%   \item /*tɬ/ > /*ɬ/ > /l/
%   \item /*l/ > /ɫ/ > /w/
% \end{itemize}


% Old Qevesa had the series /pʰ tʰ cʰ kʰ/ which has become /f θ ɕ x/ in Modern
% Qevesa, but I want a sister language to have ejective or glottalised consonants
% that are analogous to those four points of articulation: /pʼ tʼ ͡tsʼ kʼ/ or /pʼ
% tʼ kʼ qʼ/. The protolanguage has to have had one of those systems for the four
% points of articulation /*p *t *ḱ *k/. In some daughter languages /ḱ/ → /c/ →
% /tɕ/ and in others /*ḱ/ → /k/ and /*k/ → /q/. I haven't decided if the
% protolanguage had /*h/ or /*ʔ/ or both.
%
% In Qevesa /*Ch/ → /Cː/, although plosives become the geminated fricative
% analogue; but if the protolanguage had /ʔ/, then /*Cʔ/ could quite easily
% become /Cʼ/ or /Cː/. In both cases /C/ represents any consonant.

% The protolanguage inventory is (at this stage) /*p *t *ḱ *k *pʼ *tʼ *ḱʼ *kʼ *b
% *d *ǵ~*g *m *ń *n *ŋ *s *z *ɬ *ɮ *h~*ʔ *r *l/. The /*ǵ~*g/ pair were unstable
% and collapsed into /*ɟ/ or /*g/, as no daughter language has both. Likewise,
% the /*ń~*n~*ŋ/ triad collapsed into /*ɲ *n/ or /*n *ŋ/, as no daughter language
% has all four of the protolanguage's nasals. In Qevesa, /*ɬ/ → /ʂ~ʃ/ and
% /*ɮ/ → /ʐ~ʑ~ʒ/, but related languages retained (a/the) lateral fricatives
% instead.


\section{The Morphology of Proto-Teranean}
\label{sec:history:proto_teranean:morphology}

Proto-Teranean had verbs and nouns as distinct lexical categories.  Adjectival
roots were typically derived from stative verbs.


\subsection{Verbal Morphology}
\label{ssec:history:pt:verbal_morphology}

Proto-Teranean verbs had two intersecting sets of paradigms: one set of affixes
encoded the subject person-number-gender agreement; and another which encoded
the tense-aspect-mood.  

There were three main classes of verbs, so called due to their inherent vowel.
Those with \conlang{*o} were often transitive, and those with \conlang{*e} were
usually intransitive or stative in nature.  Verbs with \conlang{*a} were less
common, and were also usually intransitive as well.


\subsubsection{Personal Affixes}
\label{sssec:history:pt:vm:subject_agreement}

The subject was indicated by a series of prefixes and suffixes, listed in
\cref{tab:history:pt:verb_person}, which agreed to a varying extent with the
person, number, and occasionally gender of the verbal subject.  The first
person plural had a separate exclusive prefix, and the third person was often
not marked.  The suffixes in the table were most often found with perfective
verbs; imperfective and subjunctive verbs used \conlang{*-e} where the
perfective verbs use \conlang{*-i}.  The plural marker \conlang{*-s} was always
the final suffix added to the conjugated verb stem.


\begin{table}[h!]\small\capstart
  \begin{tabular}{BFc -c -c -c}
    \toprule
    \rowstyle{\bfseries} Person & \multicolumn{2}{-c}{Prefix} & Suffix \\
    \rowstyle{\scshape} & \acs{perf}, \acs{subj} & \acs{ipfv} & \\
    \midrule
    \acs{1sg}           & ma-     & m-     & -∅ \\
    \acs{2sg}           & to-     & t-     & -∅ \\
    \acs{3sg}           & i-, ∅-  & ∅-     & -∅ \\
    \midrule
    \acs{1du}           & we-     & w-     & -i \\
    \acs{2du}           & tē-     & t-     & -a \\
    \acs{3du}           & yē-, ∅- & y-, ∅- & -a \\
    \midrule
    \acs{1pl};\acs{inc} & sē-     & s-     & -is \\
    \acs{1pl};\acs{exc} & dye-    & dy-    & -is \\
    \acs{2pl}           & kyo-    & ky-    & -as \\
    \acs{3pl}           & ya-, ∅- & y-     & -as, -os \\
    \bottomrule
  \end{tabular}
  \caption{Proto-Teranean person affixes\label{tab:history:pt:verb_person}}
\end{table}

\subsubsection{Tense, Mood, and Aspect}
\label{sssec:history:pt:vm:tense_mood_aspect}

There were four basic verbal stems in Proto-Teranean which primarily indicated
aspectual distinctions, as shown in \cref{tab:history:pt:verb_stems}.  The
stative was used as a verbal adjective/participle, generally describing the
state or condition resulting from the action of the verb, and ranged from
passive to resultative to descriptive.  The perfective or past stem was used to
indicate completed actions occurring in the past, and developed into a range of
past tenses in daughter languages.  The imperfective or non-past stem was used
to indicate ongoing or future actions, and developed into various non-past
tenses in daughter languages.  The Eastern languages have retained the original
aspectual distinction, whereas Western languages primarily indicate tense
instead of aspect.

\begin{table}[h!]\small\capstart
  \begin{tabular}{BFc -c -c -c -c}
    \toprule
    \rowstyle{\bfseries} Stem Class & Stative & Perfective & Imperfective & Subjunctive \\
    \midrule
    B-o & ne-pʼorek & pʼōrok-a & a-pʼrōk-e-y & pʼōrk-o-w \\  % cut wood
    B-e & ne-lebeń  & lēboń-a  & a-lbēń-e-y  & lēbń-o-w  \\  % get up, stand
    B-a & ne-ħapeṡ  & ħāpoṡ-a  & a-ħpāṡ-e-y  & ħāpṡ-o-w  \\  % come close, bring close
    \bottomrule
  \end{tabular}
  \caption{Proto-Teranean verb stems\label{tab:history:pt:verb_stems}}
\end{table}

\begin{table}[h!]\small\capstart
  \begin{tabular}{KFl -c -c -c}
    \toprule
    \rowstyle{\bfseries} & Perfective  & Imperfective & Subjunctive  \\
    \rowstyle{\scshape}  & \acs{perf}  & \acs{ipfv}   & \acs{subj}  \\
    \midrule
    \acs{1sg}            & mapʼōroka   & mapʼrōkey    & mapʼōrkow    \\
    \acs{2sg}            & topʼōroka   & tapʼrōkey    & topʼōrkow    \\
    \acs{3sg}            & pʼōroka     & apʼrōkey     & pʼōrkow      \\
    \acs{1du}            & wepʼōroki   & wapʼrōkey    & wepʼōrkew    \\
    \acs{2du}            & tēpʼōroka   & tapʼrōkay    & tēpʼōrkaw    \\
    \acs{3du}            & yepʼōroka   & yapʼrōkay    & yepʼōrkaw    \\
    \acs{1pl};\acs{inc}  & sēpʼōrokis  & sapʼrōkeys   & sēpʼōrkews   \\
    \acs{1pl};\acs{exc}  & dyepʼōrokis & dyapʼrōkeys  & dyepʼōrkews  \\
    \acs{2pl}            & kyopʼōrokas & kyapʼrōkays  & kyopʼōrkaws   \\
    \acs{3pl}            & yapʼōrokas  & yapʼrōkays   & yapʼōrkaws   \\
    \midrule
    \acs{inanim}         & pʼōroko     & apʼrōkoys    & apʼōrkows   \\
    \bottomrule
  \end{tabular}
  \caption{Proto-Teranean verb conjugation for \conlang{pʼōrok} (to cut [wood, etc])\label{tab:history:pt:verb_conjugation_porok}}
\end{table}


\subsubsection{Valency, Transivity and Derived Verbal Patterns}
\label{sssec:history:pt:vm:valency_transivity_verbal_patterns}

Proto-Teranean had a split-ergative alignment, with the ergative-absolutive
alignment in certain circumstances and a nominative-accusative alignment in
others.  Inanimate subjects always received the ergative marking, and animate
subjects showed a split-S alignment loosely based around volition, with some
verbs marked as voluntary, or agentive, and others as involuntary, or
patientive.  

This split was also indicated by the verbal morphology, with separate
conjugation classes for both transitive and intransitive verbs.  A number of
regular derived stems expanded upon the basic, usually intransitive meaning,
producing a variety of verbal patterns that expressed changes in the valency
(passive, causative) or other properties in the verbal scenario (intensive,
frequentative, benefactive).  Each of these stem patterns is labelled using a
mnemonic characteristic of the derivational pattern.

\begin{description}
  \item[B-Stem] This is the basic, underived and simplest pattern.  Most verbs
    had a long vowel in the perfective, imperfective, and subjunctive stems,
    but some irregular verbs did not.
  % \item[D-Stem] This stem, formed by duplication of the second root consonant,
  %   frequently yielded a transitive from a base intransitive stem.  Most
  %   transitive verbs used this stem, and verb roots could generally increase or
  %   decrease their valency by alternating between this stem and the B-stem.
  % \item[L-Stem] A rarer variation on the D-Stem, this stem also marked an increase 
  %   in valency of the basic verb.  Few, if any verbs could possess both a D-Stem and
  %   and L-Stem.
  % \item[R-Stem] 
  \item[N-Stem] This stem is mainly passive in function, and typically reduces
    the valency of a verb by one.  Intransitive verbs following this pattern
    always receive the ergative-absolutive agreement with their subject.  It
    was formed by addition of a prefix \conlang{n-}, and is represented as the
    Stative in \cref{tab:history:pt:verb_stems}.
  \item[S-Stem] This stem increases the valency by one, and typically ascribes
    a causative meaning to the verb.  This stem has the most variation in
    daughter languages, indicated by a prefixed \conlang{š-}, \conlang{s-},
    \conlang{h-} or \conlang{l-}, all of which can be traced back to a
    Proto-Teranean \conlang{ṡ-}.  This stem is believed to derive from an older
    prefix that also indicated an increase in valency, as demonstrated by a
    number of triliteral verbs whose second and third consonants are biliteral
    roots in their own right, and whose meaning is a transitive or causative
    variant of the biliteral root.
  \item[T-Stem] This stem is mediopassive in origin, and typically indicates a
    reflexive or reciprocal action.  It is formed by the addition of a
    \conlang{-t-} infix, usually after the first consonant.  Uniquely among the
    stem types, the T-stem could be combined with the N- and S-Stems, resulting
    in reflexive or reciprocal variants of those derived verbs, though these
    were later subjected to a considerable amount of semantic drift in the
    daughter languages.
    % the T-stem could be combined with the D-, L-, N-, and S-Stems, resulting in
\end{description}

In addition to the derivational patterns, valency-changing operations may also
be have been marked with suffixes on the verb that occurred after the tense and
aspect marking.  The three most common suffixes are \conlang{-n}, \conlang{-ṡ},
and \conlang{-k}, which indicate active, passive, and mediopassive voices.
This marking is considerably rarer, and doesn't occur at all in daughter
languages in the Western branches, leading to some speculation that it is a
later innovation in the Eastern branches developing after lexicalisation of the
prefix derived patterns.

\subsection{Nominal Morphology}
\label{ssec:history:pt:nominal_morphology}

Proto-Teranean nouns were marked for case, number, and possessed an innate
animacy specific to each noun.  There was no grammatical gender, though some
derivational forms did exist to differentiate between male and female entities.

\subsubsection{Animacy and Gender}
\label{sssec:history:pt:nm:animacy}

Nouns were categorised according to animacy, with two broad classes, animate
and inanimate.  Animate nouns include all human beings, most animals (with the
notable exceptions of insects and other invertebrates), deities and their
corresponding natural phenomena; all other nouns were inanimate.  Animacy was a
fixed property of the noun, so nouns could not switch between animate and
inanimate.

Gender marking of nouns was optional and largely derivational in nature, though
a large number of nouns displayed no overt marking of gender despite referring
to male or female entities.  The most common derivational affixes to indicate
male entities were the suffixes *\conlang{-p} and *\conlang{-ħ}; female
entities were usually unmarked, but the suffixes *\conlang{-t} and
*\conlang{-ya} were occasionally used.  These suffixes were often augmented
with a vowel as *\conlang{-op}, *\conlang{-eħ}, or *\conlang{-at}.

Though not productive in any extant languages, there is strong evidence of
additional gender formants beyond just the masculine *\conlang{-p} and feminine
*\conlang{-t}.  These include the markers *\conlang{-ǵ}/*\conlang{-d} of wild
animals (e.g.  \tbw); the marker *\conlang{-s} of domesticated animals (e.g.
\tbw); the markers *\conlang{-r}/*\conlang{-l} of body parts (e.g. \tbw); and a
marker *\conlang{-n} in for spiritual or sacred items.  All of these have since
become inseperable parts of their respective stems and are only evident through
lexical comparison.

\subsubsection{Number}
\label{sssec:history:pt:nm:number}

Proto-Teranean nouns had a three-way distinction between singular, dual, and
plural.  The dual number was always marked by the suffix *\conlang{-eb}, but the
plural may be formed by the addition of suffixes, internal vowel changes, or in
rare cases, both.  The most common plural suffix was *\conlang{-ty}.

\subsubsection{Case}
\label{sssec:history:pt:nm:case}

Proto-Teranean had seven cases: absolutive, ergative, dative, genitive,
instrumental, adverbial, and vocative.  The suffixes were agglutinative in
nature, with an *\conlang{-e-} inserted after a dual or plural suffix.  These suffixes
are listed in \cref{tab:history:pt:case_suffixes}.

\Tbw

\begin{table}[h!]\small\capstart
  \begin{tabular}{BFl -Kc -l -l -l -l}
    \toprule
    \multicolumn{2}{Fc}{\rowstyle{\bfseries}Noun Case} & \multicolumn{2}{-c}{Suffixes} \\
    \rowstyle{\scshape} & & {\acs{sg}} & {\acs{du}, \acs{pl}} \\
    \midrule
    Absolutive   & \acs{abs} & -∅   & -e \\
    Ergative     & \acs{erg} & -ma  & -ema \\
    Dative       & \acs{dat} & -ṡa  & -eṡa \\
    Instrumental & \acs{ins} & -at  & -et \\
    Genitive     & \acs{gen} & -ak  & -ek \\
    Adverbial    & \acs{adv} & -da  & -eda \\
    Vocative     & \acs{voc} & -o   & -∅ \\
    \bottomrule
  \end{tabular}
  \caption{Proto-Teranean noun cases\label{tab:history:pt:case_suffixes}}
\end{table}

\section{Development of the Verbal System}
\label{sec:history:verbal_development}

\Tbw

\subsection{Tense, Mood, and Aspect}
\label{ssec:history:pt:vd:tense_mood_aspect}

The Eastern branch retained the underlying aspectual indications of the
Proto-Teranean verb stems, expanding them by the addition of suffixes which
indicated finer variations in meaning.

\Tbw

\subsection{Valency, Transivity and Derived Verbal Patterns}
\label{ssec:history:pt:vd:valency_verbal_patterns}

\Tbw

\subsection{Personal Affixes}
\label{ssec:history:pt:vd:personal_affixes}

\Tbw

\section{Development of the Nominal System}
\label{sec:history:pt:nominal_development}

\Tbw

\subsection{Animacy}
\label{ssec:history:pt:nd:animacy}

\Tbw

\subsection{Number}
\label{ssec:history:pt:nd:number}

\Tbw

\subsection{Case}
\label{ssec:history:pt:nd:case}

\Tbw

\subsection{Adjectives and Numerals}
\label{ssec:history:pt:nd:adjectives_numerals}

\Tbw

\subsection{Pronouns}
\label{ssec:history:pt:nd:pronouns}

\Tbw

\end{document}
